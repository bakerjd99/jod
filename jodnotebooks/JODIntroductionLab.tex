\documentclass[11pt,letter,landscape]{article} 
%\documentclass[11pt]{article}

    \usepackage[breakable]{tcolorbox}
    \usepackage{parskip} % Stop auto-indenting (to mimic markdown behaviour)
    
    \usepackage{iftex}
    \ifPDFTeX
    	\usepackage[T1]{fontenc}
    	\usepackage{mathpazo}
    \else
    	\usepackage{fontspec}
    \fi

    % Basic figure setup, for now with no caption control since it's done
    % automatically by Pandoc (which extracts ![](path) syntax from Markdown).
    \usepackage{graphicx}
    % Maintain compatibility with old templates. Remove in nbconvert 6.0
    \let\Oldincludegraphics\includegraphics
    % Ensure that by default, figures have no caption (until we provide a
    % proper Figure object with a Caption API and a way to capture that
    % in the conversion process - todo).
    \usepackage{caption}
    \DeclareCaptionFormat{nocaption}{}
    \captionsetup{format=nocaption,aboveskip=0pt,belowskip=0pt}

    \usepackage[Export]{adjustbox} % Used to constrain images to a maximum size
    \adjustboxset{max size={0.9\linewidth}{0.9\paperheight}}
    \usepackage{float}
    \floatplacement{figure}{H} % forces figures to be placed at the correct location
    \usepackage{xcolor} % Allow colors to be defined
    \usepackage{enumerate} % Needed for markdown enumerations to work
    \usepackage{geometry} % Used to adjust the document margins
    \usepackage{amsmath} % Equations
    \usepackage{amssymb} % Equations
    \usepackage{textcomp} % defines textquotesingle
    % Hack from http://tex.stackexchange.com/a/47451/13684:
    \AtBeginDocument{%
        \def\PYZsq{\textquotesingle}% Upright quotes in Pygmentized code
    }
    \usepackage{upquote} % Upright quotes for verbatim code
    \usepackage{eurosym} % defines \euro
    \usepackage[mathletters]{ucs} % Extended unicode (utf-8) support
    \usepackage{fancyvrb} % verbatim replacement that allows latex
    \usepackage{grffile} % extends the file name processing of package graphics 
                         % to support a larger range
    \makeatletter % fix for grffile with XeLaTeX
    \def\Gread@@xetex#1{%
      \IfFileExists{"\Gin@base".bb}%
      {\Gread@eps{\Gin@base.bb}}%
      {\Gread@@xetex@aux#1}%
    }
    \makeatother

    % The hyperref package gives us a pdf with properly built
    % internal navigation ('pdf bookmarks' for the table of contents,
    % internal cross-reference links, web links for URLs, etc.)
    \usepackage{hyperref}
    % The default LaTeX title has an obnoxious amount of whitespace. By default,
    % titling removes some of it. It also provides customization options.
    \usepackage{titling}
    \usepackage{longtable} % longtable support required by pandoc >1.10
    \usepackage{booktabs}  % table support for pandoc > 1.12.2
    \usepackage[inline]{enumitem} % IRkernel/repr support (it uses the enumerate* environment)
    \usepackage[normalem]{ulem} % ulem is needed to support strikethroughs (\sout)
                                % normalem makes italics be italics, not underlines
    \usepackage{mathrsfs}
    

    
    % Colors for the hyperref package
    \definecolor{urlcolor}{rgb}{0,.145,.698}
    \definecolor{linkcolor}{rgb}{.71,0.21,0.01}
    \definecolor{citecolor}{rgb}{.12,.54,.11}

    % ANSI colors
    \definecolor{ansi-black}{HTML}{3E424D}
    \definecolor{ansi-black-intense}{HTML}{282C36}
    \definecolor{ansi-red}{HTML}{E75C58}
    \definecolor{ansi-red-intense}{HTML}{B22B31}
    \definecolor{ansi-green}{HTML}{00A250}
    \definecolor{ansi-green-intense}{HTML}{007427}
    \definecolor{ansi-yellow}{HTML}{DDB62B}
    \definecolor{ansi-yellow-intense}{HTML}{B27D12}
    \definecolor{ansi-blue}{HTML}{208FFB}
    \definecolor{ansi-blue-intense}{HTML}{0065CA}
    \definecolor{ansi-magenta}{HTML}{D160C4}
    \definecolor{ansi-magenta-intense}{HTML}{A03196}
    \definecolor{ansi-cyan}{HTML}{60C6C8}
    \definecolor{ansi-cyan-intense}{HTML}{258F8F}
    \definecolor{ansi-white}{HTML}{C5C1B4}
    \definecolor{ansi-white-intense}{HTML}{A1A6B2}
    \definecolor{ansi-default-inverse-fg}{HTML}{FFFFFF}
    \definecolor{ansi-default-inverse-bg}{HTML}{000000}

    % commands and environments needed by pandoc snippets
    % extracted from the output of `pandoc -s`
    \providecommand{\tightlist}{%
      \setlength{\itemsep}{0pt}\setlength{\parskip}{0pt}}
    \DefineVerbatimEnvironment{Highlighting}{Verbatim}{commandchars=\\\{\}}
    % Add ',fontsize=\small' for more characters per line
    \newenvironment{Shaded}{}{}
    \newcommand{\KeywordTok}[1]{\textcolor[rgb]{0.00,0.44,0.13}{\textbf{{#1}}}}
    \newcommand{\DataTypeTok}[1]{\textcolor[rgb]{0.56,0.13,0.00}{{#1}}}
    \newcommand{\DecValTok}[1]{\textcolor[rgb]{0.25,0.63,0.44}{{#1}}}
    \newcommand{\BaseNTok}[1]{\textcolor[rgb]{0.25,0.63,0.44}{{#1}}}
    \newcommand{\FloatTok}[1]{\textcolor[rgb]{0.25,0.63,0.44}{{#1}}}
    \newcommand{\CharTok}[1]{\textcolor[rgb]{0.25,0.44,0.63}{{#1}}}
    \newcommand{\StringTok}[1]{\textcolor[rgb]{0.25,0.44,0.63}{{#1}}}
    \newcommand{\CommentTok}[1]{\textcolor[rgb]{0.38,0.63,0.69}{\textit{{#1}}}}
    \newcommand{\OtherTok}[1]{\textcolor[rgb]{0.00,0.44,0.13}{{#1}}}
    \newcommand{\AlertTok}[1]{\textcolor[rgb]{1.00,0.00,0.00}{\textbf{{#1}}}}
    \newcommand{\FunctionTok}[1]{\textcolor[rgb]{0.02,0.16,0.49}{{#1}}}
    \newcommand{\RegionMarkerTok}[1]{{#1}}
    \newcommand{\ErrorTok}[1]{\textcolor[rgb]{1.00,0.00,0.00}{\textbf{{#1}}}}
    \newcommand{\NormalTok}[1]{{#1}}
    
    % Additional commands for more recent versions of Pandoc
    \newcommand{\ConstantTok}[1]{\textcolor[rgb]{0.53,0.00,0.00}{{#1}}}
    \newcommand{\SpecialCharTok}[1]{\textcolor[rgb]{0.25,0.44,0.63}{{#1}}}
    \newcommand{\VerbatimStringTok}[1]{\textcolor[rgb]{0.25,0.44,0.63}{{#1}}}
    \newcommand{\SpecialStringTok}[1]{\textcolor[rgb]{0.73,0.40,0.53}{{#1}}}
    \newcommand{\ImportTok}[1]{{#1}}
    \newcommand{\DocumentationTok}[1]{\textcolor[rgb]{0.73,0.13,0.13}{\textit{{#1}}}}
    \newcommand{\AnnotationTok}[1]{\textcolor[rgb]{0.38,0.63,0.69}{\textbf{\textit{{#1}}}}}
    \newcommand{\CommentVarTok}[1]{\textcolor[rgb]{0.38,0.63,0.69}{\textbf{\textit{{#1}}}}}
    \newcommand{\VariableTok}[1]{\textcolor[rgb]{0.10,0.09,0.49}{{#1}}}
    \newcommand{\ControlFlowTok}[1]{\textcolor[rgb]{0.00,0.44,0.13}{\textbf{{#1}}}}
    \newcommand{\OperatorTok}[1]{\textcolor[rgb]{0.40,0.40,0.40}{{#1}}}
    \newcommand{\BuiltInTok}[1]{{#1}}
    \newcommand{\ExtensionTok}[1]{{#1}}
    \newcommand{\PreprocessorTok}[1]{\textcolor[rgb]{0.74,0.48,0.00}{{#1}}}
    \newcommand{\AttributeTok}[1]{\textcolor[rgb]{0.49,0.56,0.16}{{#1}}}
    \newcommand{\InformationTok}[1]{\textcolor[rgb]{0.38,0.63,0.69}{\textbf{\textit{{#1}}}}}
    \newcommand{\WarningTok}[1]{\textcolor[rgb]{0.38,0.63,0.69}{\textbf{\textit{{#1}}}}}
    
    
    % Define a nice break command that doesn't care if a line doesn't already
    % exist.
    \def\br{\hspace*{\fill} \\* }
    % Math Jax compatibility definitions
    \def\gt{>}
    \def\lt{<}
    \let\Oldtex\TeX
    \let\Oldlatex\LaTeX
    \renewcommand{\TeX}{\textrm{\Oldtex}}
    \renewcommand{\LaTeX}{\textrm{\Oldlatex}}
    % Document parameters
    % Document title
    \title{JOD Introduction Lab}
    
    
    
    
    
% Pygments definitions
\makeatletter
\def\PY@reset{\let\PY@it=\relax \let\PY@bf=\relax%
    \let\PY@ul=\relax \let\PY@tc=\relax%
    \let\PY@bc=\relax \let\PY@ff=\relax}
\def\PY@tok#1{\csname PY@tok@#1\endcsname}
\def\PY@toks#1+{\ifx\relax#1\empty\else%
    \PY@tok{#1}\expandafter\PY@toks\fi}
\def\PY@do#1{\PY@bc{\PY@tc{\PY@ul{%
    \PY@it{\PY@bf{\PY@ff{#1}}}}}}}
\def\PY#1#2{\PY@reset\PY@toks#1+\relax+\PY@do{#2}}

\expandafter\def\csname PY@tok@w\endcsname{\def\PY@tc##1{\textcolor[rgb]{0.73,0.73,0.73}{##1}}}
\expandafter\def\csname PY@tok@c\endcsname{\let\PY@it=\textit\def\PY@tc##1{\textcolor[rgb]{0.25,0.50,0.50}{##1}}}
\expandafter\def\csname PY@tok@cp\endcsname{\def\PY@tc##1{\textcolor[rgb]{0.74,0.48,0.00}{##1}}}
\expandafter\def\csname PY@tok@k\endcsname{\let\PY@bf=\textbf\def\PY@tc##1{\textcolor[rgb]{0.00,0.50,0.00}{##1}}}
\expandafter\def\csname PY@tok@kp\endcsname{\def\PY@tc##1{\textcolor[rgb]{0.00,0.50,0.00}{##1}}}
\expandafter\def\csname PY@tok@kt\endcsname{\def\PY@tc##1{\textcolor[rgb]{0.69,0.00,0.25}{##1}}}
\expandafter\def\csname PY@tok@o\endcsname{\def\PY@tc##1{\textcolor[rgb]{0.40,0.40,0.40}{##1}}}
\expandafter\def\csname PY@tok@ow\endcsname{\let\PY@bf=\textbf\def\PY@tc##1{\textcolor[rgb]{0.67,0.13,1.00}{##1}}}
\expandafter\def\csname PY@tok@nb\endcsname{\def\PY@tc##1{\textcolor[rgb]{0.00,0.50,0.00}{##1}}}
\expandafter\def\csname PY@tok@nf\endcsname{\def\PY@tc##1{\textcolor[rgb]{0.00,0.00,1.00}{##1}}}
\expandafter\def\csname PY@tok@nc\endcsname{\let\PY@bf=\textbf\def\PY@tc##1{\textcolor[rgb]{0.00,0.00,1.00}{##1}}}
\expandafter\def\csname PY@tok@nn\endcsname{\let\PY@bf=\textbf\def\PY@tc##1{\textcolor[rgb]{0.00,0.00,1.00}{##1}}}
\expandafter\def\csname PY@tok@ne\endcsname{\let\PY@bf=\textbf\def\PY@tc##1{\textcolor[rgb]{0.82,0.25,0.23}{##1}}}
\expandafter\def\csname PY@tok@nv\endcsname{\def\PY@tc##1{\textcolor[rgb]{0.10,0.09,0.49}{##1}}}
\expandafter\def\csname PY@tok@no\endcsname{\def\PY@tc##1{\textcolor[rgb]{0.53,0.00,0.00}{##1}}}
\expandafter\def\csname PY@tok@nl\endcsname{\def\PY@tc##1{\textcolor[rgb]{0.63,0.63,0.00}{##1}}}
\expandafter\def\csname PY@tok@ni\endcsname{\let\PY@bf=\textbf\def\PY@tc##1{\textcolor[rgb]{0.60,0.60,0.60}{##1}}}
\expandafter\def\csname PY@tok@na\endcsname{\def\PY@tc##1{\textcolor[rgb]{0.49,0.56,0.16}{##1}}}
\expandafter\def\csname PY@tok@nt\endcsname{\let\PY@bf=\textbf\def\PY@tc##1{\textcolor[rgb]{0.00,0.50,0.00}{##1}}}
\expandafter\def\csname PY@tok@nd\endcsname{\def\PY@tc##1{\textcolor[rgb]{0.67,0.13,1.00}{##1}}}
\expandafter\def\csname PY@tok@s\endcsname{\def\PY@tc##1{\textcolor[rgb]{0.73,0.13,0.13}{##1}}}
\expandafter\def\csname PY@tok@sd\endcsname{\let\PY@it=\textit\def\PY@tc##1{\textcolor[rgb]{0.73,0.13,0.13}{##1}}}
\expandafter\def\csname PY@tok@si\endcsname{\let\PY@bf=\textbf\def\PY@tc##1{\textcolor[rgb]{0.73,0.40,0.53}{##1}}}
\expandafter\def\csname PY@tok@se\endcsname{\let\PY@bf=\textbf\def\PY@tc##1{\textcolor[rgb]{0.73,0.40,0.13}{##1}}}
\expandafter\def\csname PY@tok@sr\endcsname{\def\PY@tc##1{\textcolor[rgb]{0.73,0.40,0.53}{##1}}}
\expandafter\def\csname PY@tok@ss\endcsname{\def\PY@tc##1{\textcolor[rgb]{0.10,0.09,0.49}{##1}}}
\expandafter\def\csname PY@tok@sx\endcsname{\def\PY@tc##1{\textcolor[rgb]{0.00,0.50,0.00}{##1}}}
\expandafter\def\csname PY@tok@m\endcsname{\def\PY@tc##1{\textcolor[rgb]{0.40,0.40,0.40}{##1}}}
\expandafter\def\csname PY@tok@gh\endcsname{\let\PY@bf=\textbf\def\PY@tc##1{\textcolor[rgb]{0.00,0.00,0.50}{##1}}}
\expandafter\def\csname PY@tok@gu\endcsname{\let\PY@bf=\textbf\def\PY@tc##1{\textcolor[rgb]{0.50,0.00,0.50}{##1}}}
\expandafter\def\csname PY@tok@gd\endcsname{\def\PY@tc##1{\textcolor[rgb]{0.63,0.00,0.00}{##1}}}
\expandafter\def\csname PY@tok@gi\endcsname{\def\PY@tc##1{\textcolor[rgb]{0.00,0.63,0.00}{##1}}}
\expandafter\def\csname PY@tok@gr\endcsname{\def\PY@tc##1{\textcolor[rgb]{1.00,0.00,0.00}{##1}}}
\expandafter\def\csname PY@tok@ge\endcsname{\let\PY@it=\textit}
\expandafter\def\csname PY@tok@gs\endcsname{\let\PY@bf=\textbf}
\expandafter\def\csname PY@tok@gp\endcsname{\let\PY@bf=\textbf\def\PY@tc##1{\textcolor[rgb]{0.00,0.00,0.50}{##1}}}
\expandafter\def\csname PY@tok@go\endcsname{\def\PY@tc##1{\textcolor[rgb]{0.53,0.53,0.53}{##1}}}
\expandafter\def\csname PY@tok@gt\endcsname{\def\PY@tc##1{\textcolor[rgb]{0.00,0.27,0.87}{##1}}}
\expandafter\def\csname PY@tok@err\endcsname{\def\PY@bc##1{\setlength{\fboxsep}{0pt}\fcolorbox[rgb]{1.00,0.00,0.00}{1,1,1}{\strut ##1}}}
\expandafter\def\csname PY@tok@kc\endcsname{\let\PY@bf=\textbf\def\PY@tc##1{\textcolor[rgb]{0.00,0.50,0.00}{##1}}}
\expandafter\def\csname PY@tok@kd\endcsname{\let\PY@bf=\textbf\def\PY@tc##1{\textcolor[rgb]{0.00,0.50,0.00}{##1}}}
\expandafter\def\csname PY@tok@kn\endcsname{\let\PY@bf=\textbf\def\PY@tc##1{\textcolor[rgb]{0.00,0.50,0.00}{##1}}}
\expandafter\def\csname PY@tok@kr\endcsname{\let\PY@bf=\textbf\def\PY@tc##1{\textcolor[rgb]{0.00,0.50,0.00}{##1}}}
\expandafter\def\csname PY@tok@bp\endcsname{\def\PY@tc##1{\textcolor[rgb]{0.00,0.50,0.00}{##1}}}
\expandafter\def\csname PY@tok@fm\endcsname{\def\PY@tc##1{\textcolor[rgb]{0.00,0.00,1.00}{##1}}}
\expandafter\def\csname PY@tok@vc\endcsname{\def\PY@tc##1{\textcolor[rgb]{0.10,0.09,0.49}{##1}}}
\expandafter\def\csname PY@tok@vg\endcsname{\def\PY@tc##1{\textcolor[rgb]{0.10,0.09,0.49}{##1}}}
\expandafter\def\csname PY@tok@vi\endcsname{\def\PY@tc##1{\textcolor[rgb]{0.10,0.09,0.49}{##1}}}
\expandafter\def\csname PY@tok@vm\endcsname{\def\PY@tc##1{\textcolor[rgb]{0.10,0.09,0.49}{##1}}}
\expandafter\def\csname PY@tok@sa\endcsname{\def\PY@tc##1{\textcolor[rgb]{0.73,0.13,0.13}{##1}}}
\expandafter\def\csname PY@tok@sb\endcsname{\def\PY@tc##1{\textcolor[rgb]{0.73,0.13,0.13}{##1}}}
\expandafter\def\csname PY@tok@sc\endcsname{\def\PY@tc##1{\textcolor[rgb]{0.73,0.13,0.13}{##1}}}
\expandafter\def\csname PY@tok@dl\endcsname{\def\PY@tc##1{\textcolor[rgb]{0.73,0.13,0.13}{##1}}}
\expandafter\def\csname PY@tok@s2\endcsname{\def\PY@tc##1{\textcolor[rgb]{0.73,0.13,0.13}{##1}}}
\expandafter\def\csname PY@tok@sh\endcsname{\def\PY@tc##1{\textcolor[rgb]{0.73,0.13,0.13}{##1}}}
\expandafter\def\csname PY@tok@s1\endcsname{\def\PY@tc##1{\textcolor[rgb]{0.73,0.13,0.13}{##1}}}
\expandafter\def\csname PY@tok@mb\endcsname{\def\PY@tc##1{\textcolor[rgb]{0.40,0.40,0.40}{##1}}}
\expandafter\def\csname PY@tok@mf\endcsname{\def\PY@tc##1{\textcolor[rgb]{0.40,0.40,0.40}{##1}}}
\expandafter\def\csname PY@tok@mh\endcsname{\def\PY@tc##1{\textcolor[rgb]{0.40,0.40,0.40}{##1}}}
\expandafter\def\csname PY@tok@mi\endcsname{\def\PY@tc##1{\textcolor[rgb]{0.40,0.40,0.40}{##1}}}
\expandafter\def\csname PY@tok@il\endcsname{\def\PY@tc##1{\textcolor[rgb]{0.40,0.40,0.40}{##1}}}
\expandafter\def\csname PY@tok@mo\endcsname{\def\PY@tc##1{\textcolor[rgb]{0.40,0.40,0.40}{##1}}}
\expandafter\def\csname PY@tok@ch\endcsname{\let\PY@it=\textit\def\PY@tc##1{\textcolor[rgb]{0.25,0.50,0.50}{##1}}}
\expandafter\def\csname PY@tok@cm\endcsname{\let\PY@it=\textit\def\PY@tc##1{\textcolor[rgb]{0.25,0.50,0.50}{##1}}}
\expandafter\def\csname PY@tok@cpf\endcsname{\let\PY@it=\textit\def\PY@tc##1{\textcolor[rgb]{0.25,0.50,0.50}{##1}}}
\expandafter\def\csname PY@tok@c1\endcsname{\let\PY@it=\textit\def\PY@tc##1{\textcolor[rgb]{0.25,0.50,0.50}{##1}}}
\expandafter\def\csname PY@tok@cs\endcsname{\let\PY@it=\textit\def\PY@tc##1{\textcolor[rgb]{0.25,0.50,0.50}{##1}}}

\def\PYZbs{\char`\\}
\def\PYZus{\char`\_}
\def\PYZob{\char`\{}
\def\PYZcb{\char`\}}
\def\PYZca{\char`\^}
\def\PYZam{\char`\&}
\def\PYZlt{\char`\<}
\def\PYZgt{\char`\>}
\def\PYZsh{\char`\#}
\def\PYZpc{\char`\%}
\def\PYZdl{\char`\$}
\def\PYZhy{\char`\-}
\def\PYZsq{\char`\'}
\def\PYZdq{\char`\"}
\def\PYZti{\char`\~}
% for compatibility with earlier versions
\def\PYZat{@}
\def\PYZlb{[}
\def\PYZrb{]}
\makeatother


    % For linebreaks inside Verbatim environment from package fancyvrb. 
    \makeatletter
        \newbox\Wrappedcontinuationbox 
        \newbox\Wrappedvisiblespacebox 
        \newcommand*\Wrappedvisiblespace {\textcolor{red}{\textvisiblespace}} 
        \newcommand*\Wrappedcontinuationsymbol {\textcolor{red}{\llap{\tiny$\m@th\hookrightarrow$}}} 
        \newcommand*\Wrappedcontinuationindent {3ex } 
        \newcommand*\Wrappedafterbreak {\kern\Wrappedcontinuationindent\copy\Wrappedcontinuationbox} 
        % Take advantage of the already applied Pygments mark-up to insert 
        % potential linebreaks for TeX processing. 
        %        {, <, #, %, $, ' and ": go to next line. 
        %        _, }, ^, &, >, - and ~: stay at end of broken line. 
        % Use of \textquotesingle for straight quote. 
        \newcommand*\Wrappedbreaksatspecials {% 
            \def\PYGZus{\discretionary{\char`\_}{\Wrappedafterbreak}{\char`\_}}% 
            \def\PYGZob{\discretionary{}{\Wrappedafterbreak\char`\{}{\char`\{}}% 
            \def\PYGZcb{\discretionary{\char`\}}{\Wrappedafterbreak}{\char`\}}}% 
            \def\PYGZca{\discretionary{\char`\^}{\Wrappedafterbreak}{\char`\^}}% 
            \def\PYGZam{\discretionary{\char`\&}{\Wrappedafterbreak}{\char`\&}}% 
            \def\PYGZlt{\discretionary{}{\Wrappedafterbreak\char`\<}{\char`\<}}% 
            \def\PYGZgt{\discretionary{\char`\>}{\Wrappedafterbreak}{\char`\>}}% 
            \def\PYGZsh{\discretionary{}{\Wrappedafterbreak\char`\#}{\char`\#}}% 
            \def\PYGZpc{\discretionary{}{\Wrappedafterbreak\char`\%}{\char`\%}}% 
            \def\PYGZdl{\discretionary{}{\Wrappedafterbreak\char`\$}{\char`\$}}% 
            \def\PYGZhy{\discretionary{\char`\-}{\Wrappedafterbreak}{\char`\-}}% 
            \def\PYGZsq{\discretionary{}{\Wrappedafterbreak\textquotesingle}{\textquotesingle}}% 
            \def\PYGZdq{\discretionary{}{\Wrappedafterbreak\char`\"}{\char`\"}}% 
            \def\PYGZti{\discretionary{\char`\~}{\Wrappedafterbreak}{\char`\~}}% 
        } 
        % Some characters . , ; ? ! / are not pygmentized. 
        % This macro makes them "active" and they will insert potential linebreaks 
        \newcommand*\Wrappedbreaksatpunct {% 
            \lccode`\~`\.\lowercase{\def~}{\discretionary{\hbox{\char`\.}}{\Wrappedafterbreak}{\hbox{\char`\.}}}% 
            \lccode`\~`\,\lowercase{\def~}{\discretionary{\hbox{\char`\,}}{\Wrappedafterbreak}{\hbox{\char`\,}}}% 
            \lccode`\~`\;\lowercase{\def~}{\discretionary{\hbox{\char`\;}}{\Wrappedafterbreak}{\hbox{\char`\;}}}% 
            \lccode`\~`\:\lowercase{\def~}{\discretionary{\hbox{\char`\:}}{\Wrappedafterbreak}{\hbox{\char`\:}}}% 
            \lccode`\~`\?\lowercase{\def~}{\discretionary{\hbox{\char`\?}}{\Wrappedafterbreak}{\hbox{\char`\?}}}% 
            \lccode`\~`\!\lowercase{\def~}{\discretionary{\hbox{\char`\!}}{\Wrappedafterbreak}{\hbox{\char`\!}}}% 
            \lccode`\~`\/\lowercase{\def~}{\discretionary{\hbox{\char`\/}}{\Wrappedafterbreak}{\hbox{\char`\/}}}% 
            \catcode`\.\active
            \catcode`\,\active 
            \catcode`\;\active
            \catcode`\:\active
            \catcode`\?\active
            \catcode`\!\active
            \catcode`\/\active 
            \lccode`\~`\~ 	
        }
    \makeatother

    \let\OriginalVerbatim=\Verbatim
    \makeatletter
    \renewcommand{\Verbatim}[1][1]{%
        %\parskip\z@skip
        \sbox\Wrappedcontinuationbox {\Wrappedcontinuationsymbol}%
        \sbox\Wrappedvisiblespacebox {\FV@SetupFont\Wrappedvisiblespace}%
        \def\FancyVerbFormatLine ##1{\hsize\linewidth
            \vtop{\raggedright\hyphenpenalty\z@\exhyphenpenalty\z@
                \doublehyphendemerits\z@\finalhyphendemerits\z@
                \strut ##1\strut}%
        }%
        % If the linebreak is at a space, the latter will be displayed as visible
        % space at end of first line, and a continuation symbol starts next line.
        % Stretch/shrink are however usually zero for typewriter font.
        \def\FV@Space {%
            \nobreak\hskip\z@ plus\fontdimen3\font minus\fontdimen4\font
            \discretionary{\copy\Wrappedvisiblespacebox}{\Wrappedafterbreak}
            {\kern\fontdimen2\font}%
        }%
        
        % Allow breaks at special characters using \PYG... macros.
        \Wrappedbreaksatspecials
        % Breaks at punctuation characters . , ; ? ! and / need catcode=\active 	
        \OriginalVerbatim[#1,codes*=\Wrappedbreaksatpunct]%
    }
    \makeatother

    % Exact colors from NB
    \definecolor{incolor}{HTML}{303F9F}
    \definecolor{outcolor}{HTML}{D84315}
    \definecolor{cellborder}{HTML}{CFCFCF}
    \definecolor{cellbackground}{HTML}{F7F7F7}
    
    % prompt
    \makeatletter
    \newcommand{\boxspacing}{\kern\kvtcb@left@rule\kern\kvtcb@boxsep}
    \makeatother
    \newcommand{\prompt}[4]{
        \ttfamily\llap{{\color{#2}[#3]:\hspace{3pt}#4}}\vspace{-\baselineskip}
    }
    

    
    % Prevent overflowing lines due to hard-to-break entities
    \sloppy 
    % Setup hyperref package
    \hypersetup{
      breaklinks=true,  % so long urls are correctly broken across lines
      colorlinks=true,
      urlcolor=urlcolor,
      linkcolor=linkcolor,
      citecolor=citecolor,
      }
    % Slightly bigger margins than the latex defaults
    
    \geometry{verbose,tmargin=1in,bmargin=1in,lmargin=1in,rmargin=1in}
    
    

\begin{document}
    
    \maketitle
    
    

    
    \hypertarget{jod-introduction-lab}{%
\section{JOD Introduction Lab}\label{jod-introduction-lab}}

\includegraphics{inclusions/jodteenytinycube.png}

    \hypertarget{what-is-jod}{%
\subsubsection{What is JOD?}\label{what-is-jod}}

JOD is a word storage and retrieval system. It is mainly used to
\textbf{\emph{refactor}} and reuse J words.

The basic idea behind JOD is that J programming is best viewed as
organizing collections of \textbf{\emph{words}} to perform a task.
Organized collections of words have a better name:
\textbf{\emph{dictionaries!}}

JOD is a \href{https://code.jsoftware.com/wiki/JAL/User_Guide}{J addon}.
It is installed in the
(\texttt{\textasciitilde{}addons\textbackslash{}general\textbackslash{}jod})
branch of the current J system folder by the
\href{https://code.jsoftware.com/wiki/JAL/Package_Manager}{J package
manager}.

The next lab step initializes the JOD system.

    \begin{tcolorbox}[breakable, size=fbox, boxrule=1pt, pad at break*=1mm,colback=cellbackground, colframe=cellborder]
\prompt{In}{incolor}{1}{\boxspacing}
\begin{Verbatim}[commandchars=\\\{\}]
\PY{c+c1}{NB. display J version}
\PY{l+m+mi}{9}\PY{o}{!}\PY{o}{:}\PY{l+m+mi}{14} \PY{l+s}{\PYZsq{}}\PY{l+s}{\PYZsq{}}
\end{Verbatim}
\end{tcolorbox}

    \begin{Verbatim}[commandchars=\\\{\}]

j903/j64avx2/windows/beta-w/commercial/www.jsoftware.com/2021-12-05T18:25:00/cla
ng-13-0-0/SLEEF=1
    \end{Verbatim}

    \hypertarget{start-jod}{%
\subsubsection{Start JOD}\label{start-jod}}

    \begin{tcolorbox}[breakable, size=fbox, boxrule=1pt, pad at break*=1mm,colback=cellbackground, colframe=cellborder]
\prompt{In}{incolor}{2}{\boxspacing}
\begin{Verbatim}[commandchars=\\\{\}]
\PY{c+c1}{NB. used by this lab}
\PY{n+nv}{require} \PY{l+s}{\PYZsq{}}\PY{l+s}{f}\PY{l+s}{i}\PY{l+s}{l}\PY{l+s}{e}\PY{l+s}{s}\PY{l+s}{ }\PY{l+s}{d}\PY{l+s}{i}\PY{l+s}{r}\PY{l+s}{ }\PY{l+s}{t}\PY{l+s}{a}\PY{l+s}{s}\PY{l+s}{k}\PY{l+s}{\PYZsq{}}

\PY{c+c1}{NB. start jod \PYZhy{} creates master file if necessary}
\PY{n+nv}{load} \PY{l+s}{\PYZsq{}}\PY{l+s}{g}\PY{l+s}{e}\PY{l+s}{n}\PY{l+s}{e}\PY{l+s}{r}\PY{l+s}{a}\PY{l+s}{l}\PY{l+s}{/}\PY{l+s}{j}\PY{l+s}{o}\PY{l+s}{d}\PY{l+s}{\PYZsq{}}

\PY{c+c1}{NB. use portable box drawing characters }
\PY{c+c1}{NB. simplifies rendering notebooks as (*.tex)}
\PY{n+nv}{portchars\PYZus{}ijod\PYZus{}}\PY{o}{=:}\PY{o}{[}\PY{o}{:} \PY{l+m+mi}{9}\PY{o}{!}\PY{o}{:}\PY{l+m+mi}{7} \PY{l+s}{\PYZsq{}}\PY{l+s}{+}\PY{l+s}{+}\PY{l+s}{+}\PY{l+s}{+}\PY{l+s}{+}\PY{l+s}{+}\PY{l+s}{+}\PY{l+s}{+}\PY{l+s}{+}\PY{l+s}{|}\PY{l+s}{\PYZhy{}}\PY{l+s}{\PYZsq{}}\PY{o}{\PYZdq{}}\PY{l+m}{\PYZus{}} \PY{o}{[} \PY{o}{]}
\PY{n+nv}{portchars} \PY{l+s}{\PYZsq{}}\PY{l+s}{\PYZsq{}}

\PY{c+c1}{NB. Verb to show large boxed displays in}
\PY{c+c1}{NB. the notebook without ugly wrapping.}
\PY{n+nv}{sbx}\PY{o}{=:} \PY{l+s}{\PYZsq{}}\PY{l+s}{ }\PY{l+s}{.}\PY{l+s}{.}\PY{l+s}{.}\PY{l+s}{ }\PY{l+s}{\PYZsq{}} \PY{o}{,}\PY{o}{\PYZdq{}}\PY{l+m+mi}{1}\PY{o}{\PYZti{}} \PY{l+m+mi}{73}\PY{o}{\PYZam{}}\PY{o}{\PYZob{}}\PY{o}{.}\PY{o}{\PYZdq{}}\PY{l+m+mi}{1}\PY{o}{@}\PY{o}{\PYZdq{}}\PY{o}{:}

\PY{n+nv}{smoutput} \PY{n+nv}{JODVMD\PYZus{}ajod\PYZus{}} \PY{c+c1}{NB. show JOD version}
\end{Verbatim}
\end{tcolorbox}

    \begin{Verbatim}[commandchars=\\\{\}]
+------------+--+--------------------+
|1.0.22 - dev|35|14 Dec 2021 08:32:08|
+------------+--+--------------------+
    \end{Verbatim}

    \hypertarget{remove-old-lab-dictionaries}{%
\subsubsection{Remove old lab
dictionaries}\label{remove-old-lab-dictionaries}}

JOD is installed without any dictionaries. To use JOD you must create
some dictionaries. This lab uses four example dictionaries
(\texttt{lab}), (\texttt{labdev}), (\texttt{toy}) and
(\texttt{playpen}). JOD dictionaries are created with the
(\texttt{newd}) ``new dictionary'' verb.

Before creating lab dictionaries remove any prior lab dictionaries. This
step defines a utility that will erase dictionaries from default
locations. It is run in the next step.

\textbf{\emph{WARNING: IF THE TEMPORARY LAB DICTIONARIES CONTAIN
INFORMATION YOU CARE ABOUT DO NOT EXECUTE THE NEXT LAB STEP!}}

    \begin{tcolorbox}[breakable, size=fbox, boxrule=1pt, pad at break*=1mm,colback=cellbackground, colframe=cellborder]
\prompt{In}{incolor}{3}{\boxspacing}
\begin{Verbatim}[commandchars=\\\{\}]
\PY{n+nv}{RemoveLabDictionaries\PYZus{}ijod\PYZus{}}\PY{o}{=:} \PY{n+nf}{3 : 0}
\PY{n+nv}{root}\PY{o}{=.} \PY{n+nv}{jpath} \PY{l+s}{\PYZsq{}}\PY{l+s}{\PYZti{}}\PY{l+s}{u}\PY{l+s}{s}\PY{l+s}{e}\PY{l+s}{r}\PY{l+s}{\PYZsq{}}
\PY{n+nl}{if.} \PY{n+nv}{IFWIN} \PY{n+nl}{do.}
  \PY{n+nv}{shell} \PY{l+s}{\PYZsq{}}\PY{l+s}{r}\PY{l+s}{d}\PY{l+s}{ }\PY{l+s}{/}\PY{l+s}{s}\PY{l+s}{ }\PY{l+s}{/}\PY{l+s}{q}\PY{l+s}{ }\PY{l+s}{\PYZdq{}}\PY{l+s}{\PYZsq{}}\PY{o}{,}\PY{n+nv}{root}\PY{o}{,}\PY{l+s}{\PYZsq{}}\PY{l+s}{\PYZbs{}}\PY{l+s}{j}\PY{l+s}{o}\PY{l+s}{d}\PY{l+s}{d}\PY{l+s}{i}\PY{l+s}{c}\PY{l+s}{t}\PY{l+s}{s}\PY{l+s}{\PYZbs{}}\PY{l+s}{l}\PY{l+s}{a}\PY{l+s}{b}\PY{l+s}{d}\PY{l+s}{e}\PY{l+s}{v}\PY{l+s}{\PYZdq{}}\PY{l+s}{\PYZsq{}}
  \PY{n+nv}{shell} \PY{l+s}{\PYZsq{}}\PY{l+s}{r}\PY{l+s}{d}\PY{l+s}{ }\PY{l+s}{/}\PY{l+s}{s}\PY{l+s}{ }\PY{l+s}{/}\PY{l+s}{q}\PY{l+s}{ }\PY{l+s}{\PYZdq{}}\PY{l+s}{\PYZsq{}}\PY{o}{,}\PY{n+nv}{root}\PY{o}{,}\PY{l+s}{\PYZsq{}}\PY{l+s}{\PYZbs{}}\PY{l+s}{j}\PY{l+s}{o}\PY{l+s}{d}\PY{l+s}{d}\PY{l+s}{i}\PY{l+s}{c}\PY{l+s}{t}\PY{l+s}{s}\PY{l+s}{\PYZbs{}}\PY{l+s}{l}\PY{l+s}{a}\PY{l+s}{b}\PY{l+s}{\PYZdq{}}\PY{l+s}{\PYZsq{}}
  \PY{n+nv}{shell} \PY{l+s}{\PYZsq{}}\PY{l+s}{r}\PY{l+s}{d}\PY{l+s}{ }\PY{l+s}{/}\PY{l+s}{s}\PY{l+s}{ }\PY{l+s}{/}\PY{l+s}{q}\PY{l+s}{ }\PY{l+s}{\PYZdq{}}\PY{l+s}{\PYZsq{}}\PY{o}{,}\PY{n+nv}{root}\PY{o}{,}\PY{l+s}{\PYZsq{}}\PY{l+s}{\PYZbs{}}\PY{l+s}{j}\PY{l+s}{o}\PY{l+s}{d}\PY{l+s}{d}\PY{l+s}{i}\PY{l+s}{c}\PY{l+s}{t}\PY{l+s}{s}\PY{l+s}{\PYZbs{}}\PY{l+s}{t}\PY{l+s}{o}\PY{l+s}{y}\PY{l+s}{\PYZdq{}}\PY{l+s}{\PYZsq{}}
  \PY{n+nv}{shell} \PY{l+s}{\PYZsq{}}\PY{l+s}{r}\PY{l+s}{d}\PY{l+s}{ }\PY{l+s}{/}\PY{l+s}{s}\PY{l+s}{ }\PY{l+s}{/}\PY{l+s}{q}\PY{l+s}{ }\PY{l+s}{\PYZdq{}}\PY{l+s}{\PYZsq{}}\PY{o}{,}\PY{n+nv}{root}\PY{o}{,}\PY{l+s}{\PYZsq{}}\PY{l+s}{\PYZbs{}}\PY{l+s}{j}\PY{l+s}{o}\PY{l+s}{d}\PY{l+s}{d}\PY{l+s}{i}\PY{l+s}{c}\PY{l+s}{t}\PY{l+s}{s}\PY{l+s}{\PYZbs{}}\PY{l+s}{p}\PY{l+s}{l}\PY{l+s}{a}\PY{l+s}{y}\PY{l+s}{p}\PY{l+s}{e}\PY{l+s}{n}\PY{l+s}{\PYZdq{}}\PY{l+s}{\PYZsq{}}
  \PY{n+nv}{smoutput} \PY{l+s}{\PYZsq{}}\PY{l+s}{L}\PY{l+s}{a}\PY{l+s}{b}\PY{l+s}{ }\PY{l+s}{t}\PY{l+s}{e}\PY{l+s}{m}\PY{l+s}{p}\PY{l+s}{o}\PY{l+s}{r}\PY{l+s}{a}\PY{l+s}{r}\PY{l+s}{y}\PY{l+s}{ }\PY{l+s}{(}\PY{l+s}{w}\PY{l+s}{i}\PY{l+s}{n}\PY{l+s}{)}\PY{l+s}{ }\PY{l+s}{d}\PY{l+s}{i}\PY{l+s}{c}\PY{l+s}{t}\PY{l+s}{i}\PY{l+s}{o}\PY{l+s}{n}\PY{l+s}{a}\PY{l+s}{r}\PY{l+s}{i}\PY{l+s}{e}\PY{l+s}{s}\PY{l+s}{ }\PY{l+s}{e}\PY{l+s}{r}\PY{l+s}{a}\PY{l+s}{s}\PY{l+s}{e}\PY{l+s}{d}\PY{l+s}{\PYZsq{}}
\PY{n+nl}{elseif.} \PY{n+nv}{IFUNIX} \PY{n+nl}{do.}
  \PY{c+c1}{NB. avoid blanks in paths on Linux and Mac systems}
  \PY{n+nv}{shell} \PY{l+s}{\PYZsq{}}\PY{l+s}{r}\PY{l+s}{m}\PY{l+s}{ }\PY{l+s}{\PYZhy{}}\PY{l+s}{r}\PY{l+s}{f}\PY{l+s}{ }\PY{l+s}{\PYZsq{}}\PY{o}{,}\PY{n+nv}{root}\PY{o}{,}\PY{l+s}{\PYZsq{}}\PY{l+s}{/}\PY{l+s}{j}\PY{l+s}{o}\PY{l+s}{d}\PY{l+s}{d}\PY{l+s}{i}\PY{l+s}{c}\PY{l+s}{t}\PY{l+s}{s}\PY{l+s}{/}\PY{l+s}{l}\PY{l+s}{a}\PY{l+s}{b}\PY{l+s}{d}\PY{l+s}{e}\PY{l+s}{v}\PY{l+s}{\PYZsq{}}
  \PY{n+nv}{shell} \PY{l+s}{\PYZsq{}}\PY{l+s}{r}\PY{l+s}{m}\PY{l+s}{ }\PY{l+s}{\PYZhy{}}\PY{l+s}{r}\PY{l+s}{f}\PY{l+s}{ }\PY{l+s}{\PYZsq{}}\PY{o}{,}\PY{n+nv}{root}\PY{o}{,}\PY{l+s}{\PYZsq{}}\PY{l+s}{/}\PY{l+s}{j}\PY{l+s}{o}\PY{l+s}{d}\PY{l+s}{d}\PY{l+s}{i}\PY{l+s}{c}\PY{l+s}{t}\PY{l+s}{s}\PY{l+s}{/}\PY{l+s}{l}\PY{l+s}{a}\PY{l+s}{b}\PY{l+s}{\PYZsq{}}
  \PY{n+nv}{shell} \PY{l+s}{\PYZsq{}}\PY{l+s}{r}\PY{l+s}{m}\PY{l+s}{ }\PY{l+s}{\PYZhy{}}\PY{l+s}{r}\PY{l+s}{f}\PY{l+s}{ }\PY{l+s}{\PYZsq{}}\PY{o}{,}\PY{n+nv}{root}\PY{o}{,}\PY{l+s}{\PYZsq{}}\PY{l+s}{/}\PY{l+s}{j}\PY{l+s}{o}\PY{l+s}{d}\PY{l+s}{d}\PY{l+s}{i}\PY{l+s}{c}\PY{l+s}{t}\PY{l+s}{s}\PY{l+s}{/}\PY{l+s}{t}\PY{l+s}{o}\PY{l+s}{y}\PY{l+s}{\PYZsq{}}
  \PY{n+nv}{shell} \PY{l+s}{\PYZsq{}}\PY{l+s}{r}\PY{l+s}{m}\PY{l+s}{ }\PY{l+s}{\PYZhy{}}\PY{l+s}{r}\PY{l+s}{f}\PY{l+s}{ }\PY{l+s}{\PYZsq{}}\PY{o}{,}\PY{n+nv}{root}\PY{o}{,}\PY{l+s}{\PYZsq{}}\PY{l+s}{/}\PY{l+s}{j}\PY{l+s}{o}\PY{l+s}{d}\PY{l+s}{d}\PY{l+s}{i}\PY{l+s}{c}\PY{l+s}{t}\PY{l+s}{s}\PY{l+s}{/}\PY{l+s}{p}\PY{l+s}{l}\PY{l+s}{a}\PY{l+s}{y}\PY{l+s}{p}\PY{l+s}{e}\PY{l+s}{n}\PY{l+s}{\PYZsq{}}
  \PY{n+nv}{smoutput} \PY{l+s}{\PYZsq{}}\PY{l+s}{L}\PY{l+s}{a}\PY{l+s}{b}\PY{l+s}{ }\PY{l+s}{t}\PY{l+s}{e}\PY{l+s}{m}\PY{l+s}{p}\PY{l+s}{o}\PY{l+s}{r}\PY{l+s}{a}\PY{l+s}{r}\PY{l+s}{y}\PY{l+s}{ }\PY{l+s}{(}\PY{l+s}{m}\PY{l+s}{a}\PY{l+s}{c}\PY{l+s}{/}\PY{l+s}{l}\PY{l+s}{i}\PY{l+s}{n}\PY{l+s}{u}\PY{l+s}{x}\PY{l+s}{)}\PY{l+s}{ }\PY{l+s}{d}\PY{l+s}{i}\PY{l+s}{c}\PY{l+s}{t}\PY{l+s}{i}\PY{l+s}{o}\PY{l+s}{n}\PY{l+s}{a}\PY{l+s}{r}\PY{l+s}{i}\PY{l+s}{e}\PY{l+s}{s}\PY{l+s}{ }\PY{l+s}{e}\PY{l+s}{r}\PY{l+s}{a}\PY{l+s}{s}\PY{l+s}{e}\PY{l+s}{d}\PY{l+s}{\PYZsq{}}
\PY{n+nl}{elseif.}\PY{n+nl}{do.}
  \PY{n+nv}{smoutput} \PY{l+s}{\PYZsq{}}\PY{l+s}{E}\PY{l+s}{r}\PY{l+s}{a}\PY{l+s}{s}\PY{l+s}{e}\PY{l+s}{ }\PY{l+s}{a}\PY{l+s}{n}\PY{l+s}{y}\PY{l+s}{ }\PY{l+s}{p}\PY{l+s}{r}\PY{l+s}{e}\PY{l+s}{v}\PY{l+s}{i}\PY{l+s}{o}\PY{l+s}{u}\PY{l+s}{s}\PY{l+s}{ }\PY{l+s}{t}\PY{l+s}{e}\PY{l+s}{m}\PY{l+s}{p}\PY{l+s}{o}\PY{l+s}{r}\PY{l+s}{a}\PY{l+s}{r}\PY{l+s}{y}\PY{l+s}{ }\PY{l+s}{l}\PY{l+s}{a}\PY{l+s}{b}\PY{l+s}{ }\PY{l+s}{d}\PY{l+s}{i}\PY{l+s}{c}\PY{l+s}{t}\PY{l+s}{i}\PY{l+s}{o}\PY{l+s}{n}\PY{l+s}{a}\PY{l+s}{r}\PY{l+s}{i}\PY{l+s}{e}\PY{l+s}{s}\PY{l+s}{ }\PY{l+s}{m}\PY{l+s}{a}\PY{l+s}{n}\PY{l+s}{u}\PY{l+s}{a}\PY{l+s}{l}\PY{l+s}{l}\PY{l+s}{y}\PY{l+s}{.}\PY{l+s}{\PYZsq{}}
\PY{n+nl}{end.}
\PY{n+nl}{)}
\end{Verbatim}
\end{tcolorbox}

    \hypertarget{remove-any-prior-lab-dictionaries}{%
\subsubsection{Remove any prior lab
dictionaries}\label{remove-any-prior-lab-dictionaries}}

    \begin{tcolorbox}[breakable, size=fbox, boxrule=1pt, pad at break*=1mm,colback=cellbackground, colframe=cellborder]
\prompt{In}{incolor}{4}{\boxspacing}
\begin{Verbatim}[commandchars=\\\{\}]
\PY{c+c1}{NB. close any dictionaries \PYZhy{} ignore errors}
\PY{l+m+mi}{0} \PY{l+m+mi}{0} \PY{o}{\PYZdl{}} \PY{l+m+mi}{3} \PY{n+nv}{od} \PY{l+s}{\PYZsq{}}\PY{l+s}{\PYZsq{}}

\PY{c+c1}{NB. reset master file }
\PY{n+nv}{dpset} \PY{l+s}{\PYZsq{}}\PY{l+s}{R}\PY{l+s}{E}\PY{l+s}{S}\PY{l+s}{E}\PY{l+s}{T}\PY{l+s}{M}\PY{l+s}{E}\PY{l+s}{\PYZsq{}}

\PY{c+c1}{NB. unregister any lab dictionaries \PYZhy{} ignore errors}
\PY{l+m+mi}{0} \PY{l+m+mi}{0} \PY{o}{\PYZdl{}} \PY{l+m+mi}{3} \PY{n+nv}{regd}\PY{o}{\PYZam{}}\PY{o}{\PYZgt{}} \PY{o}{;}\PY{o}{:}\PY{l+s}{\PYZsq{}}\PY{l+s}{l}\PY{l+s}{a}\PY{l+s}{b}\PY{l+s}{d}\PY{l+s}{e}\PY{l+s}{v}\PY{l+s}{ }\PY{l+s}{l}\PY{l+s}{a}\PY{l+s}{b}\PY{l+s}{ }\PY{l+s}{t}\PY{l+s}{o}\PY{l+s}{y}\PY{l+s}{ }\PY{l+s}{p}\PY{l+s}{l}\PY{l+s}{a}\PY{l+s}{y}\PY{l+s}{p}\PY{l+s}{e}\PY{l+s}{n}\PY{l+s}{\PYZsq{}}

\PY{c+c1}{NB. remove dictionary directories and all contents \PYZhy{} ignore errors}
\PY{n+nv}{RemoveLabDictionaries} \PY{l+m+mi}{0}
\end{Verbatim}
\end{tcolorbox}

    \begin{Verbatim}[commandchars=\\\{\}]
Lab temporary (win) dictionaries erased
    \end{Verbatim}

    \hypertarget{this-step-creates-the-lab-and-labdev-dictionaries}{%
\subsubsection{\texorpdfstring{This step creates the (\texttt{lab}) and
(\texttt{labdev})
dictionaries}{This step creates the (lab) and (labdev) dictionaries}}\label{this-step-creates-the-lab-and-labdev-dictionaries}}

    \begin{tcolorbox}[breakable, size=fbox, boxrule=1pt, pad at break*=1mm,colback=cellbackground, colframe=cellborder]
\prompt{In}{incolor}{5}{\boxspacing}
\begin{Verbatim}[commandchars=\\\{\}]
\PY{c+c1}{NB. close any open dictionaries}
\PY{l+m+mi}{3} \PY{n+nv}{od} \PY{l+s}{\PYZsq{}}\PY{l+s}{\PYZsq{}}

\PY{c+c1}{NB. create (lab) and  (labdev) dictionaries}
\PY{n+nv}{smoutput} \PY{n+nv}{newd} \PY{l+s}{\PYZsq{}}\PY{l+s}{l}\PY{l+s}{a}\PY{l+s}{b}\PY{l+s}{\PYZsq{}}
\PY{n+nv}{smoutput} \PY{n+nv}{newd} \PY{l+s}{\PYZsq{}}\PY{l+s}{l}\PY{l+s}{a}\PY{l+s}{b}\PY{l+s}{d}\PY{l+s}{e}\PY{l+s}{v}\PY{l+s}{\PYZsq{}}

\PY{c+c1}{NB. list available dictionaries}
\PY{n+nv}{sbx} \PY{n+nv}{od} \PY{l+s}{\PYZsq{}}\PY{l+s}{\PYZsq{}}
\end{Verbatim}
\end{tcolorbox}

    \begin{Verbatim}[commandchars=\\\{\}]
+-+---------------------+---+-------------------------------------------+
|1|dictionary created ->|lab|c:/users/john.baker/j903-user/joddicts/lab/|
+-+---------------------+---+-------------------------------------------+
+-+---------------------+------+----------------------------------------------+
|1|dictionary created ->|labdev|c:/users/john.baker/j903-user/joddicts/labdev/|
+-+---------------------+------+----------------------------------------------+
+-+----+------+------+----+----+---+-----+----+--------+---+------+------ {\ldots}
|1|barf|bpcopy|bptest|bugs|docs|gps|image|imex|jacksons|jod|joddev|joddev {\ldots}
+-+----+------+------+----+----+---+-----+----+--------+---+------+------ {\ldots}
    \end{Verbatim}

    \hypertarget{opening-and-closing-dictionaries}{%
\subsubsection{Opening and closing
dictionaries}\label{opening-and-closing-dictionaries}}

The JOD verb for opening and closing dictionaries is (\texttt{od}) or
(open dictionary). JOD verbs are short and easy to type.

(\texttt{od}) can open dictionaries \texttt{READWRITE} and
\texttt{READONLY}. As you might expect \texttt{READONLY} dictionaries
cannot be changed by JOD verbs.

    \begin{tcolorbox}[breakable, size=fbox, boxrule=1pt, pad at break*=1mm,colback=cellbackground, colframe=cellborder]
\prompt{In}{incolor}{6}{\boxspacing}
\begin{Verbatim}[commandchars=\\\{\}]
\PY{c+c1}{NB. open read/write}
\PY{n+nv}{smoutput} \PY{n+nv}{od} \PY{l+s}{\PYZsq{}}\PY{l+s}{l}\PY{l+s}{a}\PY{l+s}{b}\PY{l+s}{d}\PY{l+s}{e}\PY{l+s}{v}\PY{l+s}{\PYZsq{}}

\PY{c+c1}{NB. open read/only}
\PY{n+nv}{smoutput} \PY{l+m+mi}{2} \PY{n+nv}{od} \PY{l+s}{\PYZsq{}}\PY{l+s}{l}\PY{l+s}{a}\PY{l+s}{b}\PY{l+s}{\PYZsq{}}

\PY{c+c1}{NB. close (labdev)}
\PY{l+m+mi}{3} \PY{n+nv}{od} \PY{l+s}{\PYZsq{}}\PY{l+s}{l}\PY{l+s}{a}\PY{l+s}{b}\PY{l+s}{d}\PY{l+s}{e}\PY{l+s}{v}\PY{l+s}{\PYZsq{}}
\end{Verbatim}
\end{tcolorbox}

    \begin{Verbatim}[commandchars=\\\{\}]
+-+--------------+------+
|1|opened (rw) ->|labdev|
+-+--------------+------+
+-+--------------+---+
|1|opened (ro) ->|lab|
+-+--------------+---+
+-+---------+------+
|1|closed ->|labdev|
+-+---------+------+
    \end{Verbatim}

    \hypertarget{some-return-code-basics}{%
\subsubsection{Some return code basics}\label{some-return-code-basics}}

All JOD verbs return boxed list results. The first item is a return
code: (1) good (0) bad. Remaining items are messages and, usually, error
related information. JOD verbs perform extensive argument checking. If
you break a JOD verb please email me (\texttt{bakerjd99@gmail.com}) and
tell me what you did.

    \begin{tcolorbox}[breakable, size=fbox, boxrule=1pt, pad at break*=1mm,colback=cellbackground, colframe=cellborder]
\prompt{In}{incolor}{7}{\boxspacing}
\begin{Verbatim}[commandchars=\\\{\}]
\PY{c+c1}{NB. bad open request}
\PY{n+nv}{smoutput} \PY{n+nv}{od} \PY{l+s}{\PYZsq{}}\PY{l+s}{i}\PY{l+s}{ }\PY{l+s}{a}\PY{l+s}{m}\PY{l+s}{ }\PY{l+s}{a}\PY{l+s}{ }\PY{l+s}{m}\PY{l+s}{i}\PY{l+s}{s}\PY{l+s}{s}\PY{l+s}{i}\PY{l+s}{n}\PY{l+s}{g}\PY{l+s}{ }\PY{l+s}{d}\PY{l+s}{i}\PY{l+s}{c}\PY{l+s}{t}\PY{l+s}{i}\PY{l+s}{o}\PY{l+s}{n}\PY{l+s}{a}\PY{l+s}{r}\PY{l+s}{y}\PY{l+s}{\PYZsq{}}

\PY{c+c1}{NB. bad types}
\PY{n+nv}{smoutput} \PY{n+nv}{od} \PY{l+m+mi}{9}

\PY{c+c1}{NB. bad ranks}
\PY{n+nv}{od} \PY{l+m+mi}{3} \PY{l+m+mi}{3}\PY{o}{\PYZdl{}}\PY{l+s}{\PYZsq{}}\PY{l+s}{b}\PY{l+s}{o}\PY{l+s}{o}\PY{l+s}{\PYZsq{}}
\end{Verbatim}
\end{tcolorbox}

    \begin{Verbatim}[commandchars=\\\{\}]
+-+-------------------------------------------------+
|0|!JOD error: invalid or missing dictionary name(s)|
+-+-------------------------------------------------+
+-+-------------------------------------------------+
|0|!JOD error: invalid or missing dictionary name(s)|
+-+-------------------------------------------------+
+-+-------------------------------------------------+
|0|!JOD error: invalid or missing dictionary name(s)|
+-+-------------------------------------------------+
    \end{Verbatim}

    \hypertarget{online-jod-documentation}{%
\subsubsection{Online JOD
documentation}\label{online-jod-documentation}}

JOD has extensive (\texttt{pdf}) documentation.
\href{https://github.com/jsoftware/general_joddocument/blob/master/pdfdoc/jod.pdf}{JOD
documentation} can be accessed with the (\texttt{jodhelp}) verb.

(\texttt{jodhelp}) spawns a PDF reader task. JOD uses J's configured PDF
reader on Windows and Linux systems and the ``open'' shell command on
Macs.

    \begin{tcolorbox}[breakable, size=fbox, boxrule=1pt, pad at break*=1mm,colback=cellbackground, colframe=cellborder]
\prompt{In}{incolor}{8}{\boxspacing}
\begin{Verbatim}[commandchars=\\\{\}]
\PY{c+c1}{NB. display JOD documentation}
\PY{n+nv}{jodhelp} \PY{l+m+mi}{0}
\end{Verbatim}
\end{tcolorbox}

    \begin{Verbatim}[commandchars=\\\{\}]
+-+-------------------+
|1|starting PDF reader|
+-+-------------------+
    \end{Verbatim}

    \hypertarget{dictionary-paths}{%
\subsubsection{Dictionary paths}\label{dictionary-paths}}

The open dictionaries of JOD define a search and fetch path. The
(\texttt{did}) (dictionary identification) verb lists the path.

    \begin{tcolorbox}[breakable, size=fbox, boxrule=1pt, pad at break*=1mm,colback=cellbackground, colframe=cellborder]
\prompt{In}{incolor}{9}{\boxspacing}
\begin{Verbatim}[commandchars=\\\{\}]
\PY{c+c1}{NB. reopen only (labdev) and (lab)}
\PY{n+nv}{od} \PY{o}{;}\PY{o}{:}\PY{l+s}{\PYZsq{}}\PY{l+s}{l}\PY{l+s}{a}\PY{l+s}{b}\PY{l+s}{d}\PY{l+s}{e}\PY{l+s}{v}\PY{l+s}{ }\PY{l+s}{l}\PY{l+s}{a}\PY{l+s}{b}\PY{l+s}{\PYZsq{}} \PY{o}{[} \PY{l+m+mi}{3} \PY{n+nv}{od} \PY{l+s}{\PYZsq{}}\PY{l+s}{\PYZsq{}}

\PY{c+c1}{NB. show path}
\PY{n+nv}{did} \PY{l+m+mi}{0}
\end{Verbatim}
\end{tcolorbox}

    \begin{Verbatim}[commandchars=\\\{\}]
+-+------+---+
|1|labdev|lab|
+-+------+---+
    \end{Verbatim}

    The dyadic form of (\texttt{did}) returns details about the contents of
each dictionary on the path.

    \begin{tcolorbox}[breakable, size=fbox, boxrule=1pt, pad at break*=1mm,colback=cellbackground, colframe=cellborder]
\prompt{In}{incolor}{10}{\boxspacing}
\begin{Verbatim}[commandchars=\\\{\}]
\PY{c+c1}{NB. did \PYZti{} 0   NB. handy idiom}

\PY{c+c1}{NB. dictionary details}
\PY{l+m+mi}{0} \PY{n+nv}{did} \PY{l+m+mi}{0}
\end{Verbatim}
\end{tcolorbox}

    \begin{Verbatim}[commandchars=\\\{\}]
+-+----------------------------------------------------+
|1|+------+--+-----+-----+-------+-------+------+-----+|
| ||      |--|Words|Tests|Groups*|Suites*|Macros|Path*||
| |+------+--+-----+-----+-------+-------+------+-----+|
| ||labdev|rw|0    |0    |0      |0      |0     |/    ||
| |+------+--+-----+-----+-------+-------+------+-----+|
| ||lab   |rw|0    |0    |0      |0      |0     |/    ||
| |+------+--+-----+-----+-------+-------+------+-----+|
+-+----------------------------------------------------+
    \end{Verbatim}

    \hypertarget{some-object-orientation}{%
\subsubsection{Some object orientation}\label{some-object-orientation}}

The JOD system is a complete and detailed example of object oriented
programming in J. The system consists of a number of classes (prefixed
with \texttt{\textquotesingle{}ajod\textquotesingle{}}). When the system
loads a variety of objects are created. The basic architecture is a main
dictionary object that contains four subobjects. Each open dictionary is
also associated with a directory object. Directory objects are created
and destroyed as required. The following diagram shows JOD's class
structure.

\includegraphics{inclusions/joddot.png}

    \begin{tcolorbox}[breakable, size=fbox, boxrule=1pt, pad at break*=1mm,colback=cellbackground, colframe=cellborder]
\prompt{In}{incolor}{11}{\boxspacing}
\begin{Verbatim}[commandchars=\\\{\}]
\PY{c+c1}{NB. objects beginning with \PYZsq{}ajod\PYZsq{} are the JOD classes.}
\PY{n+nv}{smoutput} \PY{l+m+mi}{80} \PY{n+nv}{list} \PY{n+nv}{conl} \PY{l+m+mi}{0}

\PY{c+c1}{NB. JOD consists of six basic objects and as}
\PY{c+c1}{NB. many directory objects as there are path items.}
\PY{n+nv}{conl} \PY{l+m+mi}{1}
\end{Verbatim}
\end{tcolorbox}

    \begin{Verbatim}[commandchars=\\\{\}]
ajod      ajoddob   ajodmake  ajodstore ajodtools ajodutil  base      ijod
j         jal       jcompare  jdebug    jdefs     jdemo     jfif      jfile
jfiles    jfilesrc  jhs       jijs      jijx      jinter    jlogin    jregex
jsocket   json      jsp       jtask     pplatimg  z
+-+-+-+-+-+-+-+-+
|0|1|2|3|4|5|8|9|
+-+-+-+-+-+-+-+-+
    \end{Verbatim}

    \hypertarget{the-put-dictionary-concept}{%
\subsubsection{The put dictionary
concept}\label{the-put-dictionary-concept}}

The first dictionary on the path is \emph{special}. It is the only
dictionary that can be modified by JOD verbs. Because most dictionary
modifications are put's I call this dictionary the ``put'' dictionary.

It's important to understand that you can use the contents of the other
dictionaries on the path but you cannot change them in any way.

    \begin{tcolorbox}[breakable, size=fbox, boxrule=1pt, pad at break*=1mm,colback=cellbackground, colframe=cellborder]
\prompt{In}{incolor}{12}{\boxspacing}
\begin{Verbatim}[commandchars=\\\{\}]
\PY{c+c1}{NB. first path dictionary is the put dictionary}
\PY{n+nv}{did} \PY{l+m+mi}{0}
\end{Verbatim}
\end{tcolorbox}

    \begin{Verbatim}[commandchars=\\\{\}]
+-+------+---+
|1|labdev|lab|
+-+------+---+
    \end{Verbatim}

    \hypertarget{creating-new-dictionaries}{%
\subsubsection{Creating new
dictionaries}\label{creating-new-dictionaries}}

Before modifying the contents of any dictionary let's create a new
(\texttt{toy}) dictionary and make it the put dictionary.

    \begin{tcolorbox}[breakable, size=fbox, boxrule=1pt, pad at break*=1mm,colback=cellbackground, colframe=cellborder]
\prompt{In}{incolor}{13}{\boxspacing}
\begin{Verbatim}[commandchars=\\\{\}]
\PY{c+c1}{NB. close open dictionaries}
\PY{l+m+mi}{3} \PY{n+nv}{od} \PY{l+s}{\PYZsq{}}\PY{l+s}{\PYZsq{}}

\PY{c+c1}{NB. create (toy)}
\PY{n+nv}{newd} \PY{l+s}{\PYZsq{}}\PY{l+s}{t}\PY{l+s}{o}\PY{l+s}{y}\PY{l+s}{\PYZsq{}}
\end{Verbatim}
\end{tcolorbox}

    \begin{Verbatim}[commandchars=\\\{\}]
+-+---------------------+---+-------------------------------------------+
|1|dictionary created ->|toy|c:/users/john.baker/j903-user/joddicts/toy/|
+-+---------------------+---+-------------------------------------------+
    \end{Verbatim}

    Make (\texttt{toy}) a put dictionary.

    \begin{tcolorbox}[breakable, size=fbox, boxrule=1pt, pad at break*=1mm,colback=cellbackground, colframe=cellborder]
\prompt{In}{incolor}{14}{\boxspacing}
\begin{Verbatim}[commandchars=\\\{\}]
\PY{c+c1}{NB. open toy, labdev and lab \PYZhy{} toy is the put dictionary}
\PY{n+nv}{smoutput} \PY{n+nv}{od} \PY{o}{;}\PY{o}{:}\PY{l+s}{\PYZsq{}}\PY{l+s}{t}\PY{l+s}{o}\PY{l+s}{y}\PY{l+s}{ }\PY{l+s}{l}\PY{l+s}{a}\PY{l+s}{b}\PY{l+s}{d}\PY{l+s}{e}\PY{l+s}{v}\PY{l+s}{ }\PY{l+s}{l}\PY{l+s}{a}\PY{l+s}{b}\PY{l+s}{\PYZsq{}}

\PY{c+c1}{NB. insure toy is read/write}
\PY{n+nv}{dpset} \PY{l+s}{\PYZsq{}}\PY{l+s}{R}\PY{l+s}{E}\PY{l+s}{A}\PY{l+s}{D}\PY{l+s}{W}\PY{l+s}{R}\PY{l+s}{I}\PY{l+s}{T}\PY{l+s}{E}\PY{l+s}{\PYZsq{}}
\end{Verbatim}
\end{tcolorbox}

    \begin{Verbatim}[commandchars=\\\{\}]
+-+--------------------+---+------+---+
|1|opened (rw/rw/rw) ->|toy|labdev|lab|
+-+--------------------+---+------+---+
+-+--------------------------------------------+---+
|1|put dictionary read/write status restored ->|toy|
+-+--------------------------------------------+---+
    \end{Verbatim}

    \hypertarget{getting-and-putting-words}{%
\subsubsection{Getting and putting
words}\label{getting-and-putting-words}}

In the first section I said JOD is a word storage and retrieval system.
Now we are ready to (\texttt{put}) and (\texttt{get}) some words.

First create some words to store.

    \begin{tcolorbox}[breakable, size=fbox, boxrule=1pt, pad at break*=1mm,colback=cellbackground, colframe=cellborder]
\prompt{In}{incolor}{15}{\boxspacing}
\begin{Verbatim}[commandchars=\\\{\}]
\PY{c+c1}{NB. create some words}
\PY{n+nv}{random}\PY{o}{=:} \PY{err}{?}\PY{l+m+mi}{10} \PY{l+m+mi}{10}\PY{o}{\PYZdl{}}\PY{l+m+mi}{100}  \PY{c+c1}{NB. numeric noun}
\PY{n+nv}{text}\PY{o}{=:} \PY{l+s}{\PYZsq{}}\PY{l+s}{t}\PY{l+s}{h}\PY{l+s}{i}\PY{l+s}{s}\PY{l+s}{ }\PY{l+s}{i}\PY{l+s}{s}\PY{l+s}{ }\PY{l+s}{a}\PY{l+s}{ }\PY{l+s}{t}\PY{l+s}{e}\PY{l+s}{s}\PY{l+s}{t}\PY{l+s}{ }\PY{l+s}{o}\PY{l+s}{f}\PY{l+s}{ }\PY{l+s}{t}\PY{l+s}{h}\PY{l+s}{e}\PY{l+s}{ }\PY{l+s}{o}\PY{l+s}{n}\PY{l+s}{e}\PY{l+s}{ }\PY{l+s}{p}\PY{l+s}{u}\PY{l+s}{r}\PY{l+s}{e}\PY{l+s}{ }\PY{l+s}{t}\PY{l+s}{h}\PY{l+s}{i}\PY{l+s}{n}\PY{l+s}{g}\PY{l+s}{\PYZsq{}}
\PY{n+nv}{floats}\PY{o}{=:} \PY{l+m+mi}{2} \PY{o}{+} \PY{o}{\PYZpc{}} \PY{l+m+mi}{100}\PY{o}{\PYZsh{}}\PY{l+m+mi}{100}
\PY{n+nv}{symbols}\PY{o}{=:} \PY{n+nv}{s}\PY{o}{:} \PY{l+s}{\PYZsq{}}\PY{l+s}{ }\PY{l+s}{o}\PY{l+s}{n}\PY{l+s}{c}\PY{l+s}{e}\PY{l+s}{ }\PY{l+s}{m}\PY{l+s}{o}\PY{l+s}{r}\PY{l+s}{e}\PY{l+s}{ }\PY{l+s}{w}\PY{l+s}{i}\PY{l+s}{t}\PY{l+s}{h}\PY{l+s}{ }\PY{l+s}{f}\PY{l+s}{e}\PY{l+s}{e}\PY{l+s}{l}\PY{l+s}{i}\PY{l+s}{n}\PY{l+s}{g}\PY{l+s}{\PYZsq{}}
\PY{n+nv}{boxed}\PY{o}{=:} \PY{o}{\PYZlt{}}\PY{o}{\PYZdq{}}\PY{l+m+mi}{1} \PY{n+nv}{i}\PY{o}{.} \PY{l+m+mi}{2} \PY{l+m+mi}{3}
\PY{n+nv}{rationals}\PY{o}{=:} \PY{l+m+mi}{100} \PY{o}{+} \PY{o}{\PYZpc{}} \PY{p}{(}\PY{o}{\PYZgt{}}\PY{o}{:}\PY{n+nv}{i}\PY{o}{.} \PY{l+m+mi+il}{10x}\PY{p}{)} \PY{o}{\PYZca{}} \PY{l+m+mi}{50}
\PY{n+nv}{unicode}\PY{o}{=:} \PY{n+nv}{u}\PY{o}{:} \PY{l+s}{\PYZsq{}}\PY{l+s}{t}\PY{l+s}{h}\PY{l+s}{i}\PY{l+s}{s}\PY{l+s}{ }\PY{l+s}{i}\PY{l+s}{s}\PY{l+s}{ }\PY{l+s}{n}\PY{l+s}{o}\PY{l+s}{w}\PY{l+s}{ }\PY{l+s}{u}\PY{l+s}{n}\PY{l+s}{i}\PY{l+s}{c}\PY{l+s}{o}\PY{l+s}{d}\PY{l+s}{e}\PY{l+s}{\PYZsq{}}
\PY{n+nv}{each}\PY{o}{=:} \PY{o}{\PYZam{}}\PY{o}{.}\PY{o}{\PYZgt{}}  \PY{c+c1}{NB. tacit adverb}
\PY{n+nv}{explicit}\PY{o}{=:} \PY{n+nf}{4 : 0}
\PY{c+c1}{NB. explicit verb}
\PY{n+nd}{x} \PY{o}{+}\PY{o}{.} \PY{n+nd}{y}
\PY{n+nl}{)}

\PY{n+nv}{words}\PY{o}{=:} \PY{o}{;}\PY{o}{:}\PY{l+s}{\PYZsq{}}\PY{l+s}{r}\PY{l+s}{a}\PY{l+s}{n}\PY{l+s}{d}\PY{l+s}{o}\PY{l+s}{m}\PY{l+s}{ }\PY{l+s}{t}\PY{l+s}{e}\PY{l+s}{x}\PY{l+s}{t}\PY{l+s}{ }\PY{l+s}{f}\PY{l+s}{l}\PY{l+s}{o}\PY{l+s}{a}\PY{l+s}{t}\PY{l+s}{s}\PY{l+s}{ }\PY{l+s}{s}\PY{l+s}{y}\PY{l+s}{m}\PY{l+s}{b}\PY{l+s}{o}\PY{l+s}{l}\PY{l+s}{s}\PY{l+s}{ }\PY{l+s}{b}\PY{l+s}{o}\PY{l+s}{x}\PY{l+s}{e}\PY{l+s}{d}\PY{l+s}{ }\PY{l+s}{r}\PY{l+s}{a}\PY{l+s}{t}\PY{l+s}{i}\PY{l+s}{o}\PY{l+s}{n}\PY{l+s}{a}\PY{l+s}{l}\PY{l+s}{s}\PY{l+s}{ }\PY{l+s}{u}\PY{l+s}{n}\PY{l+s}{i}\PY{l+s}{c}\PY{l+s}{o}\PY{l+s}{d}\PY{l+s}{e}\PY{l+s}{ }\PY{l+s}{e}\PY{l+s}{a}\PY{l+s}{c}\PY{l+s}{h}\PY{l+s}{ }\PY{l+s}{e}\PY{l+s}{x}\PY{l+s}{p}\PY{l+s}{l}\PY{l+s}{i}\PY{l+s}{c}\PY{l+s}{i}\PY{l+s}{t}\PY{l+s}{\PYZsq{}}
\end{Verbatim}
\end{tcolorbox}

    (\texttt{put}) is the JOD command that stores words.

Save and erase the words. Take some time to convince yourself that the
words have been erased before proceeding.

    \begin{tcolorbox}[breakable, size=fbox, boxrule=1pt, pad at break*=1mm,colback=cellbackground, colframe=cellborder]
\prompt{In}{incolor}{16}{\boxspacing}
\begin{Verbatim}[commandchars=\\\{\}]
\PY{c+c1}{NB. save words}
\PY{n+nv}{smoutput} \PY{n+nv}{put} \PY{n+nv}{words}

\PY{c+c1}{NB. erase definitions}
\PY{n+nv}{erase} \PY{n+nv}{words}
\end{Verbatim}
\end{tcolorbox}

    \begin{Verbatim}[commandchars=\\\{\}]
+-+-------------------+---+
|1|9 word(s) put in ->|toy|
+-+-------------------+---+
1 1 1 1 1 1 1 1 1
    \end{Verbatim}

    Now retrieve the stored words and check that they are properly restored.

    \begin{tcolorbox}[breakable, size=fbox, boxrule=1pt, pad at break*=1mm,colback=cellbackground, colframe=cellborder]
\prompt{In}{incolor}{17}{\boxspacing}
\begin{Verbatim}[commandchars=\\\{\}]
\PY{c+c1}{NB. get words}
\PY{n+nv}{get} \PY{n+nv}{words}
\end{Verbatim}
\end{tcolorbox}

    \begin{Verbatim}[commandchars=\\\{\}]
+-+-----------------+
|1|9 word(s) defined|
+-+-----------------+
    \end{Verbatim}

    \hypertarget{documentation-101}{%
\subsubsection{Documentation 101}\label{documentation-101}}

One of my pet peeves is undocumented code!

How often have you had to face hundreds, maybe thousands, of lines of
code with nary a comment in sight. Comments are not for wimps and
girly-men. Telling comments are a hallmark of good code.

JOD provides a number of ways to document words. When a word is
introduced it's a good idea to store a short one line description of the
word.

    \begin{tcolorbox}[breakable, size=fbox, boxrule=1pt, pad at break*=1mm,colback=cellbackground, colframe=cellborder]
\prompt{In}{incolor}{18}{\boxspacing}
\begin{Verbatim}[commandchars=\\\{\}]
\PY{c+c1}{NB. store short word descriptions}
\PY{n+nv}{smoutput} \PY{l+m+mi}{0} \PY{l+m+mi}{8} \PY{n+nv}{put} \PY{l+s}{\PYZsq{}}\PY{l+s}{r}\PY{l+s}{a}\PY{l+s}{n}\PY{l+s}{d}\PY{l+s}{o}\PY{l+s}{m}\PY{l+s}{\PYZsq{}}\PY{o}{;}\PY{l+s}{\PYZsq{}}\PY{l+s}{r}\PY{l+s}{a}\PY{l+s}{n}\PY{l+s}{d}\PY{l+s}{o}\PY{l+s}{m}\PY{l+s}{ }\PY{l+s}{i}\PY{l+s}{n}\PY{l+s}{t}\PY{l+s}{e}\PY{l+s}{g}\PY{l+s}{e}\PY{l+s}{r}\PY{l+s}{ }\PY{l+s}{t}\PY{l+s}{a}\PY{l+s}{b}\PY{l+s}{l}\PY{l+s}{e}\PY{l+s}{\PYZsq{}}

\PY{l+m+mi}{0} \PY{l+m+mi}{8} \PY{n+nv}{put} \PY{l+s}{\PYZsq{}}\PY{l+s}{e}\PY{l+s}{a}\PY{l+s}{c}\PY{l+s}{h}\PY{l+s}{\PYZsq{}}\PY{o}{;}\PY{l+s}{\PYZsq{}}\PY{l+s}{a}\PY{l+s}{p}\PY{l+s}{p}\PY{l+s}{l}\PY{l+s}{i}\PY{l+s}{e}\PY{l+s}{s}\PY{l+s}{ }\PY{l+s}{l}\PY{l+s}{e}\PY{l+s}{f}\PY{l+s}{t}\PY{l+s}{ }\PY{l+s}{a}\PY{l+s}{r}\PY{l+s}{g}\PY{l+s}{u}\PY{l+s}{m}\PY{l+s}{e}\PY{l+s}{n}\PY{l+s}{t}\PY{l+s}{ }\PY{l+s}{t}\PY{l+s}{o}\PY{l+s}{ }\PY{l+s}{a}\PY{l+s}{r}\PY{l+s}{r}\PY{l+s}{a}\PY{l+s}{y}\PY{l+s}{ }\PY{l+s}{i}\PY{l+s}{t}\PY{l+s}{e}\PY{l+s}{m}\PY{l+s}{s}\PY{l+s}{\PYZsq{}}
\end{Verbatim}
\end{tcolorbox}

    \begin{Verbatim}[commandchars=\\\{\}]
+-+-------------------------------+---+
|1|1 word explanation(s) put in ->|toy|
+-+-------------------------------+---+
+-+-------------------------------+---+
|1|1 word explanation(s) put in ->|toy|
+-+-------------------------------+---+
    \end{Verbatim}

    Of course you can view your stored descriptions with (get).

    \begin{tcolorbox}[breakable, size=fbox, boxrule=1pt, pad at break*=1mm,colback=cellbackground, colframe=cellborder]
\prompt{In}{incolor}{19}{\boxspacing}
\begin{Verbatim}[commandchars=\\\{\}]
\PY{c+c1}{NB. get short explanations}
\PY{l+m+mi}{0} \PY{l+m+mi}{8} \PY{n+nv}{get} \PY{l+s}{\PYZsq{}}\PY{l+s}{r}\PY{l+s}{a}\PY{l+s}{n}\PY{l+s}{d}\PY{l+s}{o}\PY{l+s}{m}\PY{l+s}{\PYZsq{}}\PY{o}{;}\PY{l+s}{\PYZsq{}}\PY{l+s}{e}\PY{l+s}{a}\PY{l+s}{c}\PY{l+s}{h}\PY{l+s}{\PYZsq{}}
\end{Verbatim}
\end{tcolorbox}

    \begin{Verbatim}[commandchars=\\\{\}]
+-+---------------------------------------------+
|1|+------+------------------------------------+|
| ||random|random integer table                ||
| |+------+------------------------------------+|
| ||each  |applies left argument to array items||
| |+------+------------------------------------+|
+-+---------------------------------------------+
    \end{Verbatim}

    More detailed documentation can be stored and retrieved. This step loads
a realistic example of word documentation into the current put
dictionary and then displays with (\texttt{disp}).

(\texttt{disp}) is a JOD utility. It is the only verb that returns a
character list (when successful) instead of the usual boxed
(\texttt{rc;value})

    \begin{tcolorbox}[breakable, size=fbox, boxrule=1pt, pad at break*=1mm,colback=cellbackground, colframe=cellborder]
\prompt{In}{incolor}{20}{\boxspacing}
\begin{Verbatim}[commandchars=\\\{\}]
\PY{c+c1}{NB. loads (changestr) and (changestr) documentation into the current put dictionary}
\PY{n+nv}{script} \PY{l+s}{\PYZsq{}}\PY{l+s}{\PYZti{}}\PY{l+s}{a}\PY{l+s}{d}\PY{l+s}{d}\PY{l+s}{o}\PY{l+s}{n}\PY{l+s}{s}\PY{l+s}{\PYZbs{}}\PY{l+s}{g}\PY{l+s}{e}\PY{l+s}{n}\PY{l+s}{e}\PY{l+s}{r}\PY{l+s}{a}\PY{l+s}{l}\PY{l+s}{\PYZbs{}}\PY{l+s}{j}\PY{l+s}{o}\PY{l+s}{d}\PY{l+s}{\PYZbs{}}\PY{l+s}{j}\PY{l+s}{o}\PY{l+s}{d}\PY{l+s}{l}\PY{l+s}{a}\PY{l+s}{b}\PY{l+s}{s}\PY{l+s}{\PYZbs{}}\PY{l+s}{l}\PY{l+s}{a}\PY{l+s}{b}\PY{l+s}{e}\PY{l+s}{x}\PY{l+s}{a}\PY{l+s}{m}\PY{l+s}{p}\PY{l+s}{l}\PY{l+s}{e}\PY{l+s}{0}\PY{l+s}{0}\PY{l+s}{1}\PY{l+s}{.}\PY{l+s}{i}\PY{l+s}{j}\PY{l+s}{s}\PY{l+s}{\PYZsq{}}
\end{Verbatim}
\end{tcolorbox}

    This steps displays the long document loaded in the previous step.

    \begin{tcolorbox}[breakable, size=fbox, boxrule=1pt, pad at break*=1mm,colback=cellbackground, colframe=cellborder]
\prompt{In}{incolor}{21}{\boxspacing}
\begin{Verbatim}[commandchars=\\\{\}]
\PY{c+c1}{NB. show long documentation}
\PY{l+m+mi}{0} \PY{l+m+mi}{9} \PY{n+nv}{disp} \PY{l+s}{\PYZsq{}}\PY{l+s}{c}\PY{l+s}{h}\PY{l+s}{a}\PY{l+s}{n}\PY{l+s}{g}\PY{l+s}{e}\PY{l+s}{s}\PY{l+s}{t}\PY{l+s}{r}\PY{l+s}{\PYZsq{}}
\end{Verbatim}
\end{tcolorbox}

    \begin{Verbatim}[commandchars=\\\{\}]
*changestr v-- replaces substrings within a string.

This algorithm was adapted from an APL algorithm. It requires
high speed boolean bit  manipulation and is not  as effective
in current  J systems as it  is in some  APL systems. Despite
J's non-optimal booleans this verb is still fast enough to be
fruitfully applied.  On  my  400MHZ NT  machine  you can make
20,000 length increasing replacements, (the worst case), in a
1  megabyte  string  in approximately  one  second.  For  100
kilobyte  strings typical operations complete is less than  a
tenth of second.

High speed substring replacement is difficult to achieve in J
and APL environments. This verb would be a good candidate for
an external compiled routine.

dyad:  clChanged =. clTargets changestr clStr

  '/change/becomes' changestr 'change me'

  '/delete' changestr 'delete me'   NB. null replacement deletes

  NB. first character is delimiter

  '.remove..purge..wipe' changestr 'removepurgewipe'

  '/' changestr 'nothing happens'

  '' changestr 'nothing happens'

  '/nothing/happens' changestr 'no matches to change'

  NB. multiple replacements are made in left to right order

  t =. 'once all things were many'

  '/many/changes/all/at/once/ehh' changestr t

  NB. even null subtring replacements are allowed

  '//XX' changestr 'insert big x chars around us'

  NB. finally all this applies in a clean elegant
  NB. way to UNICODE strings as well

  uchars=. u: 1033 + i. 500  NB. unicode string
  datatype uchars            NB. (datatype) from j profile

  usub0=. (100+i.11)\{uchars  NB. substrings
  usub1=. (313+i.7)\{uchars
  datatype usub0
  datatype usub1

  NB. strings that will not occur in the original
  unew0=. u: 40027+i.33
  unew1=. u: 50217+i.7

  +./ unew0 E. uchars   NB. not in uchars
  +./ unew1 E. uchars

  ucharsnew=. ('/',usub0,'/',unew0,'/',usub1,'/',unew1) changestr uchars

  +./ unew0 E. ucharsnew  NB. now in string
  +./ unew1 E. ucharsnew



    \end{Verbatim}

    \hypertarget{more-putting-and-getting}{%
\subsubsection{More putting and
getting}\label{more-putting-and-getting}}

(\texttt{put}) and (\texttt{get}) are quite flexible and can store
entire locales. The locales can be named or numbered.

    \begin{tcolorbox}[breakable, size=fbox, boxrule=1pt, pad at break*=1mm,colback=cellbackground, colframe=cellborder]
\prompt{In}{incolor}{22}{\boxspacing}
\begin{Verbatim}[commandchars=\\\{\}]
\PY{c+c1}{NB. save the (ajodmake) locale/class in \PYZdq{}toy\PYZdq{}}
\PY{n+nv}{smoutput} \PY{l+s}{\PYZsq{}}\PY{l+s}{a}\PY{l+s}{j}\PY{l+s}{o}\PY{l+s}{d}\PY{l+s}{m}\PY{l+s}{a}\PY{l+s}{k}\PY{l+s}{e}\PY{l+s}{\PYZsq{}} \PY{n+nv}{put} \PY{n+nv}{zz}\PY{o}{=:} \PY{n+nv}{nl\PYZus{}ajodmake\PYZus{}} \PY{n+nv}{i}\PY{o}{.} \PY{l+m+mi}{4}

\PY{c+c1}{NB. retrieve the (ajodmake) words into an \PYZsq{}xxx\PYZsq{} locale}
\PY{l+s}{\PYZsq{}}\PY{l+s}{x}\PY{l+s}{x}\PY{l+s}{x}\PY{l+s}{\PYZsq{}} \PY{n+nv}{get} \PY{n+nv}{zz}
\end{Verbatim}
\end{tcolorbox}

    \begin{Verbatim}[commandchars=\\\{\}]
+-+--------------------+---+
|1|80 word(s) put in ->|toy|
+-+--------------------+---+
+-+------------------+
|1|80 word(s) defined|
+-+------------------+
    \end{Verbatim}

    \hypertarget{searching-for-words}{%
\subsubsection{Searching for words}\label{searching-for-words}}

Like most storage systems JOD provides facilities for searching the
contents of its database.

The main search command is (\texttt{dnl}) (dictionary name lists).

    \begin{tcolorbox}[breakable, size=fbox, boxrule=1pt, pad at break*=1mm,colback=cellbackground, colframe=cellborder]
\prompt{In}{incolor}{23}{\boxspacing}
\begin{Verbatim}[commandchars=\\\{\}]
\PY{c+c1}{NB. list all the words on the path beginning with \PYZsq{}du\PYZsq{}}
\PY{n+nv}{list} \PY{o}{\PYZcb{}}\PY{o}{.} \PY{n+nv}{dnl} \PY{l+s}{\PYZsq{}}\PY{l+s}{d}\PY{l+s}{u}\PY{l+s}{\PYZsq{}}
\end{Verbatim}
\end{tcolorbox}

    \begin{Verbatim}[commandchars=\\\{\}]
dumpdictdoc  dumpdoc      dumpgs       dumpheader   dumpntstamps dumptext
dumptm       dumptrailer  dumpwords
    \end{Verbatim}

    (\texttt{dnl}) can search for words, tests, groups, suites and macros.

This step creates some groups and then lists all the groups on the path
that begin with \texttt{\textquotesingle{}JOD\textquotesingle{}}.

    \begin{tcolorbox}[breakable, size=fbox, boxrule=1pt, pad at break*=1mm,colback=cellbackground, colframe=cellborder]
\prompt{In}{incolor}{24}{\boxspacing}
\begin{Verbatim}[commandchars=\\\{\}]
\PY{c+c1}{NB. create some groups}
\PY{n+nv}{grp} \PY{l+s}{\PYZsq{}}\PY{l+s}{s}\PY{l+s}{t}\PY{l+s}{r}\PY{l+s}{i}\PY{l+s}{n}\PY{l+s}{g}\PY{l+s}{s}\PY{l+s}{\PYZsq{}}\PY{o}{;}\PY{l+s}{\PYZsq{}}\PY{l+s}{c}\PY{l+s}{h}\PY{l+s}{a}\PY{l+s}{n}\PY{l+s}{g}\PY{l+s}{e}\PY{l+s}{s}\PY{l+s}{t}\PY{l+s}{r}\PY{l+s}{\PYZsq{}}
\PY{n+nv}{grp} \PY{l+s}{\PYZsq{}}\PY{l+s}{l}\PY{l+s}{o}\PY{l+s}{c}\PY{l+s}{t}\PY{l+s}{e}\PY{l+s}{s}\PY{l+s}{t}\PY{l+s}{\PYZsq{}}\PY{o}{;}\PY{n+nv}{nl\PYZus{}ajodmake\PYZus{}} \PY{n+nv}{i}\PY{o}{.}\PY{l+m+mi}{4}

\PY{c+c1}{NB. groups beginning with \PYZsq{}loc\PYZsq{}}
\PY{l+m+mi}{2} \PY{l+m+mi}{1} \PY{n+nv}{dnl} \PY{l+s}{\PYZsq{}}\PY{l+s}{l}\PY{l+s}{o}\PY{l+s}{c}\PY{l+s}{\PYZsq{}}
\end{Verbatim}
\end{tcolorbox}

    \begin{Verbatim}[commandchars=\\\{\}]
+-+-------+
|1|loctest|
+-+-------+
    \end{Verbatim}

    \hypertarget{what-are-these-funny-argument-numbers}{%
\subsubsection{What are these funny argument
numbers?}\label{what-are-these-funny-argument-numbers}}

By now you have probably noticed that many JOD verbs take integer
arguments. JOD argument codes are of basically three types, object
codes, option codes and qualifiers.

The objects JOD stores and retrieves all have object codes. The next
table displays JOD object codes.

    \begin{tcolorbox}[breakable, size=fbox, boxrule=1pt, pad at break*=1mm,colback=cellbackground, colframe=cellborder]
\prompt{In}{incolor}{25}{\boxspacing}
\begin{Verbatim}[commandchars=\\\{\}]
\PY{c+c1}{NB. JOD object codes}
\PY{p}{(}\PY{o}{\PYZlt{}}\PY{o}{\PYZdq{}}\PY{l+m+mi}{0} \PY{n+nv}{i}\PY{o}{.} \PY{l+m+mi}{6}\PY{p}{)} \PY{o}{,}\PY{o}{.} \PY{o}{;}\PY{o}{:}\PY{l+s}{\PYZsq{}}\PY{l+s}{W}\PY{l+s}{O}\PY{l+s}{R}\PY{l+s}{D}\PY{l+s}{ }\PY{l+s}{T}\PY{l+s}{E}\PY{l+s}{S}\PY{l+s}{T}\PY{l+s}{ }\PY{l+s}{G}\PY{l+s}{R}\PY{l+s}{O}\PY{l+s}{U}\PY{l+s}{P}\PY{l+s}{ }\PY{l+s}{S}\PY{l+s}{U}\PY{l+s}{I}\PY{l+s}{T}\PY{l+s}{E}\PY{l+s}{ }\PY{l+s}{M}\PY{l+s}{A}\PY{l+s}{C}\PY{l+s}{R}\PY{l+s}{O}\PY{l+s}{ }\PY{l+s}{D}\PY{l+s}{I}\PY{l+s}{C}\PY{l+s}{T}\PY{l+s}{I}\PY{l+s}{O}\PY{l+s}{N}\PY{l+s}{A}\PY{l+s}{R}\PY{l+s}{Y}\PY{l+s}{\PYZsq{}}
\end{Verbatim}
\end{tcolorbox}

    \begin{Verbatim}[commandchars=\\\{\}]
+-+----------+
|0|WORD      |
+-+----------+
|1|TEST      |
+-+----------+
|2|GROUP     |
+-+----------+
|3|SUITE     |
+-+----------+
|4|MACRO     |
+-+----------+
|5|DICTIONARY|
+-+----------+
    \end{Verbatim}

    Option and qualifier codes select and modify options. They are all
integers. For more information about argument codes read
\href{https://github.com/jsoftware/general_joddocument/blob/master/pdfdoc/jod.pdf}{JOD's
documentation}.

Now look at some more (\texttt{dnl}) commands.

    \begin{tcolorbox}[breakable, size=fbox, boxrule=1pt, pad at break*=1mm,colback=cellbackground, colframe=cellborder]
\prompt{In}{incolor}{26}{\boxspacing}
\begin{Verbatim}[commandchars=\\\{\}]
\PY{c+c1}{NB. (group, option 1 \PYZhy{} match prefix) \PYZhy{} case matters}
\PY{n+nv}{smoutput} \PY{l+m+mi}{2} \PY{l+m+mi}{1} \PY{n+nv}{dnl} \PY{l+s}{\PYZsq{}}\PY{l+s}{l}\PY{l+s}{\PYZsq{}}

\PY{c+c1}{NB. (macro, option 2 \PYZhy{} name contains string) \PYZhy{} no macros yet}
\PY{n+nv}{smoutput} \PY{l+m+mi}{4} \PY{l+m+mi}{2} \PY{n+nv}{dnl} \PY{l+s}{\PYZsq{}}\PY{l+s}{a}\PY{l+s}{r}\PY{l+s}{\PYZsq{}}

\PY{c+c1}{NB. make a macro and search again}
\PY{n+nv}{smoutput} \PY{l+m+mi}{4} \PY{n+nv}{put} \PY{l+s}{\PYZsq{}}\PY{l+s}{a}\PY{l+s}{r}\PY{l+s}{r}\PY{l+s}{g}\PY{l+s}{h}\PY{l+s}{\PYZsq{}}\PY{o}{;}\PY{n+nv}{JSCRIPT\PYZus{}ajod\PYZus{}}\PY{o}{;}\PY{l+s}{\PYZsq{}}\PY{l+s}{N}\PY{l+s}{B}\PY{l+s}{.}\PY{l+s}{ }\PY{l+s}{m}\PY{l+s}{y}\PY{l+s}{ }\PY{l+s}{d}\PY{l+s}{o}\PY{l+s}{ }\PY{l+s}{n}\PY{l+s}{o}\PY{l+s}{t}\PY{l+s}{h}\PY{l+s}{i}\PY{l+s}{n}\PY{l+s}{g}\PY{l+s}{ }\PY{l+s}{J}\PY{l+s}{ }\PY{l+s}{m}\PY{l+s}{a}\PY{l+s}{c}\PY{l+s}{r}\PY{l+s}{o}\PY{l+s}{\PYZsq{}}

\PY{l+m+mi}{4} \PY{l+m+mi}{2} \PY{n+nv}{dnl} \PY{l+s}{\PYZsq{}}\PY{l+s}{a}\PY{l+s}{r}\PY{l+s}{\PYZsq{}}
\end{Verbatim}
\end{tcolorbox}

    \begin{Verbatim}[commandchars=\\\{\}]
+-+-------+
|1|loctest|
+-+-------+
+-++
|1||
+-++
+-+--------------------+---+
|1|1 macro(s) put in ->|toy|
+-+--------------------+---+
+-+-----+
|1|arrgh|
+-+-----+
    \end{Verbatim}

    \hypertarget{groups-and-suites}{%
\subsubsection{Groups and suites}\label{groups-and-suites}}

JOD provides a simple way to group words and tests. A group is a
collection of J words. A suite is a collection of J test scripts. You
create and modify groups and suites with the (\texttt{grp}) verb.

    \begin{tcolorbox}[breakable, size=fbox, boxrule=1pt, pad at break*=1mm,colback=cellbackground, colframe=cellborder]
\prompt{In}{incolor}{27}{\boxspacing}
\begin{Verbatim}[commandchars=\\\{\}]
\PY{c+c1}{NB. create a group of words with names beginning with \PYZsq{}ch\PYZsq{}}
\PY{n+nv}{grp} \PY{l+s}{\PYZsq{}}\PY{l+s}{t}\PY{l+s}{e}\PY{l+s}{s}\PY{l+s}{t}\PY{l+s}{g}\PY{l+s}{r}\PY{l+s}{o}\PY{l+s}{u}\PY{l+s}{p}\PY{l+s}{\PYZsq{}} \PY{o}{;} \PY{o}{\PYZcb{}}\PY{o}{.} \PY{n+nv}{dnl} \PY{l+s}{\PYZsq{}}\PY{l+s}{c}\PY{l+s}{h}\PY{l+s}{\PYZsq{}}

\PY{c+c1}{NB. create a test}
\PY{l+m+mi}{1} \PY{n+nv}{put} \PY{l+s}{\PYZsq{}}\PY{l+s}{h}\PY{l+s}{e}\PY{l+s}{l}\PY{l+s}{l}\PY{l+s}{o}\PY{l+s}{w}\PY{l+s}{o}\PY{l+s}{r}\PY{l+s}{l}\PY{l+s}{d}\PY{l+s}{\PYZsq{}}\PY{o}{;}\PY{l+s}{\PYZsq{}}\PY{l+s}{1}\PY{l+s}{ }\PY{l+s}{[}\PY{l+s}{ }\PY{l+s}{\PYZsq{}\PYZsq{}}\PY{l+s}{J}\PY{l+s}{O}\PY{l+s}{D}\PY{l+s}{ }\PY{l+s}{t}\PY{l+s}{e}\PY{l+s}{s}\PY{l+s}{t}\PY{l+s}{s}\PY{l+s}{ }\PY{l+s}{a}\PY{l+s}{r}\PY{l+s}{e}\PY{l+s}{ }\PY{l+s}{J}\PY{l+s}{ }\PY{l+s}{s}\PY{l+s}{c}\PY{l+s}{r}\PY{l+s}{i}\PY{l+s}{p}\PY{l+s}{t}\PY{l+s}{s}\PY{l+s}{ }\PY{l+s}{t}\PY{l+s}{h}\PY{l+s}{a}\PY{l+s}{t}\PY{l+s}{ }\PY{l+s}{r}\PY{l+s}{e}\PY{l+s}{t}\PY{l+s}{u}\PY{l+s}{r}\PY{l+s}{n}\PY{l+s}{ }\PY{l+s}{1}\PY{l+s}{s}\PY{l+s}{\PYZsq{}\PYZsq{}}\PY{l+s}{\PYZsq{}}

\PY{c+c1}{NB. create a test suite \PYZhy{} note left argument code}
\PY{l+m+mi}{3} \PY{n+nv}{grp} \PY{l+s}{\PYZsq{}}\PY{l+s}{t}\PY{l+s}{e}\PY{l+s}{s}\PY{l+s}{t}\PY{l+s}{s}\PY{l+s}{u}\PY{l+s}{i}\PY{l+s}{t}\PY{l+s}{e}\PY{l+s}{\PYZsq{}} \PY{o}{;} \PY{o}{\PYZcb{}}\PY{o}{.} \PY{l+m+mi}{1} \PY{n+nv}{dnl} \PY{l+s}{\PYZsq{}}\PY{l+s}{\PYZsq{}}
\end{Verbatim}
\end{tcolorbox}

    \begin{Verbatim}[commandchars=\\\{\}]
+-+---------------------------+---+
|1|suite <testsuite> put in ->|toy|
+-+---------------------------+---+
    \end{Verbatim}

    \hypertarget{you-can-list-the-contents-of-groups-or-suites-with-grp}{%
\subsubsection{\texorpdfstring{You can list the contents of groups or
suites with
(\texttt{grp})}{You can list the contents of groups or suites with (grp)}}\label{you-can-list-the-contents-of-groups-or-suites-with-grp}}

    \begin{tcolorbox}[breakable, size=fbox, boxrule=1pt, pad at break*=1mm,colback=cellbackground, colframe=cellborder]
\prompt{In}{incolor}{28}{\boxspacing}
\begin{Verbatim}[commandchars=\\\{\}]
\PY{c+c1}{NB. list contents of testgroup group}
\PY{n+nv}{smoutput} \PY{n+nv}{grp} \PY{l+s}{\PYZsq{}}\PY{l+s}{t}\PY{l+s}{e}\PY{l+s}{s}\PY{l+s}{t}\PY{l+s}{g}\PY{l+s}{r}\PY{l+s}{o}\PY{l+s}{u}\PY{l+s}{p}\PY{l+s}{\PYZsq{}}

\PY{c+c1}{NB. contents of testsuite, note suite code left argument}
\PY{l+m+mi}{3} \PY{n+nv}{grp} \PY{l+s}{\PYZsq{}}\PY{l+s}{t}\PY{l+s}{e}\PY{l+s}{s}\PY{l+s}{t}\PY{l+s}{s}\PY{l+s}{u}\PY{l+s}{i}\PY{l+s}{t}\PY{l+s}{e}\PY{l+s}{\PYZsq{}}
\end{Verbatim}
\end{tcolorbox}

    \begin{Verbatim}[commandchars=\\\{\}]
+-+---------+
|1|changestr|
+-+---------+
+-+----------+
|1|helloworld|
+-+----------+
    \end{Verbatim}

    \hypertarget{making-groups-and-suites}{%
\subsubsection{Making groups and
suites}\label{making-groups-and-suites}}

One of the main advantages of storing J code in JOD vs.~a plain script
is that you can maintain a \emph{single} version of a word, test, group
or suite and then generate many J load scripts that use dictionary
objects. Database designers call this ``one version of the truth.''

The following inserts a single word in a (\texttt{toy}) group and then
generates scripts.

    \begin{tcolorbox}[breakable, size=fbox, boxrule=1pt, pad at break*=1mm,colback=cellbackground, colframe=cellborder]
\prompt{In}{incolor}{29}{\boxspacing}
\begin{Verbatim}[commandchars=\\\{\}]
\PY{c+c1}{NB. left justify table verb}
\PY{n+nv}{ljust}\PY{o}{=:}\PY{l+s}{\PYZsq{}}\PY{l+s}{ }\PY{l+s}{\PYZsq{}}\PY{o}{\PYZam{}}\PY{o}{\PYZdl{}}\PY{o}{:} \PY{o}{:}\PY{p}{(}\PY{o}{]} \PY{o}{|}\PY{o}{.}\PY{o}{\PYZdq{}}\PY{l+m+mi}{\PYZus{}1}\PY{o}{\PYZti{}} \PY{n+nv}{i}\PY{o}{.}\PY{o}{\PYZdq{}}\PY{l+m+mi}{1}\PY{o}{\PYZam{}}\PY{l+m+mi}{0}\PY{o}{@}\PY{p}{(}\PY{o}{]} \PY{n+nv}{e}\PY{o}{.} \PY{o}{[}\PY{p}{)}\PY{p}{)}

\PY{c+c1}{NB. store in put dictionary}
\PY{n+nv}{put} \PY{l+s}{\PYZsq{}}\PY{l+s}{l}\PY{l+s}{j}\PY{l+s}{u}\PY{l+s}{s}\PY{l+s}{t}\PY{l+s}{\PYZsq{}}

\PY{c+c1}{NB. insert in all put dictionary groups }
\PY{p}{(}\PY{o}{\PYZcb{}}\PY{o}{.} \PY{l+m+mi}{2} \PY{n+nv}{revo} \PY{l+s}{\PYZsq{}}\PY{l+s}{\PYZsq{}}\PY{p}{)} \PY{n+nv}{addgrp}\PY{o}{\PYZam{}}\PY{o}{\PYZgt{}} \PY{o}{\PYZlt{}}\PY{l+s}{\PYZsq{}}\PY{l+s}{l}\PY{l+s}{j}\PY{l+s}{u}\PY{l+s}{s}\PY{l+s}{t}\PY{l+s}{\PYZsq{}}

\PY{c+c1}{NB. lookup (revo) in jod.pdf with (jodhelp)}

\PY{c+c1}{NB. generate all put dictionary groups }
\PY{n+nv}{smoutput} \PY{n+nv}{sbx} \PY{l+m+mi}{0} \PY{n+nv}{mls}\PY{o}{\PYZam{}}\PY{o}{\PYZgt{}} \PY{o}{\PYZcb{}}\PY{o}{.} \PY{l+m+mi}{2} \PY{n+nv}{revo}\PY{l+s}{\PYZsq{}}\PY{l+s}{\PYZsq{}}

\PY{c+c1}{NB. if the left argument is elided the groups are made into (load) scripts}
\PY{c+c1}{NB. mls\PYZam{}\PYZgt{} \PYZcb{}. 2 revo\PYZsq{}\PYZsq{}}
\end{Verbatim}
\end{tcolorbox}

    \begin{Verbatim}[commandchars=\\\{\}]
+-+-------------+-------------------------------------------------------- {\ldots}
|1|file saved ->|c:/users/john.baker/j903-user/joddicts/toy/script/string {\ldots}
+-+-------------+-------------------------------------------------------- {\ldots}
|1|file saved ->|c:/users/john.baker/j903-user/joddicts/toy/script/loctes {\ldots}
+-+-------------+-------------------------------------------------------- {\ldots}
|1|file saved ->|c:/users/john.baker/j903-user/joddicts/toy/script/testgr {\ldots}
+-+-------------+-------------------------------------------------------- {\ldots}
    \end{Verbatim}

    \hypertarget{macros}{%
\subsubsection{Macros}\label{macros}}

Tasks, like updating generated scripts, can be simplified with JOD
macros. A JOD macro is an arbitrary J script that can be fetched and
executed with (\texttt{rm}).

    \begin{tcolorbox}[breakable, size=fbox, boxrule=1pt, pad at break*=1mm,colback=cellbackground, colframe=cellborder]
\prompt{In}{incolor}{30}{\boxspacing}
\begin{Verbatim}[commandchars=\\\{\}]
\PY{c+c1}{NB. macro that generates all put dictionary groups}
\PY{n+nv}{jodmacro}\PY{o}{=:} \PY{l+s}{\PYZsq{}}\PY{l+s}{N}\PY{l+s}{B}\PY{l+s}{.}\PY{l+s}{ }\PY{l+s}{g}\PY{l+s}{e}\PY{l+s}{n}\PY{l+s}{e}\PY{l+s}{r}\PY{l+s}{a}\PY{l+s}{t}\PY{l+s}{e}\PY{l+s}{ }\PY{l+s}{a}\PY{l+s}{l}\PY{l+s}{l}\PY{l+s}{ }\PY{l+s}{p}\PY{l+s}{u}\PY{l+s}{t}\PY{l+s}{ }\PY{l+s}{d}\PY{l+s}{i}\PY{l+s}{c}\PY{l+s}{t}\PY{l+s}{i}\PY{l+s}{o}\PY{l+s}{n}\PY{l+s}{a}\PY{l+s}{r}\PY{l+s}{y}\PY{l+s}{ }\PY{l+s}{g}\PY{l+s}{r}\PY{l+s}{o}\PY{l+s}{u}\PY{l+s}{p}\PY{l+s}{s}\PY{l+s}{\PYZsq{}}\PY{o}{,}\PY{n+nv}{LF}\PY{o}{,}\PY{l+s}{\PYZsq{}}\PY{l+s}{0}\PY{l+s}{ }\PY{l+s}{m}\PY{l+s}{l}\PY{l+s}{s}\PY{l+s}{\PYZam{}}\PY{l+s}{\PYZgt{}}\PY{l+s}{ }\PY{l+s}{\PYZcb{}}\PY{l+s}{.}\PY{l+s}{ }\PY{l+s}{2}\PY{l+s}{ }\PY{l+s}{r}\PY{l+s}{e}\PY{l+s}{v}\PY{l+s}{o}\PY{l+s}{\PYZsq{}\PYZsq{}}\PY{l+s}{\PYZsq{}\PYZsq{}}\PY{l+s}{ }\PY{l+s}{\PYZsq{}}

\PY{c+c1}{NB. store macro \PYZhy{} code (JSCRIPT\PYZus{}ajod\PYZus{}) tells JOD this is a J script}
\PY{l+m+mi}{4} \PY{n+nv}{put} \PY{l+s}{\PYZsq{}}\PY{l+s}{m}\PY{l+s}{a}\PY{l+s}{k}\PY{l+s}{e}\PY{l+s}{p}\PY{l+s}{u}\PY{l+s}{t}\PY{l+s}{g}\PY{l+s}{r}\PY{l+s}{p}\PY{l+s}{s}\PY{l+s}{\PYZsq{}}\PY{o}{;}\PY{n+nv}{JSCRIPT\PYZus{}ajod\PYZus{}}\PY{o}{;}\PY{n+nv}{jodmacro}
\end{Verbatim}
\end{tcolorbox}

    \begin{Verbatim}[commandchars=\\\{\}]
+-+--------------------+---+
|1|1 macro(s) put in ->|toy|
+-+--------------------+---+
    \end{Verbatim}

    Running a JOD macro is a simple matter of opening the appropriate
dictionaries and using (\texttt{rm}) - run macro.

    \begin{tcolorbox}[breakable, size=fbox, boxrule=1pt, pad at break*=1mm,colback=cellbackground, colframe=cellborder]
\prompt{In}{incolor}{31}{\boxspacing}
\begin{Verbatim}[commandchars=\\\{\}]
\PY{c+c1}{NB. fetch and execute silently \PYZhy{} will only display explicit code output}
\PY{c+c1}{NB. 1 rm \PYZsq{}makeputgrps\PYZsq{}}

\PY{c+c1}{NB. fetch and execute }
\PY{n+nv}{sbx} \PY{n+nv}{rm} \PY{l+s}{\PYZsq{}}\PY{l+s}{m}\PY{l+s}{a}\PY{l+s}{k}\PY{l+s}{e}\PY{l+s}{p}\PY{l+s}{u}\PY{l+s}{t}\PY{l+s}{g}\PY{l+s}{r}\PY{l+s}{p}\PY{l+s}{s}\PY{l+s}{\PYZsq{}}
\end{Verbatim}
\end{tcolorbox}

    \begin{Verbatim}[commandchars=\\\{\}]
   NB. generate all put dictionary groups
   0 mls\&> \}. 2 revo''
+-+-------------+---------------------------------------------------------------
+
|1|file saved ->|c:/users/john.baker/j903-user/joddicts/toy/script/strings.ijs
|
+-+-------------+---------------------------------------------------------------
+
|1|file saved ->|c:/users/john.baker/j903-user/joddicts/toy/script/loctest.ijs
|
+-+-------------+---------------------------------------------------------------
+
|1|file saved
->|c:/users/john.baker/j903-user/joddicts/toy/script/testgroup.ijs|
+-+-------------+---------------------------------------------------------------
+
    \end{Verbatim}

    Macros are not restricted to J scripts. You can also store HTML, LaTeX,
XML, TEXT, BTYE, MARKDOWN, UTF8, SQL, PYTHON and JSON scripts in JOD
dictionaries. Only J scripts can be run however.

    \begin{tcolorbox}[breakable, size=fbox, boxrule=1pt, pad at break*=1mm,colback=cellbackground, colframe=cellborder]
\prompt{In}{incolor}{32}{\boxspacing}
\begin{Verbatim}[commandchars=\\\{\}]
\PY{c+c1}{NB. store LaTeX (22) and HTML (23) texts}
\PY{l+m+mi}{4} \PY{n+nv}{put} \PY{l+s}{\PYZsq{}}\PY{l+s}{l}\PY{l+s}{a}\PY{l+s}{t}\PY{l+s}{e}\PY{l+s}{x}\PY{l+s}{\PYZsq{}}\PY{o}{;}\PY{l+m+mi}{22}\PY{o}{;}\PY{l+s}{\PYZsq{}}\PY{l+s}{.}\PY{l+s}{.}\PY{l+s}{.}\PY{l+s}{ }\PY{l+s}{L}\PY{l+s}{a}\PY{l+s}{T}\PY{l+s}{e}\PY{l+s}{X}\PY{l+s}{ }\PY{l+s}{c}\PY{l+s}{o}\PY{l+s}{d}\PY{l+s}{e}\PY{l+s}{ }\PY{l+s}{.}\PY{l+s}{.}\PY{l+s}{.}\PY{l+s}{\PYZsq{}}

\PY{l+m+mi}{4} \PY{n+nv}{put} \PY{l+s}{\PYZsq{}}\PY{l+s}{h}\PY{l+s}{t}\PY{l+s}{m}\PY{l+s}{l}\PY{l+s}{\PYZsq{}}\PY{o}{;}\PY{l+m+mi}{23}\PY{o}{;}\PY{l+s}{\PYZsq{}}\PY{l+s}{ }\PY{l+s}{.}\PY{l+s}{.}\PY{l+s}{.}\PY{l+s}{ }\PY{l+s}{H}\PY{l+s}{T}\PY{l+s}{M}\PY{l+s}{L}\PY{l+s}{ }\PY{l+s}{c}\PY{l+s}{o}\PY{l+s}{d}\PY{l+s}{e}\PY{l+s}{ }\PY{l+s}{.}\PY{l+s}{.}\PY{l+s}{.}\PY{l+s}{\PYZsq{}}

\PY{c+c1}{NB. store XML and arbitrary TEXT (bytes).}
\PY{l+m+mi}{4} \PY{n+nv}{put} \PY{l+s}{\PYZsq{}}\PY{l+s}{x}\PY{l+s}{m}\PY{l+s}{l}\PY{l+s}{\PYZsq{}}\PY{o}{;}\PY{n+nv}{XML\PYZus{}ajod\PYZus{}}\PY{o}{;}\PY{l+s}{\PYZsq{}}\PY{l+s}{\PYZlt{}}\PY{l+s}{t}\PY{l+s}{e}\PY{l+s}{s}\PY{l+s}{t}\PY{l+s}{\PYZgt{}}\PY{l+s}{t}\PY{l+s}{h}\PY{l+s}{i}\PY{l+s}{s}\PY{l+s}{ }\PY{l+s}{i}\PY{l+s}{s}\PY{l+s}{ }\PY{l+s}{l}\PY{l+s}{a}\PY{l+s}{m}\PY{l+s}{e}\PY{l+s}{ }\PY{l+s}{x}\PY{l+s}{m}\PY{l+s}{l}\PY{l+s}{\PYZlt{}}\PY{l+s}{/}\PY{l+s}{t}\PY{l+s}{e}\PY{l+s}{s}\PY{l+s}{t}\PY{l+s}{\PYZgt{}}\PY{l+s}{\PYZsq{}}

\PY{c+c1}{NB. BYTE is uninterpreted bytes and can store binaries \PYZhy{} not recommended for large files.}
\PY{l+m+mi}{4} \PY{n+nv}{put} \PY{l+s}{\PYZsq{}}\PY{l+s}{B}\PY{l+s}{I}\PY{l+s}{N}\PY{l+s}{\PYZsq{}}\PY{o}{;}\PY{l+m+mi}{26}\PY{o}{;}\PY{n+nv}{read\PYZus{}ajod\PYZus{}} \PY{n+nv}{jpath} \PY{l+s}{\PYZsq{}}\PY{l+s}{\PYZti{}}\PY{l+s}{a}\PY{l+s}{d}\PY{l+s}{d}\PY{l+s}{o}\PY{l+s}{n}\PY{l+s}{s}\PY{l+s}{\PYZbs{}}\PY{l+s}{g}\PY{l+s}{e}\PY{l+s}{n}\PY{l+s}{e}\PY{l+s}{r}\PY{l+s}{a}\PY{l+s}{l}\PY{l+s}{\PYZbs{}}\PY{l+s}{j}\PY{l+s}{o}\PY{l+s}{d}\PY{l+s}{\PYZbs{}}\PY{l+s}{j}\PY{l+s}{m}\PY{l+s}{a}\PY{l+s}{s}\PY{l+s}{t}\PY{l+s}{e}\PY{l+s}{r}\PY{l+s}{.}\PY{l+s}{i}\PY{l+s}{j}\PY{l+s}{f}\PY{l+s}{\PYZsq{}}

\PY{c+c1}{NB. byte size of macro}
\PY{n+nv}{smoutput} \PY{l+m+mi}{4} \PY{l+m+mi}{15} \PY{n+nv}{get} \PY{l+s}{\PYZsq{}}\PY{l+s}{B}\PY{l+s}{I}\PY{l+s}{N}\PY{l+s}{\PYZsq{}}

\PY{c+c1}{NB. macro text types are contants in the main JOD class}
\PY{n+nv}{JSCRIPT\PYZus{}ajod\PYZus{}}\PY{o}{,} \PY{n+nv}{LATEX\PYZus{}ajod\PYZus{}}\PY{o}{,} \PY{n+nv}{HTML\PYZus{}ajod\PYZus{}}\PY{o}{,} \PY{n+nv}{XML\PYZus{}ajod\PYZus{}}\PY{o}{,} \PY{n+nv}{TEXT\PYZus{}ajod\PYZus{}}\PY{o}{,} \PY{n+nv}{BYTE\PYZus{}ajod\PYZus{}}\PY{o}{,} \PY{n+nv}{MARKDOWN\PYZus{}ajod\PYZus{}}\PY{o}{,} \PY{n+nv}{UTF8\PYZus{}ajod\PYZus{}}\PY{o}{,} \PY{n+nv}{PYTHON\PYZus{}ajod\PYZus{}}\PY{o}{,} \PY{n+nv}{SQL\PYZus{}ajod\PYZus{}}\PY{o}{,} \PY{n+nv}{JSON\PYZus{}ajod\PYZus{}}
\end{Verbatim}
\end{tcolorbox}

    \begin{Verbatim}[commandchars=\\\{\}]
+-+-----+
|1|57472|
+-+-----+
21 22 23 24 25 26 27 28 29 30 31
    \end{Verbatim}

    \hypertarget{loading-dictionary-dump-scripts}{%
\subsubsection{Loading dictionary dump
scripts}\label{loading-dictionary-dump-scripts}}

To demonstrate other JOD features we need some words in our dictionary.
The next step loads (\texttt{labdump.ijs}).

    \begin{tcolorbox}[breakable, size=fbox, boxrule=1pt, pad at break*=1mm,colback=cellbackground, colframe=cellborder]
\prompt{In}{incolor}{33}{\boxspacing}
\begin{Verbatim}[commandchars=\\\{\}]
\PY{c+c1}{NB. insure correct path}
\PY{n+nv}{od} \PY{o}{;}\PY{o}{:}\PY{l+s}{\PYZsq{}}\PY{l+s}{t}\PY{l+s}{o}\PY{l+s}{y}\PY{l+s}{ }\PY{l+s}{l}\PY{l+s}{a}\PY{l+s}{b}\PY{l+s}{d}\PY{l+s}{e}\PY{l+s}{v}\PY{l+s}{ }\PY{l+s}{l}\PY{l+s}{a}\PY{l+s}{b}\PY{l+s}{\PYZsq{}} \PY{o}{[} \PY{l+m+mi}{3} \PY{n+nv}{od} \PY{l+s}{\PYZsq{}}\PY{l+s}{\PYZsq{}}

\PY{c+c1}{NB. load dump script}
\PY{l+m+mi}{0}\PY{o}{!}\PY{o}{:}\PY{l+m+mi}{0} \PY{o}{\PYZlt{}}\PY{n+nv}{jpath} \PY{l+s}{\PYZsq{}}\PY{l+s}{\PYZti{}}\PY{l+s}{a}\PY{l+s}{d}\PY{l+s}{d}\PY{l+s}{o}\PY{l+s}{n}\PY{l+s}{s}\PY{l+s}{/}\PY{l+s}{g}\PY{l+s}{e}\PY{l+s}{n}\PY{l+s}{e}\PY{l+s}{r}\PY{l+s}{a}\PY{l+s}{l}\PY{l+s}{/}\PY{l+s}{j}\PY{l+s}{o}\PY{l+s}{d}\PY{l+s}{/}\PY{l+s}{j}\PY{l+s}{o}\PY{l+s}{d}\PY{l+s}{l}\PY{l+s}{a}\PY{l+s}{b}\PY{l+s}{s}\PY{l+s}{/}\PY{l+s}{l}\PY{l+s}{a}\PY{l+s}{b}\PY{l+s}{d}\PY{l+s}{u}\PY{l+s}{m}\PY{l+s}{p}\PY{l+s}{.}\PY{l+s}{i}\PY{l+s}{j}\PY{l+s}{s}\PY{l+s}{\PYZsq{}}
\end{Verbatim}
\end{tcolorbox}

    \begin{Verbatim}[commandchars=\\\{\}]
+-+-------------------+---+
|1|1 word(s) put in ->|toy|
+-+-------------------+---+
+-+--------------------+---+
|1|35 word(s) put in ->|toy|
+-+--------------------+---+
+-+--------------------------------+---+
|1|36 word explanation(s) put in ->|toy|
+-+--------------------------------+---+
+-+----------------------------+---+
|1|2 word document(s) put in ->|toy|
+-+----------------------------+---+
+-+-------------------------+---+
|1|group <bstats> put in -> |toy|
+-+-------------------------+---+
|1|group <sunmoon> put in ->|toy|
+-+-------------------------+---+
NB. end-of-JOD-dump-file regenerate cross references with:  0 globs\&> \}. revo ''
    \end{Verbatim}

    Dump scripts do not store word references. They must be generated.

    \begin{tcolorbox}[breakable, size=fbox, boxrule=1pt, pad at break*=1mm,colback=cellbackground, colframe=cellborder]
\prompt{In}{incolor}{34}{\boxspacing}
\begin{Verbatim}[commandchars=\\\{\}]
\PY{c+c1}{NB. update word references \PYZhy{} show first 5 messages}
\PY{l+m+mi}{5} \PY{o}{\PYZob{}}\PY{o}{.} \PY{l+m+mi}{0} \PY{n+nv}{globs}\PY{o}{\PYZam{}}\PY{o}{\PYZgt{}} \PY{o}{\PYZcb{}}\PY{o}{.} \PY{n+nv}{revo}\PY{l+s}{\PYZsq{}}\PY{l+s}{\PYZsq{}}
\end{Verbatim}
\end{tcolorbox}

    \begin{Verbatim}[commandchars=\\\{\}]
+-+--------------------------------+---++++++++++++
|1|<antimode> references put in -> |toy||||||||||||
+-+--------------------------------+---++++++++++++
|1|<arctan> references put in ->   |toy||||||||||||
+-+--------------------------------+---++++++++++++
|1|<calmoons> references put in -> |toy||||||||||||
+-+--------------------------------+---++++++++++++
|1|<cos> references put in ->      |toy||||||||||||
+-+--------------------------------+---++++++++++++
|1|<datecheck> references put in ->|toy||||||||||||
+-+--------------------------------+---++++++++++++
    \end{Verbatim}

    \hypertarget{global-references}{%
\subsubsection{Global references}\label{global-references}}

JOD has facilities for carrying out static name analysis on J words and
tests.

The (\texttt{globs}) and (\texttt{uses}) verbs analyze and stored name
references.

    \begin{tcolorbox}[breakable, size=fbox, boxrule=1pt, pad at break*=1mm,colback=cellbackground, colframe=cellborder]
\prompt{In}{incolor}{35}{\boxspacing}
\begin{Verbatim}[commandchars=\\\{\}]
\PY{c+c1}{NB. analyze names}
\PY{n+nv}{get} \PY{l+s}{\PYZsq{}}\PY{l+s}{d}\PY{l+s}{s}\PY{l+s}{t}\PY{l+s}{a}\PY{l+s}{t}\PY{l+s}{\PYZsq{}}

\PY{c+c1}{NB. classify name use in base locale word}
\PY{l+m+mi}{11} \PY{n+nv}{globs} \PY{l+s}{\PYZsq{}}\PY{l+s}{d}\PY{l+s}{s}\PY{l+s}{t}\PY{l+s}{a}\PY{l+s}{t}\PY{l+s}{\PYZsq{}}
\end{Verbatim}
\end{tcolorbox}

    \begin{Verbatim}[commandchars=\\\{\}]
+-+--------------------------------------------------------------------+
|1|+------+-----------------------------------------------------------+|
| ||Global|+--------+--------+----+------+-----+--+--+--------+------+||
| ||      ||antimode|kurtosis|mean|median|mode2|q1|q3|skewness|stddev|||
| ||      |+--------+--------+----+------+-----+--+--+--------+------+||
| |+------+-----------------------------------------------------------+|
| ||Local |+---+---+-+-+                                              ||
| ||      ||max|min|t|v|                                              ||
| ||      |+---+---+-+-+                                              ||
| |+------+-----------------------------------------------------------+|
| ||(*)=: |                                                           ||
| |+------+-----------------------------------------------------------+|
| ||(*)=. |                                                           ||
| |+------+-----------------------------------------------------------+|
| ||for.  |                                                           ||
| |+------+-----------------------------------------------------------+|
+-+--------------------------------------------------------------------+
    \end{Verbatim}

    You can update global word references.

    \begin{tcolorbox}[breakable, size=fbox, boxrule=1pt, pad at break*=1mm,colback=cellbackground, colframe=cellborder]
\prompt{In}{incolor}{36}{\boxspacing}
\begin{Verbatim}[commandchars=\\\{\}]
\PY{c+c1}{NB. 0 is the word code \PYZhy{} stores global references}
\PY{l+m+mi}{0} \PY{n+nv}{globs} \PY{l+s}{\PYZsq{}}\PY{l+s}{d}\PY{l+s}{s}\PY{l+s}{t}\PY{l+s}{a}\PY{l+s}{t}\PY{l+s}{\PYZsq{}}
\end{Verbatim}
\end{tcolorbox}

    \begin{Verbatim}[commandchars=\\\{\}]
+-+----------------------------+---+
|1|<dstat> references put in ->|toy|
+-+----------------------------+---+
    \end{Verbatim}

    (\texttt{uses}) retrieves stored references.

    \begin{tcolorbox}[breakable, size=fbox, boxrule=1pt, pad at break*=1mm,colback=cellbackground, colframe=cellborder]
\prompt{In}{incolor}{37}{\boxspacing}
\begin{Verbatim}[commandchars=\\\{\}]
\PY{c+c1}{NB. global references for (dstat)}
\PY{n+nv}{uses} \PY{l+s}{\PYZsq{}}\PY{l+s}{d}\PY{l+s}{s}\PY{l+s}{t}\PY{l+s}{a}\PY{l+s}{t}\PY{l+s}{\PYZsq{}}
\end{Verbatim}
\end{tcolorbox}

    \begin{Verbatim}[commandchars=\\\{\}]
+-+--------------------------------------------------------------------+
|1|+-----+-----------------------------------------------------------++|
| ||dstat|+--------+--------+----+------+-----+--+--+--------+------+|||
| ||     ||antimode|kurtosis|mean|median|mode2|q1|q3|skewness|stddev||||
| ||     |+--------+--------+----+------+-----+--+--+--------+------+|||
| |+-----+-----------------------------------------------------------++|
+-+--------------------------------------------------------------------+
    \end{Verbatim}

    (\texttt{uses}) becomes very ``useful'' when all words have stored
references.

    \begin{tcolorbox}[breakable, size=fbox, boxrule=1pt, pad at break*=1mm,colback=cellbackground, colframe=cellborder]
\prompt{In}{incolor}{38}{\boxspacing}
\begin{Verbatim}[commandchars=\\\{\}]
\PY{c+c1}{NB. insure toy is the put dictionary}
\PY{n+nv}{od} \PY{o}{;}\PY{o}{:}\PY{l+s}{\PYZsq{}}\PY{l+s}{t}\PY{l+s}{o}\PY{l+s}{y}\PY{l+s}{ }\PY{l+s}{l}\PY{l+s}{a}\PY{l+s}{b}\PY{l+s}{d}\PY{l+s}{e}\PY{l+s}{v}\PY{l+s}{ }\PY{l+s}{l}\PY{l+s}{a}\PY{l+s}{b}\PY{l+s}{\PYZsq{}} \PY{o}{[} \PY{l+m+mi}{3} \PY{n+nv}{od} \PY{l+s}{\PYZsq{}}\PY{l+s}{\PYZsq{}}
\end{Verbatim}
\end{tcolorbox}

    \begin{Verbatim}[commandchars=\\\{\}]
+-+--------------------+---+------+---+
|1|opened (rw/rw/rw) ->|toy|labdev|lab|
+-+--------------------+---+------+---+
    \end{Verbatim}

    (\texttt{uses}) can return many reference lists at once. The same path
search mechanism is used for retrieving references.

    \begin{tcolorbox}[breakable, size=fbox, boxrule=1pt, pad at break*=1mm,colback=cellbackground, colframe=cellborder]
\prompt{In}{incolor}{39}{\boxspacing}
\begin{Verbatim}[commandchars=\\\{\}]
\PY{c+c1}{NB. global references of words beginning with \PYZsq{}m\PYZsq{}}
\PY{c+c1}{NB. uses \PYZcb{}. dnl \PYZsq{}m\PYZsq{}}

\PY{c+c1}{NB. global references of words ending with \PYZsq{}s\PYZsq{}}
\PY{n+nv}{uses} \PY{o}{\PYZcb{}}\PY{o}{.} \PY{l+m+mi}{0} \PY{l+m+mi}{3} \PY{n+nv}{dnl}\PY{l+s}{\PYZsq{}}\PY{l+s}{s}\PY{l+s}{\PYZsq{}}
\end{Verbatim}
\end{tcolorbox}

    \begin{Verbatim}[commandchars=\\\{\}]
+-+-------------------------------------+
|1|+------------+---------------------++|
| ||calmoons    |+----------+-----+   |||
| ||            ||fromjulian|moons|   |||
| ||            |+----------+-----+   |||
| |+------------+---------------------++|
| ||cos         |                     |||
| |+------------+---------------------++|
| ||dumpgs      |                     |||
| |+------------+---------------------++|
| ||dumpntstamps|                     |||
| |+------------+---------------------++|
| ||dumpwords   |                     |||
| |+------------+---------------------++|
| ||extscopes   |                     |||
| |+------------+---------------------++|
| ||floats      |                     |||
| |+------------+---------------------++|
| ||fuserows    |                     |||
| |+------------+---------------------++|
| ||getallts    |                     |||
| |+------------+---------------------++|
| ||halfbits    |                     |||
| |+------------+---------------------++|
| ||jscriptdefs |                     |||
| |+------------+---------------------++|
| ||kurtosis    |+---+-----+          |||
| ||            ||dev|ssdev|          |||
| ||            |+---+-----+          |||
| |+------------+---------------------++|
| ||makegs      |                     |||
| |+------------+---------------------++|
| ||moons       |+---+                |||
| ||            ||sin|                |||
| ||            |+---+                |||
| |+------------+---------------------++|
| ||namecats    |                     |||
| |+------------+---------------------++|
| ||opaqnames   |                     |||
| |+------------+---------------------++|
| ||putallts    |                     |||
| |+------------+---------------------++|
| ||rationals   |                     |||
| |+------------+---------------------++|
| ||skewness    |+---+-----+          |||
| ||            ||dev|ssdev|          |||
| ||            |+---+-----+          |||
| |+------------+---------------------++|
| ||symbols     |                     |||
| |+------------+---------------------++|
| ||wrdglobals  |                     |||
| |+------------+---------------------++|
| ||writeijs    |                     |||
| |+------------+---------------------++|
| ||yeardates   |+---------+---------+|||
| ||            ||datecheck|yeardates||||
| ||            |+---------+---------+|||
| |+------------+---------------------++|
+-+-------------------------------------+
    \end{Verbatim}

    \hypertarget{the-uses-union}{%
\subsubsection{\texorpdfstring{The \texttt{uses}
union}{The uses union}}\label{the-uses-union}}

Option 31 of (\texttt{uses}) returns the \textbf{\emph{uses-union}} of a
word. The uses-union is basically a unique list of all the words on the
call tree of a word.

    \begin{tcolorbox}[breakable, size=fbox, boxrule=1pt, pad at break*=1mm,colback=cellbackground, colframe=cellborder]
\prompt{In}{incolor}{40}{\boxspacing}
\begin{Verbatim}[commandchars=\\\{\}]
\PY{c+c1}{NB. uses union of two words}
\PY{l+m+mi}{31} \PY{n+nv}{uses} \PY{o}{;}\PY{o}{:}\PY{l+s}{\PYZsq{}}\PY{l+s}{c}\PY{l+s}{a}\PY{l+s}{l}\PY{l+s}{m}\PY{l+s}{o}\PY{l+s}{o}\PY{l+s}{n}\PY{l+s}{s}\PY{l+s}{ }\PY{l+s}{s}\PY{l+s}{u}\PY{l+s}{n}\PY{l+s}{r}\PY{l+s}{i}\PY{l+s}{s}\PY{l+s}{e}\PY{l+s}{s}\PY{l+s}{e}\PY{l+s}{t}\PY{l+s}{0}\PY{l+s}{\PYZsq{}}
\end{Verbatim}
\end{tcolorbox}

    \begin{Verbatim}[commandchars=\\\{\}]
+-+---------------------------------------------------+
|1|+-----------+------------------------------------++|
| ||calmoons   |+----------+-----+---+              |||
| ||           ||fromjulian|moons|sin|              |||
| ||           |+----------+-----+---+              |||
| |+-----------+------------------------------------++|
| ||sunriseset0|+---------+------+---+---+-----+---+|||
| ||           ||NORISESET|arctan|cos|sin|tabit|tan||||
| ||           |+---------+------+---+---+-----+---+|||
| |+-----------+------------------------------------++|
+-+---------------------------------------------------+
    \end{Verbatim}

    \hypertarget{generating-load-scripts}{%
\subsubsection{Generating load scripts}\label{generating-load-scripts}}

JOD can generate J load scripts from dictionary groups. The generated
scripts are written to the put dictionary's script subdirectory.

    \begin{tcolorbox}[breakable, size=fbox, boxrule=1pt, pad at break*=1mm,colback=cellbackground, colframe=cellborder]
\prompt{In}{incolor}{41}{\boxspacing}
\begin{Verbatim}[commandchars=\\\{\}]
\PY{c+c1}{NB. generate load script}
\PY{n+nv}{sbx} \PY{n+nv}{mls} \PY{l+s}{\PYZsq{}}\PY{l+s}{s}\PY{l+s}{u}\PY{l+s}{n}\PY{l+s}{m}\PY{l+s}{o}\PY{l+s}{o}\PY{l+s}{n}\PY{l+s}{\PYZsq{}}   \PY{c+c1}{NB. sun/moon rise set}
\end{Verbatim}
\end{tcolorbox}

    \begin{Verbatim}[commandchars=\\\{\}]
+-+--------------------+------------------------------------------------- {\ldots}
|1|load script saved ->|c:/users/john.baker/j903-user/joddicts/toy/script {\ldots}
+-+--------------------+------------------------------------------------- {\ldots}
    \end{Verbatim}

    (\texttt{mls}) appends generated scripts to the current user's
\texttt{startup.ijs} file so they can be loaded independently of JOD.

Note: mls scripts are added to \texttt{PUBLIC\_j\_} or
\texttt{Public\_j\_} for the current user.

    \begin{tcolorbox}[breakable, size=fbox, boxrule=1pt, pad at break*=1mm,colback=cellbackground, colframe=cellborder]
\prompt{In}{incolor}{42}{\boxspacing}
\begin{Verbatim}[commandchars=\\\{\}]
\PY{c+c1}{NB. load generated script}
\PY{n+nv}{load} \PY{l+s}{\PYZsq{}}\PY{l+s}{s}\PY{l+s}{u}\PY{l+s}{n}\PY{l+s}{m}\PY{l+s}{o}\PY{l+s}{o}\PY{l+s}{n}\PY{l+s}{\PYZsq{}}

\PY{n+nv}{calmoons} \PY{l+m+mi}{2019}  \PY{c+c1}{NB. full (1)  and new (0) moons in 2019}
\end{Verbatim}
\end{tcolorbox}

    \begin{Verbatim}[commandchars=\\\{\}]
0 2019  1  5
1 2019  1 20
0 2019  2  4
1 2019  2 19
0 2019  3  6
1 2019  3 20
0 2019  4  4
1 2019  4 18
0 2019  5  4
1 2019  5 18
0 2019  6  2
1 2019  6 16
0 2019  7  2
1 2019  7 16
0 2019  7 31
1 2019  8 15
0 2019  8 29
1 2019  9 13
0 2019  9 28
1 2019 10 13
0 2019 10 27
1 2019 11 12
0 2019 11 26
1 2019 12 11
0 2019 12 25
    \end{Verbatim}

    \hypertarget{generating-scripts-on-demand}{%
\subsubsection{Generating scripts on
demand}\label{generating-scripts-on-demand}}

JOD can also generate and load scripts without creating load scripts.

    \begin{tcolorbox}[breakable, size=fbox, boxrule=1pt, pad at break*=1mm,colback=cellbackground, colframe=cellborder]
\prompt{In}{incolor}{43}{\boxspacing}
\begin{Verbatim}[commandchars=\\\{\}]
\PY{c+c1}{NB. load basic statistics group}
\PY{n+nv}{lg} \PY{l+s}{\PYZsq{}}\PY{l+s}{b}\PY{l+s}{s}\PY{l+s}{t}\PY{l+s}{a}\PY{l+s}{t}\PY{l+s}{s}\PY{l+s}{\PYZsq{}}
\end{Verbatim}
\end{tcolorbox}

    \begin{Verbatim}[commandchars=\\\{\}]
+-+-------------------+
|1|bstats group loaded|
+-+-------------------+
    \end{Verbatim}

    (\texttt{getrx}) loads all the words called by a given word.

    \begin{tcolorbox}[breakable, size=fbox, boxrule=1pt, pad at break*=1mm,colback=cellbackground, colframe=cellborder]
\prompt{In}{incolor}{44}{\boxspacing}
\begin{Verbatim}[commandchars=\\\{\}]
\PY{c+c1}{NB. load into arbitrary locales}
\PY{c+c1}{NB. \PYZsq{}statloc\PYZsq{} getrx \PYZsq{}dstat\PYZsq{}}
\PY{c+c1}{NB. \PYZsq{}99\PYZsq{} getrx \PYZsq{}dstat\PYZsq{}}

\PY{c+c1}{NB. load all words needed to run (dstat)}
\PY{n+nv}{getrx} \PY{l+s}{\PYZsq{}}\PY{l+s}{d}\PY{l+s}{s}\PY{l+s}{t}\PY{l+s}{a}\PY{l+s}{t}\PY{l+s}{\PYZsq{}}
\end{Verbatim}
\end{tcolorbox}

    \begin{Verbatim}[commandchars=\\\{\}]
+-+------------------------------+
|1|(14) words loaded into -> base|
+-+------------------------------+
    \end{Verbatim}

    \hypertarget{backing-up-and-restoring-dictionaries}{%
\subsubsection{Backing up and restoring
dictionaries}\label{backing-up-and-restoring-dictionaries}}

JOD is database for J words, scripts and other precious program texts.
Most database systems have means for backing up and restoring databases
and JOD does as well. The (\texttt{packd}) verb backups up a database.

    \begin{tcolorbox}[breakable, size=fbox, boxrule=1pt, pad at break*=1mm,colback=cellbackground, colframe=cellborder]
\prompt{In}{incolor}{45}{\boxspacing}
\begin{Verbatim}[commandchars=\\\{\}]
\PY{c+c1}{NB. save a backup of the current put dictionary}
\PY{n+nv}{packd} \PY{l+s}{\PYZsq{}}\PY{l+s}{t}\PY{l+s}{o}\PY{l+s}{y}\PY{l+s}{\PYZsq{}}
\end{Verbatim}
\end{tcolorbox}

    \begin{Verbatim}[commandchars=\\\{\}]
+-+--------------------+---+-+
|1|dictionary packed ->|toy|0|
+-+--------------------+---+-+
    \end{Verbatim}

    (\texttt{packd}) copies the current dictionary files to the backup
subdirectory and prefixes all the files with a unique ever increasing
backup number.

    \begin{tcolorbox}[breakable, size=fbox, boxrule=1pt, pad at break*=1mm,colback=cellbackground, colframe=cellborder]
\prompt{In}{incolor}{46}{\boxspacing}
\begin{Verbatim}[commandchars=\\\{\}]
\PY{c+c1}{NB. list put dictionary backup files}
\PY{n+nv}{BDIR}\PY{o}{=:} \PY{o}{\PYZob{}}\PY{o}{:}\PY{o}{\PYZob{}}\PY{o}{.}  \PY{n+nv}{DPATH\PYZus{}\PYZus{}ST\PYZus{}\PYZus{}JODobj}   \PY{c+c1}{NB. put directory}
\PY{n+nv}{dir} \PY{n+nv}{BAK\PYZus{}\PYZus{}BDIR}\PY{o}{,}\PY{l+s}{\PYZsq{}}\PY{l+s}{*}\PY{l+s}{.}\PY{l+s}{i}\PY{l+s}{j}\PY{l+s}{f}\PY{l+s}{\PYZsq{}}            \PY{c+c1}{NB. backup files}
\end{Verbatim}
\end{tcolorbox}

    \begin{Verbatim}[commandchars=\\\{\}]
0jgroups.ijf       22400 14-Dec-21 09:50:51
0jmacros.ijf       76672 14-Dec-21 09:50:50
0jsuites.ijf        6272 14-Dec-21 09:50:50
0jtests.ijf         6016 14-Dec-21 09:50:50
0juses.ijf         21632 14-Dec-21 09:50:51
0jwords.ijf       204608 14-Dec-21 09:50:51
    \end{Verbatim}

    (\texttt{restd}) restores the last backup by selecting backup files with
the highest prefix.

    \begin{tcolorbox}[breakable, size=fbox, boxrule=1pt, pad at break*=1mm,colback=cellbackground, colframe=cellborder]
\prompt{In}{incolor}{47}{\boxspacing}
\begin{Verbatim}[commandchars=\\\{\}]
\PY{c+c1}{NB. restore last backup}
\PY{n+nv}{restd} \PY{l+s}{\PYZsq{}}\PY{l+s}{t}\PY{l+s}{o}\PY{l+s}{y}\PY{l+s}{\PYZsq{}}
\end{Verbatim}
\end{tcolorbox}

    \begin{Verbatim}[commandchars=\\\{\}]
+-+----------------------+---+-+
|1|dictionary restored ->|toy|0|
+-+----------------------+---+-+
    \end{Verbatim}

    In addition to restoring entire backups JOD supports fetching individual
objects from particular backups.

    \begin{tcolorbox}[breakable, size=fbox, boxrule=1pt, pad at break*=1mm,colback=cellbackground, colframe=cellborder]
\prompt{In}{incolor}{48}{\boxspacing}
\begin{Verbatim}[commandchars=\\\{\}]
\PY{c+c1}{NB. open (toy) and create new backup}
\PY{n+nv}{od} \PY{l+s}{\PYZsq{}}\PY{l+s}{t}\PY{l+s}{o}\PY{l+s}{y}\PY{l+s}{\PYZsq{}} \PY{o}{[} \PY{l+m+mi}{3} \PY{n+nv}{od} \PY{l+s}{\PYZsq{}}\PY{l+s}{\PYZsq{}}
\PY{n+nv}{smoutput} \PY{n+nv}{packd} \PY{l+s}{\PYZsq{}}\PY{l+s}{t}\PY{l+s}{o}\PY{l+s}{y}\PY{l+s}{\PYZsq{}}

\PY{c+c1}{NB. display available backup numbers}
\PY{n+nv}{smoutput} \PY{n+nv}{bnl} \PY{l+s}{\PYZsq{}}\PY{l+s}{.}\PY{l+s}{\PYZsq{}}

\PY{c+c1}{NB. all words in last backup}
\PY{n+nv}{sbx} \PY{n+nv}{bnl} \PY{l+s}{\PYZsq{}}\PY{l+s}{\PYZsq{}}
\end{Verbatim}
\end{tcolorbox}

    \begin{Verbatim}[commandchars=\\\{\}]
+-+--------------------+---+-+
|1|dictionary packed ->|toy|2|
+-+--------------------+---+-+
+-+--+--+
|1|.2|.0|
+-+--+--+
+-+--------+--------+--------+--------+--------+--------+-------+-------+ {\ldots}
|1|DDEFESCS|DUMPMSG0|DUMPMSG1|DUMPMSG2|DUMPMSG3|DUMPMSG4|DUMPTAG|ERR0150| {\ldots}
+-+--------+--------+--------+--------+--------+--------+-------+-------+ {\ldots}
    \end{Verbatim}

    Objects fetched from backups are not defined in locales for the simple
reason that many versions of the same object may be retrieved. Backup
text and binaries are recovered by editing the fetched data and
selecting what you need.

    \begin{tcolorbox}[breakable, size=fbox, boxrule=1pt, pad at break*=1mm,colback=cellbackground, colframe=cellborder]
\prompt{In}{incolor}{49}{\boxspacing}
\begin{Verbatim}[commandchars=\\\{\}]
\PY{c+c1}{NB. fetch all words from last backup}
\PY{l+s}{\PYZsq{}}\PY{l+s}{r}\PY{l+s}{c}\PY{l+s}{ }\PY{l+s}{n}\PY{l+s}{c}\PY{l+s}{v}\PY{l+s}{\PYZsq{}}\PY{o}{=.} \PY{n+nv}{bget} \PY{o}{\PYZcb{}}\PY{o}{.} \PY{n+nv}{bnl} \PY{l+s}{\PYZsq{}}\PY{l+s}{\PYZsq{}}

\PY{c+c1}{NB. edit first five objects \PYZhy{} opens JQT or JHS editor}
\PY{c+c1}{NB. requires browser/file permissions and pop ups enabled}
\PY{c+c1}{NB. ed 5 \PYZob{}. ncv}
\end{Verbatim}
\end{tcolorbox}

    (\texttt{packd}) creates binary backups. You can also backup
dictionaries as dump scripts. Dump scripts are single J scripts that can
be used to backup, copy and merge dictionaries.

    \begin{tcolorbox}[breakable, size=fbox, boxrule=1pt, pad at break*=1mm,colback=cellbackground, colframe=cellborder]
\prompt{In}{incolor}{50}{\boxspacing}
\begin{Verbatim}[commandchars=\\\{\}]
\PY{c+c1}{NB. dump all the words on the path as a single dump script.}
\PY{n+nv}{sbx} \PY{n+nv}{toydump}\PY{o}{=:} \PY{n+nv}{make} \PY{l+s}{\PYZsq{}}\PY{l+s}{\PYZsq{}}
\end{Verbatim}
\end{tcolorbox}

    \begin{Verbatim}[commandchars=\\\{\}]
+-+---------------------------+------------------------------------------ {\ldots}
|1|object(s) on path dumped ->|c:/users/john.baker/j903-user/joddicts/toy {\ldots}
+-+---------------------------+------------------------------------------ {\ldots}
    \end{Verbatim}

    When we load (\texttt{toydump}) into a new dictionary observe how the
path is changed. The dictionaries have been merged.

    \begin{tcolorbox}[breakable, size=fbox, boxrule=1pt, pad at break*=1mm,colback=cellbackground, colframe=cellborder]
\prompt{In}{incolor}{51}{\boxspacing}
\begin{Verbatim}[commandchars=\\\{\}]
\PY{c+c1}{NB. new dictionary}
\PY{n+nv}{newd} \PY{l+s}{\PYZsq{}}\PY{l+s}{p}\PY{l+s}{l}\PY{l+s}{a}\PY{l+s}{y}\PY{l+s}{p}\PY{l+s}{e}\PY{l+s}{n}\PY{l+s}{\PYZsq{}} \PY{o}{[} \PY{l+m+mi}{3} \PY{n+nv}{od} \PY{l+s}{\PYZsq{}}\PY{l+s}{\PYZsq{}}

\PY{c+c1}{NB. open}
\PY{n+nv}{od} \PY{l+s}{\PYZsq{}}\PY{l+s}{p}\PY{l+s}{l}\PY{l+s}{a}\PY{l+s}{y}\PY{l+s}{p}\PY{l+s}{e}\PY{l+s}{n}\PY{l+s}{\PYZsq{}}

\PY{c+c1}{NB. load (toydump)}
\PY{l+m+mi}{0}\PY{o}{!}\PY{o}{:}\PY{l+m+mi}{0} \PY{o}{\PYZlt{}} \PY{o}{;}\PY{o}{\PYZob{}}\PY{o}{:} \PY{n+nv}{toydump}

\PY{c+c1}{NB. dictionary information}
\PY{n+nv}{did}\PY{o}{\PYZti{}} \PY{l+m+mi}{0}
\end{Verbatim}
\end{tcolorbox}

    \begin{Verbatim}[commandchars=\\\{\}]
+-+--------------------+-------+
|1|43 word(s) put in ->|playpen|
+-+--------------------+-------+
+-+--------------------+-------+
|1|50 word(s) put in ->|playpen|
+-+--------------------+-------+
+-+--------------------+-------+
|1|33 word(s) put in ->|playpen|
+-+--------------------+-------+
+-+--------------------------------+-------+
|1|38 word explanation(s) put in ->|playpen|
+-+--------------------------------+-------+
+-+----------------------------+-------+
|1|1 word document(s) put in ->|playpen|
+-+----------------------------+-------+
+-+----------------------------+-------+
|1|2 word document(s) put in ->|playpen|
+-+----------------------------+-------+
+-+-------------------+-------+
|1|1 test(s) put in ->|playpen|
+-+-------------------+-------+
+-+--------------------+-------+
|1|6 macro(s) put in ->|playpen|
+-+--------------------+-------+
+-+---------------------------+-------+
|1|group <bstats> put in ->   |playpen|
+-+---------------------------+-------+
|1|group <loctest> put in ->  |playpen|
+-+---------------------------+-------+
|1|group <strings> put in ->  |playpen|
+-+---------------------------+-------+
|1|group <sunmoon> put in ->  |playpen|
+-+---------------------------+-------+
|1|group <testgroup> put in ->|playpen|
+-+---------------------------+-------+
+-+---------------------------+-------+
|1|suite <testsuite> put in ->|playpen|
+-+---------------------------+-------+
NB. end-of-JOD-dump-file regenerate cross references with:  0 globs\&> \}. revo ''
+-+--------------------------------------------------------+
|1|+-------+--+-----+-----+-------+-------+------+--------+|
| ||       |--|Words|Tests|Groups*|Suites*|Macros|Path*   ||
| |+-------+--+-----+-----+-------+-------+------+--------+|
| ||playpen|rw|126  |1    |5      |1      |6     |/playpen||
| |+-------+--+-----+-----+-------+-------+------+--------+|
+-+--------------------------------------------------------+
    \end{Verbatim}

    \hypertarget{final-words}{%
\subsubsection{Final words}\label{final-words}}

You now have some idea of what JOD is all about. To learn more read
\href{https://github.com/jsoftware/general_joddocument/blob/master/pdfdoc/jod.pdf}{JOD's
documentation} and run the other JOD labs. If you have any problems,
questions or complaints please email me at bakerjd99@gmail.com

\begin{verbatim}
 John Baker
 bakerjd99@gmail.com
 December 2021
\end{verbatim}

    \begin{tcolorbox}[breakable, size=fbox, boxrule=1pt, pad at break*=1mm,colback=cellbackground, colframe=cellborder]
\prompt{In}{incolor}{52}{\boxspacing}
\begin{Verbatim}[commandchars=\\\{\}]
\PY{c+c1}{NB. close any open dictionaries}
\PY{l+m+mi}{3} \PY{n+nv}{od} \PY{l+s}{\PYZsq{}}\PY{l+s}{\PYZsq{}}
\end{Verbatim}
\end{tcolorbox}

    \begin{Verbatim}[commandchars=\\\{\}]
+-+---------+-------+
|1|closed ->|playpen|
+-+---------+-------+
    \end{Verbatim}


    % Add a bibliography block to the postdoc
    
    
    
\end{document}
