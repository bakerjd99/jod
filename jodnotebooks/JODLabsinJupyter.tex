
% Default to the notebook output style

    


% Inherit from the specified cell style.




\documentclass[11pt,letter,landscape]{article}   
%\documentclass[11pt]{article}

    
    
    \usepackage[T1]{fontenc}
    % Nicer default font (+ math font) than Computer Modern for most use cases
    \usepackage{mathpazo}

    % Basic figure setup, for now with no caption control since it's done
    % automatically by Pandoc (which extracts ![](path) syntax from Markdown).
    \usepackage{graphicx}
    % We will generate all images so they have a width \maxwidth. This means
    % that they will get their normal width if they fit onto the page, but
    % are scaled down if they would overflow the margins.
    \makeatletter
    \def\maxwidth{\ifdim\Gin@nat@width>\linewidth\linewidth
    \else\Gin@nat@width\fi}
    \makeatother
    \let\Oldincludegraphics\includegraphics
    % Set max figure width to be 80% of text width, for now hardcoded.
    \renewcommand{\includegraphics}[1]{\Oldincludegraphics[width=.8\maxwidth]{#1}}
    % Ensure that by default, figures have no caption (until we provide a
    % proper Figure object with a Caption API and a way to capture that
    % in the conversion process - todo).
    \usepackage{caption}
    \DeclareCaptionLabelFormat{nolabel}{}
    \captionsetup{labelformat=nolabel}

    \usepackage{adjustbox} % Used to constrain images to a maximum size 
    \usepackage{xcolor} % Allow colors to be defined
    \usepackage{enumerate} % Needed for markdown enumerations to work
    \usepackage{geometry} % Used to adjust the document margins
    \usepackage{amsmath} % Equations
    \usepackage{amssymb} % Equations
    \usepackage{textcomp} % defines textquotesingle
    % Hack from http://tex.stackexchange.com/a/47451/13684:
    \AtBeginDocument{%
        \def\PYZsq{\textquotesingle}% Upright quotes in Pygmentized code
    }
    \usepackage{upquote} % Upright quotes for verbatim code
    \usepackage{eurosym} % defines \euro
    \usepackage[mathletters]{ucs} % Extended unicode (utf-8) support
    \usepackage[utf8x]{inputenc} % Allow utf-8 characters in the tex document
    \usepackage{fancyvrb} % verbatim replacement that allows latex
    \usepackage{grffile} % extends the file name processing of package graphics 
                         % to support a larger range 
    % The hyperref package gives us a pdf with properly built
    % internal navigation ('pdf bookmarks' for the table of contents,
    % internal cross-reference links, web links for URLs, etc.)
    \usepackage{hyperref}
    \usepackage{longtable} % longtable support required by pandoc >1.10
    \usepackage{booktabs}  % table support for pandoc > 1.12.2
    \usepackage[inline]{enumitem} % IRkernel/repr support (it uses the enumerate* environment)
    \usepackage[normalem]{ulem} % ulem is needed to support strikethroughs (\sout)
                                % normalem makes italics be italics, not underlines
    

    
    
    % Colors for the hyperref package
    \definecolor{urlcolor}{rgb}{0,.145,.698}
    \definecolor{linkcolor}{rgb}{.71,0.21,0.01}
    \definecolor{citecolor}{rgb}{.12,.54,.11}

    % ANSI colors
    \definecolor{ansi-black}{HTML}{3E424D}
    \definecolor{ansi-black-intense}{HTML}{282C36}
    \definecolor{ansi-red}{HTML}{E75C58}
    \definecolor{ansi-red-intense}{HTML}{B22B31}
    \definecolor{ansi-green}{HTML}{00A250}
    \definecolor{ansi-green-intense}{HTML}{007427}
    \definecolor{ansi-yellow}{HTML}{DDB62B}
    \definecolor{ansi-yellow-intense}{HTML}{B27D12}
    \definecolor{ansi-blue}{HTML}{208FFB}
    \definecolor{ansi-blue-intense}{HTML}{0065CA}
    \definecolor{ansi-magenta}{HTML}{D160C4}
    \definecolor{ansi-magenta-intense}{HTML}{A03196}
    \definecolor{ansi-cyan}{HTML}{60C6C8}
    \definecolor{ansi-cyan-intense}{HTML}{258F8F}
    \definecolor{ansi-white}{HTML}{C5C1B4}
    \definecolor{ansi-white-intense}{HTML}{A1A6B2}

    % commands and environments needed by pandoc snippets
    % extracted from the output of `pandoc -s`
    \providecommand{\tightlist}{%
      \setlength{\itemsep}{0pt}\setlength{\parskip}{0pt}}
    \DefineVerbatimEnvironment{Highlighting}{Verbatim}{commandchars=\\\{\}}
    % Add ',fontsize=\small' for more characters per line
    \newenvironment{Shaded}{}{}
    \newcommand{\KeywordTok}[1]{\textcolor[rgb]{0.00,0.44,0.13}{\textbf{{#1}}}}
    \newcommand{\DataTypeTok}[1]{\textcolor[rgb]{0.56,0.13,0.00}{{#1}}}
    \newcommand{\DecValTok}[1]{\textcolor[rgb]{0.25,0.63,0.44}{{#1}}}
    \newcommand{\BaseNTok}[1]{\textcolor[rgb]{0.25,0.63,0.44}{{#1}}}
    \newcommand{\FloatTok}[1]{\textcolor[rgb]{0.25,0.63,0.44}{{#1}}}
    \newcommand{\CharTok}[1]{\textcolor[rgb]{0.25,0.44,0.63}{{#1}}}
    \newcommand{\StringTok}[1]{\textcolor[rgb]{0.25,0.44,0.63}{{#1}}}
    \newcommand{\CommentTok}[1]{\textcolor[rgb]{0.38,0.63,0.69}{\textit{{#1}}}}
    \newcommand{\OtherTok}[1]{\textcolor[rgb]{0.00,0.44,0.13}{{#1}}}
    \newcommand{\AlertTok}[1]{\textcolor[rgb]{1.00,0.00,0.00}{\textbf{{#1}}}}
    \newcommand{\FunctionTok}[1]{\textcolor[rgb]{0.02,0.16,0.49}{{#1}}}
    \newcommand{\RegionMarkerTok}[1]{{#1}}
    \newcommand{\ErrorTok}[1]{\textcolor[rgb]{1.00,0.00,0.00}{\textbf{{#1}}}}
    \newcommand{\NormalTok}[1]{{#1}}
    
    % Additional commands for more recent versions of Pandoc
    \newcommand{\ConstantTok}[1]{\textcolor[rgb]{0.53,0.00,0.00}{{#1}}}
    \newcommand{\SpecialCharTok}[1]{\textcolor[rgb]{0.25,0.44,0.63}{{#1}}}
    \newcommand{\VerbatimStringTok}[1]{\textcolor[rgb]{0.25,0.44,0.63}{{#1}}}
    \newcommand{\SpecialStringTok}[1]{\textcolor[rgb]{0.73,0.40,0.53}{{#1}}}
    \newcommand{\ImportTok}[1]{{#1}}
    \newcommand{\DocumentationTok}[1]{\textcolor[rgb]{0.73,0.13,0.13}{\textit{{#1}}}}
    \newcommand{\AnnotationTok}[1]{\textcolor[rgb]{0.38,0.63,0.69}{\textbf{\textit{{#1}}}}}
    \newcommand{\CommentVarTok}[1]{\textcolor[rgb]{0.38,0.63,0.69}{\textbf{\textit{{#1}}}}}
    \newcommand{\VariableTok}[1]{\textcolor[rgb]{0.10,0.09,0.49}{{#1}}}
    \newcommand{\ControlFlowTok}[1]{\textcolor[rgb]{0.00,0.44,0.13}{\textbf{{#1}}}}
    \newcommand{\OperatorTok}[1]{\textcolor[rgb]{0.40,0.40,0.40}{{#1}}}
    \newcommand{\BuiltInTok}[1]{{#1}}
    \newcommand{\ExtensionTok}[1]{{#1}}
    \newcommand{\PreprocessorTok}[1]{\textcolor[rgb]{0.74,0.48,0.00}{{#1}}}
    \newcommand{\AttributeTok}[1]{\textcolor[rgb]{0.49,0.56,0.16}{{#1}}}
    \newcommand{\InformationTok}[1]{\textcolor[rgb]{0.38,0.63,0.69}{\textbf{\textit{{#1}}}}}
    \newcommand{\WarningTok}[1]{\textcolor[rgb]{0.38,0.63,0.69}{\textbf{\textit{{#1}}}}}
    
    
    % Define a nice break command that doesn't care if a line doesn't already
    % exist.
    \def\br{\hspace*{\fill} \\* }
    % Math Jax compatability definitions
    \def\gt{>}
    \def\lt{<}
    % Document parameters
    \title{JOD Labs in Jupyter}
    
    
    

    % Pygments definitions
    
\makeatletter
\def\PY@reset{\let\PY@it=\relax \let\PY@bf=\relax%
    \let\PY@ul=\relax \let\PY@tc=\relax%
    \let\PY@bc=\relax \let\PY@ff=\relax}
\def\PY@tok#1{\csname PY@tok@#1\endcsname}
\def\PY@toks#1+{\ifx\relax#1\empty\else%
    \PY@tok{#1}\expandafter\PY@toks\fi}
\def\PY@do#1{\PY@bc{\PY@tc{\PY@ul{%
    \PY@it{\PY@bf{\PY@ff{#1}}}}}}}
\def\PY#1#2{\PY@reset\PY@toks#1+\relax+\PY@do{#2}}

\expandafter\def\csname PY@tok@w\endcsname{\def\PY@tc##1{\textcolor[rgb]{0.73,0.73,0.73}{##1}}}
\expandafter\def\csname PY@tok@c\endcsname{\let\PY@it=\textit\def\PY@tc##1{\textcolor[rgb]{0.25,0.50,0.50}{##1}}}
\expandafter\def\csname PY@tok@cp\endcsname{\def\PY@tc##1{\textcolor[rgb]{0.74,0.48,0.00}{##1}}}
\expandafter\def\csname PY@tok@k\endcsname{\let\PY@bf=\textbf\def\PY@tc##1{\textcolor[rgb]{0.00,0.50,0.00}{##1}}}
\expandafter\def\csname PY@tok@kp\endcsname{\def\PY@tc##1{\textcolor[rgb]{0.00,0.50,0.00}{##1}}}
\expandafter\def\csname PY@tok@kt\endcsname{\def\PY@tc##1{\textcolor[rgb]{0.69,0.00,0.25}{##1}}}
\expandafter\def\csname PY@tok@o\endcsname{\def\PY@tc##1{\textcolor[rgb]{0.40,0.40,0.40}{##1}}}
\expandafter\def\csname PY@tok@ow\endcsname{\let\PY@bf=\textbf\def\PY@tc##1{\textcolor[rgb]{0.67,0.13,1.00}{##1}}}
\expandafter\def\csname PY@tok@nb\endcsname{\def\PY@tc##1{\textcolor[rgb]{0.00,0.50,0.00}{##1}}}
\expandafter\def\csname PY@tok@nf\endcsname{\def\PY@tc##1{\textcolor[rgb]{0.00,0.00,1.00}{##1}}}
\expandafter\def\csname PY@tok@nc\endcsname{\let\PY@bf=\textbf\def\PY@tc##1{\textcolor[rgb]{0.00,0.00,1.00}{##1}}}
\expandafter\def\csname PY@tok@nn\endcsname{\let\PY@bf=\textbf\def\PY@tc##1{\textcolor[rgb]{0.00,0.00,1.00}{##1}}}
\expandafter\def\csname PY@tok@ne\endcsname{\let\PY@bf=\textbf\def\PY@tc##1{\textcolor[rgb]{0.82,0.25,0.23}{##1}}}
\expandafter\def\csname PY@tok@nv\endcsname{\def\PY@tc##1{\textcolor[rgb]{0.10,0.09,0.49}{##1}}}
\expandafter\def\csname PY@tok@no\endcsname{\def\PY@tc##1{\textcolor[rgb]{0.53,0.00,0.00}{##1}}}
\expandafter\def\csname PY@tok@nl\endcsname{\def\PY@tc##1{\textcolor[rgb]{0.63,0.63,0.00}{##1}}}
\expandafter\def\csname PY@tok@ni\endcsname{\let\PY@bf=\textbf\def\PY@tc##1{\textcolor[rgb]{0.60,0.60,0.60}{##1}}}
\expandafter\def\csname PY@tok@na\endcsname{\def\PY@tc##1{\textcolor[rgb]{0.49,0.56,0.16}{##1}}}
\expandafter\def\csname PY@tok@nt\endcsname{\let\PY@bf=\textbf\def\PY@tc##1{\textcolor[rgb]{0.00,0.50,0.00}{##1}}}
\expandafter\def\csname PY@tok@nd\endcsname{\def\PY@tc##1{\textcolor[rgb]{0.67,0.13,1.00}{##1}}}
\expandafter\def\csname PY@tok@s\endcsname{\def\PY@tc##1{\textcolor[rgb]{0.73,0.13,0.13}{##1}}}
\expandafter\def\csname PY@tok@sd\endcsname{\let\PY@it=\textit\def\PY@tc##1{\textcolor[rgb]{0.73,0.13,0.13}{##1}}}
\expandafter\def\csname PY@tok@si\endcsname{\let\PY@bf=\textbf\def\PY@tc##1{\textcolor[rgb]{0.73,0.40,0.53}{##1}}}
\expandafter\def\csname PY@tok@se\endcsname{\let\PY@bf=\textbf\def\PY@tc##1{\textcolor[rgb]{0.73,0.40,0.13}{##1}}}
\expandafter\def\csname PY@tok@sr\endcsname{\def\PY@tc##1{\textcolor[rgb]{0.73,0.40,0.53}{##1}}}
\expandafter\def\csname PY@tok@ss\endcsname{\def\PY@tc##1{\textcolor[rgb]{0.10,0.09,0.49}{##1}}}
\expandafter\def\csname PY@tok@sx\endcsname{\def\PY@tc##1{\textcolor[rgb]{0.00,0.50,0.00}{##1}}}
\expandafter\def\csname PY@tok@m\endcsname{\def\PY@tc##1{\textcolor[rgb]{0.40,0.40,0.40}{##1}}}
\expandafter\def\csname PY@tok@gh\endcsname{\let\PY@bf=\textbf\def\PY@tc##1{\textcolor[rgb]{0.00,0.00,0.50}{##1}}}
\expandafter\def\csname PY@tok@gu\endcsname{\let\PY@bf=\textbf\def\PY@tc##1{\textcolor[rgb]{0.50,0.00,0.50}{##1}}}
\expandafter\def\csname PY@tok@gd\endcsname{\def\PY@tc##1{\textcolor[rgb]{0.63,0.00,0.00}{##1}}}
\expandafter\def\csname PY@tok@gi\endcsname{\def\PY@tc##1{\textcolor[rgb]{0.00,0.63,0.00}{##1}}}
\expandafter\def\csname PY@tok@gr\endcsname{\def\PY@tc##1{\textcolor[rgb]{1.00,0.00,0.00}{##1}}}
\expandafter\def\csname PY@tok@ge\endcsname{\let\PY@it=\textit}
\expandafter\def\csname PY@tok@gs\endcsname{\let\PY@bf=\textbf}
\expandafter\def\csname PY@tok@gp\endcsname{\let\PY@bf=\textbf\def\PY@tc##1{\textcolor[rgb]{0.00,0.00,0.50}{##1}}}
\expandafter\def\csname PY@tok@go\endcsname{\def\PY@tc##1{\textcolor[rgb]{0.53,0.53,0.53}{##1}}}
\expandafter\def\csname PY@tok@gt\endcsname{\def\PY@tc##1{\textcolor[rgb]{0.00,0.27,0.87}{##1}}}
\expandafter\def\csname PY@tok@err\endcsname{\def\PY@bc##1{\setlength{\fboxsep}{0pt}\fcolorbox[rgb]{1.00,0.00,0.00}{1,1,1}{\strut ##1}}}
\expandafter\def\csname PY@tok@kc\endcsname{\let\PY@bf=\textbf\def\PY@tc##1{\textcolor[rgb]{0.00,0.50,0.00}{##1}}}
\expandafter\def\csname PY@tok@kd\endcsname{\let\PY@bf=\textbf\def\PY@tc##1{\textcolor[rgb]{0.00,0.50,0.00}{##1}}}
\expandafter\def\csname PY@tok@kn\endcsname{\let\PY@bf=\textbf\def\PY@tc##1{\textcolor[rgb]{0.00,0.50,0.00}{##1}}}
\expandafter\def\csname PY@tok@kr\endcsname{\let\PY@bf=\textbf\def\PY@tc##1{\textcolor[rgb]{0.00,0.50,0.00}{##1}}}
\expandafter\def\csname PY@tok@bp\endcsname{\def\PY@tc##1{\textcolor[rgb]{0.00,0.50,0.00}{##1}}}
\expandafter\def\csname PY@tok@fm\endcsname{\def\PY@tc##1{\textcolor[rgb]{0.00,0.00,1.00}{##1}}}
\expandafter\def\csname PY@tok@vc\endcsname{\def\PY@tc##1{\textcolor[rgb]{0.10,0.09,0.49}{##1}}}
\expandafter\def\csname PY@tok@vg\endcsname{\def\PY@tc##1{\textcolor[rgb]{0.10,0.09,0.49}{##1}}}
\expandafter\def\csname PY@tok@vi\endcsname{\def\PY@tc##1{\textcolor[rgb]{0.10,0.09,0.49}{##1}}}
\expandafter\def\csname PY@tok@vm\endcsname{\def\PY@tc##1{\textcolor[rgb]{0.10,0.09,0.49}{##1}}}
\expandafter\def\csname PY@tok@sa\endcsname{\def\PY@tc##1{\textcolor[rgb]{0.73,0.13,0.13}{##1}}}
\expandafter\def\csname PY@tok@sb\endcsname{\def\PY@tc##1{\textcolor[rgb]{0.73,0.13,0.13}{##1}}}
\expandafter\def\csname PY@tok@sc\endcsname{\def\PY@tc##1{\textcolor[rgb]{0.73,0.13,0.13}{##1}}}
\expandafter\def\csname PY@tok@dl\endcsname{\def\PY@tc##1{\textcolor[rgb]{0.73,0.13,0.13}{##1}}}
\expandafter\def\csname PY@tok@s2\endcsname{\def\PY@tc##1{\textcolor[rgb]{0.73,0.13,0.13}{##1}}}
\expandafter\def\csname PY@tok@sh\endcsname{\def\PY@tc##1{\textcolor[rgb]{0.73,0.13,0.13}{##1}}}
\expandafter\def\csname PY@tok@s1\endcsname{\def\PY@tc##1{\textcolor[rgb]{0.73,0.13,0.13}{##1}}}
\expandafter\def\csname PY@tok@mb\endcsname{\def\PY@tc##1{\textcolor[rgb]{0.40,0.40,0.40}{##1}}}
\expandafter\def\csname PY@tok@mf\endcsname{\def\PY@tc##1{\textcolor[rgb]{0.40,0.40,0.40}{##1}}}
\expandafter\def\csname PY@tok@mh\endcsname{\def\PY@tc##1{\textcolor[rgb]{0.40,0.40,0.40}{##1}}}
\expandafter\def\csname PY@tok@mi\endcsname{\def\PY@tc##1{\textcolor[rgb]{0.40,0.40,0.40}{##1}}}
\expandafter\def\csname PY@tok@il\endcsname{\def\PY@tc##1{\textcolor[rgb]{0.40,0.40,0.40}{##1}}}
\expandafter\def\csname PY@tok@mo\endcsname{\def\PY@tc##1{\textcolor[rgb]{0.40,0.40,0.40}{##1}}}
\expandafter\def\csname PY@tok@ch\endcsname{\let\PY@it=\textit\def\PY@tc##1{\textcolor[rgb]{0.25,0.50,0.50}{##1}}}
\expandafter\def\csname PY@tok@cm\endcsname{\let\PY@it=\textit\def\PY@tc##1{\textcolor[rgb]{0.25,0.50,0.50}{##1}}}
\expandafter\def\csname PY@tok@cpf\endcsname{\let\PY@it=\textit\def\PY@tc##1{\textcolor[rgb]{0.25,0.50,0.50}{##1}}}
\expandafter\def\csname PY@tok@c1\endcsname{\let\PY@it=\textit\def\PY@tc##1{\textcolor[rgb]{0.25,0.50,0.50}{##1}}}
\expandafter\def\csname PY@tok@cs\endcsname{\let\PY@it=\textit\def\PY@tc##1{\textcolor[rgb]{0.25,0.50,0.50}{##1}}}

\def\PYZbs{\char`\\}
\def\PYZus{\char`\_}
\def\PYZob{\char`\{}
\def\PYZcb{\char`\}}
\def\PYZca{\char`\^}
\def\PYZam{\char`\&}
\def\PYZlt{\char`\<}
\def\PYZgt{\char`\>}
\def\PYZsh{\char`\#}
\def\PYZpc{\char`\%}
\def\PYZdl{\char`\$}
\def\PYZhy{\char`\-}
\def\PYZsq{\char`\'}
\def\PYZdq{\char`\"}
\def\PYZti{\char`\~}
% for compatibility with earlier versions
\def\PYZat{@}
\def\PYZlb{[}
\def\PYZrb{]}
\makeatother


    % Exact colors from NB
    \definecolor{incolor}{rgb}{0.0, 0.0, 0.5}
    \definecolor{outcolor}{rgb}{0.545, 0.0, 0.0}



    
    % Prevent overflowing lines due to hard-to-break entities
    \sloppy 
    % Setup hyperref package
    \hypersetup{
      breaklinks=true,  % so long urls are correctly broken across lines
      colorlinks=true,
      urlcolor=urlcolor,
      linkcolor=linkcolor,
      citecolor=citecolor,
      }
    % Slightly bigger margins than the latex defaults
    
    \geometry{verbose,tmargin=1in,bmargin=1in,lmargin=1in,rmargin=1in}
    
    

    \begin{document}
    
    
    \maketitle
    
    

    
    \section{JOD Labs in Jupyter}\label{jod-labs-in-jupyter}

\begin{figure}
\centering
\includegraphics{inclusions/jodteenytinycube.png}
\caption{}
\end{figure}

    \subsubsection{J Labs and Jupyter}\label{j-labs-and-jupyter}

For many years \href{http://www.jsoftware.com/}{J systems} included a
"labs" facility. J labs descended from even earlier APL labs. Labs
exploited the interactive nature of APL, a radical facility in the
1960s, and the pioneers of APL, particularly
\href{https://en.wikipedia.org/wiki/Kenneth_E._Iverson}{Kenneth
Iverson}, made heavy use of labs for teaching APL and related subjects.

\emph{The idea behind labs is simple: we learn best by doing!}

Nobody expects portrait painters to master painting by reading about it,
attending tedious lectures, or watching boob-tube videos. Painters have
to get their hands dirty and paint! Yet, in the early days of computing,
programmers were expected to master programming by reading about it and
attending even more tedious lectures.

Labs were created to ameliorate this ridiculous situation.

    \subsubsection{Jupyter Notebooks are Labs on
Steroids}\label{jupyter-notebooks-are-labs-on-steroids}

The modern Jupyter notebook has extended interactive "Labs". Well
crafted Jupyter notebooks are excellent live texts. They encourage wide
randing interactive exploration, documentation, and illustration.

When I first encountered Jupyter notebooks I immediately thought, "Hey
this would be a great delivery mechanism for J labs."

To use a particular programming language with Jupyter you a need a
"kernel." \textbf{\emph{The single best design decision the Jupyter
notebook developers made was to create a "kernel agnostic notebook."}}

Notebooks, like labs, are not new.
\href{https://www.wolfram.com/mathematica/}{Mathematica} notebooks have
been around for decades and there were \emph{notebook'ey} software
systems even before Mathematica. Remember, the word "notebook" refers to
\emph{ancient} paper, parchment and papyrus documents.

What distinguishes Jupyter from other notebooks systems is its
open-source nature. The Jupyter project encourages developers to create
kernels for a variety of programming languages and many have responded.
You can find Jupyter kernels for scores or programming languages
including J.

    \subsubsection{Martin Saurer's J Kernel}\label{martin-saurers-j-kernel}

Martin Sauer implemented a J 8.04 kernel:
\href{https://github.com/martin-saurer/jkernel}{see this Github
repository}.

His kernel works with later versions of J. I am using it with J 8.07. To
execute JOD Jupyter labs you need to:

\begin{enumerate}
\def\labelenumi{\arabic{enumi}.}
\item
  Install Jupyter. I recommend
  \href{https://www.anaconda.com/download/}{Anaconda}.
\item
  Install a
  \href{https://code.jsoftware.com/wiki/System/Installation}{current
  version of J}.
\item
  Use the \href{https://code.jsoftware.com/wiki/JAL/User_Guide}{J
  package manager} to install the J addons,
  \href{https://code.jsoftware.com/wiki/Addons/general/jod}{jod},
  \href{https://code.jsoftware.com/wiki/Addons/general/jodsource}{jodsource},
  and
  \href{https://code.jsoftware.com/wiki/Addons/general/joddocument}{joddocument}.
\item
  Follow Martin's
  \href{https://github.com/martin-saurer/jkernel/blob/master/README.md}{J
  kernel instructions} to install a J kernel on your machine.
\item
  Download the
  \href{https://github.com/bakerjd99/jod/tree/master/jodnotebooks}{JOD
  Jupyter labs} and save them in a Jupyter directory.
\end{enumerate}

I know it's a lot of hoops to jump through but having a working
J/Jupyter environment is worth it. You can judge for yourself by
browsing the following JOD addon labs that have been converted to
Jupyter notebooks.

\begin{enumerate}
\def\labelenumi{\arabic{enumi}.}
\item
  \href{https://github.com/bakerjd99/jod/blob/master/jodnotebooks/JOD\%20Introduction\%20Lab.ipynb}{JOD
  Introduction Lab}
\item
  \href{https://github.com/bakerjd99/jod/blob/master/jodnotebooks/JOD\%20Source\%20Code\%20Dump\%20Scripts\%20Lab.ipynb}{JOD
  Source Code Dump Scripts Lab}
\item
  \href{https://github.com/bakerjd99/jod/blob/master/jodnotebooks/JOD\%20Best\%20Practices\%20Lab.ipynb}{JOD
  Best Practices Lab}
\end{enumerate}

    \begin{Verbatim}[commandchars=\\\{\}]
{\color{incolor}In [{\color{incolor}1}]:} \PY{c+c1}{NB. show J version}
        \PY{l+m+mi}{9}\PY{o}{!}\PY{o}{:}\PY{l+m+mi}{14} \PY{l+s}{\PYZsq{}}\PY{l+s}{\PYZsq{}}
\end{Verbatim}


    \begin{Verbatim}[commandchars=\\\{\}]
j901/j64avx/windows/beta-s/commercial/www.jsoftware.com/2019-12-02T12:51:33

    \end{Verbatim}

    \subsubsection{Some differences between JOD labs in J and
Jupyter.}\label{some-differences-between-jod-labs-in-j-and-jupyter.}

If you are familiar with how J labs work in J front ends like JQT or JHS
you will see some differences when running J labs in Jupyter.

\paragraph{Use portable box drawing
characters.}\label{use-portable-box-drawing-characters.}

J box characters, used to display the structure of nested arrays, are a
persistent pain. Many fonts do not have box characters and if present
they may not line up properly.

In Jupyter notebooks box characters make it more difficult to
\emph{download} notebooks as \texttt{LateX} files. Check out the options
under the Jupyter \texttt{File} menu.

The simplest way to avoid box character hell is to use standard ASCII
characters. (\texttt{portchars}) defines a set of box characters that
work in most contexts.

    \begin{Verbatim}[commandchars=\\\{\}]
{\color{incolor}In [{\color{incolor}2}]:} \PY{c+c1}{NB. use portable box drawing characters }
        \PY{c+c1}{NB. simplifies rendering notebooks as (*.tex)}
        \PY{n+nv}{portchars\PYZus{}ijod\PYZus{}}\PY{o}{=:}\PY{o}{[}\PY{o}{:} \PY{l+m+mi}{9}\PY{o}{!}\PY{o}{:}\PY{l+m+mi}{7} \PY{l+s}{\PYZsq{}}\PY{l+s}{+}\PY{l+s}{+}\PY{l+s}{+}\PY{l+s}{+}\PY{l+s}{+}\PY{l+s}{+}\PY{l+s}{+}\PY{l+s}{+}\PY{l+s}{+}\PY{l+s}{|}\PY{l+s}{\PYZhy{}}\PY{l+s}{\PYZsq{}}\PY{o}{\PYZdq{}}\PY{l+m}{\PYZus{}} \PY{o}{[} \PY{o}{]}
        \PY{n+nv}{portchars\PYZus{}ijod\PYZus{}} \PY{l+s}{\PYZsq{}}\PY{l+s}{\PYZsq{}}
\end{Verbatim}


    \begin{Verbatim}[commandchars=\\\{\}]


    \end{Verbatim}

    \paragraph{Jupyter notebooks can be easily converted to a variety of
formats.}\label{jupyter-notebooks-can-be-easily-converted-to-a-variety-of-formats.}

If you have taken care to use portable box drawing characters Jupyter
notebooks can be easily converted to various formats. I've converted
these notebooks to \texttt{tex} and \texttt{pdf}. See this Git
directory.

\url{https://github.com/bakerjd99/jod/blob/master/jodnotebooks/}

    \paragraph{Jupyter only displays the last result of a code
block.}\label{jupyter-only-displays-the-last-result-of-a-code-block.}

    \begin{Verbatim}[commandchars=\\\{\}]
{\color{incolor}In [{\color{incolor}3}]:} \PY{c+c1}{NB. no visible ouput}
        \PY{o}{;}\PY{o}{:}\PY{l+s}{\PYZsq{}}\PY{l+s}{n}\PY{l+s}{o}\PY{l+s}{t}\PY{l+s}{ }\PY{l+s}{b}\PY{l+s}{y}\PY{l+s}{ }\PY{l+s}{t}\PY{l+s}{h}\PY{l+s}{e}\PY{l+s}{ }\PY{l+s}{h}\PY{l+s}{a}\PY{l+s}{i}\PY{l+s}{r}\PY{l+s}{ }\PY{l+s}{o}\PY{l+s}{f}\PY{l+s}{ }\PY{l+s}{m}\PY{l+s}{y}\PY{l+s}{ }\PY{l+s}{c}\PY{l+s}{h}\PY{l+s}{i}\PY{l+s}{n}\PY{l+s}{n}\PY{l+s}{y}\PY{l+s}{ }\PY{l+s}{c}\PY{l+s}{h}\PY{l+s}{i}\PY{l+s}{n}\PY{l+s}{ }\PY{l+s}{c}\PY{l+s}{h}\PY{l+s}{i}\PY{l+s}{n}\PY{l+s}{\PYZsq{}}
        
        \PY{c+c1}{NB. only the last statement displays output}
        \PY{n+nv}{i}\PY{o}{.} \PY{l+m+mi}{2} \PY{l+m+mi}{3}
\end{Verbatim}


    \begin{Verbatim}[commandchars=\\\{\}]
0 1 2
3 4 5

    \end{Verbatim}

    \paragraph{\texorpdfstring{To see other statements use
\texttt{smoutput}}{To see other statements use smoutput}}\label{to-see-other-statements-use-smoutput}

    \begin{Verbatim}[commandchars=\\\{\}]
{\color{incolor}In [{\color{incolor}4}]:} \PY{c+c1}{NB. use smoutput to show result}
        \PY{n+nv}{smoutput} \PY{o}{;}\PY{o}{:}\PY{l+s}{\PYZsq{}}\PY{l+s}{n}\PY{l+s}{o}\PY{l+s}{t}\PY{l+s}{ }\PY{l+s}{b}\PY{l+s}{y}\PY{l+s}{ }\PY{l+s}{t}\PY{l+s}{h}\PY{l+s}{e}\PY{l+s}{ }\PY{l+s}{h}\PY{l+s}{a}\PY{l+s}{i}\PY{l+s}{r}\PY{l+s}{ }\PY{l+s}{o}\PY{l+s}{f}\PY{l+s}{ }\PY{l+s}{m}\PY{l+s}{y}\PY{l+s}{ }\PY{l+s}{c}\PY{l+s}{h}\PY{l+s}{i}\PY{l+s}{n}\PY{l+s}{n}\PY{l+s}{y}\PY{l+s}{ }\PY{l+s}{c}\PY{l+s}{h}\PY{l+s}{i}\PY{l+s}{n}\PY{l+s}{ }\PY{l+s}{c}\PY{l+s}{h}\PY{l+s}{i}\PY{l+s}{n}\PY{l+s}{\PYZsq{}}
        
        \PY{c+c1}{NB. only the last statement displays output}
        \PY{n+nv}{i}\PY{o}{.} \PY{l+m+mi}{2} \PY{l+m+mi}{3}
\end{Verbatim}


    \begin{Verbatim}[commandchars=\\\{\}]
+---+--+---+----+--+--+------+----+----+
|not|by|the|hair|of|my|chinny|chin|chin|
+---+--+---+----+--+--+------+----+----+
0 1 2
3 4 5

    \end{Verbatim}

    \paragraph{Large deeply nested outputs may wrap
poorly}\label{large-deeply-nested-outputs-may-wrap-poorly}

    \begin{Verbatim}[commandchars=\\\{\}]
{\color{incolor}In [{\color{incolor}5}]:} \PY{c+c1}{NB. \PYZlt{}\PYZca{}:10 ] (i. 30 50) ; \PYZlt{} ;:\PYZsq{}+/@, 34 5\PYZsq{}}
\end{Verbatim}


    \paragraph{J plot graphics work but forms that request user input may
not}\label{j-plot-graphics-work-but-forms-that-request-user-input-may-not}

J graphics in Jupyter depend on J's JHS IDE. If a graphic works in JHS
and does not require user input there is a good chance it will work in
Jupyter notebooks.

    \begin{Verbatim}[commandchars=\\\{\}]
{\color{incolor}In [{\color{incolor}6}]:} \PY{n+nv}{load} \PY{l+s}{\PYZsq{}}\PY{l+s}{p}\PY{l+s}{l}\PY{l+s}{o}\PY{l+s}{t}\PY{l+s}{\PYZsq{}}
        \PY{l+s}{\PYZsq{}}\PY{l+s}{b}\PY{l+s}{a}\PY{l+s}{r}\PY{l+s}{\PYZsq{}} \PY{n+nv}{plot} \PY{n+nv}{i}\PY{o}{.} \PY{l+m+mi}{5}
\end{Verbatim}


    
    
    \subsubsection{Final words}\label{final-words}

\textbf{\emph{Jupyter versions of J labs provide a richer environment
for experimenting with and documenting J systems.}}

It isn't difficult to convert existing J labs to Jupyter notebooks and
since both J labs and Jupyter notebooks are stored as simple text it
wouldn't be difficult to implement an addon that converted J labs to
Jupyter notebooks. The inverse, while more challenging, is also
feasible.


    % Add a bibliography block to the postdoc
    
    
    
    \end{document}
