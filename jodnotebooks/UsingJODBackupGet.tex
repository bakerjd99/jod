\documentclass[11pt,letter,landscape]{article} 
%\documentclass[11pt]{article}

    \usepackage[breakable]{tcolorbox}
    \usepackage{parskip} % Stop auto-indenting (to mimic markdown behaviour)
    
    \usepackage{iftex}
    \ifPDFTeX
    	\usepackage[T1]{fontenc}
    	\usepackage{mathpazo}
    \else
    	\usepackage{fontspec}
    \fi

    % Basic figure setup, for now with no caption control since it's done
    % automatically by Pandoc (which extracts ![](path) syntax from Markdown).
    \usepackage{graphicx}
    % Maintain compatibility with old templates. Remove in nbconvert 6.0
    \let\Oldincludegraphics\includegraphics
    % Ensure that by default, figures have no caption (until we provide a
    % proper Figure object with a Caption API and a way to capture that
    % in the conversion process - todo).
    \usepackage{caption}
    \DeclareCaptionFormat{nocaption}{}
    \captionsetup{format=nocaption,aboveskip=0pt,belowskip=0pt}

    \usepackage[Export]{adjustbox} % Used to constrain images to a maximum size
    \adjustboxset{max size={0.9\linewidth}{0.9\paperheight}}
    \usepackage{float}
    \floatplacement{figure}{H} % forces figures to be placed at the correct location
    \usepackage{xcolor} % Allow colors to be defined
    \usepackage{enumerate} % Needed for markdown enumerations to work
    \usepackage{geometry} % Used to adjust the document margins
    \usepackage{amsmath} % Equations
    \usepackage{amssymb} % Equations
    \usepackage{textcomp} % defines textquotesingle
    % Hack from http://tex.stackexchange.com/a/47451/13684:
    \AtBeginDocument{%
        \def\PYZsq{\textquotesingle}% Upright quotes in Pygmentized code
    }
    \usepackage{upquote} % Upright quotes for verbatim code
    \usepackage{eurosym} % defines \euro
    \usepackage[mathletters]{ucs} % Extended unicode (utf-8) support
    \usepackage{fancyvrb} % verbatim replacement that allows latex
    \usepackage{grffile} % extends the file name processing of package graphics 
                         % to support a larger range
    \makeatletter % fix for grffile with XeLaTeX
    \def\Gread@@xetex#1{%
      \IfFileExists{"\Gin@base".bb}%
      {\Gread@eps{\Gin@base.bb}}%
      {\Gread@@xetex@aux#1}%
    }
    \makeatother

    % The hyperref package gives us a pdf with properly built
    % internal navigation ('pdf bookmarks' for the table of contents,
    % internal cross-reference links, web links for URLs, etc.)
    \usepackage{hyperref}
    % The default LaTeX title has an obnoxious amount of whitespace. By default,
    % titling removes some of it. It also provides customization options.
    \usepackage{titling}
    \usepackage{longtable} % longtable support required by pandoc >1.10
    \usepackage{booktabs}  % table support for pandoc > 1.12.2
    \usepackage[inline]{enumitem} % IRkernel/repr support (it uses the enumerate* environment)
    \usepackage[normalem]{ulem} % ulem is needed to support strikethroughs (\sout)
                                % normalem makes italics be italics, not underlines
    \usepackage{mathrsfs}
    

    
    % Colors for the hyperref package
    \definecolor{urlcolor}{rgb}{0,.145,.698}
    \definecolor{linkcolor}{rgb}{.71,0.21,0.01}
    \definecolor{citecolor}{rgb}{.12,.54,.11}

    % ANSI colors
    \definecolor{ansi-black}{HTML}{3E424D}
    \definecolor{ansi-black-intense}{HTML}{282C36}
    \definecolor{ansi-red}{HTML}{E75C58}
    \definecolor{ansi-red-intense}{HTML}{B22B31}
    \definecolor{ansi-green}{HTML}{00A250}
    \definecolor{ansi-green-intense}{HTML}{007427}
    \definecolor{ansi-yellow}{HTML}{DDB62B}
    \definecolor{ansi-yellow-intense}{HTML}{B27D12}
    \definecolor{ansi-blue}{HTML}{208FFB}
    \definecolor{ansi-blue-intense}{HTML}{0065CA}
    \definecolor{ansi-magenta}{HTML}{D160C4}
    \definecolor{ansi-magenta-intense}{HTML}{A03196}
    \definecolor{ansi-cyan}{HTML}{60C6C8}
    \definecolor{ansi-cyan-intense}{HTML}{258F8F}
    \definecolor{ansi-white}{HTML}{C5C1B4}
    \definecolor{ansi-white-intense}{HTML}{A1A6B2}
    \definecolor{ansi-default-inverse-fg}{HTML}{FFFFFF}
    \definecolor{ansi-default-inverse-bg}{HTML}{000000}

    % commands and environments needed by pandoc snippets
    % extracted from the output of `pandoc -s`
    \providecommand{\tightlist}{%
      \setlength{\itemsep}{0pt}\setlength{\parskip}{0pt}}
    \DefineVerbatimEnvironment{Highlighting}{Verbatim}{commandchars=\\\{\}}
    % Add ',fontsize=\small' for more characters per line
    \newenvironment{Shaded}{}{}
    \newcommand{\KeywordTok}[1]{\textcolor[rgb]{0.00,0.44,0.13}{\textbf{{#1}}}}
    \newcommand{\DataTypeTok}[1]{\textcolor[rgb]{0.56,0.13,0.00}{{#1}}}
    \newcommand{\DecValTok}[1]{\textcolor[rgb]{0.25,0.63,0.44}{{#1}}}
    \newcommand{\BaseNTok}[1]{\textcolor[rgb]{0.25,0.63,0.44}{{#1}}}
    \newcommand{\FloatTok}[1]{\textcolor[rgb]{0.25,0.63,0.44}{{#1}}}
    \newcommand{\CharTok}[1]{\textcolor[rgb]{0.25,0.44,0.63}{{#1}}}
    \newcommand{\StringTok}[1]{\textcolor[rgb]{0.25,0.44,0.63}{{#1}}}
    \newcommand{\CommentTok}[1]{\textcolor[rgb]{0.38,0.63,0.69}{\textit{{#1}}}}
    \newcommand{\OtherTok}[1]{\textcolor[rgb]{0.00,0.44,0.13}{{#1}}}
    \newcommand{\AlertTok}[1]{\textcolor[rgb]{1.00,0.00,0.00}{\textbf{{#1}}}}
    \newcommand{\FunctionTok}[1]{\textcolor[rgb]{0.02,0.16,0.49}{{#1}}}
    \newcommand{\RegionMarkerTok}[1]{{#1}}
    \newcommand{\ErrorTok}[1]{\textcolor[rgb]{1.00,0.00,0.00}{\textbf{{#1}}}}
    \newcommand{\NormalTok}[1]{{#1}}
    
    % Additional commands for more recent versions of Pandoc
    \newcommand{\ConstantTok}[1]{\textcolor[rgb]{0.53,0.00,0.00}{{#1}}}
    \newcommand{\SpecialCharTok}[1]{\textcolor[rgb]{0.25,0.44,0.63}{{#1}}}
    \newcommand{\VerbatimStringTok}[1]{\textcolor[rgb]{0.25,0.44,0.63}{{#1}}}
    \newcommand{\SpecialStringTok}[1]{\textcolor[rgb]{0.73,0.40,0.53}{{#1}}}
    \newcommand{\ImportTok}[1]{{#1}}
    \newcommand{\DocumentationTok}[1]{\textcolor[rgb]{0.73,0.13,0.13}{\textit{{#1}}}}
    \newcommand{\AnnotationTok}[1]{\textcolor[rgb]{0.38,0.63,0.69}{\textbf{\textit{{#1}}}}}
    \newcommand{\CommentVarTok}[1]{\textcolor[rgb]{0.38,0.63,0.69}{\textbf{\textit{{#1}}}}}
    \newcommand{\VariableTok}[1]{\textcolor[rgb]{0.10,0.09,0.49}{{#1}}}
    \newcommand{\ControlFlowTok}[1]{\textcolor[rgb]{0.00,0.44,0.13}{\textbf{{#1}}}}
    \newcommand{\OperatorTok}[1]{\textcolor[rgb]{0.40,0.40,0.40}{{#1}}}
    \newcommand{\BuiltInTok}[1]{{#1}}
    \newcommand{\ExtensionTok}[1]{{#1}}
    \newcommand{\PreprocessorTok}[1]{\textcolor[rgb]{0.74,0.48,0.00}{{#1}}}
    \newcommand{\AttributeTok}[1]{\textcolor[rgb]{0.49,0.56,0.16}{{#1}}}
    \newcommand{\InformationTok}[1]{\textcolor[rgb]{0.38,0.63,0.69}{\textbf{\textit{{#1}}}}}
    \newcommand{\WarningTok}[1]{\textcolor[rgb]{0.38,0.63,0.69}{\textbf{\textit{{#1}}}}}
    
    
    % Define a nice break command that doesn't care if a line doesn't already
    % exist.
    \def\br{\hspace*{\fill} \\* }
    % Math Jax compatibility definitions
    \def\gt{>}
    \def\lt{<}
    \let\Oldtex\TeX
    \let\Oldlatex\LaTeX
    \renewcommand{\TeX}{\textrm{\Oldtex}}
    \renewcommand{\LaTeX}{\textrm{\Oldlatex}}
    % Document parameters
    % Document title
    \title{Using JOD Backup Get}
    
    
    
    
    
% Pygments definitions
\makeatletter
\def\PY@reset{\let\PY@it=\relax \let\PY@bf=\relax%
    \let\PY@ul=\relax \let\PY@tc=\relax%
    \let\PY@bc=\relax \let\PY@ff=\relax}
\def\PY@tok#1{\csname PY@tok@#1\endcsname}
\def\PY@toks#1+{\ifx\relax#1\empty\else%
    \PY@tok{#1}\expandafter\PY@toks\fi}
\def\PY@do#1{\PY@bc{\PY@tc{\PY@ul{%
    \PY@it{\PY@bf{\PY@ff{#1}}}}}}}
\def\PY#1#2{\PY@reset\PY@toks#1+\relax+\PY@do{#2}}

\expandafter\def\csname PY@tok@w\endcsname{\def\PY@tc##1{\textcolor[rgb]{0.73,0.73,0.73}{##1}}}
\expandafter\def\csname PY@tok@c\endcsname{\let\PY@it=\textit\def\PY@tc##1{\textcolor[rgb]{0.25,0.50,0.50}{##1}}}
\expandafter\def\csname PY@tok@cp\endcsname{\def\PY@tc##1{\textcolor[rgb]{0.74,0.48,0.00}{##1}}}
\expandafter\def\csname PY@tok@k\endcsname{\let\PY@bf=\textbf\def\PY@tc##1{\textcolor[rgb]{0.00,0.50,0.00}{##1}}}
\expandafter\def\csname PY@tok@kp\endcsname{\def\PY@tc##1{\textcolor[rgb]{0.00,0.50,0.00}{##1}}}
\expandafter\def\csname PY@tok@kt\endcsname{\def\PY@tc##1{\textcolor[rgb]{0.69,0.00,0.25}{##1}}}
\expandafter\def\csname PY@tok@o\endcsname{\def\PY@tc##1{\textcolor[rgb]{0.40,0.40,0.40}{##1}}}
\expandafter\def\csname PY@tok@ow\endcsname{\let\PY@bf=\textbf\def\PY@tc##1{\textcolor[rgb]{0.67,0.13,1.00}{##1}}}
\expandafter\def\csname PY@tok@nb\endcsname{\def\PY@tc##1{\textcolor[rgb]{0.00,0.50,0.00}{##1}}}
\expandafter\def\csname PY@tok@nf\endcsname{\def\PY@tc##1{\textcolor[rgb]{0.00,0.00,1.00}{##1}}}
\expandafter\def\csname PY@tok@nc\endcsname{\let\PY@bf=\textbf\def\PY@tc##1{\textcolor[rgb]{0.00,0.00,1.00}{##1}}}
\expandafter\def\csname PY@tok@nn\endcsname{\let\PY@bf=\textbf\def\PY@tc##1{\textcolor[rgb]{0.00,0.00,1.00}{##1}}}
\expandafter\def\csname PY@tok@ne\endcsname{\let\PY@bf=\textbf\def\PY@tc##1{\textcolor[rgb]{0.82,0.25,0.23}{##1}}}
\expandafter\def\csname PY@tok@nv\endcsname{\def\PY@tc##1{\textcolor[rgb]{0.10,0.09,0.49}{##1}}}
\expandafter\def\csname PY@tok@no\endcsname{\def\PY@tc##1{\textcolor[rgb]{0.53,0.00,0.00}{##1}}}
\expandafter\def\csname PY@tok@nl\endcsname{\def\PY@tc##1{\textcolor[rgb]{0.63,0.63,0.00}{##1}}}
\expandafter\def\csname PY@tok@ni\endcsname{\let\PY@bf=\textbf\def\PY@tc##1{\textcolor[rgb]{0.60,0.60,0.60}{##1}}}
\expandafter\def\csname PY@tok@na\endcsname{\def\PY@tc##1{\textcolor[rgb]{0.49,0.56,0.16}{##1}}}
\expandafter\def\csname PY@tok@nt\endcsname{\let\PY@bf=\textbf\def\PY@tc##1{\textcolor[rgb]{0.00,0.50,0.00}{##1}}}
\expandafter\def\csname PY@tok@nd\endcsname{\def\PY@tc##1{\textcolor[rgb]{0.67,0.13,1.00}{##1}}}
\expandafter\def\csname PY@tok@s\endcsname{\def\PY@tc##1{\textcolor[rgb]{0.73,0.13,0.13}{##1}}}
\expandafter\def\csname PY@tok@sd\endcsname{\let\PY@it=\textit\def\PY@tc##1{\textcolor[rgb]{0.73,0.13,0.13}{##1}}}
\expandafter\def\csname PY@tok@si\endcsname{\let\PY@bf=\textbf\def\PY@tc##1{\textcolor[rgb]{0.73,0.40,0.53}{##1}}}
\expandafter\def\csname PY@tok@se\endcsname{\let\PY@bf=\textbf\def\PY@tc##1{\textcolor[rgb]{0.73,0.40,0.13}{##1}}}
\expandafter\def\csname PY@tok@sr\endcsname{\def\PY@tc##1{\textcolor[rgb]{0.73,0.40,0.53}{##1}}}
\expandafter\def\csname PY@tok@ss\endcsname{\def\PY@tc##1{\textcolor[rgb]{0.10,0.09,0.49}{##1}}}
\expandafter\def\csname PY@tok@sx\endcsname{\def\PY@tc##1{\textcolor[rgb]{0.00,0.50,0.00}{##1}}}
\expandafter\def\csname PY@tok@m\endcsname{\def\PY@tc##1{\textcolor[rgb]{0.40,0.40,0.40}{##1}}}
\expandafter\def\csname PY@tok@gh\endcsname{\let\PY@bf=\textbf\def\PY@tc##1{\textcolor[rgb]{0.00,0.00,0.50}{##1}}}
\expandafter\def\csname PY@tok@gu\endcsname{\let\PY@bf=\textbf\def\PY@tc##1{\textcolor[rgb]{0.50,0.00,0.50}{##1}}}
\expandafter\def\csname PY@tok@gd\endcsname{\def\PY@tc##1{\textcolor[rgb]{0.63,0.00,0.00}{##1}}}
\expandafter\def\csname PY@tok@gi\endcsname{\def\PY@tc##1{\textcolor[rgb]{0.00,0.63,0.00}{##1}}}
\expandafter\def\csname PY@tok@gr\endcsname{\def\PY@tc##1{\textcolor[rgb]{1.00,0.00,0.00}{##1}}}
\expandafter\def\csname PY@tok@ge\endcsname{\let\PY@it=\textit}
\expandafter\def\csname PY@tok@gs\endcsname{\let\PY@bf=\textbf}
\expandafter\def\csname PY@tok@gp\endcsname{\let\PY@bf=\textbf\def\PY@tc##1{\textcolor[rgb]{0.00,0.00,0.50}{##1}}}
\expandafter\def\csname PY@tok@go\endcsname{\def\PY@tc##1{\textcolor[rgb]{0.53,0.53,0.53}{##1}}}
\expandafter\def\csname PY@tok@gt\endcsname{\def\PY@tc##1{\textcolor[rgb]{0.00,0.27,0.87}{##1}}}
\expandafter\def\csname PY@tok@err\endcsname{\def\PY@bc##1{\setlength{\fboxsep}{0pt}\fcolorbox[rgb]{1.00,0.00,0.00}{1,1,1}{\strut ##1}}}
\expandafter\def\csname PY@tok@kc\endcsname{\let\PY@bf=\textbf\def\PY@tc##1{\textcolor[rgb]{0.00,0.50,0.00}{##1}}}
\expandafter\def\csname PY@tok@kd\endcsname{\let\PY@bf=\textbf\def\PY@tc##1{\textcolor[rgb]{0.00,0.50,0.00}{##1}}}
\expandafter\def\csname PY@tok@kn\endcsname{\let\PY@bf=\textbf\def\PY@tc##1{\textcolor[rgb]{0.00,0.50,0.00}{##1}}}
\expandafter\def\csname PY@tok@kr\endcsname{\let\PY@bf=\textbf\def\PY@tc##1{\textcolor[rgb]{0.00,0.50,0.00}{##1}}}
\expandafter\def\csname PY@tok@bp\endcsname{\def\PY@tc##1{\textcolor[rgb]{0.00,0.50,0.00}{##1}}}
\expandafter\def\csname PY@tok@fm\endcsname{\def\PY@tc##1{\textcolor[rgb]{0.00,0.00,1.00}{##1}}}
\expandafter\def\csname PY@tok@vc\endcsname{\def\PY@tc##1{\textcolor[rgb]{0.10,0.09,0.49}{##1}}}
\expandafter\def\csname PY@tok@vg\endcsname{\def\PY@tc##1{\textcolor[rgb]{0.10,0.09,0.49}{##1}}}
\expandafter\def\csname PY@tok@vi\endcsname{\def\PY@tc##1{\textcolor[rgb]{0.10,0.09,0.49}{##1}}}
\expandafter\def\csname PY@tok@vm\endcsname{\def\PY@tc##1{\textcolor[rgb]{0.10,0.09,0.49}{##1}}}
\expandafter\def\csname PY@tok@sa\endcsname{\def\PY@tc##1{\textcolor[rgb]{0.73,0.13,0.13}{##1}}}
\expandafter\def\csname PY@tok@sb\endcsname{\def\PY@tc##1{\textcolor[rgb]{0.73,0.13,0.13}{##1}}}
\expandafter\def\csname PY@tok@sc\endcsname{\def\PY@tc##1{\textcolor[rgb]{0.73,0.13,0.13}{##1}}}
\expandafter\def\csname PY@tok@dl\endcsname{\def\PY@tc##1{\textcolor[rgb]{0.73,0.13,0.13}{##1}}}
\expandafter\def\csname PY@tok@s2\endcsname{\def\PY@tc##1{\textcolor[rgb]{0.73,0.13,0.13}{##1}}}
\expandafter\def\csname PY@tok@sh\endcsname{\def\PY@tc##1{\textcolor[rgb]{0.73,0.13,0.13}{##1}}}
\expandafter\def\csname PY@tok@s1\endcsname{\def\PY@tc##1{\textcolor[rgb]{0.73,0.13,0.13}{##1}}}
\expandafter\def\csname PY@tok@mb\endcsname{\def\PY@tc##1{\textcolor[rgb]{0.40,0.40,0.40}{##1}}}
\expandafter\def\csname PY@tok@mf\endcsname{\def\PY@tc##1{\textcolor[rgb]{0.40,0.40,0.40}{##1}}}
\expandafter\def\csname PY@tok@mh\endcsname{\def\PY@tc##1{\textcolor[rgb]{0.40,0.40,0.40}{##1}}}
\expandafter\def\csname PY@tok@mi\endcsname{\def\PY@tc##1{\textcolor[rgb]{0.40,0.40,0.40}{##1}}}
\expandafter\def\csname PY@tok@il\endcsname{\def\PY@tc##1{\textcolor[rgb]{0.40,0.40,0.40}{##1}}}
\expandafter\def\csname PY@tok@mo\endcsname{\def\PY@tc##1{\textcolor[rgb]{0.40,0.40,0.40}{##1}}}
\expandafter\def\csname PY@tok@ch\endcsname{\let\PY@it=\textit\def\PY@tc##1{\textcolor[rgb]{0.25,0.50,0.50}{##1}}}
\expandafter\def\csname PY@tok@cm\endcsname{\let\PY@it=\textit\def\PY@tc##1{\textcolor[rgb]{0.25,0.50,0.50}{##1}}}
\expandafter\def\csname PY@tok@cpf\endcsname{\let\PY@it=\textit\def\PY@tc##1{\textcolor[rgb]{0.25,0.50,0.50}{##1}}}
\expandafter\def\csname PY@tok@c1\endcsname{\let\PY@it=\textit\def\PY@tc##1{\textcolor[rgb]{0.25,0.50,0.50}{##1}}}
\expandafter\def\csname PY@tok@cs\endcsname{\let\PY@it=\textit\def\PY@tc##1{\textcolor[rgb]{0.25,0.50,0.50}{##1}}}

\def\PYZbs{\char`\\}
\def\PYZus{\char`\_}
\def\PYZob{\char`\{}
\def\PYZcb{\char`\}}
\def\PYZca{\char`\^}
\def\PYZam{\char`\&}
\def\PYZlt{\char`\<}
\def\PYZgt{\char`\>}
\def\PYZsh{\char`\#}
\def\PYZpc{\char`\%}
\def\PYZdl{\char`\$}
\def\PYZhy{\char`\-}
\def\PYZsq{\char`\'}
\def\PYZdq{\char`\"}
\def\PYZti{\char`\~}
% for compatibility with earlier versions
\def\PYZat{@}
\def\PYZlb{[}
\def\PYZrb{]}
\makeatother


    % For linebreaks inside Verbatim environment from package fancyvrb. 
    \makeatletter
        \newbox\Wrappedcontinuationbox 
        \newbox\Wrappedvisiblespacebox 
        \newcommand*\Wrappedvisiblespace {\textcolor{red}{\textvisiblespace}} 
        \newcommand*\Wrappedcontinuationsymbol {\textcolor{red}{\llap{\tiny$\m@th\hookrightarrow$}}} 
        \newcommand*\Wrappedcontinuationindent {3ex } 
        \newcommand*\Wrappedafterbreak {\kern\Wrappedcontinuationindent\copy\Wrappedcontinuationbox} 
        % Take advantage of the already applied Pygments mark-up to insert 
        % potential linebreaks for TeX processing. 
        %        {, <, #, %, $, ' and ": go to next line. 
        %        _, }, ^, &, >, - and ~: stay at end of broken line. 
        % Use of \textquotesingle for straight quote. 
        \newcommand*\Wrappedbreaksatspecials {% 
            \def\PYGZus{\discretionary{\char`\_}{\Wrappedafterbreak}{\char`\_}}% 
            \def\PYGZob{\discretionary{}{\Wrappedafterbreak\char`\{}{\char`\{}}% 
            \def\PYGZcb{\discretionary{\char`\}}{\Wrappedafterbreak}{\char`\}}}% 
            \def\PYGZca{\discretionary{\char`\^}{\Wrappedafterbreak}{\char`\^}}% 
            \def\PYGZam{\discretionary{\char`\&}{\Wrappedafterbreak}{\char`\&}}% 
            \def\PYGZlt{\discretionary{}{\Wrappedafterbreak\char`\<}{\char`\<}}% 
            \def\PYGZgt{\discretionary{\char`\>}{\Wrappedafterbreak}{\char`\>}}% 
            \def\PYGZsh{\discretionary{}{\Wrappedafterbreak\char`\#}{\char`\#}}% 
            \def\PYGZpc{\discretionary{}{\Wrappedafterbreak\char`\%}{\char`\%}}% 
            \def\PYGZdl{\discretionary{}{\Wrappedafterbreak\char`\$}{\char`\$}}% 
            \def\PYGZhy{\discretionary{\char`\-}{\Wrappedafterbreak}{\char`\-}}% 
            \def\PYGZsq{\discretionary{}{\Wrappedafterbreak\textquotesingle}{\textquotesingle}}% 
            \def\PYGZdq{\discretionary{}{\Wrappedafterbreak\char`\"}{\char`\"}}% 
            \def\PYGZti{\discretionary{\char`\~}{\Wrappedafterbreak}{\char`\~}}% 
        } 
        % Some characters . , ; ? ! / are not pygmentized. 
        % This macro makes them "active" and they will insert potential linebreaks 
        \newcommand*\Wrappedbreaksatpunct {% 
            \lccode`\~`\.\lowercase{\def~}{\discretionary{\hbox{\char`\.}}{\Wrappedafterbreak}{\hbox{\char`\.}}}% 
            \lccode`\~`\,\lowercase{\def~}{\discretionary{\hbox{\char`\,}}{\Wrappedafterbreak}{\hbox{\char`\,}}}% 
            \lccode`\~`\;\lowercase{\def~}{\discretionary{\hbox{\char`\;}}{\Wrappedafterbreak}{\hbox{\char`\;}}}% 
            \lccode`\~`\:\lowercase{\def~}{\discretionary{\hbox{\char`\:}}{\Wrappedafterbreak}{\hbox{\char`\:}}}% 
            \lccode`\~`\?\lowercase{\def~}{\discretionary{\hbox{\char`\?}}{\Wrappedafterbreak}{\hbox{\char`\?}}}% 
            \lccode`\~`\!\lowercase{\def~}{\discretionary{\hbox{\char`\!}}{\Wrappedafterbreak}{\hbox{\char`\!}}}% 
            \lccode`\~`\/\lowercase{\def~}{\discretionary{\hbox{\char`\/}}{\Wrappedafterbreak}{\hbox{\char`\/}}}% 
            \catcode`\.\active
            \catcode`\,\active 
            \catcode`\;\active
            \catcode`\:\active
            \catcode`\?\active
            \catcode`\!\active
            \catcode`\/\active 
            \lccode`\~`\~ 	
        }
    \makeatother

    \let\OriginalVerbatim=\Verbatim
    \makeatletter
    \renewcommand{\Verbatim}[1][1]{%
        %\parskip\z@skip
        \sbox\Wrappedcontinuationbox {\Wrappedcontinuationsymbol}%
        \sbox\Wrappedvisiblespacebox {\FV@SetupFont\Wrappedvisiblespace}%
        \def\FancyVerbFormatLine ##1{\hsize\linewidth
            \vtop{\raggedright\hyphenpenalty\z@\exhyphenpenalty\z@
                \doublehyphendemerits\z@\finalhyphendemerits\z@
                \strut ##1\strut}%
        }%
        % If the linebreak is at a space, the latter will be displayed as visible
        % space at end of first line, and a continuation symbol starts next line.
        % Stretch/shrink are however usually zero for typewriter font.
        \def\FV@Space {%
            \nobreak\hskip\z@ plus\fontdimen3\font minus\fontdimen4\font
            \discretionary{\copy\Wrappedvisiblespacebox}{\Wrappedafterbreak}
            {\kern\fontdimen2\font}%
        }%
        
        % Allow breaks at special characters using \PYG... macros.
        \Wrappedbreaksatspecials
        % Breaks at punctuation characters . , ; ? ! and / need catcode=\active 	
        \OriginalVerbatim[#1,codes*=\Wrappedbreaksatpunct]%
    }
    \makeatother

    % Exact colors from NB
    \definecolor{incolor}{HTML}{303F9F}
    \definecolor{outcolor}{HTML}{D84315}
    \definecolor{cellborder}{HTML}{CFCFCF}
    \definecolor{cellbackground}{HTML}{F7F7F7}
    
    % prompt
    \makeatletter
    \newcommand{\boxspacing}{\kern\kvtcb@left@rule\kern\kvtcb@boxsep}
    \makeatother
    \newcommand{\prompt}[4]{
        \ttfamily\llap{{\color{#2}[#3]:\hspace{3pt}#4}}\vspace{-\baselineskip}
    }
    

    
    % Prevent overflowing lines due to hard-to-break entities
    \sloppy 
    % Setup hyperref package
    \hypersetup{
      breaklinks=true,  % so long urls are correctly broken across lines
      colorlinks=true,
      urlcolor=urlcolor,
      linkcolor=linkcolor,
      citecolor=citecolor,
      }
    % Slightly bigger margins than the latex defaults
    
    \geometry{verbose,tmargin=1in,bmargin=1in,lmargin=1in,rmargin=1in}
    
    

\begin{document}
    
    \maketitle
    
    

    
    \hypertarget{using-jod-backup-get-bget}{%
\section{\texorpdfstring{Using JOD Backup Get
\texttt{bget}}{Using JOD Backup Get bget}}\label{using-jod-backup-get-bget}}

\includegraphics{inclusions/jodteenytinycube.png}

    \hypertarget{introduction}{%
\subsubsection{Introduction}\label{introduction}}

In addition to restoring entire dictionary backups with \texttt{restd}
JOD also supports fetching individual objects from particular backups
with \texttt{bget} and \texttt{bnl}.

If you screw up part of a larger system restoring \emph{all the code}
may create more problems than it solves. Usually you only want
\href{https://www.youtube.com/watch?v=wPiHQ37gXnE}{\emph{the good bits}}
of a backup.

\textbf{\texttt{bget} is your good bits
\href{https://www.youtube.com/watch?v=R8OWNspU_yE}{huckleberry}.}

    \begin{tcolorbox}[breakable, size=fbox, boxrule=1pt, pad at break*=1mm,colback=cellbackground, colframe=cellborder]
\prompt{In}{incolor}{1}{\boxspacing}
\begin{Verbatim}[commandchars=\\\{\}]
\PY{c+c1}{NB. display J version}
\PY{l+m+mi}{9}\PY{o}{!}\PY{o}{:}\PY{l+m+mi}{14}\PY{l+s}{\PYZsq{}}\PY{l+s}{\PYZsq{}}
\end{Verbatim}
\end{tcolorbox}

    \begin{Verbatim}[commandchars=\\\{\}]

j903/j64avx2/windows/beta-w/commercial/www.jsoftware.com/2021-12-05T18:25:00/cla
ng-13-0-0/SLEEF=1
    \end{Verbatim}

    The following examples assume you have installed the JOD development
dictionaries (\texttt{joddev}), (\texttt{jod}), and (\texttt{utils}).

Use \href{https://code.jsoftware.com/wiki/JAL/User_Guide}{Pacman} to
install the
\href{https://code.jsoftware.com/wiki/Addons/general/jodsource}{JODSOURCE
addon} and follow the instructions to load (\texttt{joddev}),
(\texttt{jod}), and (\texttt{utils}).

    \begin{tcolorbox}[breakable, size=fbox, boxrule=1pt, pad at break*=1mm,colback=cellbackground, colframe=cellborder]
\prompt{In}{incolor}{2}{\boxspacing}
\begin{Verbatim}[commandchars=\\\{\}]
\PY{c+c1}{NB. load JOD in a clear base locale}
\PY{n+nv}{load} \PY{l+s}{\PYZsq{}}\PY{l+s}{g}\PY{l+s}{e}\PY{l+s}{n}\PY{l+s}{e}\PY{l+s}{r}\PY{l+s}{a}\PY{l+s}{l}\PY{l+s}{/}\PY{l+s}{j}\PY{l+s}{o}\PY{l+s}{d}\PY{l+s}{\PYZsq{}} \PY{o}{[} \PY{n+nv}{clear} \PY{l+s}{\PYZsq{}}\PY{l+s}{\PYZsq{}}

\PY{c+c1}{NB. Converting Jupyter notebooks to LaTeX is }
\PY{c+c1}{NB. simplified by ASCII box characters.}
\PY{n+nv}{portchars} \PY{l+s}{\PYZsq{}}\PY{l+s}{\PYZsq{}}

\PY{c+c1}{NB. show JOD version}
\PY{n+nv}{smoutput} \PY{n+nv}{JODVMD\PYZus{}ajod\PYZus{}}
\end{Verbatim}
\end{tcolorbox}

    \begin{Verbatim}[commandchars=\\\{\}]
+------------+--+--------------------+
|1.0.22 - dev|35|14 Dec 2021 08:32:08|
+------------+--+--------------------+
    \end{Verbatim}

    \begin{tcolorbox}[breakable, size=fbox, boxrule=1pt, pad at break*=1mm,colback=cellbackground, colframe=cellborder]
\prompt{In}{incolor}{3}{\boxspacing}
\begin{Verbatim}[commandchars=\\\{\}]
\PY{c+c1}{NB. The distributed JOD profile automatically RESETME\PYZsq{}s.}
\PY{c+c1}{NB. To safely use dictionaries with many J tasks they must}
\PY{c+c1}{NB. be READONLY. To prevent opening the same put dictionary}
\PY{c+c1}{NB. READWRITE comment out (dpset) and restart this notebook.}
\PY{n+nv}{dpset} \PY{l+s}{\PYZsq{}}\PY{l+s}{R}\PY{l+s}{E}\PY{l+s}{S}\PY{l+s}{E}\PY{l+s}{T}\PY{l+s}{M}\PY{l+s}{E}\PY{l+s}{\PYZsq{}}

\PY{c+c1}{NB. Verb to show large boxed displays in}
\PY{c+c1}{NB. the notebook without ugly wrapping.}
\PY{n+nv}{sbx}\PY{o}{=:} \PY{l+s}{\PYZsq{}}\PY{l+s}{ }\PY{l+s}{.}\PY{l+s}{.}\PY{l+s}{.}\PY{l+s}{ }\PY{l+s}{\PYZsq{}} \PY{o}{,}\PY{o}{\PYZdq{}}\PY{l+m+mi}{1}\PY{o}{\PYZti{}} \PY{l+m+mi}{73}\PY{o}{\PYZam{}}\PY{o}{\PYZob{}}\PY{o}{.}\PY{o}{\PYZdq{}}\PY{l+m+mi}{1}\PY{o}{@}\PY{o}{\PYZdq{}}\PY{o}{:}

\PY{c+c1}{NB. open some JOD dictionaries to search}
\PY{n+nv}{od} \PY{o}{;}\PY{o}{:}\PY{l+s}{\PYZsq{}}\PY{l+s}{j}\PY{l+s}{o}\PY{l+s}{d}\PY{l+s}{d}\PY{l+s}{e}\PY{l+s}{v}\PY{l+s}{ }\PY{l+s}{j}\PY{l+s}{o}\PY{l+s}{d}\PY{l+s}{ }\PY{l+s}{u}\PY{l+s}{t}\PY{l+s}{i}\PY{l+s}{l}\PY{l+s}{s}\PY{l+s}{\PYZsq{}} \PY{o}{[} \PY{l+m+mi}{3} \PY{n+nv}{od} \PY{l+s}{\PYZsq{}}\PY{l+s}{\PYZsq{}}
\PY{n+nv}{did} \PY{o}{\PYZti{}} \PY{l+m+mi}{0}
\end{Verbatim}
\end{tcolorbox}

    \begin{Verbatim}[commandchars=\\\{\}]
+-+----------------------------------------------------------------+
|1|+------+--+-----+-----+-------+-------+------+-----------------+|
| ||      |--|Words|Tests|Groups*|Suites*|Macros|Path*            ||
| |+------+--+-----+-----+-------+-------+------+-----------------+|
| ||joddev|rw|31   |4    |6      |2      |21    |/joddev/jod/utils||
| |+------+--+-----+-----+-------+-------+------+-----------------+|
| ||jod   |ro|841  |76   |23     |13     |74    |/jod/utils       ||
| |+------+--+-----+-----+-------+-------+------+-----------------+|
| ||utils |ro|419  |8    |23     |0      |19    |/utils           ||
| |+------+--+-----+-----+-------+-------+------+-----------------+|
+-+----------------------------------------------------------------+
    \end{Verbatim}

    \hypertarget{bnl-lists-available-backups}{%
\subsubsection{\texorpdfstring{\texttt{bnl} lists available
backups}{bnl lists available backups}}\label{bnl-lists-available-backups}}

JOD binary backups are created with \texttt{packd}. When \texttt{packd}
is run it copies current dictionary files to the backup folder and
renames them with an ever increasing backup number prefix.

    \begin{tcolorbox}[breakable, size=fbox, boxrule=1pt, pad at break*=1mm,colback=cellbackground, colframe=cellborder]
\prompt{In}{incolor}{4}{\boxspacing}
\begin{Verbatim}[commandchars=\\\{\}]
\PY{c+c1}{NB. list all available put dictionary backups}
\PY{n+nv}{sbx} \PY{n+nv}{bnl} \PY{l+s}{\PYZsq{}}\PY{l+s}{.}\PY{l+s}{\PYZsq{}}
\end{Verbatim}
\end{tcolorbox}

    \begin{Verbatim}[commandchars=\\\{\}]
+-+---+---+---+---+---+---+---+---+---+                                   {\ldots}
|1|.38|.37|.36|.35|.34|.33|.32|.31|.30|                                   {\ldots}
+-+---+---+---+---+---+---+---+---+---+                                   {\ldots}
    \end{Verbatim}

    \begin{tcolorbox}[breakable, size=fbox, boxrule=1pt, pad at break*=1mm,colback=cellbackground, colframe=cellborder]
\prompt{In}{incolor}{5}{\boxspacing}
\begin{Verbatim}[commandchars=\\\{\}]
\PY{c+c1}{NB. dates of backups}
\PY{l+m+mi}{14} \PY{n+nv}{bnl} \PY{l+s}{\PYZsq{}}\PY{l+s}{.}\PY{l+s}{\PYZsq{}}
\end{Verbatim}
\end{tcolorbox}

    \begin{Verbatim}[commandchars=\\\{\}]
+-+-------------------------+
|1|+--+--------------------+|
| ||38|11 Dec 2021 14:38:19||
| |+--+--------------------+|
| ||37|10 Dec 2021 11:53:14||
| |+--+--------------------+|
| ||36|03 Dec 2021 09:02:39||
| |+--+--------------------+|
| ||35|12 Sep 2021 11:27:55||
| |+--+--------------------+|
| ||34|05 Aug 2021 10:37:17||
| |+--+--------------------+|
| ||33|05 Dec 2020 09:12:45||
| |+--+--------------------+|
| ||32|02 Dec 2020 11:03:12||
| |+--+--------------------+|
| ||31|01 Dec 2020 10:46:44||
| |+--+--------------------+|
| ||30|30 Nov 2020 13:55:56||
| |+--+--------------------+|
+-+-------------------------+
    \end{Verbatim}

    \hypertarget{bnl-lists-objects-in-backups}{%
\subsubsection{\texorpdfstring{\texttt{bnl} lists objects in
backups}{bnl lists objects in backups}}\label{bnl-lists-objects-in-backups}}

\texttt{bnl} lists objects in backups. See jod.pdf for more details.

    \begin{tcolorbox}[breakable, size=fbox, boxrule=1pt, pad at break*=1mm,colback=cellbackground, colframe=cellborder]
\prompt{In}{incolor}{6}{\boxspacing}
\begin{Verbatim}[commandchars=\\\{\}]
\PY{c+c1}{NB. list all words in last backup}
\PY{n+nv}{smoutout} \PY{n+nv}{sbx} \PY{n+nv}{bnl} \PY{l+s}{\PYZsq{}}\PY{l+s}{\PYZsq{}}

\PY{c+c1}{NB. list all test cases in last backup}
\PY{n+nv}{smoutput} \PY{n+nv}{sbx} \PY{l+m+mi}{1} \PY{n+nv}{bnl} \PY{l+s}{\PYZsq{}}\PY{l+s}{\PYZsq{}}

\PY{c+c1}{NB. oldest backup}
\PY{n+nv}{smoutput} \PY{n+nv}{OldestBnum}\PY{o}{=:} \PY{o}{;} \PY{o}{\PYZob{}}\PY{o}{:} \PY{n+nv}{bnl} \PY{l+s}{\PYZsq{}}\PY{l+s}{.}\PY{l+s}{\PYZsq{}}

\PY{c+c1}{NB. list all macros in oldest backup}
\PY{n+nv}{smoutput} \PY{n+nv}{sbx} \PY{l+m+mi}{4} \PY{n+nv}{bnl} \PY{n+nv}{OldestBnum}

\PY{c+c1}{NB. list all macros with names starting with \PYZdq{}JOD\PYZdq{} in the oldest backup}
\PY{l+m+mi}{4} \PY{n+nv}{bnl} \PY{l+s}{\PYZsq{}}\PY{l+s}{J}\PY{l+s}{O}\PY{l+s}{D}\PY{l+s}{\PYZsq{}}\PY{o}{,}\PY{n+nv}{OldestBnum}
\end{Verbatim}
\end{tcolorbox}

    \begin{Verbatim}[commandchars=\\\{\}]
+-+----------+----------------+----------------+--------------+           {\ldots}
|1|abvSmoke00|buildjodliterate|globsBasicDDef00|globsSmokeDD00|           {\ldots}
+-+----------+----------------+----------------+--------------+           {\ldots}
.30
+-+---------------+--------------------+-----------+-------------------+- {\ldots}
|1|JODBUILDHISTORY|JODTOOLSBUILDHISTORY|TODO\_JOD\_md|TODO\_jodliterate\_md|b {\ldots}
+-+---------------+--------------------+-----------+-------------------+- {\ldots}
+-+---------------+--------------------+
|1|JODBUILDHISTORY|JODTOOLSBUILDHISTORY|
+-+---------------+--------------------+
    \end{Verbatim}

    \hypertarget{abv-a-backup-name-list-helper-verb}{%
\subsubsection{\texorpdfstring{\texttt{abv} a backup name list helper
verb}{abv a backup name list helper verb}}\label{abv-a-backup-name-list-helper-verb}}

To streamline the search for backup objects recent JOD versions
(1.0.22+) provides \texttt{abv} (all backup versions). \texttt{abv}
returns backup names from the most recent to oldest. The order derives
from backup file numbers.

    \begin{tcolorbox}[breakable, size=fbox, boxrule=1pt, pad at break*=1mm,colback=cellbackground, colframe=cellborder]
\prompt{In}{incolor}{7}{\boxspacing}
\begin{Verbatim}[commandchars=\\\{\}]
\PY{c+c1}{NB. show first ten rows of all backup versions of all words in all backups}
\PY{n+nv}{sbx} \PY{l+m+mi}{10} \PY{o}{\PYZob{}}\PY{o}{.} \PY{n+nv}{de} \PY{o}{\PYZcb{}}\PY{o}{.} \PY{n+nv}{abv} \PY{l+s}{\PYZsq{}}\PY{l+s}{\PYZsq{}}
\end{Verbatim}
\end{tcolorbox}

    \begin{Verbatim}[commandchars=\\\{\}]
uwlatexfrwords.37          uwlatexfrwords.36          uwlatexfrwords.35   {\ldots}
uwlatexfrwords.32          uwlatexfrwords.31          usedby.38           {\ldots}
usedby.35                  ppcodelatex2.38            ppcodelatex2.37     {\ldots}
ppcodelatex2.34            ppcodelatex2.33            ppcodelatex2.32     {\ldots}
ppcodelatex.38             ppcodelatex.37             ppcodelatex.36      {\ldots}
ppcodelatex.33             ppcodelatex.32             ppcodelatex.31      {\ldots}
parsecode.37               markgassign.38             markgassign.37      {\ldots}
markgassign.34             markgassign.33             markgassign.32      {\ldots}
markdfrwords.37            markdfrwords.36            markdfrwords.35     {\ldots}
markdfrwords.32            markdfrwords.31            jodon.38            {\ldots}
    \end{Verbatim}

    \begin{tcolorbox}[breakable, size=fbox, boxrule=1pt, pad at break*=1mm,colback=cellbackground, colframe=cellborder]
\prompt{In}{incolor}{8}{\boxspacing}
\begin{Verbatim}[commandchars=\\\{\}]
\PY{c+c1}{NB. all words in all backups beginning with \PYZdq{}m\PYZdq{}}
\PY{n+nv}{sbx} \PY{n+nv}{abv} \PY{l+s}{\PYZsq{}}\PY{l+s}{m}\PY{l+s}{\PYZsq{}}
\end{Verbatim}
\end{tcolorbox}

    \begin{Verbatim}[commandchars=\\\{\}]
+-+--------------+--------------+--------------+--------------+---------- {\ldots}
|1|markgassign.38|markgassign.37|markgassign.36|markgassign.35|markgassig {\ldots}
+-+--------------+--------------+--------------+--------------+---------- {\ldots}
    \end{Verbatim}

    \begin{tcolorbox}[breakable, size=fbox, boxrule=1pt, pad at break*=1mm,colback=cellbackground, colframe=cellborder]
\prompt{In}{incolor}{9}{\boxspacing}
\begin{Verbatim}[commandchars=\\\{\}]
\PY{c+c1}{NB. all groups in all backups starting with \PYZdq{}jod\PYZdq{} \PYZhy{} note the backup suffix numbers}
\PY{l+m+mi}{80} \PY{n+nv}{list} \PY{o}{\PYZcb{}}\PY{o}{.} \PY{l+m+mi}{2} \PY{n+nv}{abv} \PY{l+s}{\PYZsq{}}\PY{l+s}{j}\PY{l+s}{o}\PY{l+s}{d}\PY{l+s}{\PYZsq{}}
\end{Verbatim}
\end{tcolorbox}

    \begin{Verbatim}[commandchars=\\\{\}]
jodutil.38                  jodutil.37
jodutil.36                  jodutil.35
jodutil.34                  jodutil.33
jodutil.32                  jodutil.31
jodutil.30                  jodtools.38
jodtools.37                 jodtools.36
jodtools.35                 jodtools.34
jodtools.33                 jodtools.32
jodtools.31                 jodtools.30
jodmake.38                  jodmake.37
jodliterateTroubleMakers.38 jodliterateTroubleMakers.37
jodliterateTroubleMakers.36 jodliterateTroubleMakers.35
jodliterateTroubleMakers.34 jodliterateTroubleMakers.33
jodliterateTroubleMakers.32 jodliterateTroubleMakers.31
jodliterateTroubleMakers.30 jodliterate.38
jodliterate.37              jodliterate.36
jodliterate.35              jodliterate.34
jodliterate.33              jodliterate.32
jodliterate.31              jodliterate.30
jod.38                      jod.37
jod.36                      jod.35
jod.34                      jod.33
jod.32                      jod.31
jod.30
    \end{Verbatim}

    \hypertarget{bget-fetches-objects-from-backups}{%
\subsubsection{\texorpdfstring{\texttt{bget} fetches objects from
backups}{bget fetches objects from backups}}\label{bget-fetches-objects-from-backups}}

\texttt{bget} fetches objects from backups. See jod.pdf for more
details.

Unlike \texttt{get} the verb \texttt{bget} does not define fetched
objects in locales. The reason for this is simple, \texttt{bget} may
return many versions of the same object. Which one should you restore?
JOD lets the user decide.

    \begin{tcolorbox}[breakable, size=fbox, boxrule=1pt, pad at break*=1mm,colback=cellbackground, colframe=cellborder]
\prompt{In}{incolor}{10}{\boxspacing}
\begin{Verbatim}[commandchars=\\\{\}]
\PY{c+c1}{NB. select a random word from the last backup}
\PY{n+nv}{AllWords}\PY{o}{=:} \PY{o}{\PYZcb{}}\PY{o}{.} \PY{n+nv}{bnl}\PY{l+s}{\PYZsq{}}\PY{l+s}{\PYZsq{}}
\PY{n+nv}{smoutput} \PY{n+nv}{RandWord}\PY{o}{=:} \PY{o}{;}\PY{p}{(}\PY{n}{?}\PY{o}{\PYZsh{}}\PY{n+nv}{AllWords}\PY{p}{)} \PY{o}{\PYZob{}} \PY{n+nv}{AllWords}

\PY{c+c1}{NB. fetch the word}
\PY{n+nv}{sbx} \PY{n+nv}{bget} \PY{n+nv}{RandWord}
\end{Verbatim}
\end{tcolorbox}

    \begin{Verbatim}[commandchars=\\\{\}]
jodhelp
+-+---------------------------------------------------------------------- {\ldots}
|1|+----------+-+-------------------------------------------------------- {\ldots}
| ||jodhelp\_38|3|3 : 0  NB.*jodhelp v-- display PDF JOD help. NB. NB. mon {\ldots}
| |+----------+-+-------------------------------------------------------- {\ldots}
+-+---------------------------------------------------------------------- {\ldots}
    \end{Verbatim}

    \hypertarget{bget-can-fetch-many-objects}{%
\subsubsection{\texorpdfstring{\texttt{bget} can fetch many
objects}{bget can fetch many objects}}\label{bget-can-fetch-many-objects}}

\texttt{bget} can fetch many objects. The objects are returned in boxed
name, class and value tables. For words and macros these tables have
three columns for other objects there are two columns.

    \begin{tcolorbox}[breakable, size=fbox, boxrule=1pt, pad at break*=1mm,colback=cellbackground, colframe=cellborder]
\prompt{In}{incolor}{11}{\boxspacing}
\begin{Verbatim}[commandchars=\\\{\}]
\PY{c+c1}{NB. fetch the last five versions of words starting with \PYZdq{}jodo\PYZdq{}}
\PY{c+c1}{NB. NOTE: the backup number suffixes of the names in the first column of (ncv)}
\PY{l+s}{\PYZsq{}}\PY{l+s}{r}\PY{l+s}{c}\PY{l+s}{ }\PY{l+s}{n}\PY{l+s}{c}\PY{l+s}{v}\PY{l+s}{\PYZsq{}}\PY{o}{=:} \PY{n+nv}{bget} \PY{l+m+mi}{5} \PY{o}{\PYZob{}}\PY{o}{.} \PY{o}{\PYZcb{}}\PY{o}{.} \PY{n+nv}{abv} \PY{l+s}{\PYZsq{}}\PY{l+s}{j}\PY{l+s}{o}\PY{l+s}{d}\PY{l+s}{o}\PY{l+s}{\PYZsq{}}
\PY{n+nv}{sbx} \PY{n+nv}{ncv}
\end{Verbatim}
\end{tcolorbox}

    \begin{Verbatim}[commandchars=\\\{\}]
+--------+-+------------------------------------------------------------- {\ldots}
|jodon\_38|3|3 : 0  NB.*jodon v--  turn  JOD on  result  is 1  if  success {\ldots}
+--------+-+------------------------------------------------------------- {\ldots}
|jodon\_37|3|3 : 0  NB.*jodon v--  turn  JOD on  result  is 1  if  success {\ldots}
+--------+-+------------------------------------------------------------- {\ldots}
|jodon\_36|3|3 : 0  NB.*jodon v--  turn  JOD on  result  is 1  if  success {\ldots}
+--------+-+------------------------------------------------------------- {\ldots}
|jodon\_35|3|3 : 0  NB.*jodon v--  turn  JOD on  result  is 1  if  success {\ldots}
+--------+-+------------------------------------------------------------- {\ldots}
|jodon\_34|3|3 : 0  NB.*jodon v--  turn  JOD on  result  is 1  if  success {\ldots}
+--------+-+------------------------------------------------------------- {\ldots}
    \end{Verbatim}

    \begin{tcolorbox}[breakable, size=fbox, boxrule=1pt, pad at break*=1mm,colback=cellbackground, colframe=cellborder]
\prompt{In}{incolor}{12}{\boxspacing}
\begin{Verbatim}[commandchars=\\\{\}]
\PY{c+c1}{NB. fetch the last ten versions of the short explanations of all words starting with \PYZdq{}bc\PYZdq{}}
\PY{c+c1}{NB. smoutput \PYZcb{}. abv \PYZsq{}jo\PYZsq{}}
\PY{l+s}{\PYZsq{}}\PY{l+s}{r}\PY{l+s}{c}\PY{l+s}{ }\PY{l+s}{n}\PY{l+s}{c}\PY{l+s}{v}\PY{l+s}{\PYZsq{}}\PY{o}{=:} \PY{l+m+mi}{0} \PY{l+m+mi}{8} \PY{n+nv}{bget} \PY{l+m+mi}{10} \PY{o}{\PYZob{}}\PY{o}{.} \PY{o}{\PYZcb{}}\PY{o}{.} \PY{n+nv}{abv} \PY{l+s}{\PYZsq{}}\PY{l+s}{j}\PY{l+s}{o}\PY{l+s}{\PYZsq{}}
\PY{n+nv}{sbx} \PY{n+nv}{ncv}
\end{Verbatim}
\end{tcolorbox}

    \begin{Verbatim}[commandchars=\\\{\}]
+-------------------+---------------------------------------------------- {\ldots}
|jodon\_38           |turn JOD on result is 1 if successful and 0 otherwis {\ldots}
+-------------------+---------------------------------------------------- {\ldots}
|jodon\_37           |turn JOD on result is 1 if successful and 0 otherwis {\ldots}
+-------------------+---------------------------------------------------- {\ldots}
|jodon\_36           |turn JOD on result is 1 if successful and 0 otherwis {\ldots}
+-------------------+---------------------------------------------------- {\ldots}
|jodon\_35           |turn JOD on result is 1 if successful and 0 otherwis {\ldots}
+-------------------+---------------------------------------------------- {\ldots}
|jodon\_34           |turn JOD on result is 1 if successful and 0 otherwis {\ldots}
+-------------------+---------------------------------------------------- {\ldots}
|jodon\_33           |turn JOD on result is 1 if successful and 0 otherwis {\ldots}
+-------------------+---------------------------------------------------- {\ldots}
|jodon\_32           |turn JOD on result is 1 if successful and 0 otherwis {\ldots}
+-------------------+---------------------------------------------------- {\ldots}
|jodon\_31           |turn JOD on result is 1 if successful and 0 otherwis {\ldots}
+-------------------+---------------------------------------------------- {\ldots}
|jodon\_30           |turn JOD on result is 1 if successful and 0 otherwis {\ldots}
+-------------------+---------------------------------------------------- {\ldots}
|jodliterate\_root\_38|                                                     {\ldots}
+-------------------+---------------------------------------------------- {\ldots}
    \end{Verbatim}

    \hypertarget{bget-results-can-be-edited-with-ed}{%
\subsubsection{\texorpdfstring{\texttt{bget} results can be edited with
\texttt{ed}}{bget results can be edited with ed}}\label{bget-results-can-be-edited-with-ed}}

\texttt{bget} name, class and value tables can be edited with the JOD
\texttt{ed} verb. JOD \texttt{ed} works differently depending on what J
IDE you are using but the basic approach is to generate script text from
object data, write the script to a file, and then open a J editor. The
following examples illustrate typical uses.

    \begin{tcolorbox}[breakable, size=fbox, boxrule=1pt, pad at break*=1mm,colback=cellbackground, colframe=cellborder]
\prompt{In}{incolor}{13}{\boxspacing}
\begin{Verbatim}[commandchars=\\\{\}]
\PY{c+c1}{NB. edit the most recent version of a word}
\PY{l+s}{\PYZsq{}}\PY{l+s}{r}\PY{l+s}{c}\PY{l+s}{ }\PY{l+s}{n}\PY{l+s}{c}\PY{l+s}{v}\PY{l+s}{\PYZsq{}}\PY{o}{=:} \PY{n+nv}{bget} \PY{l+s}{\PYZsq{}}\PY{l+s}{b}\PY{l+s}{c}\PY{l+s}{h}\PY{l+s}{e}\PY{l+s}{c}\PY{l+s}{k}\PY{l+s}{n}\PY{l+s}{a}\PY{l+s}{m}\PY{l+s}{e}\PY{l+s}{s}\PY{l+s}{\PYZsq{}}

\PY{c+c1}{NB. requires browser/file permissions and pop ups enabled}
\PY{c+c1}{NB. (ed) is typically used in JQT and JHS environments and  }
\PY{c+c1}{NB. may or may not work in Jupyter notebooks.}
\PY{c+c1}{NB. ed ncv}
\end{Verbatim}
\end{tcolorbox}

    \begin{tcolorbox}[breakable, size=fbox, boxrule=1pt, pad at break*=1mm,colback=cellbackground, colframe=cellborder]
\prompt{In}{incolor}{14}{\boxspacing}
\begin{Verbatim}[commandchars=\\\{\}]
\PY{c+c1}{NB. edit all group header text}
\PY{l+s}{\PYZsq{}}\PY{l+s}{r}\PY{l+s}{c}\PY{l+s}{ }\PY{l+s}{n}\PY{l+s}{c}\PY{l+s}{v}\PY{l+s}{\PYZsq{}}\PY{o}{=:} \PY{l+m+mi}{2} \PY{n+nv}{bget} \PY{o}{\PYZcb{}}\PY{o}{.} \PY{l+m+mi}{2} \PY{n+nv}{bnl} \PY{l+s}{\PYZsq{}}\PY{l+s}{\PYZsq{}}
\PY{n+nv}{smoutput} \PY{n+nv}{sbx} \PY{n+nv}{ncv}

\PY{c+c1}{NB. ed ncv}
\end{Verbatim}
\end{tcolorbox}

    \begin{Verbatim}[commandchars=\\\{\}]
+---------------------------+-------------------------------------------- {\ldots}
|jod\_38                     |NB. *jod c-- main JOD dictionary class.  NB. {\ldots}
+---------------------------+-------------------------------------------- {\ldots}
|jodliterate\_38             |NB.*jodliterate s-- generates literate sourc {\ldots}
+---------------------------+-------------------------------------------- {\ldots}
|jodliterateTroubleMakers\_38|    NB. WARNING: long line breaking GROUP HE {\ldots}
+---------------------------+-------------------------------------------- {\ldots}
|jodmake\_38                 |NB. *jodmake c-- script making \& code manipu {\ldots}
+---------------------------+-------------------------------------------- {\ldots}
|jodtools\_38                |NB.*jodtools c-- derived tools class: extens {\ldots}
+---------------------------+-------------------------------------------- {\ldots}
|jodutil\_38                 |NB. *jodutil c-- a collection of JOD utility {\ldots}
+---------------------------+-------------------------------------------- {\ldots}
    \end{Verbatim}

    \begin{tcolorbox}[breakable, size=fbox, boxrule=1pt, pad at break*=1mm,colback=cellbackground, colframe=cellborder]
\prompt{In}{incolor}{15}{\boxspacing}
\begin{Verbatim}[commandchars=\\\{\}]
\PY{c+c1}{NB. edit all versions of all macros that begin with \PYZdq{}manifestjoddoc\PYZdq{}}
\PY{l+s}{\PYZsq{}}\PY{l+s}{r}\PY{l+s}{c}\PY{l+s}{ }\PY{l+s}{n}\PY{l+s}{c}\PY{l+s}{v}\PY{l+s}{\PYZsq{}}\PY{o}{=:} \PY{l+m+mi}{4} \PY{n+nv}{bget} \PY{o}{\PYZcb{}}\PY{o}{.} \PY{l+m+mi}{4} \PY{n+nv}{abv} \PY{l+s}{\PYZsq{}}\PY{l+s}{m}\PY{l+s}{a}\PY{l+s}{n}\PY{l+s}{i}\PY{l+s}{f}\PY{l+s}{e}\PY{l+s}{s}\PY{l+s}{t}\PY{l+s}{j}\PY{l+s}{o}\PY{l+s}{d}\PY{l+s}{d}\PY{l+s}{o}\PY{l+s}{c}\PY{l+s}{\PYZsq{}}
\PY{n+nv}{smoutput} \PY{l+m+mi}{80} \PY{n+nv}{list} \PY{l+m+mi}{0} \PY{o}{\PYZob{}}\PY{o}{\PYZdq{}}\PY{l+m+mi}{1} \PY{n+nv}{ncv}

\PY{c+c1}{NB. ed ncv}
\end{Verbatim}
\end{tcolorbox}

    \begin{Verbatim}[commandchars=\\\{\}]
manifestjoddocument\_38 manifestjoddocument\_37 manifestjoddocument\_36
manifestjoddocument\_35 manifestjoddocument\_34 manifestjoddocument\_33
manifestjoddocument\_32 manifestjoddocument\_31 manifestjoddocument\_30
    \end{Verbatim}

    \hypertarget{final-remarks}{%
\subsubsection{Final remarks}\label{final-remarks}}

\texttt{bget} and \texttt{bnl} make it easy to recover objects in JOD
binary backups. The objects are not defined in locales. You must edit
them with \texttt{ed} and select what you need. Object names are
suffixed with backup numbers. This is done to distinquish object
versions. Typically these backup numbers will be removed when editing
recovered objects.

\begin{verbatim}
John Baker 
December 2021
bakerjd99@gmail.com
\end{verbatim}


    % Add a bibliography block to the postdoc
    
    
    
\end{document}
