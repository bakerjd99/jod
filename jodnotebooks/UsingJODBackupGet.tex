
% Default to the notebook output style

    


% Inherit from the specified cell style.




\documentclass[11pt,letter,landscape]{article}  
%\documentclass[11pt]{article}

    
    
    \usepackage[T1]{fontenc}
    % Nicer default font (+ math font) than Computer Modern for most use cases
    \usepackage{mathpazo}

    % Basic figure setup, for now with no caption control since it's done
    % automatically by Pandoc (which extracts ![](path) syntax from Markdown).
    \usepackage{graphicx}
    % We will generate all images so they have a width \maxwidth. This means
    % that they will get their normal width if they fit onto the page, but
    % are scaled down if they would overflow the margins.
    \makeatletter
    \def\maxwidth{\ifdim\Gin@nat@width>\linewidth\linewidth
    \else\Gin@nat@width\fi}
    \makeatother
    \let\Oldincludegraphics\includegraphics
    % Set max figure width to be 80% of text width, for now hardcoded.
    \renewcommand{\includegraphics}[1]{\Oldincludegraphics[width=.8\maxwidth]{#1}}
    % Ensure that by default, figures have no caption (until we provide a
    % proper Figure object with a Caption API and a way to capture that
    % in the conversion process - todo).
    \usepackage{caption}
    \DeclareCaptionLabelFormat{nolabel}{}
    \captionsetup{labelformat=nolabel}

    \usepackage{adjustbox} % Used to constrain images to a maximum size 
    \usepackage{xcolor} % Allow colors to be defined
    \usepackage{enumerate} % Needed for markdown enumerations to work
    \usepackage{geometry} % Used to adjust the document margins
    \usepackage{amsmath} % Equations
    \usepackage{amssymb} % Equations
    \usepackage{textcomp} % defines textquotesingle
    % Hack from http://tex.stackexchange.com/a/47451/13684:
    \AtBeginDocument{%
        \def\PYZsq{\textquotesingle}% Upright quotes in Pygmentized code
    }
    \usepackage{upquote} % Upright quotes for verbatim code
    \usepackage{eurosym} % defines \euro
    \usepackage[mathletters]{ucs} % Extended unicode (utf-8) support
    \usepackage[utf8x]{inputenc} % Allow utf-8 characters in the tex document
    \usepackage{fancyvrb} % verbatim replacement that allows latex
    \usepackage{grffile} % extends the file name processing of package graphics 
                         % to support a larger range 
    % The hyperref package gives us a pdf with properly built
    % internal navigation ('pdf bookmarks' for the table of contents,
    % internal cross-reference links, web links for URLs, etc.)
    \usepackage{hyperref}
    \usepackage{longtable} % longtable support required by pandoc >1.10
    \usepackage{booktabs}  % table support for pandoc > 1.12.2
    \usepackage[inline]{enumitem} % IRkernel/repr support (it uses the enumerate* environment)
    \usepackage[normalem]{ulem} % ulem is needed to support strikethroughs (\sout)
                                % normalem makes italics be italics, not underlines
    

    
    
    % Colors for the hyperref package
    \definecolor{urlcolor}{rgb}{0,.145,.698}
    \definecolor{linkcolor}{rgb}{.71,0.21,0.01}
    \definecolor{citecolor}{rgb}{.12,.54,.11}

    % ANSI colors
    \definecolor{ansi-black}{HTML}{3E424D}
    \definecolor{ansi-black-intense}{HTML}{282C36}
    \definecolor{ansi-red}{HTML}{E75C58}
    \definecolor{ansi-red-intense}{HTML}{B22B31}
    \definecolor{ansi-green}{HTML}{00A250}
    \definecolor{ansi-green-intense}{HTML}{007427}
    \definecolor{ansi-yellow}{HTML}{DDB62B}
    \definecolor{ansi-yellow-intense}{HTML}{B27D12}
    \definecolor{ansi-blue}{HTML}{208FFB}
    \definecolor{ansi-blue-intense}{HTML}{0065CA}
    \definecolor{ansi-magenta}{HTML}{D160C4}
    \definecolor{ansi-magenta-intense}{HTML}{A03196}
    \definecolor{ansi-cyan}{HTML}{60C6C8}
    \definecolor{ansi-cyan-intense}{HTML}{258F8F}
    \definecolor{ansi-white}{HTML}{C5C1B4}
    \definecolor{ansi-white-intense}{HTML}{A1A6B2}

    % commands and environments needed by pandoc snippets
    % extracted from the output of `pandoc -s`
    \providecommand{\tightlist}{%
      \setlength{\itemsep}{0pt}\setlength{\parskip}{0pt}}
    \DefineVerbatimEnvironment{Highlighting}{Verbatim}{commandchars=\\\{\}}
    % Add ',fontsize=\small' for more characters per line
    \newenvironment{Shaded}{}{}
    \newcommand{\KeywordTok}[1]{\textcolor[rgb]{0.00,0.44,0.13}{\textbf{{#1}}}}
    \newcommand{\DataTypeTok}[1]{\textcolor[rgb]{0.56,0.13,0.00}{{#1}}}
    \newcommand{\DecValTok}[1]{\textcolor[rgb]{0.25,0.63,0.44}{{#1}}}
    \newcommand{\BaseNTok}[1]{\textcolor[rgb]{0.25,0.63,0.44}{{#1}}}
    \newcommand{\FloatTok}[1]{\textcolor[rgb]{0.25,0.63,0.44}{{#1}}}
    \newcommand{\CharTok}[1]{\textcolor[rgb]{0.25,0.44,0.63}{{#1}}}
    \newcommand{\StringTok}[1]{\textcolor[rgb]{0.25,0.44,0.63}{{#1}}}
    \newcommand{\CommentTok}[1]{\textcolor[rgb]{0.38,0.63,0.69}{\textit{{#1}}}}
    \newcommand{\OtherTok}[1]{\textcolor[rgb]{0.00,0.44,0.13}{{#1}}}
    \newcommand{\AlertTok}[1]{\textcolor[rgb]{1.00,0.00,0.00}{\textbf{{#1}}}}
    \newcommand{\FunctionTok}[1]{\textcolor[rgb]{0.02,0.16,0.49}{{#1}}}
    \newcommand{\RegionMarkerTok}[1]{{#1}}
    \newcommand{\ErrorTok}[1]{\textcolor[rgb]{1.00,0.00,0.00}{\textbf{{#1}}}}
    \newcommand{\NormalTok}[1]{{#1}}
    
    % Additional commands for more recent versions of Pandoc
    \newcommand{\ConstantTok}[1]{\textcolor[rgb]{0.53,0.00,0.00}{{#1}}}
    \newcommand{\SpecialCharTok}[1]{\textcolor[rgb]{0.25,0.44,0.63}{{#1}}}
    \newcommand{\VerbatimStringTok}[1]{\textcolor[rgb]{0.25,0.44,0.63}{{#1}}}
    \newcommand{\SpecialStringTok}[1]{\textcolor[rgb]{0.73,0.40,0.53}{{#1}}}
    \newcommand{\ImportTok}[1]{{#1}}
    \newcommand{\DocumentationTok}[1]{\textcolor[rgb]{0.73,0.13,0.13}{\textit{{#1}}}}
    \newcommand{\AnnotationTok}[1]{\textcolor[rgb]{0.38,0.63,0.69}{\textbf{\textit{{#1}}}}}
    \newcommand{\CommentVarTok}[1]{\textcolor[rgb]{0.38,0.63,0.69}{\textbf{\textit{{#1}}}}}
    \newcommand{\VariableTok}[1]{\textcolor[rgb]{0.10,0.09,0.49}{{#1}}}
    \newcommand{\ControlFlowTok}[1]{\textcolor[rgb]{0.00,0.44,0.13}{\textbf{{#1}}}}
    \newcommand{\OperatorTok}[1]{\textcolor[rgb]{0.40,0.40,0.40}{{#1}}}
    \newcommand{\BuiltInTok}[1]{{#1}}
    \newcommand{\ExtensionTok}[1]{{#1}}
    \newcommand{\PreprocessorTok}[1]{\textcolor[rgb]{0.74,0.48,0.00}{{#1}}}
    \newcommand{\AttributeTok}[1]{\textcolor[rgb]{0.49,0.56,0.16}{{#1}}}
    \newcommand{\InformationTok}[1]{\textcolor[rgb]{0.38,0.63,0.69}{\textbf{\textit{{#1}}}}}
    \newcommand{\WarningTok}[1]{\textcolor[rgb]{0.38,0.63,0.69}{\textbf{\textit{{#1}}}}}
    
    
    % Define a nice break command that doesn't care if a line doesn't already
    % exist.
    \def\br{\hspace*{\fill} \\* }
    % Math Jax compatability definitions
    \def\gt{>}
    \def\lt{<}
    % Document parameters
    \title{Using JOD Backup Get}
    
    
    

    % Pygments definitions
    
\makeatletter
\def\PY@reset{\let\PY@it=\relax \let\PY@bf=\relax%
    \let\PY@ul=\relax \let\PY@tc=\relax%
    \let\PY@bc=\relax \let\PY@ff=\relax}
\def\PY@tok#1{\csname PY@tok@#1\endcsname}
\def\PY@toks#1+{\ifx\relax#1\empty\else%
    \PY@tok{#1}\expandafter\PY@toks\fi}
\def\PY@do#1{\PY@bc{\PY@tc{\PY@ul{%
    \PY@it{\PY@bf{\PY@ff{#1}}}}}}}
\def\PY#1#2{\PY@reset\PY@toks#1+\relax+\PY@do{#2}}

\expandafter\def\csname PY@tok@w\endcsname{\def\PY@tc##1{\textcolor[rgb]{0.73,0.73,0.73}{##1}}}
\expandafter\def\csname PY@tok@c\endcsname{\let\PY@it=\textit\def\PY@tc##1{\textcolor[rgb]{0.25,0.50,0.50}{##1}}}
\expandafter\def\csname PY@tok@cp\endcsname{\def\PY@tc##1{\textcolor[rgb]{0.74,0.48,0.00}{##1}}}
\expandafter\def\csname PY@tok@k\endcsname{\let\PY@bf=\textbf\def\PY@tc##1{\textcolor[rgb]{0.00,0.50,0.00}{##1}}}
\expandafter\def\csname PY@tok@kp\endcsname{\def\PY@tc##1{\textcolor[rgb]{0.00,0.50,0.00}{##1}}}
\expandafter\def\csname PY@tok@kt\endcsname{\def\PY@tc##1{\textcolor[rgb]{0.69,0.00,0.25}{##1}}}
\expandafter\def\csname PY@tok@o\endcsname{\def\PY@tc##1{\textcolor[rgb]{0.40,0.40,0.40}{##1}}}
\expandafter\def\csname PY@tok@ow\endcsname{\let\PY@bf=\textbf\def\PY@tc##1{\textcolor[rgb]{0.67,0.13,1.00}{##1}}}
\expandafter\def\csname PY@tok@nb\endcsname{\def\PY@tc##1{\textcolor[rgb]{0.00,0.50,0.00}{##1}}}
\expandafter\def\csname PY@tok@nf\endcsname{\def\PY@tc##1{\textcolor[rgb]{0.00,0.00,1.00}{##1}}}
\expandafter\def\csname PY@tok@nc\endcsname{\let\PY@bf=\textbf\def\PY@tc##1{\textcolor[rgb]{0.00,0.00,1.00}{##1}}}
\expandafter\def\csname PY@tok@nn\endcsname{\let\PY@bf=\textbf\def\PY@tc##1{\textcolor[rgb]{0.00,0.00,1.00}{##1}}}
\expandafter\def\csname PY@tok@ne\endcsname{\let\PY@bf=\textbf\def\PY@tc##1{\textcolor[rgb]{0.82,0.25,0.23}{##1}}}
\expandafter\def\csname PY@tok@nv\endcsname{\def\PY@tc##1{\textcolor[rgb]{0.10,0.09,0.49}{##1}}}
\expandafter\def\csname PY@tok@no\endcsname{\def\PY@tc##1{\textcolor[rgb]{0.53,0.00,0.00}{##1}}}
\expandafter\def\csname PY@tok@nl\endcsname{\def\PY@tc##1{\textcolor[rgb]{0.63,0.63,0.00}{##1}}}
\expandafter\def\csname PY@tok@ni\endcsname{\let\PY@bf=\textbf\def\PY@tc##1{\textcolor[rgb]{0.60,0.60,0.60}{##1}}}
\expandafter\def\csname PY@tok@na\endcsname{\def\PY@tc##1{\textcolor[rgb]{0.49,0.56,0.16}{##1}}}
\expandafter\def\csname PY@tok@nt\endcsname{\let\PY@bf=\textbf\def\PY@tc##1{\textcolor[rgb]{0.00,0.50,0.00}{##1}}}
\expandafter\def\csname PY@tok@nd\endcsname{\def\PY@tc##1{\textcolor[rgb]{0.67,0.13,1.00}{##1}}}
\expandafter\def\csname PY@tok@s\endcsname{\def\PY@tc##1{\textcolor[rgb]{0.73,0.13,0.13}{##1}}}
\expandafter\def\csname PY@tok@sd\endcsname{\let\PY@it=\textit\def\PY@tc##1{\textcolor[rgb]{0.73,0.13,0.13}{##1}}}
\expandafter\def\csname PY@tok@si\endcsname{\let\PY@bf=\textbf\def\PY@tc##1{\textcolor[rgb]{0.73,0.40,0.53}{##1}}}
\expandafter\def\csname PY@tok@se\endcsname{\let\PY@bf=\textbf\def\PY@tc##1{\textcolor[rgb]{0.73,0.40,0.13}{##1}}}
\expandafter\def\csname PY@tok@sr\endcsname{\def\PY@tc##1{\textcolor[rgb]{0.73,0.40,0.53}{##1}}}
\expandafter\def\csname PY@tok@ss\endcsname{\def\PY@tc##1{\textcolor[rgb]{0.10,0.09,0.49}{##1}}}
\expandafter\def\csname PY@tok@sx\endcsname{\def\PY@tc##1{\textcolor[rgb]{0.00,0.50,0.00}{##1}}}
\expandafter\def\csname PY@tok@m\endcsname{\def\PY@tc##1{\textcolor[rgb]{0.40,0.40,0.40}{##1}}}
\expandafter\def\csname PY@tok@gh\endcsname{\let\PY@bf=\textbf\def\PY@tc##1{\textcolor[rgb]{0.00,0.00,0.50}{##1}}}
\expandafter\def\csname PY@tok@gu\endcsname{\let\PY@bf=\textbf\def\PY@tc##1{\textcolor[rgb]{0.50,0.00,0.50}{##1}}}
\expandafter\def\csname PY@tok@gd\endcsname{\def\PY@tc##1{\textcolor[rgb]{0.63,0.00,0.00}{##1}}}
\expandafter\def\csname PY@tok@gi\endcsname{\def\PY@tc##1{\textcolor[rgb]{0.00,0.63,0.00}{##1}}}
\expandafter\def\csname PY@tok@gr\endcsname{\def\PY@tc##1{\textcolor[rgb]{1.00,0.00,0.00}{##1}}}
\expandafter\def\csname PY@tok@ge\endcsname{\let\PY@it=\textit}
\expandafter\def\csname PY@tok@gs\endcsname{\let\PY@bf=\textbf}
\expandafter\def\csname PY@tok@gp\endcsname{\let\PY@bf=\textbf\def\PY@tc##1{\textcolor[rgb]{0.00,0.00,0.50}{##1}}}
\expandafter\def\csname PY@tok@go\endcsname{\def\PY@tc##1{\textcolor[rgb]{0.53,0.53,0.53}{##1}}}
\expandafter\def\csname PY@tok@gt\endcsname{\def\PY@tc##1{\textcolor[rgb]{0.00,0.27,0.87}{##1}}}
\expandafter\def\csname PY@tok@err\endcsname{\def\PY@bc##1{\setlength{\fboxsep}{0pt}\fcolorbox[rgb]{1.00,0.00,0.00}{1,1,1}{\strut ##1}}}
\expandafter\def\csname PY@tok@kc\endcsname{\let\PY@bf=\textbf\def\PY@tc##1{\textcolor[rgb]{0.00,0.50,0.00}{##1}}}
\expandafter\def\csname PY@tok@kd\endcsname{\let\PY@bf=\textbf\def\PY@tc##1{\textcolor[rgb]{0.00,0.50,0.00}{##1}}}
\expandafter\def\csname PY@tok@kn\endcsname{\let\PY@bf=\textbf\def\PY@tc##1{\textcolor[rgb]{0.00,0.50,0.00}{##1}}}
\expandafter\def\csname PY@tok@kr\endcsname{\let\PY@bf=\textbf\def\PY@tc##1{\textcolor[rgb]{0.00,0.50,0.00}{##1}}}
\expandafter\def\csname PY@tok@bp\endcsname{\def\PY@tc##1{\textcolor[rgb]{0.00,0.50,0.00}{##1}}}
\expandafter\def\csname PY@tok@fm\endcsname{\def\PY@tc##1{\textcolor[rgb]{0.00,0.00,1.00}{##1}}}
\expandafter\def\csname PY@tok@vc\endcsname{\def\PY@tc##1{\textcolor[rgb]{0.10,0.09,0.49}{##1}}}
\expandafter\def\csname PY@tok@vg\endcsname{\def\PY@tc##1{\textcolor[rgb]{0.10,0.09,0.49}{##1}}}
\expandafter\def\csname PY@tok@vi\endcsname{\def\PY@tc##1{\textcolor[rgb]{0.10,0.09,0.49}{##1}}}
\expandafter\def\csname PY@tok@vm\endcsname{\def\PY@tc##1{\textcolor[rgb]{0.10,0.09,0.49}{##1}}}
\expandafter\def\csname PY@tok@sa\endcsname{\def\PY@tc##1{\textcolor[rgb]{0.73,0.13,0.13}{##1}}}
\expandafter\def\csname PY@tok@sb\endcsname{\def\PY@tc##1{\textcolor[rgb]{0.73,0.13,0.13}{##1}}}
\expandafter\def\csname PY@tok@sc\endcsname{\def\PY@tc##1{\textcolor[rgb]{0.73,0.13,0.13}{##1}}}
\expandafter\def\csname PY@tok@dl\endcsname{\def\PY@tc##1{\textcolor[rgb]{0.73,0.13,0.13}{##1}}}
\expandafter\def\csname PY@tok@s2\endcsname{\def\PY@tc##1{\textcolor[rgb]{0.73,0.13,0.13}{##1}}}
\expandafter\def\csname PY@tok@sh\endcsname{\def\PY@tc##1{\textcolor[rgb]{0.73,0.13,0.13}{##1}}}
\expandafter\def\csname PY@tok@s1\endcsname{\def\PY@tc##1{\textcolor[rgb]{0.73,0.13,0.13}{##1}}}
\expandafter\def\csname PY@tok@mb\endcsname{\def\PY@tc##1{\textcolor[rgb]{0.40,0.40,0.40}{##1}}}
\expandafter\def\csname PY@tok@mf\endcsname{\def\PY@tc##1{\textcolor[rgb]{0.40,0.40,0.40}{##1}}}
\expandafter\def\csname PY@tok@mh\endcsname{\def\PY@tc##1{\textcolor[rgb]{0.40,0.40,0.40}{##1}}}
\expandafter\def\csname PY@tok@mi\endcsname{\def\PY@tc##1{\textcolor[rgb]{0.40,0.40,0.40}{##1}}}
\expandafter\def\csname PY@tok@il\endcsname{\def\PY@tc##1{\textcolor[rgb]{0.40,0.40,0.40}{##1}}}
\expandafter\def\csname PY@tok@mo\endcsname{\def\PY@tc##1{\textcolor[rgb]{0.40,0.40,0.40}{##1}}}
\expandafter\def\csname PY@tok@ch\endcsname{\let\PY@it=\textit\def\PY@tc##1{\textcolor[rgb]{0.25,0.50,0.50}{##1}}}
\expandafter\def\csname PY@tok@cm\endcsname{\let\PY@it=\textit\def\PY@tc##1{\textcolor[rgb]{0.25,0.50,0.50}{##1}}}
\expandafter\def\csname PY@tok@cpf\endcsname{\let\PY@it=\textit\def\PY@tc##1{\textcolor[rgb]{0.25,0.50,0.50}{##1}}}
\expandafter\def\csname PY@tok@c1\endcsname{\let\PY@it=\textit\def\PY@tc##1{\textcolor[rgb]{0.25,0.50,0.50}{##1}}}
\expandafter\def\csname PY@tok@cs\endcsname{\let\PY@it=\textit\def\PY@tc##1{\textcolor[rgb]{0.25,0.50,0.50}{##1}}}

\def\PYZbs{\char`\\}
\def\PYZus{\char`\_}
\def\PYZob{\char`\{}
\def\PYZcb{\char`\}}
\def\PYZca{\char`\^}
\def\PYZam{\char`\&}
\def\PYZlt{\char`\<}
\def\PYZgt{\char`\>}
\def\PYZsh{\char`\#}
\def\PYZpc{\char`\%}
\def\PYZdl{\char`\$}
\def\PYZhy{\char`\-}
\def\PYZsq{\char`\'}
\def\PYZdq{\char`\"}
\def\PYZti{\char`\~}
% for compatibility with earlier versions
\def\PYZat{@}
\def\PYZlb{[}
\def\PYZrb{]}
\makeatother


    % Exact colors from NB
    \definecolor{incolor}{rgb}{0.0, 0.0, 0.5}
    \definecolor{outcolor}{rgb}{0.545, 0.0, 0.0}



    
    % Prevent overflowing lines due to hard-to-break entities
    \sloppy 
    % Setup hyperref package
    \hypersetup{
      breaklinks=true,  % so long urls are correctly broken across lines
      colorlinks=true,
      urlcolor=urlcolor,
      linkcolor=linkcolor,
      citecolor=citecolor,
      }
    % Slightly bigger margins than the latex defaults
    
    \geometry{verbose,tmargin=1in,bmargin=1in,lmargin=1in,rmargin=1in}
    
    

    \begin{document}
    
    
    \maketitle
    
    

    
    \section{\texorpdfstring{Using JOD Backup Get
\texttt{bget}}{Using JOD Backup Get bget}}\label{using-jod-backup-get-bget}

\begin{figure}
\centering
\includegraphics{inclusions/jodteenytinycube.png}
\caption{}
\end{figure}

    \subsubsection{Introduction}\label{introduction}

In addition to restoring entire dictionary backups with \texttt{restd}
JOD also supports fetching individual objects from particular backups
with \texttt{bget} and \texttt{bnl}.

If you screw up part of a larger system restoring \emph{all the code}
may create more problems than it solves. Usually you only want
\href{https://www.youtube.com/watch?v=wPiHQ37gXnE}{\emph{the good bits}}
of a backup.

\textbf{\texttt{bget} is your good bits
\href{https://www.youtube.com/watch?v=R8OWNspU_yE}{huckleberry}.}

    \begin{Verbatim}[commandchars=\\\{\}]
{\color{incolor}In [{\color{incolor}1}]:} \PY{c+c1}{NB. display J version}
        \PY{l+m+mi}{9}\PY{o}{!}\PY{o}{:}\PY{l+m+mi}{14}\PY{l+s}{\PYZsq{}}\PY{l+s}{\PYZsq{}}
\end{Verbatim}


    \begin{Verbatim}[commandchars=\\\{\}]
j901/j64avx/windows/beta-s/commercial/www.jsoftware.com/2019-12-02T12:51:33

    \end{Verbatim}

    The following examples assume you have installed the JOD development
dictionaries (\texttt{joddev}), (\texttt{jod}), and (\texttt{utils}).

Use \href{https://code.jsoftware.com/wiki/JAL/User_Guide}{JAL} to
install the
\href{https://code.jsoftware.com/wiki/Addons/general/jodsource}{JODSOURCE
addon} and follow the instructions to load (\texttt{joddev}),
(\texttt{jod}), and (\texttt{utils}).

    \begin{Verbatim}[commandchars=\\\{\}]
{\color{incolor}In [{\color{incolor}2}]:} \PY{c+c1}{NB. load JOD in a clear base locale}
        \PY{n+nv}{load} \PY{l+s}{\PYZsq{}}\PY{l+s}{g}\PY{l+s}{e}\PY{l+s}{n}\PY{l+s}{e}\PY{l+s}{r}\PY{l+s}{a}\PY{l+s}{l}\PY{l+s}{/}\PY{l+s}{j}\PY{l+s}{o}\PY{l+s}{d}\PY{l+s}{\PYZsq{}} \PY{o}{[} \PY{n+nv}{clear} \PY{l+s}{\PYZsq{}}\PY{l+s}{\PYZsq{}}
        
        \PY{c+c1}{NB. The distributed JOD profile automatically RESETME\PYZsq{}s.}
        \PY{c+c1}{NB. To safely use dictionaries with many J tasks they must}
        \PY{c+c1}{NB. be READONLY. To prevent opening the same put dictionary}
        \PY{c+c1}{NB. READWRITE comment out (dpset) and restart this notebook.}
        \PY{n+nv}{dpset} \PY{l+s}{\PYZsq{}}\PY{l+s}{R}\PY{l+s}{E}\PY{l+s}{S}\PY{l+s}{E}\PY{l+s}{T}\PY{l+s}{M}\PY{l+s}{E}\PY{l+s}{\PYZsq{}}
        
        \PY{c+c1}{NB. Converting Jupyter notebooks to LaTeX is }
        \PY{c+c1}{NB. simplified by ASCII box characters.}
        \PY{n+nv}{portchars} \PY{l+s}{\PYZsq{}}\PY{l+s}{\PYZsq{}}
        
        \PY{c+c1}{NB. Verb to show large boxed displays in}
        \PY{c+c1}{NB. the notebook without ugly wrapping.}
        \PY{n+nv}{sbx}\PY{o}{=:} \PY{l+s}{\PYZsq{}}\PY{l+s}{ }\PY{l+s}{.}\PY{l+s}{.}\PY{l+s}{.}\PY{l+s}{ }\PY{l+s}{\PYZsq{}} \PY{o}{,}\PY{o}{\PYZdq{}}\PY{l+m+mi}{1}\PY{o}{\PYZti{}} \PY{l+m+mi}{90}\PY{o}{\PYZam{}}\PY{o}{\PYZob{}}\PY{o}{.}\PY{o}{\PYZdq{}}\PY{l+m+mi}{1}\PY{o}{@}\PY{o}{\PYZdq{}}\PY{o}{:}
        
        \PY{c+c1}{NB. open some JOD dictionaries to search}
        \PY{n+nv}{od} \PY{o}{;}\PY{o}{:}\PY{l+s}{\PYZsq{}}\PY{l+s}{j}\PY{l+s}{o}\PY{l+s}{d}\PY{l+s}{d}\PY{l+s}{e}\PY{l+s}{v}\PY{l+s}{ }\PY{l+s}{j}\PY{l+s}{o}\PY{l+s}{d}\PY{l+s}{ }\PY{l+s}{u}\PY{l+s}{t}\PY{l+s}{i}\PY{l+s}{l}\PY{l+s}{s}\PY{l+s}{\PYZsq{}} \PY{o}{[} \PY{l+m+mi}{3} \PY{n+nv}{od} \PY{l+s}{\PYZsq{}}\PY{l+s}{\PYZsq{}}
        \PY{n+nv}{did} \PY{o}{\PYZti{}} \PY{l+m+mi}{0}
\end{Verbatim}


    \begin{Verbatim}[commandchars=\\\{\}]
+-+----------------------------------------------------------------+
|1|+------+--+-----+-----+-------+-------+------+-----------------+|
| ||      |--|Words|Tests|Groups*|Suites*|Macros|Path*            ||
| |+------+--+-----+-----+-------+-------+------+-----------------+|
| ||joddev|rw|32   |8    |2      |1      |12    |/joddev/jod/utils||
| |+------+--+-----+-----+-------+-------+------+-----------------+|
| ||jod   |ro|782  |67   |22     |13     |61    |/jod/utils       ||
| |+------+--+-----+-----+-------+-------+------+-----------------+|
| ||utils |ro|392  |7    |21     |0      |17    |/utils           ||
| |+------+--+-----+-----+-------+-------+------+-----------------+|
+-+----------------------------------------------------------------+

    \end{Verbatim}

    \subsubsection{\texorpdfstring{\texttt{bnl} lists available
backups}{bnl lists available backups}}\label{bnl-lists-available-backups}

JOD binary backups are created with \texttt{packd}. When \texttt{packd}
is run it copies current dictionary files to the backup folder and
renames them with an ever increasing backup number prefix.

    \begin{Verbatim}[commandchars=\\\{\}]
{\color{incolor}In [{\color{incolor}3}]:} \PY{c+c1}{NB. list all available put dictionary backups}
        \PY{n+nv}{sbx} \PY{n+nv}{bnl} \PY{l+s}{\PYZsq{}}\PY{l+s}{.}\PY{l+s}{\PYZsq{}}
\end{Verbatim}


    \begin{Verbatim}[commandchars=\\\{\}]
+-+---+---+---+---+---+---+---+---+---+---+---+---+---+---+---+---+---+---+---+---+---+--- {\ldots} 
|1|.84|.83|.82|.81|.80|.79|.78|.77|.76|.75|.74|.73|.72|.71|.70|.69|.68|.67|.66|.65|.64|.63 {\ldots} 
+-+---+---+---+---+---+---+---+---+---+---+---+---+---+---+---+---+---+---+---+---+---+--- {\ldots} 

    \end{Verbatim}

    \subsubsection{\texorpdfstring{\texttt{bnl} lists objects in
backups}{bnl lists objects in backups}}\label{bnl-lists-objects-in-backups}

\texttt{bnl} lists objects in backups. See jod.pdf for more details.

    \begin{Verbatim}[commandchars=\\\{\}]
{\color{incolor}In [{\color{incolor}4}]:} \PY{c+c1}{NB. list all words in last backup}
        \PY{n+nv}{smoutout} \PY{n+nv}{sbx} \PY{n+nv}{bnl} \PY{l+s}{\PYZsq{}}\PY{l+s}{\PYZsq{}}
        
        \PY{c+c1}{NB. list all test cases in last backup}
        \PY{n+nv}{smoutput} \PY{n+nv}{sbx} \PY{l+m+mi}{1} \PY{n+nv}{bnl} \PY{l+s}{\PYZsq{}}\PY{l+s}{\PYZsq{}}
        
        \PY{c+c1}{NB. oldest backup}
        \PY{n+nv}{smoutput} \PY{n+nv}{OldestBnum}\PY{o}{=:} \PY{o}{;} \PY{o}{\PYZob{}}\PY{o}{:} \PY{n+nv}{bnl} \PY{l+s}{\PYZsq{}}\PY{l+s}{.}\PY{l+s}{\PYZsq{}}
        
        \PY{c+c1}{NB. list all macros in oldest backup}
        \PY{n+nv}{smoutput} \PY{n+nv}{sbx} \PY{l+m+mi}{4} \PY{n+nv}{bnl} \PY{n+nv}{OldestBnum}
        
        \PY{c+c1}{NB. list all macros with names starting with \PYZdq{}JOD\PYZdq{} in the oldest backup}
        \PY{l+m+mi}{4} \PY{n+nv}{bnl} \PY{l+s}{\PYZsq{}}\PY{l+s}{J}\PY{l+s}{O}\PY{l+s}{D}\PY{l+s}{\PYZsq{}}\PY{o}{,}\PY{n+nv}{OldestBnum}
\end{Verbatim}


    \begin{Verbatim}[commandchars=\\\{\}]
+-+-----------+----------+----------+----------+-----------------+---------------------+-- {\ldots} 
|1|bgetSmoke00|bnlBasic01|bnlSmoke00|bnlSmoke01|getDictextSmoke00|loadtest100dictionary|pu {\ldots} 
+-+-----------+----------+----------+----------+-----------------+---------------------+-- {\ldots} 
.10
+-+---------------+--------------------+----------+-----------+-------------------+------- {\ldots} 
|1|JODBUILDHISTORY|JODTOOLSBUILDHISTORY|historyjod|manifestjod|manifestjoddocument|manifes {\ldots} 
+-+---------------+--------------------+----------+-----------+-------------------+------- {\ldots} 
+-+---------------+--------------------+
|1|JODBUILDHISTORY|JODTOOLSBUILDHISTORY|
+-+---------------+--------------------+

    \end{Verbatim}

    \subsubsection{\texorpdfstring{\texttt{abv} a backup name list helper
verb}{abv a backup name list helper verb}}\label{abv-a-backup-name-list-helper-verb}

To streamline the search for backup objects the following helper verb
uses \texttt{bnl} to return valid backup name lists for objects.

    \begin{Verbatim}[commandchars=\\\{\}]
{\color{incolor}In [{\color{incolor}5}]:} \PY{c+c1}{NB. define in JOD\PYZsq{}s interface locale}
        \PY{n+nv}{abv\PYZus{}ijod\PYZus{}}\PY{o}{=:}\PY{n+nf}{3 : 0}
        
        \PY{c+c1}{NB.*abv v\PYZhy{}\PYZhy{} all backup version names.}
        \PY{c+c1}{NB.}
        \PY{c+c1}{NB. Returns all valid backup names matching name prefix (y). }
        \PY{c+c1}{NB.}
        \PY{c+c1}{NB. monad:  blclBNames =. abv zl|clPfx}
        \PY{c+c1}{NB.}
        \PY{c+c1}{NB.   abv \PYZsq{}ch\PYZsq{}  NB. all words in all backups starting with \PYZsq{}ch\PYZsq{}}
        \PY{c+c1}{NB.   abv \PYZsq{}\PYZsq{}    NB. all words in all backups}
        \PY{c+c1}{NB.}
        \PY{c+c1}{NB. dyad:   blclBNames =. il abv zl|clPfx}
        \PY{c+c1}{NB.}
        \PY{c+c1}{NB.   2 abv \PYZsq{}jod\PYZsq{}  NB. all group names in all backups starting with \PYZsq{}jod\PYZsq{}}
        \PY{c+c1}{NB.   4 abv \PYZsq{}\PYZsq{}     NB. all macros in all backups}
        
        \PY{l+m+mi}{0} \PY{n+nv}{abv} \PY{n+nd}{y} \PY{c+c1}{NB. word default}
        \PY{o}{:}
        \PY{n+nl}{if.} \PY{n+nv}{badcl\PYZus{}ajod\PYZus{}} \PY{n+nd}{y} \PY{n+nl}{do.} \PY{n+nv}{jderr\PYZus{}ajod\PYZus{}} \PY{n+nv}{ERR002\PYZus{}ajod\PYZus{}} \PY{n+nl}{return.} \PY{n+nl}{end.}
        \PY{n+nl}{if.} \PY{o}{\PYZhy{}}\PY{o}{.}\PY{n+nv}{isempty\PYZus{}ajod\PYZus{}} \PY{n+nd}{y} \PY{n+nl}{do.} \PY{n+nl}{if.} \PY{n+nv}{badrc\PYZus{}ajod\PYZus{}} \PY{n+nv}{uv}\PY{o}{=.}  \PY{n+nv}{checknames\PYZus{}ajod\PYZus{}} \PY{n+nd}{y} \PY{n+nl}{do.} \PY{n+nv}{uv} \PY{n+nl}{return.} \PY{n+nl}{else.} \PY{n+nd}{y}\PY{o}{=.} \PY{n+nv}{rv\PYZus{}ajod\PYZus{}} \PY{n+nv}{uv} \PY{n+nl}{end.} \PY{n+nl}{end.} 
        \PY{n+nl}{if.} \PY{n+nv}{badrc\PYZus{}ajod\PYZus{}} \PY{n+nv}{uv}\PY{o}{=.} \PY{n+nd}{x} \PY{n+nv}{bnl} \PY{l+s}{\PYZsq{}}\PY{l+s}{.}\PY{l+s}{\PYZsq{}} \PY{n+nl}{do.} \PY{n+nv}{uv} \PY{n+nl}{return.} \PY{n+nl}{else.} \PY{n+nv}{bn}\PY{o}{=.} \PY{o}{\PYZcb{}}\PY{o}{.}\PY{n+nv}{uv} \PY{n+nl}{end.}
        
        \PY{c+c1}{NB. names matching prefix in all backups}
        \PY{n+nv}{pfx}\PY{o}{=.} \PY{p}{(}\PY{o}{\PYZlt{}}\PY{n+nv}{a}\PY{o}{:}\PY{p}{)} \PY{o}{\PYZhy{}}\PY{o}{.}\PY{o}{\PYZam{}}\PY{o}{.}\PY{o}{\PYZgt{}}\PY{o}{\PYZti{}} \PY{o}{\PYZcb{}}\PY{o}{.}\PY{o}{@}\PY{p}{(}\PY{n+nd}{x}\PY{o}{\PYZam{}}\PY{n+nv}{bnl}\PY{p}{)}\PY{o}{\PYZam{}}\PY{o}{.}\PY{o}{\PYZgt{}} \PY{p}{(}\PY{o}{\PYZlt{}}\PY{n+nd}{y}\PY{p}{)} \PY{o}{,}\PY{o}{\PYZam{}}\PY{o}{.}\PY{o}{\PYZgt{}} \PY{n+nv}{bn}
        \PY{n+nv}{b}\PY{o}{=.} \PY{l+m+mi}{0} \PY{o}{\PYZlt{}} \PY{o}{\PYZsh{}}\PY{o}{\PYZam{}}\PY{o}{\PYZgt{}} \PY{n+nv}{pfx} 
        
        \PY{c+c1}{NB. return backup names from most recent to older backups}
        \PY{o}{\PYZbs{}}\PY{o}{:}\PY{o}{\PYZti{}} \PY{o}{;}\PY{o}{\PYZlt{}}\PY{o}{\PYZdq{}}\PY{l+m+mi}{1}\PY{o}{@}\PY{o}{;}\PY{o}{\PYZdq{}}\PY{l+m+mi}{1}\PY{o}{\PYZam{}}\PY{o}{.}\PY{o}{\PYZgt{}} \PY{p}{(}\PY{n+nv}{b} \PY{o}{\PYZsh{}} \PY{n+nv}{pfx}\PY{p}{)} \PY{o}{,}\PY{o}{\PYZdq{}}\PY{l+m+mi}{0}\PY{o}{\PYZam{}}\PY{o}{.}\PY{o}{\PYZgt{}} \PY{o}{\PYZlt{}}\PY{o}{\PYZdq{}}\PY{l+m+mi}{0} \PY{n+nv}{b} \PY{o}{\PYZsh{}} \PY{n+nv}{bn}
        \PY{n+nl}{)}
\end{Verbatim}


    \texttt{abv} returns backup names from most recent to oldest. The order
derives from backup numbers.

    \begin{Verbatim}[commandchars=\\\{\}]
{\color{incolor}In [{\color{incolor}6}]:} \PY{c+c1}{NB. all words in all backups beginning with \PYZsq{}b\PYZsq{}}
        \PY{n+nv}{sbx} \PY{n+nv}{abv} \PY{l+s}{\PYZsq{}}\PY{l+s}{b}\PY{l+s}{u}\PY{l+s}{\PYZsq{}}
\end{Verbatim}


    \begin{Verbatim}[commandchars=\\\{\}]
+--------------------------+--------------------------+--------------------------+-------- {\ldots} 
|buildjodtoolscompressed.67|buildjodtoolscompressed.66|buildjodtoolscompressed.65|buildjod {\ldots} 
+--------------------------+--------------------------+--------------------------+-------- {\ldots} 

    \end{Verbatim}

    \begin{Verbatim}[commandchars=\\\{\}]
{\color{incolor}In [{\color{incolor}7}]:} \PY{c+c1}{NB. all groups in all backups starting with \PYZdq{}build\PYZdq{} \PYZhy{} note the backup suffix numbers}
        \PY{l+m+mi}{80} \PY{n+nv}{list} \PY{o}{\PYZcb{}}\PY{o}{.} \PY{l+m+mi}{2} \PY{n+nv}{abv} \PY{l+s}{\PYZsq{}}\PY{l+s}{b}\PY{l+s}{u}\PY{l+s}{i}\PY{l+s}{l}\PY{l+s}{d}\PY{l+s}{\PYZsq{}}
\end{Verbatim}


    \begin{Verbatim}[commandchars=\\\{\}]
buildjod.66 buildjod.65 buildjod.64 buildjod.63 buildjod.62 buildjod.61 
buildjod.60 buildjod.59 buildjod.58 buildjod.57 buildjod.56 buildjod.55 
buildjod.54 buildjod.53 buildjod.52 buildjod.51 buildjod.50 buildjod.49 
buildjod.48 buildjod.47 buildjod.46 buildjod.45 buildjod.44 buildjod.43 
buildjod.42 buildjod.41 buildjod.40 buildjod.39 buildjod.38 buildjod.37 
buildjod.36 buildjod.35 buildjod.34                                     

    \end{Verbatim}

    \subsubsection{\texorpdfstring{\texttt{bget} fetches objects from
backups}{bget fetches objects from backups}}\label{bget-fetches-objects-from-backups}

\texttt{bget} fetches objects from backups. See jod.pdf for more
details.

Unlike \texttt{get} the verb \texttt{bget} does not define fetched
objects in locales. The reason for this is simple, \texttt{bget} may
return many versions of the same object. Which one should you restore?
JOD lets the user decide.

    \begin{Verbatim}[commandchars=\\\{\}]
{\color{incolor}In [{\color{incolor}8}]:} \PY{c+c1}{NB. select a random word from the last backup}
        \PY{n+nv}{AllWords}\PY{o}{=:} \PY{o}{\PYZcb{}}\PY{o}{.} \PY{n+nv}{bnl}\PY{l+s}{\PYZsq{}}\PY{l+s}{\PYZsq{}}
        \PY{n+nv}{smoutput} \PY{n+nv}{RandWord}\PY{o}{=:} \PY{o}{;}\PY{p}{(}\PY{n}{?}\PY{o}{\PYZsh{}}\PY{n+nv}{AllWords}\PY{p}{)} \PY{o}{\PYZob{}} \PY{n+nv}{AllWords}
        
        \PY{c+c1}{NB. fetch the word}
        \PY{n+nv}{sbx} \PY{n+nv}{bget} \PY{n+nv}{RandWord}
\end{Verbatim}


    \begin{Verbatim}[commandchars=\\\{\}]
bgetobjects
+-+--------------------------------------------------------------------------------------- {\ldots} 
|1|+--------------+-+--------------------------------------------------------------------- {\ldots} 
| ||bgetobjects\_84|3|4 : 0  NB.*bgetobjects v-- get objects from backups. NB. NB. dyad: il {\ldots} 
| |+--------------+-+--------------------------------------------------------------------- {\ldots} 
+-+--------------------------------------------------------------------------------------- {\ldots} 

    \end{Verbatim}

    \subsubsection{\texorpdfstring{\texttt{bget} can fetch many
objects}{bget can fetch many objects}}\label{bget-can-fetch-many-objects}

\texttt{bget} can fetch many objects. The objects are returned in boxed
name, class and value tables. For words and macros these tables have
three columns for other objects there are two columns.

    \begin{Verbatim}[commandchars=\\\{\}]
{\color{incolor}In [{\color{incolor}9}]:} \PY{c+c1}{NB. fetch the last five versions of words starting with \PYZdq{}release\PYZdq{}}
        \PY{c+c1}{NB. NOTE: the backup number suffixes of the names in the first column of (ncv)}
        \PY{l+s}{\PYZsq{}}\PY{l+s}{r}\PY{l+s}{c}\PY{l+s}{ }\PY{l+s}{n}\PY{l+s}{c}\PY{l+s}{v}\PY{l+s}{\PYZsq{}}\PY{o}{=:} \PY{n+nv}{bget} \PY{l+m+mi}{5} \PY{o}{\PYZob{}}\PY{o}{.} \PY{n+nv}{abv} \PY{l+s}{\PYZsq{}}\PY{l+s}{r}\PY{l+s}{e}\PY{l+s}{l}\PY{l+s}{e}\PY{l+s}{a}\PY{l+s}{s}\PY{l+s}{e}\PY{l+s}{\PYZsq{}}
        \PY{n+nv}{sbx} \PY{n+nv}{ncv}
\end{Verbatim}


    \begin{Verbatim}[commandchars=\\\{\}]
+-------------+-+------------------------------------------------------------------------- {\ldots} 
|releasejod\_84|3|4 : 0  NB.*releasejod v-- final JOD release steps. NB. NB. dyad:  blVersi {\ldots} 
+-------------+-+------------------------------------------------------------------------- {\ldots} 
|releasejod\_83|3|4 : 0  NB.*releasejod v-- final JOD release steps. NB. NB. dyad:  blVersi {\ldots} 
+-------------+-+------------------------------------------------------------------------- {\ldots} 
|releasejod\_82|3|4 : 0  NB.*releasejod v-- final JOD release steps. NB. NB. dyad:  blVersi {\ldots} 
+-------------+-+------------------------------------------------------------------------- {\ldots} 
|releasejod\_81|3|4 : 0  NB.*releasejod v-- final JOD release steps. NB. NB. dyad:  blVersi {\ldots} 
+-------------+-+------------------------------------------------------------------------- {\ldots} 
|releasejod\_80|3|4 : 0  NB.*releasejod v-- final JOD release steps. NB. NB. dyad:  blVersi {\ldots} 
+-------------+-+------------------------------------------------------------------------- {\ldots} 

    \end{Verbatim}

    \begin{Verbatim}[commandchars=\\\{\}]
{\color{incolor}In [{\color{incolor}10}]:} \PY{c+c1}{NB. fetch the last ten versions of the short explanations of all words starting with \PYZdq{}bc\PYZdq{}}
         \PY{l+s}{\PYZsq{}}\PY{l+s}{r}\PY{l+s}{c}\PY{l+s}{ }\PY{l+s}{n}\PY{l+s}{c}\PY{l+s}{v}\PY{l+s}{\PYZsq{}}\PY{o}{=:} \PY{l+m+mi}{0} \PY{l+m+mi}{8} \PY{n+nv}{bget} \PY{l+m+mi}{10} \PY{o}{\PYZob{}}\PY{o}{.} \PY{n+nv}{abv} \PY{l+s}{\PYZsq{}}\PY{l+s}{b}\PY{l+s}{c}\PY{l+s}{\PYZsq{}}
         \PY{n+nv}{sbx} \PY{n+nv}{ncv}
\end{Verbatim}


    \begin{Verbatim}[commandchars=\\\{\}]
+--------------+---------------------------+                                               {\ldots} 
|bchecknames\_84|checks backup name patterns|                                               {\ldots} 
+--------------+---------------------------+                                               {\ldots} 
|bchecknames\_83|checks backup name patterns|                                               {\ldots} 
+--------------+---------------------------+                                               {\ldots} 
|bchecknames\_82|checks backup name patterns|                                               {\ldots} 
+--------------+---------------------------+                                               {\ldots} 
|bchecknames\_81|checks backup name patterns|                                               {\ldots} 
+--------------+---------------------------+                                               {\ldots} 
|bchecknames\_80|checks backup name patterns|                                               {\ldots} 
+--------------+---------------------------+                                               {\ldots} 
|bchecknames\_79|checks backup name patterns|                                               {\ldots} 
+--------------+---------------------------+                                               {\ldots} 
|bchecknames\_78|checks backup name patterns|                                               {\ldots} 
+--------------+---------------------------+                                               {\ldots} 
|bchecknames\_77|checks backup name patterns|                                               {\ldots} 
+--------------+---------------------------+                                               {\ldots} 
|bchecknames\_76|checks backup name patterns|                                               {\ldots} 
+--------------+---------------------------+                                               {\ldots} 
|bchecknames\_75|checks backup name patterns|                                               {\ldots} 
+--------------+---------------------------+                                               {\ldots} 

    \end{Verbatim}

    \subsubsection{\texorpdfstring{\texttt{bget} results can be edited with
\texttt{ed}}{bget results can be edited with ed}}\label{bget-results-can-be-edited-with-ed}

\texttt{bget} name, class and value tables can be edited with the JOD
\texttt{ed} verb. JOD \texttt{ed} works differently depending on what J
IDE you are using but the basic approach is to generate script text from
object data, write the script to a file, and then open a J editor. The
following examples illustrate typical uses.

    \begin{Verbatim}[commandchars=\\\{\}]
{\color{incolor}In [{\color{incolor}11}]:} \PY{c+c1}{NB. edit the most recent version of a word}
         \PY{l+s}{\PYZsq{}}\PY{l+s}{r}\PY{l+s}{c}\PY{l+s}{ }\PY{l+s}{n}\PY{l+s}{c}\PY{l+s}{v}\PY{l+s}{\PYZsq{}}\PY{o}{=:} \PY{n+nv}{bget} \PY{l+s}{\PYZsq{}}\PY{l+s}{b}\PY{l+s}{c}\PY{l+s}{h}\PY{l+s}{e}\PY{l+s}{c}\PY{l+s}{k}\PY{l+s}{n}\PY{l+s}{a}\PY{l+s}{m}\PY{l+s}{e}\PY{l+s}{s}\PY{l+s}{\PYZsq{}}
         \PY{n+nv}{ed} \PY{n+nv}{ncv}
\end{Verbatim}


    \begin{Verbatim}[commandchars=\\\{\}]
\$\$\$edit\$\$\$c:/users/john/j901-user/temp/bchecknames\_84.ijs

    \end{Verbatim}

    \begin{Verbatim}[commandchars=\\\{\}]
{\color{incolor}In [{\color{incolor}12}]:} \PY{c+c1}{NB. edit all group header text}
         \PY{l+s}{\PYZsq{}}\PY{l+s}{r}\PY{l+s}{c}\PY{l+s}{ }\PY{l+s}{n}\PY{l+s}{c}\PY{l+s}{v}\PY{l+s}{\PYZsq{}}\PY{o}{=:} \PY{l+m+mi}{2} \PY{n+nv}{bget} \PY{o}{\PYZcb{}}\PY{o}{.} \PY{l+m+mi}{2} \PY{n+nv}{bnl} \PY{l+s}{\PYZsq{}}\PY{l+s}{\PYZsq{}}
         \PY{n+nv}{smoutput} \PY{n+nv}{sbx} \PY{n+nv}{ncv}
         \PY{n+nv}{ed} \PY{n+nv}{ncv}
\end{Verbatim}


    \begin{Verbatim}[commandchars=\\\{\}]
+-----------+----------------------------------------------------------------------------- {\ldots} 
|jod\_84     |NB. *jod c-- main JOD dictionary class.  NB.  NB. All other dictionary classe {\ldots} 
+-----------+----------------------------------------------------------------------------- {\ldots} 
|jodstore\_84|NB.*jodstore c-- storage object class: extension of (jod).  NB.  NB. Hides th {\ldots} 
+-----------+----------------------------------------------------------------------------- {\ldots} 
\$\$\$edit\$\$\$c:/users/john/j901-user/temp/jod\_84.ijs

    \end{Verbatim}

    \begin{Verbatim}[commandchars=\\\{\}]
{\color{incolor}In [{\color{incolor}13}]:} \PY{c+c1}{NB. edit all versions of all macros that begin with \PYZdq{}manifestjoddoc\PYZdq{}}
         \PY{l+s}{\PYZsq{}}\PY{l+s}{r}\PY{l+s}{c}\PY{l+s}{ }\PY{l+s}{n}\PY{l+s}{c}\PY{l+s}{v}\PY{l+s}{\PYZsq{}}\PY{o}{=:} \PY{l+m+mi}{4} \PY{n+nv}{bget} \PY{l+m+mi}{4} \PY{n+nv}{abv} \PY{l+s}{\PYZsq{}}\PY{l+s}{m}\PY{l+s}{a}\PY{l+s}{n}\PY{l+s}{i}\PY{l+s}{f}\PY{l+s}{e}\PY{l+s}{s}\PY{l+s}{t}\PY{l+s}{j}\PY{l+s}{o}\PY{l+s}{d}\PY{l+s}{d}\PY{l+s}{o}\PY{l+s}{c}\PY{l+s}{\PYZsq{}}
         \PY{n+nv}{smoutput} \PY{l+m+mi}{80} \PY{n+nv}{list} \PY{l+m+mi}{0} \PY{o}{\PYZob{}}\PY{o}{\PYZdq{}}\PY{l+m+mi}{1} \PY{n+nv}{ncv}
         \PY{n+nv}{ed} \PY{n+nv}{ncv}
\end{Verbatim}


    \begin{Verbatim}[commandchars=\\\{\}]
manifestjoddocument\_84 manifestjoddocument\_83 manifestjoddocument\_82 
manifestjoddocument\_81 manifestjoddocument\_80 manifestjoddocument\_79 
manifestjoddocument\_78 manifestjoddocument\_77 manifestjoddocument\_76 
manifestjoddocument\_75 manifestjoddocument\_74 manifestjoddocument\_73 
manifestjoddocument\_72 manifestjoddocument\_71 manifestjoddocument\_70 
manifestjoddocument\_69 manifestjoddocument\_68 manifestjoddocument\_67 
manifestjoddocument\_66 manifestjoddocument\_65 manifestjoddocument\_64 
manifestjoddocument\_63 manifestjoddocument\_62 manifestjoddocument\_61 
manifestjoddocument\_60 manifestjoddocument\_59 manifestjoddocument\_58 
manifestjoddocument\_57 manifestjoddocument\_56 manifestjoddocument\_55 
manifestjoddocument\_54 manifestjoddocument\_53 manifestjoddocument\_52 
manifestjoddocument\_51 manifestjoddocument\_50 manifestjoddocument\_49 
manifestjoddocument\_48 manifestjoddocument\_47 manifestjoddocument\_46 
manifestjoddocument\_45 manifestjoddocument\_44 manifestjoddocument\_43 
manifestjoddocument\_42 manifestjoddocument\_41 manifestjoddocument\_40 
manifestjoddocument\_39 manifestjoddocument\_38 manifestjoddocument\_37 
manifestjoddocument\_36 manifestjoddocument\_35 manifestjoddocument\_34 
manifestjoddocument\_33 manifestjoddocument\_32 manifestjoddocument\_31 
manifestjoddocument\_30 manifestjoddocument\_29 manifestjoddocument\_28 
manifestjoddocument\_27 manifestjoddocument\_26 manifestjoddocument\_25 
manifestjoddocument\_24 manifestjoddocument\_23 manifestjoddocument\_22 
manifestjoddocument\_21 manifestjoddocument\_20 manifestjoddocument\_19 
manifestjoddocument\_18 manifestjoddocument\_17 manifestjoddocument\_16 
manifestjoddocument\_15 manifestjoddocument\_14 manifestjoddocument\_13 
manifestjoddocument\_12 manifestjoddocument\_11 manifestjoddocument\_10 
\$\$\$edit\$\$\$c:/users/john/j901-user/temp/manifestjoddocument\_84.ijs

    \end{Verbatim}

    \subsubsection{Final remarks}\label{final-remarks}

\texttt{bget} and \texttt{bnl} make it easy to recover objects in JOD
binary backups. The objects are not defined in locales. You must edit
them with \texttt{ed} and select what you need. Object names are
suffixed with backup numbers. This is done to distinquish object
versions. Typically these backup numbers will be removed when editing
recovered objects.

\begin{verbatim}
John Baker 
December 2019
bakerjd99@gmail.com
\end{verbatim}


    % Add a bibliography block to the postdoc
    
    
    
    \end{document}
