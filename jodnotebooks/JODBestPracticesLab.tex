\documentclass[11pt,letter,landscape]{article} 
%\documentclass[11pt]{article}

    \usepackage[breakable]{tcolorbox}
    \usepackage{parskip} % Stop auto-indenting (to mimic markdown behaviour)
    
    \usepackage{iftex}
    \ifPDFTeX
    	\usepackage[T1]{fontenc}
    	\usepackage{mathpazo}
    \else
    	\usepackage{fontspec}
    \fi

    % Basic figure setup, for now with no caption control since it's done
    % automatically by Pandoc (which extracts ![](path) syntax from Markdown).
    \usepackage{graphicx}
    % Maintain compatibility with old templates. Remove in nbconvert 6.0
    \let\Oldincludegraphics\includegraphics
    % Ensure that by default, figures have no caption (until we provide a
    % proper Figure object with a Caption API and a way to capture that
    % in the conversion process - todo).
    \usepackage{caption}
    \DeclareCaptionFormat{nocaption}{}
    \captionsetup{format=nocaption,aboveskip=0pt,belowskip=0pt}

    \usepackage[Export]{adjustbox} % Used to constrain images to a maximum size
    \adjustboxset{max size={0.9\linewidth}{0.9\paperheight}}
    \usepackage{float}
    \floatplacement{figure}{H} % forces figures to be placed at the correct location
    \usepackage{xcolor} % Allow colors to be defined
    \usepackage{enumerate} % Needed for markdown enumerations to work
    \usepackage{geometry} % Used to adjust the document margins
    \usepackage{amsmath} % Equations
    \usepackage{amssymb} % Equations
    \usepackage{textcomp} % defines textquotesingle
    % Hack from http://tex.stackexchange.com/a/47451/13684:
    \AtBeginDocument{%
        \def\PYZsq{\textquotesingle}% Upright quotes in Pygmentized code
    }
    \usepackage{upquote} % Upright quotes for verbatim code
    \usepackage{eurosym} % defines \euro
    \usepackage[mathletters]{ucs} % Extended unicode (utf-8) support
    \usepackage{fancyvrb} % verbatim replacement that allows latex
    \usepackage{grffile} % extends the file name processing of package graphics 
                         % to support a larger range
    \makeatletter % fix for grffile with XeLaTeX
    \def\Gread@@xetex#1{%
      \IfFileExists{"\Gin@base".bb}%
      {\Gread@eps{\Gin@base.bb}}%
      {\Gread@@xetex@aux#1}%
    }
    \makeatother

    % The hyperref package gives us a pdf with properly built
    % internal navigation ('pdf bookmarks' for the table of contents,
    % internal cross-reference links, web links for URLs, etc.)
    \usepackage{hyperref}
    % The default LaTeX title has an obnoxious amount of whitespace. By default,
    % titling removes some of it. It also provides customization options.
    \usepackage{titling}
    \usepackage{longtable} % longtable support required by pandoc >1.10
    \usepackage{booktabs}  % table support for pandoc > 1.12.2
    \usepackage[inline]{enumitem} % IRkernel/repr support (it uses the enumerate* environment)
    \usepackage[normalem]{ulem} % ulem is needed to support strikethroughs (\sout)
                                % normalem makes italics be italics, not underlines
    \usepackage{mathrsfs}
    

    
    % Colors for the hyperref package
    \definecolor{urlcolor}{rgb}{0,.145,.698}
    \definecolor{linkcolor}{rgb}{.71,0.21,0.01}
    \definecolor{citecolor}{rgb}{.12,.54,.11}

    % ANSI colors
    \definecolor{ansi-black}{HTML}{3E424D}
    \definecolor{ansi-black-intense}{HTML}{282C36}
    \definecolor{ansi-red}{HTML}{E75C58}
    \definecolor{ansi-red-intense}{HTML}{B22B31}
    \definecolor{ansi-green}{HTML}{00A250}
    \definecolor{ansi-green-intense}{HTML}{007427}
    \definecolor{ansi-yellow}{HTML}{DDB62B}
    \definecolor{ansi-yellow-intense}{HTML}{B27D12}
    \definecolor{ansi-blue}{HTML}{208FFB}
    \definecolor{ansi-blue-intense}{HTML}{0065CA}
    \definecolor{ansi-magenta}{HTML}{D160C4}
    \definecolor{ansi-magenta-intense}{HTML}{A03196}
    \definecolor{ansi-cyan}{HTML}{60C6C8}
    \definecolor{ansi-cyan-intense}{HTML}{258F8F}
    \definecolor{ansi-white}{HTML}{C5C1B4}
    \definecolor{ansi-white-intense}{HTML}{A1A6B2}
    \definecolor{ansi-default-inverse-fg}{HTML}{FFFFFF}
    \definecolor{ansi-default-inverse-bg}{HTML}{000000}

    % commands and environments needed by pandoc snippets
    % extracted from the output of `pandoc -s`
    \providecommand{\tightlist}{%
      \setlength{\itemsep}{0pt}\setlength{\parskip}{0pt}}
    \DefineVerbatimEnvironment{Highlighting}{Verbatim}{commandchars=\\\{\}}
    % Add ',fontsize=\small' for more characters per line
    \newenvironment{Shaded}{}{}
    \newcommand{\KeywordTok}[1]{\textcolor[rgb]{0.00,0.44,0.13}{\textbf{{#1}}}}
    \newcommand{\DataTypeTok}[1]{\textcolor[rgb]{0.56,0.13,0.00}{{#1}}}
    \newcommand{\DecValTok}[1]{\textcolor[rgb]{0.25,0.63,0.44}{{#1}}}
    \newcommand{\BaseNTok}[1]{\textcolor[rgb]{0.25,0.63,0.44}{{#1}}}
    \newcommand{\FloatTok}[1]{\textcolor[rgb]{0.25,0.63,0.44}{{#1}}}
    \newcommand{\CharTok}[1]{\textcolor[rgb]{0.25,0.44,0.63}{{#1}}}
    \newcommand{\StringTok}[1]{\textcolor[rgb]{0.25,0.44,0.63}{{#1}}}
    \newcommand{\CommentTok}[1]{\textcolor[rgb]{0.38,0.63,0.69}{\textit{{#1}}}}
    \newcommand{\OtherTok}[1]{\textcolor[rgb]{0.00,0.44,0.13}{{#1}}}
    \newcommand{\AlertTok}[1]{\textcolor[rgb]{1.00,0.00,0.00}{\textbf{{#1}}}}
    \newcommand{\FunctionTok}[1]{\textcolor[rgb]{0.02,0.16,0.49}{{#1}}}
    \newcommand{\RegionMarkerTok}[1]{{#1}}
    \newcommand{\ErrorTok}[1]{\textcolor[rgb]{1.00,0.00,0.00}{\textbf{{#1}}}}
    \newcommand{\NormalTok}[1]{{#1}}
    
    % Additional commands for more recent versions of Pandoc
    \newcommand{\ConstantTok}[1]{\textcolor[rgb]{0.53,0.00,0.00}{{#1}}}
    \newcommand{\SpecialCharTok}[1]{\textcolor[rgb]{0.25,0.44,0.63}{{#1}}}
    \newcommand{\VerbatimStringTok}[1]{\textcolor[rgb]{0.25,0.44,0.63}{{#1}}}
    \newcommand{\SpecialStringTok}[1]{\textcolor[rgb]{0.73,0.40,0.53}{{#1}}}
    \newcommand{\ImportTok}[1]{{#1}}
    \newcommand{\DocumentationTok}[1]{\textcolor[rgb]{0.73,0.13,0.13}{\textit{{#1}}}}
    \newcommand{\AnnotationTok}[1]{\textcolor[rgb]{0.38,0.63,0.69}{\textbf{\textit{{#1}}}}}
    \newcommand{\CommentVarTok}[1]{\textcolor[rgb]{0.38,0.63,0.69}{\textbf{\textit{{#1}}}}}
    \newcommand{\VariableTok}[1]{\textcolor[rgb]{0.10,0.09,0.49}{{#1}}}
    \newcommand{\ControlFlowTok}[1]{\textcolor[rgb]{0.00,0.44,0.13}{\textbf{{#1}}}}
    \newcommand{\OperatorTok}[1]{\textcolor[rgb]{0.40,0.40,0.40}{{#1}}}
    \newcommand{\BuiltInTok}[1]{{#1}}
    \newcommand{\ExtensionTok}[1]{{#1}}
    \newcommand{\PreprocessorTok}[1]{\textcolor[rgb]{0.74,0.48,0.00}{{#1}}}
    \newcommand{\AttributeTok}[1]{\textcolor[rgb]{0.49,0.56,0.16}{{#1}}}
    \newcommand{\InformationTok}[1]{\textcolor[rgb]{0.38,0.63,0.69}{\textbf{\textit{{#1}}}}}
    \newcommand{\WarningTok}[1]{\textcolor[rgb]{0.38,0.63,0.69}{\textbf{\textit{{#1}}}}}
    
    
    % Define a nice break command that doesn't care if a line doesn't already
    % exist.
    \def\br{\hspace*{\fill} \\* }
    % Math Jax compatibility definitions
    \def\gt{>}
    \def\lt{<}
    \let\Oldtex\TeX
    \let\Oldlatex\LaTeX
    \renewcommand{\TeX}{\textrm{\Oldtex}}
    \renewcommand{\LaTeX}{\textrm{\Oldlatex}}
    % Document parameters
    % Document title
    \title{JOD Best Practices Lab}
    
    
    
    
    
% Pygments definitions
\makeatletter
\def\PY@reset{\let\PY@it=\relax \let\PY@bf=\relax%
    \let\PY@ul=\relax \let\PY@tc=\relax%
    \let\PY@bc=\relax \let\PY@ff=\relax}
\def\PY@tok#1{\csname PY@tok@#1\endcsname}
\def\PY@toks#1+{\ifx\relax#1\empty\else%
    \PY@tok{#1}\expandafter\PY@toks\fi}
\def\PY@do#1{\PY@bc{\PY@tc{\PY@ul{%
    \PY@it{\PY@bf{\PY@ff{#1}}}}}}}
\def\PY#1#2{\PY@reset\PY@toks#1+\relax+\PY@do{#2}}

\expandafter\def\csname PY@tok@w\endcsname{\def\PY@tc##1{\textcolor[rgb]{0.73,0.73,0.73}{##1}}}
\expandafter\def\csname PY@tok@c\endcsname{\let\PY@it=\textit\def\PY@tc##1{\textcolor[rgb]{0.25,0.50,0.50}{##1}}}
\expandafter\def\csname PY@tok@cp\endcsname{\def\PY@tc##1{\textcolor[rgb]{0.74,0.48,0.00}{##1}}}
\expandafter\def\csname PY@tok@k\endcsname{\let\PY@bf=\textbf\def\PY@tc##1{\textcolor[rgb]{0.00,0.50,0.00}{##1}}}
\expandafter\def\csname PY@tok@kp\endcsname{\def\PY@tc##1{\textcolor[rgb]{0.00,0.50,0.00}{##1}}}
\expandafter\def\csname PY@tok@kt\endcsname{\def\PY@tc##1{\textcolor[rgb]{0.69,0.00,0.25}{##1}}}
\expandafter\def\csname PY@tok@o\endcsname{\def\PY@tc##1{\textcolor[rgb]{0.40,0.40,0.40}{##1}}}
\expandafter\def\csname PY@tok@ow\endcsname{\let\PY@bf=\textbf\def\PY@tc##1{\textcolor[rgb]{0.67,0.13,1.00}{##1}}}
\expandafter\def\csname PY@tok@nb\endcsname{\def\PY@tc##1{\textcolor[rgb]{0.00,0.50,0.00}{##1}}}
\expandafter\def\csname PY@tok@nf\endcsname{\def\PY@tc##1{\textcolor[rgb]{0.00,0.00,1.00}{##1}}}
\expandafter\def\csname PY@tok@nc\endcsname{\let\PY@bf=\textbf\def\PY@tc##1{\textcolor[rgb]{0.00,0.00,1.00}{##1}}}
\expandafter\def\csname PY@tok@nn\endcsname{\let\PY@bf=\textbf\def\PY@tc##1{\textcolor[rgb]{0.00,0.00,1.00}{##1}}}
\expandafter\def\csname PY@tok@ne\endcsname{\let\PY@bf=\textbf\def\PY@tc##1{\textcolor[rgb]{0.82,0.25,0.23}{##1}}}
\expandafter\def\csname PY@tok@nv\endcsname{\def\PY@tc##1{\textcolor[rgb]{0.10,0.09,0.49}{##1}}}
\expandafter\def\csname PY@tok@no\endcsname{\def\PY@tc##1{\textcolor[rgb]{0.53,0.00,0.00}{##1}}}
\expandafter\def\csname PY@tok@nl\endcsname{\def\PY@tc##1{\textcolor[rgb]{0.63,0.63,0.00}{##1}}}
\expandafter\def\csname PY@tok@ni\endcsname{\let\PY@bf=\textbf\def\PY@tc##1{\textcolor[rgb]{0.60,0.60,0.60}{##1}}}
\expandafter\def\csname PY@tok@na\endcsname{\def\PY@tc##1{\textcolor[rgb]{0.49,0.56,0.16}{##1}}}
\expandafter\def\csname PY@tok@nt\endcsname{\let\PY@bf=\textbf\def\PY@tc##1{\textcolor[rgb]{0.00,0.50,0.00}{##1}}}
\expandafter\def\csname PY@tok@nd\endcsname{\def\PY@tc##1{\textcolor[rgb]{0.67,0.13,1.00}{##1}}}
\expandafter\def\csname PY@tok@s\endcsname{\def\PY@tc##1{\textcolor[rgb]{0.73,0.13,0.13}{##1}}}
\expandafter\def\csname PY@tok@sd\endcsname{\let\PY@it=\textit\def\PY@tc##1{\textcolor[rgb]{0.73,0.13,0.13}{##1}}}
\expandafter\def\csname PY@tok@si\endcsname{\let\PY@bf=\textbf\def\PY@tc##1{\textcolor[rgb]{0.73,0.40,0.53}{##1}}}
\expandafter\def\csname PY@tok@se\endcsname{\let\PY@bf=\textbf\def\PY@tc##1{\textcolor[rgb]{0.73,0.40,0.13}{##1}}}
\expandafter\def\csname PY@tok@sr\endcsname{\def\PY@tc##1{\textcolor[rgb]{0.73,0.40,0.53}{##1}}}
\expandafter\def\csname PY@tok@ss\endcsname{\def\PY@tc##1{\textcolor[rgb]{0.10,0.09,0.49}{##1}}}
\expandafter\def\csname PY@tok@sx\endcsname{\def\PY@tc##1{\textcolor[rgb]{0.00,0.50,0.00}{##1}}}
\expandafter\def\csname PY@tok@m\endcsname{\def\PY@tc##1{\textcolor[rgb]{0.40,0.40,0.40}{##1}}}
\expandafter\def\csname PY@tok@gh\endcsname{\let\PY@bf=\textbf\def\PY@tc##1{\textcolor[rgb]{0.00,0.00,0.50}{##1}}}
\expandafter\def\csname PY@tok@gu\endcsname{\let\PY@bf=\textbf\def\PY@tc##1{\textcolor[rgb]{0.50,0.00,0.50}{##1}}}
\expandafter\def\csname PY@tok@gd\endcsname{\def\PY@tc##1{\textcolor[rgb]{0.63,0.00,0.00}{##1}}}
\expandafter\def\csname PY@tok@gi\endcsname{\def\PY@tc##1{\textcolor[rgb]{0.00,0.63,0.00}{##1}}}
\expandafter\def\csname PY@tok@gr\endcsname{\def\PY@tc##1{\textcolor[rgb]{1.00,0.00,0.00}{##1}}}
\expandafter\def\csname PY@tok@ge\endcsname{\let\PY@it=\textit}
\expandafter\def\csname PY@tok@gs\endcsname{\let\PY@bf=\textbf}
\expandafter\def\csname PY@tok@gp\endcsname{\let\PY@bf=\textbf\def\PY@tc##1{\textcolor[rgb]{0.00,0.00,0.50}{##1}}}
\expandafter\def\csname PY@tok@go\endcsname{\def\PY@tc##1{\textcolor[rgb]{0.53,0.53,0.53}{##1}}}
\expandafter\def\csname PY@tok@gt\endcsname{\def\PY@tc##1{\textcolor[rgb]{0.00,0.27,0.87}{##1}}}
\expandafter\def\csname PY@tok@err\endcsname{\def\PY@bc##1{\setlength{\fboxsep}{0pt}\fcolorbox[rgb]{1.00,0.00,0.00}{1,1,1}{\strut ##1}}}
\expandafter\def\csname PY@tok@kc\endcsname{\let\PY@bf=\textbf\def\PY@tc##1{\textcolor[rgb]{0.00,0.50,0.00}{##1}}}
\expandafter\def\csname PY@tok@kd\endcsname{\let\PY@bf=\textbf\def\PY@tc##1{\textcolor[rgb]{0.00,0.50,0.00}{##1}}}
\expandafter\def\csname PY@tok@kn\endcsname{\let\PY@bf=\textbf\def\PY@tc##1{\textcolor[rgb]{0.00,0.50,0.00}{##1}}}
\expandafter\def\csname PY@tok@kr\endcsname{\let\PY@bf=\textbf\def\PY@tc##1{\textcolor[rgb]{0.00,0.50,0.00}{##1}}}
\expandafter\def\csname PY@tok@bp\endcsname{\def\PY@tc##1{\textcolor[rgb]{0.00,0.50,0.00}{##1}}}
\expandafter\def\csname PY@tok@fm\endcsname{\def\PY@tc##1{\textcolor[rgb]{0.00,0.00,1.00}{##1}}}
\expandafter\def\csname PY@tok@vc\endcsname{\def\PY@tc##1{\textcolor[rgb]{0.10,0.09,0.49}{##1}}}
\expandafter\def\csname PY@tok@vg\endcsname{\def\PY@tc##1{\textcolor[rgb]{0.10,0.09,0.49}{##1}}}
\expandafter\def\csname PY@tok@vi\endcsname{\def\PY@tc##1{\textcolor[rgb]{0.10,0.09,0.49}{##1}}}
\expandafter\def\csname PY@tok@vm\endcsname{\def\PY@tc##1{\textcolor[rgb]{0.10,0.09,0.49}{##1}}}
\expandafter\def\csname PY@tok@sa\endcsname{\def\PY@tc##1{\textcolor[rgb]{0.73,0.13,0.13}{##1}}}
\expandafter\def\csname PY@tok@sb\endcsname{\def\PY@tc##1{\textcolor[rgb]{0.73,0.13,0.13}{##1}}}
\expandafter\def\csname PY@tok@sc\endcsname{\def\PY@tc##1{\textcolor[rgb]{0.73,0.13,0.13}{##1}}}
\expandafter\def\csname PY@tok@dl\endcsname{\def\PY@tc##1{\textcolor[rgb]{0.73,0.13,0.13}{##1}}}
\expandafter\def\csname PY@tok@s2\endcsname{\def\PY@tc##1{\textcolor[rgb]{0.73,0.13,0.13}{##1}}}
\expandafter\def\csname PY@tok@sh\endcsname{\def\PY@tc##1{\textcolor[rgb]{0.73,0.13,0.13}{##1}}}
\expandafter\def\csname PY@tok@s1\endcsname{\def\PY@tc##1{\textcolor[rgb]{0.73,0.13,0.13}{##1}}}
\expandafter\def\csname PY@tok@mb\endcsname{\def\PY@tc##1{\textcolor[rgb]{0.40,0.40,0.40}{##1}}}
\expandafter\def\csname PY@tok@mf\endcsname{\def\PY@tc##1{\textcolor[rgb]{0.40,0.40,0.40}{##1}}}
\expandafter\def\csname PY@tok@mh\endcsname{\def\PY@tc##1{\textcolor[rgb]{0.40,0.40,0.40}{##1}}}
\expandafter\def\csname PY@tok@mi\endcsname{\def\PY@tc##1{\textcolor[rgb]{0.40,0.40,0.40}{##1}}}
\expandafter\def\csname PY@tok@il\endcsname{\def\PY@tc##1{\textcolor[rgb]{0.40,0.40,0.40}{##1}}}
\expandafter\def\csname PY@tok@mo\endcsname{\def\PY@tc##1{\textcolor[rgb]{0.40,0.40,0.40}{##1}}}
\expandafter\def\csname PY@tok@ch\endcsname{\let\PY@it=\textit\def\PY@tc##1{\textcolor[rgb]{0.25,0.50,0.50}{##1}}}
\expandafter\def\csname PY@tok@cm\endcsname{\let\PY@it=\textit\def\PY@tc##1{\textcolor[rgb]{0.25,0.50,0.50}{##1}}}
\expandafter\def\csname PY@tok@cpf\endcsname{\let\PY@it=\textit\def\PY@tc##1{\textcolor[rgb]{0.25,0.50,0.50}{##1}}}
\expandafter\def\csname PY@tok@c1\endcsname{\let\PY@it=\textit\def\PY@tc##1{\textcolor[rgb]{0.25,0.50,0.50}{##1}}}
\expandafter\def\csname PY@tok@cs\endcsname{\let\PY@it=\textit\def\PY@tc##1{\textcolor[rgb]{0.25,0.50,0.50}{##1}}}

\def\PYZbs{\char`\\}
\def\PYZus{\char`\_}
\def\PYZob{\char`\{}
\def\PYZcb{\char`\}}
\def\PYZca{\char`\^}
\def\PYZam{\char`\&}
\def\PYZlt{\char`\<}
\def\PYZgt{\char`\>}
\def\PYZsh{\char`\#}
\def\PYZpc{\char`\%}
\def\PYZdl{\char`\$}
\def\PYZhy{\char`\-}
\def\PYZsq{\char`\'}
\def\PYZdq{\char`\"}
\def\PYZti{\char`\~}
% for compatibility with earlier versions
\def\PYZat{@}
\def\PYZlb{[}
\def\PYZrb{]}
\makeatother


    % For linebreaks inside Verbatim environment from package fancyvrb. 
    \makeatletter
        \newbox\Wrappedcontinuationbox 
        \newbox\Wrappedvisiblespacebox 
        \newcommand*\Wrappedvisiblespace {\textcolor{red}{\textvisiblespace}} 
        \newcommand*\Wrappedcontinuationsymbol {\textcolor{red}{\llap{\tiny$\m@th\hookrightarrow$}}} 
        \newcommand*\Wrappedcontinuationindent {3ex } 
        \newcommand*\Wrappedafterbreak {\kern\Wrappedcontinuationindent\copy\Wrappedcontinuationbox} 
        % Take advantage of the already applied Pygments mark-up to insert 
        % potential linebreaks for TeX processing. 
        %        {, <, #, %, $, ' and ": go to next line. 
        %        _, }, ^, &, >, - and ~: stay at end of broken line. 
        % Use of \textquotesingle for straight quote. 
        \newcommand*\Wrappedbreaksatspecials {% 
            \def\PYGZus{\discretionary{\char`\_}{\Wrappedafterbreak}{\char`\_}}% 
            \def\PYGZob{\discretionary{}{\Wrappedafterbreak\char`\{}{\char`\{}}% 
            \def\PYGZcb{\discretionary{\char`\}}{\Wrappedafterbreak}{\char`\}}}% 
            \def\PYGZca{\discretionary{\char`\^}{\Wrappedafterbreak}{\char`\^}}% 
            \def\PYGZam{\discretionary{\char`\&}{\Wrappedafterbreak}{\char`\&}}% 
            \def\PYGZlt{\discretionary{}{\Wrappedafterbreak\char`\<}{\char`\<}}% 
            \def\PYGZgt{\discretionary{\char`\>}{\Wrappedafterbreak}{\char`\>}}% 
            \def\PYGZsh{\discretionary{}{\Wrappedafterbreak\char`\#}{\char`\#}}% 
            \def\PYGZpc{\discretionary{}{\Wrappedafterbreak\char`\%}{\char`\%}}% 
            \def\PYGZdl{\discretionary{}{\Wrappedafterbreak\char`\$}{\char`\$}}% 
            \def\PYGZhy{\discretionary{\char`\-}{\Wrappedafterbreak}{\char`\-}}% 
            \def\PYGZsq{\discretionary{}{\Wrappedafterbreak\textquotesingle}{\textquotesingle}}% 
            \def\PYGZdq{\discretionary{}{\Wrappedafterbreak\char`\"}{\char`\"}}% 
            \def\PYGZti{\discretionary{\char`\~}{\Wrappedafterbreak}{\char`\~}}% 
        } 
        % Some characters . , ; ? ! / are not pygmentized. 
        % This macro makes them "active" and they will insert potential linebreaks 
        \newcommand*\Wrappedbreaksatpunct {% 
            \lccode`\~`\.\lowercase{\def~}{\discretionary{\hbox{\char`\.}}{\Wrappedafterbreak}{\hbox{\char`\.}}}% 
            \lccode`\~`\,\lowercase{\def~}{\discretionary{\hbox{\char`\,}}{\Wrappedafterbreak}{\hbox{\char`\,}}}% 
            \lccode`\~`\;\lowercase{\def~}{\discretionary{\hbox{\char`\;}}{\Wrappedafterbreak}{\hbox{\char`\;}}}% 
            \lccode`\~`\:\lowercase{\def~}{\discretionary{\hbox{\char`\:}}{\Wrappedafterbreak}{\hbox{\char`\:}}}% 
            \lccode`\~`\?\lowercase{\def~}{\discretionary{\hbox{\char`\?}}{\Wrappedafterbreak}{\hbox{\char`\?}}}% 
            \lccode`\~`\!\lowercase{\def~}{\discretionary{\hbox{\char`\!}}{\Wrappedafterbreak}{\hbox{\char`\!}}}% 
            \lccode`\~`\/\lowercase{\def~}{\discretionary{\hbox{\char`\/}}{\Wrappedafterbreak}{\hbox{\char`\/}}}% 
            \catcode`\.\active
            \catcode`\,\active 
            \catcode`\;\active
            \catcode`\:\active
            \catcode`\?\active
            \catcode`\!\active
            \catcode`\/\active 
            \lccode`\~`\~ 	
        }
    \makeatother

    \let\OriginalVerbatim=\Verbatim
    \makeatletter
    \renewcommand{\Verbatim}[1][1]{%
        %\parskip\z@skip
        \sbox\Wrappedcontinuationbox {\Wrappedcontinuationsymbol}%
        \sbox\Wrappedvisiblespacebox {\FV@SetupFont\Wrappedvisiblespace}%
        \def\FancyVerbFormatLine ##1{\hsize\linewidth
            \vtop{\raggedright\hyphenpenalty\z@\exhyphenpenalty\z@
                \doublehyphendemerits\z@\finalhyphendemerits\z@
                \strut ##1\strut}%
        }%
        % If the linebreak is at a space, the latter will be displayed as visible
        % space at end of first line, and a continuation symbol starts next line.
        % Stretch/shrink are however usually zero for typewriter font.
        \def\FV@Space {%
            \nobreak\hskip\z@ plus\fontdimen3\font minus\fontdimen4\font
            \discretionary{\copy\Wrappedvisiblespacebox}{\Wrappedafterbreak}
            {\kern\fontdimen2\font}%
        }%
        
        % Allow breaks at special characters using \PYG... macros.
        \Wrappedbreaksatspecials
        % Breaks at punctuation characters . , ; ? ! and / need catcode=\active 	
        \OriginalVerbatim[#1,codes*=\Wrappedbreaksatpunct]%
    }
    \makeatother

    % Exact colors from NB
    \definecolor{incolor}{HTML}{303F9F}
    \definecolor{outcolor}{HTML}{D84315}
    \definecolor{cellborder}{HTML}{CFCFCF}
    \definecolor{cellbackground}{HTML}{F7F7F7}
    
    % prompt
    \makeatletter
    \newcommand{\boxspacing}{\kern\kvtcb@left@rule\kern\kvtcb@boxsep}
    \makeatother
    \newcommand{\prompt}[4]{
        \ttfamily\llap{{\color{#2}[#3]:\hspace{3pt}#4}}\vspace{-\baselineskip}
    }
    

    
    % Prevent overflowing lines due to hard-to-break entities
    \sloppy 
    % Setup hyperref package
    \hypersetup{
      breaklinks=true,  % so long urls are correctly broken across lines
      colorlinks=true,
      urlcolor=urlcolor,
      linkcolor=linkcolor,
      citecolor=citecolor,
      }
    % Slightly bigger margins than the latex defaults
    
    \geometry{verbose,tmargin=1in,bmargin=1in,lmargin=1in,rmargin=1in}
    
    

\begin{document}
    
    \maketitle
    
    

    
    \hypertarget{jod-best-practices-lab}{%
\section{JOD Best Practices Lab}\label{jod-best-practices-lab}}

\includegraphics{inclusions/jodteenytinycube.png}

    \hypertarget{introduction}{%
\subsubsection{Introduction}\label{introduction}}

Software tools are like loaded guns: powerful weapons for slaying your
coding demons but also dangerous when used improperly. Have you ever
shot yourself in the foot? I know I'm missing a few toes and I bet you
are as well.

This lab outlines a number of ``best practices'' or guidelines for using
JOD. I've learned the hard way; if you take my advice to heart you
\emph{might} be spared!

    \begin{tcolorbox}[breakable, size=fbox, boxrule=1pt, pad at break*=1mm,colback=cellbackground, colframe=cellborder]
\prompt{In}{incolor}{1}{\boxspacing}
\begin{Verbatim}[commandchars=\\\{\}]
\PY{c+c1}{NB. show J version}
\PY{l+m+mi}{9}\PY{o}{!}\PY{o}{:}\PY{l+m+mi}{14} \PY{l+s}{\PYZsq{}}\PY{l+s}{\PYZsq{}}
\end{Verbatim}
\end{tcolorbox}

    \begin{Verbatim}[commandchars=\\\{\}]

j903/j64avx2/windows/beta-w/commercial/www.jsoftware.com/2021-12-05T18:25:00/cla
ng-13-0-0/SLEEF=1
    \end{Verbatim}

    \begin{tcolorbox}[breakable, size=fbox, boxrule=1pt, pad at break*=1mm,colback=cellbackground, colframe=cellborder]
\prompt{In}{incolor}{2}{\boxspacing}
\begin{Verbatim}[commandchars=\\\{\}]
\PY{n+nv}{load} \PY{l+s}{\PYZsq{}}\PY{l+s}{g}\PY{l+s}{e}\PY{l+s}{n}\PY{l+s}{e}\PY{l+s}{r}\PY{l+s}{a}\PY{l+s}{l}\PY{l+s}{/}\PY{l+s}{j}\PY{l+s}{o}\PY{l+s}{d}\PY{l+s}{\PYZsq{}}

\PY{c+c1}{NB. use portable box drawing characters}
\PY{n+nv}{portchars} \PY{l+s}{\PYZsq{}}\PY{l+s}{\PYZsq{}}

\PY{c+c1}{NB. Verb to show large boxed displays}
\PY{c+c1}{NB. in notebook without ugly wrapping}
\PY{n+nv}{sbx}\PY{o}{=:} \PY{l+s}{\PYZsq{}}\PY{l+s}{ }\PY{l+s}{.}\PY{l+s}{.}\PY{l+s}{.}\PY{l+s}{ }\PY{l+s}{\PYZsq{}} \PY{o}{,}\PY{o}{\PYZdq{}}\PY{l+m+mi}{1}\PY{o}{\PYZti{}} \PY{l+m+mi}{73}\PY{o}{\PYZam{}}\PY{o}{\PYZob{}}\PY{o}{.}\PY{o}{\PYZdq{}}\PY{l+m+mi}{1}\PY{o}{@}\PY{o}{\PYZdq{}}\PY{o}{:}

\PY{c+c1}{NB. Verb to show first 12 lines of character}
\PY{c+c1}{NB. lists in notebook without wrapping}
\PY{n+nv}{tlf}\PY{o}{=:} \PY{o}{]} \PY{o}{,} \PY{p}{(}\PY{p}{(}\PY{l+m+mi}{10}\PY{o}{\PYZob{}}\PY{n+nv}{a}\PY{o}{.}\PY{p}{)}\PY{o}{\PYZdq{}}\PY{l+m}{\PYZus{}} \PY{o}{=} \PY{o}{\PYZob{}}\PY{o}{:}\PY{p}{)} \PY{o}{\PYZcb{}}\PY{o}{.} \PY{p}{(}\PY{l+m+mi}{10}\PY{o}{\PYZob{}}\PY{n+nv}{a}\PY{o}{.}\PY{p}{)}\PY{o}{\PYZdq{}}\PY{l+m}{\PYZus{}}
\PY{n+nv}{stx}\PY{o}{=:} \PY{o}{\PYZob{}}\PY{o}{\PYZob{}}\PY{n+nv}{sbx} \PY{l+m+mi}{12} \PY{o}{\PYZob{}}\PY{o}{.} \PY{o}{]}\PY{o}{;}\PY{o}{.}\PY{l+m+mi}{\PYZus{}2} \PY{n+nv}{tlf} \PY{n+nv}{y} \PY{o}{\PYZhy{}}\PY{o}{.} \PY{n+nv}{CR}\PY{o}{\PYZcb{}}\PY{o}{\PYZcb{}}

\PY{c+c1}{NB. show JOD version}
\PY{n+nv}{smoutput} \PY{n+nv}{JODVMD\PYZus{}ajod\PYZus{}}
\end{Verbatim}
\end{tcolorbox}

    \begin{Verbatim}[commandchars=\\\{\}]
+------------+-+--------------------+
|1.0.23 - dev|1|14 Dec 2021 10:09:40|
+------------+-+--------------------+
    \end{Verbatim}

    \hypertarget{test-for-jod-user-defined-folders}{%
\subsubsection{Test for JOD user defined
folders}\label{test-for-jod-user-defined-folders}}

\textbf{WARNING: THIS LAB REQUIRES A NUMBER OF USER DEFINED JOD
FOLDERS.}

This lab requires a number of JOD folders to run. The next step tests
for the existence of these folders.

If (\texttt{TestJODDirectories}) lists any undefined JOD directories
configure these directories before running this lab.
\href{https://github.com/jsoftware/general_joddocument/blob/master/pdfdoc/jod.pdf}{Instructions
on how to do this} can be found in (\texttt{jod.pdf}). Install the
\href{https://code.jsoftware.com/wiki/Addons/general/joddocument}{joddocument}
addon to get (\texttt{jod.pdf}).

    \begin{tcolorbox}[breakable, size=fbox, boxrule=1pt, pad at break*=1mm,colback=cellbackground, colframe=cellborder]
\prompt{In}{incolor}{3}{\boxspacing}
\begin{Verbatim}[commandchars=\\\{\}]
\PY{n+nv}{TestJODDirectories\PYZus{}ijod\PYZus{}}\PY{o}{=:}\PY{n+nf}{3 : 0}

\PY{c+c1}{NB.*TestJODDirectories v\PYZhy{}\PYZhy{} test user configured JOD directories.}
\PY{c+c1}{NB.}
\PY{c+c1}{NB. This  verb  checks  that  required  JOD  lab directories  are}
\PY{c+c1}{NB. defined. \PYZdq{}Defined\PYZdq{} does not mean the  directories  exist only}
\PY{c+c1}{NB. that (jpath) expands to something other than its default.}
\PY{c+c1}{NB.}
\PY{c+c1}{NB. monad:  clMsg =. TestJODDirectories uuIgnore}

\PY{c+c1}{NB. when a relative directory does not exist (jpath) echoes its argument}
\PY{n+nv}{jodudirs} \PY{o}{=.} \PY{l+s}{\PYZsq{}}\PY{l+s}{\PYZti{}}\PY{l+s}{\PYZsq{}}\PY{o}{\PYZam{}}\PY{o}{,}\PY{o}{\PYZam{}}\PY{o}{.}\PY{o}{\PYZgt{}} \PY{o}{;}\PY{o}{:}\PY{l+s}{\PYZsq{}}\PY{l+s}{J}\PY{l+s}{O}\PY{l+s}{D}\PY{l+s}{ }\PY{l+s}{J}\PY{l+s}{O}\PY{l+s}{D}\PY{l+s}{D}\PY{l+s}{U}\PY{l+s}{M}\PY{l+s}{P}\PY{l+s}{S}\PY{l+s}{ }\PY{l+s}{J}\PY{l+s}{O}\PY{l+s}{D}\PY{l+s}{S}\PY{l+s}{O}\PY{l+s}{U}\PY{l+s}{R}\PY{l+s}{C}\PY{l+s}{E}\PY{l+s}{ }\PY{l+s}{J}\PY{l+s}{O}\PY{l+s}{D}\PY{l+s}{T}\PY{l+s}{E}\PY{l+s}{S}\PY{l+s}{T}\PY{l+s}{\PYZsq{}}
\PY{n+nv}{jodudefs} \PY{o}{=.} \PY{n+nv}{jpath}\PY{o}{\PYZam{}}\PY{o}{.}\PY{o}{\PYZgt{}} \PY{n+nv}{jodudirs}  \PY{c+c1}{NB. !(*)=. jpath}
\PY{n+nv}{mask} \PY{o}{=.} \PY{n+nv}{jodudirs} \PY{o}{=} \PY{n+nv}{jodudefs}
\PY{n+nl}{if.} \PY{l+m+mi}{1} \PY{n+nv}{e}\PY{o}{.} \PY{n+nv}{mask} \PY{n+nl}{do.}
  \PY{o}{\PYZgt{}} \PY{l+s}{\PYZsq{}}\PY{l+s}{m}\PY{l+s}{i}\PY{l+s}{s}\PY{l+s}{s}\PY{l+s}{i}\PY{l+s}{n}\PY{l+s}{g}\PY{l+s}{ }\PY{l+s}{J}\PY{l+s}{O}\PY{l+s}{D}\PY{l+s}{ }\PY{l+s}{f}\PY{l+s}{o}\PY{l+s}{l}\PY{l+s}{d}\PY{l+s}{e}\PY{l+s}{r}\PY{l+s}{s}\PY{l+s}{ }\PY{l+s}{\PYZhy{}}\PY{l+s}{ }\PY{l+s}{d}\PY{l+s}{e}\PY{l+s}{f}\PY{l+s}{i}\PY{l+s}{n}\PY{l+s}{e}\PY{l+s}{ }\PY{l+s}{b}\PY{l+s}{e}\PY{l+s}{f}\PY{l+s}{o}\PY{l+s}{r}\PY{l+s}{e}\PY{l+s}{ }\PY{l+s}{r}\PY{l+s}{u}\PY{l+s}{n}\PY{l+s}{n}\PY{l+s}{i}\PY{l+s}{n}\PY{l+s}{g}\PY{l+s}{ }\PY{l+s}{J}\PY{l+s}{O}\PY{l+s}{D}\PY{l+s}{ }\PY{l+s}{l}\PY{l+s}{a}\PY{l+s}{b}\PY{l+s}{s}\PY{l+s}{\PYZsq{}} \PY{o}{;} \PY{n+nv}{mask} \PY{o}{\PYZsh{}} \PY{n+nv}{jodudirs}
\PY{n+nl}{else.}
  \PY{o}{\PYZgt{}} \PY{l+s}{\PYZsq{}}\PY{l+s}{a}\PY{l+s}{l}\PY{l+s}{l}\PY{l+s}{ }\PY{l+s}{J}\PY{l+s}{O}\PY{l+s}{D}\PY{l+s}{ }\PY{l+s}{f}\PY{l+s}{o}\PY{l+s}{l}\PY{l+s}{d}\PY{l+s}{e}\PY{l+s}{r}\PY{l+s}{s}\PY{l+s}{ }\PY{l+s}{a}\PY{l+s}{r}\PY{l+s}{e}\PY{l+s}{ }\PY{l+s}{d}\PY{l+s}{e}\PY{l+s}{f}\PY{l+s}{i}\PY{l+s}{n}\PY{l+s}{e}\PY{l+s}{d}\PY{l+s}{\PYZsq{}} \PY{o}{;} \PY{n+nv}{jodudefs}
\PY{n+nl}{end.}
\PY{n+nl}{)}

\PY{n+nv}{TestJODDirectories} \PY{l+m+mi}{0}
\end{Verbatim}
\end{tcolorbox}

    \begin{Verbatim}[commandchars=\\\{\}]
all JOD folders are defined
c:/users/john.baker/onedrive - jackson companies/jod
c:/users/john.baker/onedrive - jackson companies/jod/joddumps
c:/jodtest/labtesting
c:/jodtest/test
    \end{Verbatim}

    \hypertarget{lab-cleanup}{%
\subsubsection{Lab cleanup}\label{lab-cleanup}}

\textbf{WARNING: DICTIONARIES CREATED BY PRIOR RUNS OF LAB THIS WILL BE
DELETED IN THE NEXT STEP.}

When the Best Practices lab runs it creates a number of dictionaries in
the (\texttt{\textasciitilde{}JODSOURCE}) directory. The next step will
remove any of these dictionaries. \textbf{IF YOU CARE ABOUT THESE
DICTIONARIES STOP NOW.}

    \begin{tcolorbox}[breakable, size=fbox, boxrule=1pt, pad at break*=1mm,colback=cellbackground, colframe=cellborder]
\prompt{In}{incolor}{4}{\boxspacing}
\begin{Verbatim}[commandchars=\\\{\}]
\PY{n+nv}{RemoveLabBestDictionaries\PYZus{}ijod\PYZus{}}\PY{o}{=:} \PY{n+nf}{3 : 0}
\PY{n+nl}{if.} \PY{n+nv}{IFWIN} \PY{n+nl}{do.}
  \PY{n+nv}{shell} \PY{l+s}{\PYZsq{}}\PY{l+s}{r}\PY{l+s}{d}\PY{l+s}{ }\PY{l+s}{/}\PY{l+s}{s}\PY{l+s}{ }\PY{l+s}{/}\PY{l+s}{q}\PY{l+s}{ }\PY{l+s}{\PYZdq{}}\PY{l+s}{\PYZsq{}}\PY{o}{,}\PY{p}{(}\PY{n+nv}{jpath} \PY{l+s}{\PYZsq{}}\PY{l+s}{\PYZti{}}\PY{l+s}{J}\PY{l+s}{O}\PY{l+s}{D}\PY{l+s}{S}\PY{l+s}{O}\PY{l+s}{U}\PY{l+s}{R}\PY{l+s}{C}\PY{l+s}{E}\PY{l+s}{\PYZsq{}}\PY{p}{)}\PY{o}{,}\PY{l+s}{\PYZsq{}}\PY{l+s}{\PYZdq{}}\PY{l+s}{\PYZsq{}}
  \PY{n+nv}{smoutput} \PY{l+s}{\PYZsq{}}\PY{l+s}{L}\PY{l+s}{a}\PY{l+s}{b}\PY{l+s}{ }\PY{l+s}{t}\PY{l+s}{e}\PY{l+s}{m}\PY{l+s}{p}\PY{l+s}{o}\PY{l+s}{r}\PY{l+s}{a}\PY{l+s}{r}\PY{l+s}{y}\PY{l+s}{ }\PY{l+s}{b}\PY{l+s}{e}\PY{l+s}{s}\PY{l+s}{t}\PY{l+s}{ }\PY{l+s}{p}\PY{l+s}{r}\PY{l+s}{a}\PY{l+s}{c}\PY{l+s}{t}\PY{l+s}{i}\PY{l+s}{c}\PY{l+s}{e}\PY{l+s}{s}\PY{l+s}{ }\PY{l+s}{(}\PY{l+s}{w}\PY{l+s}{i}\PY{l+s}{n}\PY{l+s}{)}\PY{l+s}{ }\PY{l+s}{d}\PY{l+s}{i}\PY{l+s}{c}\PY{l+s}{t}\PY{l+s}{i}\PY{l+s}{o}\PY{l+s}{n}\PY{l+s}{a}\PY{l+s}{r}\PY{l+s}{i}\PY{l+s}{e}\PY{l+s}{s}\PY{l+s}{ }\PY{l+s}{e}\PY{l+s}{r}\PY{l+s}{a}\PY{l+s}{s}\PY{l+s}{e}\PY{l+s}{d}\PY{l+s}{\PYZsq{}}
\PY{n+nl}{elseif.} \PY{n+nv}{IFUNIX} \PY{n+nl}{do.}
  \PY{c+c1}{NB. avoid blanks in Mac and Linux paths}
  \PY{n+nv}{shell} \PY{l+s}{\PYZsq{}}\PY{l+s}{r}\PY{l+s}{m}\PY{l+s}{ }\PY{l+s}{\PYZhy{}}\PY{l+s}{r}\PY{l+s}{f}\PY{l+s}{ }\PY{l+s}{\PYZsq{}}\PY{o}{,}\PY{n+nv}{jpath} \PY{l+s}{\PYZsq{}}\PY{l+s}{\PYZti{}}\PY{l+s}{J}\PY{l+s}{O}\PY{l+s}{D}\PY{l+s}{S}\PY{l+s}{O}\PY{l+s}{U}\PY{l+s}{R}\PY{l+s}{C}\PY{l+s}{E}\PY{l+s}{\PYZsq{}}
  \PY{n+nv}{smoutput} \PY{l+s}{\PYZsq{}}\PY{l+s}{L}\PY{l+s}{a}\PY{l+s}{b}\PY{l+s}{ }\PY{l+s}{t}\PY{l+s}{e}\PY{l+s}{m}\PY{l+s}{p}\PY{l+s}{o}\PY{l+s}{r}\PY{l+s}{a}\PY{l+s}{r}\PY{l+s}{y}\PY{l+s}{ }\PY{l+s}{b}\PY{l+s}{e}\PY{l+s}{s}\PY{l+s}{t}\PY{l+s}{ }\PY{l+s}{p}\PY{l+s}{r}\PY{l+s}{a}\PY{l+s}{c}\PY{l+s}{t}\PY{l+s}{i}\PY{l+s}{c}\PY{l+s}{e}\PY{l+s}{s}\PY{l+s}{ }\PY{l+s}{(}\PY{l+s}{m}\PY{l+s}{a}\PY{l+s}{c}\PY{l+s}{/}\PY{l+s}{l}\PY{l+s}{i}\PY{l+s}{n}\PY{l+s}{u}\PY{l+s}{x}\PY{l+s}{)}\PY{l+s}{ }\PY{l+s}{d}\PY{l+s}{i}\PY{l+s}{c}\PY{l+s}{t}\PY{l+s}{i}\PY{l+s}{o}\PY{l+s}{n}\PY{l+s}{a}\PY{l+s}{r}\PY{l+s}{i}\PY{l+s}{e}\PY{l+s}{s}\PY{l+s}{ }\PY{l+s}{e}\PY{l+s}{r}\PY{l+s}{a}\PY{l+s}{s}\PY{l+s}{e}\PY{l+s}{d}\PY{l+s}{\PYZsq{}}
\PY{n+nl}{elseif.}\PY{n+nl}{do.}
  \PY{n+nv}{smoutput} \PY{l+s}{\PYZsq{}}\PY{l+s}{E}\PY{l+s}{r}\PY{l+s}{a}\PY{l+s}{s}\PY{l+s}{e}\PY{l+s}{ }\PY{l+s}{a}\PY{l+s}{n}\PY{l+s}{y}\PY{l+s}{ }\PY{l+s}{p}\PY{l+s}{r}\PY{l+s}{e}\PY{l+s}{v}\PY{l+s}{i}\PY{l+s}{o}\PY{l+s}{u}\PY{l+s}{s}\PY{l+s}{ }\PY{l+s}{t}\PY{l+s}{e}\PY{l+s}{m}\PY{l+s}{p}\PY{l+s}{o}\PY{l+s}{r}\PY{l+s}{a}\PY{l+s}{r}\PY{l+s}{y}\PY{l+s}{ }\PY{l+s}{b}\PY{l+s}{e}\PY{l+s}{s}\PY{l+s}{t}\PY{l+s}{ }\PY{l+s}{p}\PY{l+s}{r}\PY{l+s}{a}\PY{l+s}{c}\PY{l+s}{t}\PY{l+s}{i}\PY{l+s}{c}\PY{l+s}{e}\PY{l+s}{s}\PY{l+s}{ }\PY{l+s}{d}\PY{l+s}{i}\PY{l+s}{c}\PY{l+s}{t}\PY{l+s}{i}\PY{l+s}{o}\PY{l+s}{n}\PY{l+s}{a}\PY{l+s}{r}\PY{l+s}{i}\PY{l+s}{e}\PY{l+s}{s}\PY{l+s}{ }\PY{l+s}{m}\PY{l+s}{a}\PY{l+s}{n}\PY{l+s}{u}\PY{l+s}{a}\PY{l+s}{l}\PY{l+s}{l}\PY{l+s}{y}\PY{l+s}{.}\PY{l+s}{\PYZsq{}}
\PY{n+nl}{end.}
\PY{n+nl}{)}
\end{Verbatim}
\end{tcolorbox}

    Remove Best Practices lab dictionaries.

    \begin{tcolorbox}[breakable, size=fbox, boxrule=1pt, pad at break*=1mm,colback=cellbackground, colframe=cellborder]
\prompt{In}{incolor}{5}{\boxspacing}
\begin{Verbatim}[commandchars=\\\{\}]
\PY{c+c1}{NB. clear any previously opened dictionaries in master file}
\PY{n+nv}{dpset} \PY{l+s}{\PYZsq{}}\PY{l+s}{R}\PY{l+s}{E}\PY{l+s}{S}\PY{l+s}{E}\PY{l+s}{T}\PY{l+s}{M}\PY{l+s}{E}\PY{l+s}{\PYZsq{}}

\PY{c+c1}{NB. unregister dictionaries \PYZhy{} IGNORE ERRORS}
\PY{l+m+mi}{3} \PY{n+nv}{regd}\PY{o}{\PYZam{}}\PY{o}{\PYZgt{}} \PY{o}{;}\PY{o}{:}\PY{l+s}{\PYZsq{}}\PY{l+s}{b}\PY{l+s}{p}\PY{l+s}{c}\PY{l+s}{o}\PY{l+s}{p}\PY{l+s}{y}\PY{l+s}{ }\PY{l+s}{b}\PY{l+s}{p}\PY{l+s}{t}\PY{l+s}{e}\PY{l+s}{s}\PY{l+s}{t}\PY{l+s}{\PYZsq{}} \PY{o}{[} \PY{l+m+mi}{3} \PY{n+nv}{od} \PY{l+s}{\PYZsq{}}\PY{l+s}{\PYZsq{}}

\PY{c+c1}{NB. delete dictionary files}
\PY{n+nv}{RemoveLabBestDictionaries} \PY{l+m+mi}{0}
\end{Verbatim}
\end{tcolorbox}

    \begin{Verbatim}[commandchars=\\\{\}]
Lab temporary best practices (win) dictionaries erased
    \end{Verbatim}

    \hypertarget{jod-does-not-belong-in-the-j-tree}{%
\subsubsection{JOD does not belong in the J
tree}\label{jod-does-not-belong-in-the-j-tree}}

My first and most important bit of advice is simply:

\textbf{NEVER NEVER NEVER store your JOD dictionaries in J install
directories!}

I would also avoid any OS managed directories like

\begin{verbatim}
c:\Program Files\..
\end{verbatim}

Create a JOD master dictionary directory root that is completely
independent of J. It's also a good idea to define a subdirectory
structure that mirrors J's versions. JOD creates binary \texttt{jfiles}.
These files are fairly stable but binaries can change when J changes.

    \begin{tcolorbox}[breakable, size=fbox, boxrule=1pt, pad at break*=1mm,colback=cellbackground, colframe=cellborder]
\prompt{In}{incolor}{6}{\boxspacing}
\begin{Verbatim}[commandchars=\\\{\}]
\PY{c+c1}{NB. create a master JOD directory root outside of J\PYZsq{}s directories.}
\PY{c+c1}{NB. This creation depends on a configured directory (\PYZti{}JODSOURCE).}
\PY{n+nv}{smoutput} \PY{n+nv}{newd} \PY{l+s}{\PYZsq{}}\PY{l+s}{b}\PY{l+s}{p}\PY{l+s}{t}\PY{l+s}{e}\PY{l+s}{s}\PY{l+s}{t}\PY{l+s}{\PYZsq{}}\PY{o}{;}\PY{p}{(}\PY{n+nv}{jpath} \PY{l+s}{\PYZsq{}}\PY{l+s}{\PYZti{}}\PY{l+s}{J}\PY{l+s}{O}\PY{l+s}{D}\PY{l+s}{S}\PY{l+s}{O}\PY{l+s}{U}\PY{l+s}{R}\PY{l+s}{C}\PY{l+s}{E}\PY{l+s}{/}\PY{l+s}{b}\PY{l+s}{p}\PY{l+s}{t}\PY{l+s}{e}\PY{l+s}{s}\PY{l+s}{t}\PY{l+s}{\PYZsq{}}\PY{p}{)}\PY{o}{;}\PY{l+s}{\PYZsq{}}\PY{l+s}{b}\PY{l+s}{e}\PY{l+s}{s}\PY{l+s}{t}\PY{l+s}{ }\PY{l+s}{p}\PY{l+s}{r}\PY{l+s}{a}\PY{l+s}{c}\PY{l+s}{t}\PY{l+s}{i}\PY{l+s}{c}\PY{l+s}{e}\PY{l+s}{s}\PY{l+s}{ }\PY{l+s}{t}\PY{l+s}{e}\PY{l+s}{s}\PY{l+s}{t}\PY{l+s}{ }\PY{l+s}{d}\PY{l+s}{i}\PY{l+s}{c}\PY{l+s}{t}\PY{l+s}{i}\PY{l+s}{o}\PY{l+s}{n}\PY{l+s}{a}\PY{l+s}{r}\PY{l+s}{y}\PY{l+s}{\PYZsq{}}

\PY{c+c1}{NB. This is an example \PYZhy{} use another root for your dictionaries.}
\end{Verbatim}
\end{tcolorbox}

    \begin{Verbatim}[commandchars=\\\{\}]
+-+---------------------+------+-----------------------------+
|1|dictionary created ->|bptest|c:/jodtest/labtesting/bptest/|
+-+---------------------+------+-----------------------------+
    \end{Verbatim}

    \hypertarget{load-lab-dictionary-from-dump-script}{%
\subsubsection{Load lab dictionary from dump
script}\label{load-lab-dictionary-from-dump-script}}

To illustrate key features of JOD we need a nonempty dictionary. This
next step loads the (\texttt{bptest}) dictionary from a dump script
distributed with JOD.

    \begin{tcolorbox}[breakable, size=fbox, boxrule=1pt, pad at break*=1mm,colback=cellbackground, colframe=cellborder]
\prompt{In}{incolor}{7}{\boxspacing}
\begin{Verbatim}[commandchars=\\\{\}]
\PY{c+c1}{NB. open new dictionary}
\PY{n+nv}{od} \PY{l+s}{\PYZsq{}}\PY{l+s}{b}\PY{l+s}{p}\PY{l+s}{t}\PY{l+s}{e}\PY{l+s}{s}\PY{l+s}{t}\PY{l+s}{\PYZsq{}} \PY{o}{[} \PY{l+m+mi}{3} \PY{n+nv}{od} \PY{l+s}{\PYZsq{}}\PY{l+s}{\PYZsq{}}

\PY{c+c1}{NB. load from example dump script}
\PY{l+m+mi}{0}\PY{o}{!}\PY{o}{:}\PY{l+m+mi}{0} \PY{o}{\PYZlt{}}\PY{n+nv}{jpath} \PY{l+s}{\PYZsq{}}\PY{l+s}{\PYZti{}}\PY{l+s}{a}\PY{l+s}{d}\PY{l+s}{d}\PY{l+s}{o}\PY{l+s}{n}\PY{l+s}{s}\PY{l+s}{/}\PY{l+s}{g}\PY{l+s}{e}\PY{l+s}{n}\PY{l+s}{e}\PY{l+s}{r}\PY{l+s}{a}\PY{l+s}{l}\PY{l+s}{/}\PY{l+s}{j}\PY{l+s}{o}\PY{l+s}{d}\PY{l+s}{/}\PY{l+s}{j}\PY{l+s}{o}\PY{l+s}{d}\PY{l+s}{l}\PY{l+s}{a}\PY{l+s}{b}\PY{l+s}{s}\PY{l+s}{/}\PY{l+s}{l}\PY{l+s}{a}\PY{l+s}{b}\PY{l+s}{d}\PY{l+s}{u}\PY{l+s}{m}\PY{l+s}{p}\PY{l+s}{.}\PY{l+s}{i}\PY{l+s}{j}\PY{l+s}{s}\PY{l+s}{\PYZsq{}}

\PY{c+c1}{NB. regenerate all references \PYZhy{} show last three messages}
\PY{l+m+mi}{\PYZus{}3} \PY{o}{\PYZob{}}\PY{o}{.} \PY{l+m+mi}{0} \PY{n+nv}{globs}\PY{o}{\PYZam{}}\PY{o}{\PYZgt{}} \PY{o}{\PYZcb{}}\PY{o}{.} \PY{n+nv}{revo}\PY{l+s}{\PYZsq{}}\PY{l+s}{\PYZsq{}}
\end{Verbatim}
\end{tcolorbox}

    \begin{Verbatim}[commandchars=\\\{\}]
+-+-------------------+------+
|1|1 word(s) put in ->|bptest|
+-+-------------------+------+
+-+--------------------+------+
|1|35 word(s) put in ->|bptest|
+-+--------------------+------+
+-+--------------------------------+------+
|1|36 word explanation(s) put in ->|bptest|
+-+--------------------------------+------+
+-+----------------------------+------+
|1|2 word document(s) put in ->|bptest|
+-+----------------------------+------+
+-+-------------------------+------+
|1|group <bstats> put in -> |bptest|
+-+-------------------------+------+
|1|group <sunmoon> put in ->|bptest|
+-+-------------------------+------+
NB. end-of-JOD-dump-file regenerate cross references with:  0 globs\&> \}. revo ''
+-+-------------------------------------+------+
|1|<var> references put in ->           |bptest|
+-+-------------------------------------+------+
|1|<yeardates> references put in ->     |bptest|
+-+-------------------------------------+------+
|1|<NORISESET> is a noun - no references|      |
+-+-------------------------------------+------+
    \end{Verbatim}

    \hypertarget{backup-backup-backup}{%
\subsubsection{Backup backup backup}\label{backup-backup-backup}}

A wise man once said, \emph{``You're either backed up or f\&\%ked up!''}

\emph{Care to go over the options again?}

It is literally a snap to make a backup with JOD. So backup often!

    \begin{tcolorbox}[breakable, size=fbox, boxrule=1pt, pad at break*=1mm,colback=cellbackground, colframe=cellborder]
\prompt{In}{incolor}{8}{\boxspacing}
\begin{Verbatim}[commandchars=\\\{\}]
\PY{c+c1}{NB. open the best practice dictionary}
\PY{n+nv}{od} \PY{l+s}{\PYZsq{}}\PY{l+s}{b}\PY{l+s}{p}\PY{l+s}{t}\PY{l+s}{e}\PY{l+s}{s}\PY{l+s}{t}\PY{l+s}{\PYZsq{}} \PY{o}{[} \PY{l+m+mi}{3} \PY{n+nv}{od} \PY{l+s}{\PYZsq{}}\PY{l+s}{\PYZsq{}}

\PY{c+c1}{NB. back it up}
\PY{n+nv}{smoutput} \PY{n+nv}{packd} \PY{l+s}{\PYZsq{}}\PY{l+s}{b}\PY{l+s}{p}\PY{l+s}{t}\PY{l+s}{e}\PY{l+s}{s}\PY{l+s}{t}\PY{l+s}{\PYZsq{}}

\PY{c+c1}{NB. restd recovers the most current backup}
\PY{n+nv}{restd} \PY{l+s}{\PYZsq{}}\PY{l+s}{b}\PY{l+s}{p}\PY{l+s}{t}\PY{l+s}{e}\PY{l+s}{s}\PY{l+s}{t}\PY{l+s}{\PYZsq{}}
\end{Verbatim}
\end{tcolorbox}

    \begin{Verbatim}[commandchars=\\\{\}]
+-+--------------------+------+-+
|1|dictionary packed ->|bptest|0|
+-+--------------------+------+-+
+-+----------------------+------+-+
|1|dictionary restored ->|bptest|0|
+-+----------------------+------+-+
    \end{Verbatim}

    \hypertarget{take-a-script-dump}{%
\subsubsection{Take a script dump}\label{take-a-script-dump}}

(\texttt{packd}) backs up binary \texttt{jfiles} but it's also a good
idea to ``dump'' your dictionaries as plain text. JOD can dump all open
dictionaries as a single J script. Script dumps are the most stable way
to store J dictionaries. The \texttt{jodsource} addon distributes all
JOD source code in this form.

    \begin{tcolorbox}[breakable, size=fbox, boxrule=1pt, pad at break*=1mm,colback=cellbackground, colframe=cellborder]
\prompt{In}{incolor}{9}{\boxspacing}
\begin{Verbatim}[commandchars=\\\{\}]
\PY{c+c1}{NB. dump only (bptest)}
\PY{n+nv}{od} \PY{l+s}{\PYZsq{}}\PY{l+s}{b}\PY{l+s}{p}\PY{l+s}{t}\PY{l+s}{e}\PY{l+s}{s}\PY{l+s}{t}\PY{l+s}{\PYZsq{}} \PY{o}{[} \PY{l+m+mi}{3} \PY{n+nv}{od} \PY{l+s}{\PYZsq{}}\PY{l+s}{\PYZsq{}}

\PY{c+c1}{NB. (make) creates a dictionary dump in the dump subdirectory}
\PY{n+nv}{bptestdump}\PY{o}{=:} \PY{n+nv}{showpass\PYZus{}ajod\PYZus{}} \PY{n+nv}{make} \PY{l+s}{\PYZsq{}}\PY{l+s}{\PYZsq{}}
\end{Verbatim}
\end{tcolorbox}

    \begin{Verbatim}[commandchars=\\\{\}]
+-+---------------------------+--------------------------------------------+
|1|object(s) on path dumped ->|c:/jodtest/labtesting/bptest/dump/bptest.ijs|
+-+---------------------------+--------------------------------------------+
    \end{Verbatim}

    \hypertarget{some-uses-of-dump-scripts}{%
\subsubsection{Some uses of dump
scripts}\label{some-uses-of-dump-scripts}}

JOD dump scripts can be used to reload, copy and merge dictionaries.

    \begin{tcolorbox}[breakable, size=fbox, boxrule=1pt, pad at break*=1mm,colback=cellbackground, colframe=cellborder]
\prompt{In}{incolor}{10}{\boxspacing}
\begin{Verbatim}[commandchars=\\\{\}]
\PY{c+c1}{NB. reload bptest}
\PY{n+nv}{od} \PY{l+s}{\PYZsq{}}\PY{l+s}{b}\PY{l+s}{p}\PY{l+s}{t}\PY{l+s}{e}\PY{l+s}{s}\PY{l+s}{t}\PY{l+s}{\PYZsq{}} \PY{o}{[} \PY{l+m+mi}{3} \PY{n+nv}{od} \PY{l+s}{\PYZsq{}}\PY{l+s}{\PYZsq{}}
\PY{l+m+mi}{0}\PY{o}{!}\PY{o}{:}\PY{l+m+mi}{0} \PY{o}{\PYZob{}}\PY{o}{:}\PY{n+nv}{bptestdump}
\end{Verbatim}
\end{tcolorbox}

    \begin{Verbatim}[commandchars=\\\{\}]
+-+-------------------+------+
|1|1 word(s) put in ->|bptest|
+-+-------------------+------+
+-+--------------------+------+
|1|35 word(s) put in ->|bptest|
+-+--------------------+------+
+-+--------------------------------+------+
|1|36 word explanation(s) put in ->|bptest|
+-+--------------------------------+------+
+-+----------------------------+------+
|1|2 word document(s) put in ->|bptest|
+-+----------------------------+------+
+-+-------------------------+------+
|1|group <bstats> put in -> |bptest|
+-+-------------------------+------+
|1|group <sunmoon> put in ->|bptest|
+-+-------------------------+------+
+-+------------------------------+------+
|1|dictionary document updated ->|bptest|
+-+------------------------------+------+
NB. end-of-JOD-dump-file regenerate cross references with:  0 globs\&> \}. revo ''
    \end{Verbatim}

    Copy a dictionary.

    \begin{tcolorbox}[breakable, size=fbox, boxrule=1pt, pad at break*=1mm,colback=cellbackground, colframe=cellborder]
\prompt{In}{incolor}{11}{\boxspacing}
\begin{Verbatim}[commandchars=\\\{\}]
\PY{c+c1}{NB. copy/merge bptest dictionary}
\PY{n+nv}{smoutput} \PY{n+nv}{newd} \PY{l+s}{\PYZsq{}}\PY{l+s}{b}\PY{l+s}{p}\PY{l+s}{c}\PY{l+s}{o}\PY{l+s}{p}\PY{l+s}{y}\PY{l+s}{\PYZsq{}}\PY{o}{;}\PY{n+nv}{jpath} \PY{l+s}{\PYZsq{}}\PY{l+s}{\PYZti{}}\PY{l+s}{J}\PY{l+s}{O}\PY{l+s}{D}\PY{l+s}{S}\PY{l+s}{O}\PY{l+s}{U}\PY{l+s}{R}\PY{l+s}{C}\PY{l+s}{E}\PY{l+s}{/}\PY{l+s}{b}\PY{l+s}{p}\PY{l+s}{c}\PY{l+s}{o}\PY{l+s}{p}\PY{l+s}{y}\PY{l+s}{\PYZsq{}}
\PY{n+nv}{od} \PY{l+s}{\PYZsq{}}\PY{l+s}{b}\PY{l+s}{p}\PY{l+s}{c}\PY{l+s}{o}\PY{l+s}{p}\PY{l+s}{y}\PY{l+s}{\PYZsq{}} \PY{o}{[} \PY{l+m+mi}{3} \PY{n+nv}{od} \PY{l+s}{\PYZsq{}}\PY{l+s}{\PYZsq{}}
\PY{l+m+mi}{0}\PY{o}{!}\PY{o}{:}\PY{l+m+mi}{0} \PY{o}{\PYZob{}}\PY{o}{:}\PY{n+nv}{bptestdump}


\PY{c+c1}{NB. clear path}
\PY{n+nv}{dpset} \PY{l+s}{\PYZsq{}}\PY{l+s}{C}\PY{l+s}{L}\PY{l+s}{E}\PY{l+s}{A}\PY{l+s}{R}\PY{l+s}{P}\PY{l+s}{A}\PY{l+s}{T}\PY{l+s}{H}\PY{l+s}{\PYZsq{}}
\end{Verbatim}
\end{tcolorbox}

    \begin{Verbatim}[commandchars=\\\{\}]
+-+---------------------+------+-----------------------------+
|1|dictionary created ->|bpcopy|c:/jodtest/labtesting/bpcopy/|
+-+---------------------+------+-----------------------------+
+-+-------------------+------+
|1|1 word(s) put in ->|bpcopy|
+-+-------------------+------+
+-+--------------------+------+
|1|35 word(s) put in ->|bpcopy|
+-+--------------------+------+
+-+--------------------------------+------+
|1|36 word explanation(s) put in ->|bpcopy|
+-+--------------------------------+------+
+-+----------------------------+------+
|1|2 word document(s) put in ->|bpcopy|
+-+----------------------------+------+
+-+-------------------------+------+
|1|group <bstats> put in -> |bpcopy|
+-+-------------------------+------+
|1|group <sunmoon> put in ->|bpcopy|
+-+-------------------------+------+
+-+------------------------------+------+
|1|dictionary document updated ->|bpcopy|
+-+------------------------------+------+
NB. end-of-JOD-dump-file regenerate cross references with:  0 globs\&> \}. revo ''
+-+---------------+------+
|1|path cleared ->|bpcopy|
+-+---------------+------+
    \end{Verbatim}

    \hypertarget{dump-scripts-are-the-best-way-to-share-and-version-control-dictionaries}{%
\subsubsection{Dump scripts are the best way to share and version
control
dictionaries}\label{dump-scripts-are-the-best-way-to-share-and-version-control-dictionaries}}

Dump scripts can be used to share and version control dictionaries. See
this Git repository for examples.

\href{https://github.com/bakerjd99/joddumps}{Example JOD Dump Script
Repository}

    \hypertarget{make-a-master-re-register-script}{%
\subsubsection{Make a master re-register
script}\label{make-a-master-re-register-script}}

JOD only sees the dictionaries registered in the \texttt{jmaster.ijf}
file so maintaining a list of registered dictionaries is a good idea.
JOD can generate a re-register script that you can edit.

Generate a re-register script and put it in your main JOD dictionary
directory root.

    \begin{tcolorbox}[breakable, size=fbox, boxrule=1pt, pad at break*=1mm,colback=cellbackground, colframe=cellborder]
\prompt{In}{incolor}{12}{\boxspacing}
\begin{Verbatim}[commandchars=\\\{\}]
\PY{c+c1}{NB. generate re\PYZhy{}register script}
\PY{n+nv}{rereg}\PY{o}{=:} \PY{o}{;}\PY{o}{\PYZob{}}\PY{o}{:} \PY{l+m+mi}{5} \PY{n+nv}{od} \PY{l+s}{\PYZsq{}}\PY{l+s}{\PYZsq{}}

\PY{c+c1}{NB. save it in the master root}
\PY{n+nv}{rereg} \PY{n+nv}{write\PYZus{}ajod\PYZus{}} \PY{n+nv}{jpath} \PY{l+s}{\PYZsq{}}\PY{l+s}{\PYZti{}}\PY{l+s}{J}\PY{l+s}{O}\PY{l+s}{D}\PY{l+s}{S}\PY{l+s}{O}\PY{l+s}{U}\PY{l+s}{R}\PY{l+s}{C}\PY{l+s}{E}\PY{l+s}{/}\PY{l+s}{j}\PY{l+s}{o}\PY{l+s}{d}\PY{l+s}{r}\PY{l+s}{e}\PY{l+s}{g}\PY{l+s}{i}\PY{l+s}{s}\PY{l+s}{t}\PY{l+s}{e}\PY{l+s}{r}\PY{l+s}{.}\PY{l+s}{i}\PY{l+s}{j}\PY{l+s}{s}\PY{l+s}{\PYZsq{}}
\end{Verbatim}
\end{tcolorbox}

    \hypertarget{set-library-dictionaries-to-readonly}{%
\subsubsection{\texorpdfstring{Set library dictionaries to
\texttt{READONLY}}{Set library dictionaries to READONLY}}\label{set-library-dictionaries-to-readonly}}

Open JOD dictionaries define a search path. The first dictionary on the
path is the only dictionary that can be changed. It is called the
``put'' dictionary. Even though nonput dictionaries cannot be changed by
JOD it's a good idea to set them \texttt{READONLY} because:

\texttt{READONLY} dictionaries can be accessed by any number of JOD
tasks. \texttt{READWRITE} dictionaries can only be accessed by one task.

Keeping libraries \texttt{READONLY} prevents accidental put's as you
open and close dictionaries.

    \begin{tcolorbox}[breakable, size=fbox, boxrule=1pt, pad at break*=1mm,colback=cellbackground, colframe=cellborder]
\prompt{In}{incolor}{13}{\boxspacing}
\begin{Verbatim}[commandchars=\\\{\}]
\PY{c+c1}{NB. make bptest READONLY}
\PY{n+nv}{od} \PY{l+s}{\PYZsq{}}\PY{l+s}{b}\PY{l+s}{p}\PY{l+s}{t}\PY{l+s}{e}\PY{l+s}{s}\PY{l+s}{t}\PY{l+s}{\PYZsq{}} \PY{o}{[} \PY{l+m+mi}{3} \PY{n+nv}{od} \PY{l+s}{\PYZsq{}}\PY{l+s}{\PYZsq{}}
\PY{n+nv}{dpset} \PY{l+s}{\PYZsq{}}\PY{l+s}{R}\PY{l+s}{E}\PY{l+s}{A}\PY{l+s}{D}\PY{l+s}{O}\PY{l+s}{N}\PY{l+s}{L}\PY{l+s}{Y}\PY{l+s}{\PYZsq{}}

\PY{c+c1}{NB. bpcopy is now the put dictionary}
\PY{n+nv}{od} \PY{o}{;}\PY{o}{:}\PY{l+s}{\PYZsq{}}\PY{l+s}{b}\PY{l+s}{p}\PY{l+s}{c}\PY{l+s}{o}\PY{l+s}{p}\PY{l+s}{y}\PY{l+s}{ }\PY{l+s}{b}\PY{l+s}{p}\PY{l+s}{t}\PY{l+s}{e}\PY{l+s}{s}\PY{l+s}{t}\PY{l+s}{\PYZsq{}} \PY{o}{[} \PY{l+m+mi}{3} \PY{n+nv}{od} \PY{l+s}{\PYZsq{}}\PY{l+s}{\PYZsq{}}

\PY{c+c1}{NB. first group/suite sets path}
\PY{n+nv}{grp} \PY{l+s}{\PYZsq{}}\PY{l+s}{a}\PY{l+s}{g}\PY{l+s}{r}\PY{l+s}{o}\PY{l+s}{u}\PY{l+s}{p}\PY{l+s}{\PYZsq{}}\PY{o}{;} \PY{o}{;}\PY{o}{:}\PY{l+s}{\PYZsq{}}\PY{l+s}{d}\PY{l+s}{a}\PY{l+s}{t}\PY{l+s}{e}\PY{l+s}{c}\PY{l+s}{h}\PY{l+s}{e}\PY{l+s}{c}\PY{l+s}{k}\PY{l+s}{ }\PY{l+s}{y}\PY{l+s}{e}\PY{l+s}{a}\PY{l+s}{r}\PY{l+s}{d}\PY{l+s}{a}\PY{l+s}{t}\PY{l+s}{e}\PY{l+s}{s}\PY{l+s}{ }\PY{l+s}{t}\PY{l+s}{o}\PY{l+s}{d}\PY{l+s}{a}\PY{l+s}{y}\PY{l+s}{ }\PY{l+s}{s}\PY{l+s}{u}\PY{l+s}{n}\PY{l+s}{r}\PY{l+s}{i}\PY{l+s}{s}\PY{l+s}{e}\PY{l+s}{s}\PY{l+s}{e}\PY{l+s}{t}\PY{l+s}{0}\PY{l+s}{\PYZsq{}}

\PY{c+c1}{NB. note dictionary path}
\PY{n+nv}{did} \PY{o}{\PYZti{}} \PY{l+m+mi}{0}
\end{Verbatim}
\end{tcolorbox}

    \begin{Verbatim}[commandchars=\\\{\}]
+-+-------------------------------------------------------------+
|1|+------+--+-----+-----+-------+-------+------+--------------+|
| ||      |--|Words|Tests|Groups*|Suites*|Macros|Path*         ||
| |+------+--+-----+-----+-------+-------+------+--------------+|
| ||bpcopy|rw|36   |0    |3      |0      |0     |/bpcopy/bptest||
| |+------+--+-----+-----+-------+-------+------+--------------+|
| ||bptest|ro|36   |0    |2      |0      |0     |/bptest       ||
| |+------+--+-----+-----+-------+-------+------+--------------+|
+-+-------------------------------------------------------------+
    \end{Verbatim}

    \hypertarget{keep-references-updated}{%
\subsubsection{Keep references updated}\label{keep-references-updated}}

JOD stores word references. References enable many useful operations.
References allow (\texttt{getrx}) to load words that call other words in
new contexts.

    \begin{tcolorbox}[breakable, size=fbox, boxrule=1pt, pad at break*=1mm,colback=cellbackground, colframe=cellborder]
\prompt{In}{incolor}{14}{\boxspacing}
\begin{Verbatim}[commandchars=\\\{\}]
\PY{c+c1}{NB. only put dictionary references need updating \PYZhy{} show last five messages}
\PY{l+m+mi}{\PYZus{}5} \PY{o}{\PYZob{}}\PY{o}{.} \PY{l+m+mi}{0} \PY{n+nv}{globs}\PY{o}{\PYZam{}}\PY{o}{\PYZgt{}} \PY{o}{\PYZcb{}}\PY{o}{.} \PY{n+nv}{revo} \PY{l+s}{\PYZsq{}}\PY{l+s}{\PYZsq{}}
\end{Verbatim}
\end{tcolorbox}

    \begin{Verbatim}[commandchars=\\\{\}]
+-+-------------------------------------+------+
|1|<tan> references put in ->           |bpcopy|
+-+-------------------------------------+------+
|1|<today> references put in ->         |bpcopy|
+-+-------------------------------------+------+
|1|<var> references put in ->           |bpcopy|
+-+-------------------------------------+------+
|1|<yeardates> references put in ->     |bpcopy|
+-+-------------------------------------+------+
|1|<NORISESET> is a noun - no references|      |
+-+-------------------------------------+------+
    \end{Verbatim}

    \hypertarget{document-dictionary-objects}{%
\subsubsection{Document dictionary
objects}\label{document-dictionary-objects}}

Documentation is a long standing sore point for programmers. Most of
them hate it. Some claim it's unnecessary and even distracting. Others
put in half baked efforts. In my opinion this ``isn't even wrong!'' Good
documentation elevates code. In
\href{http://www.literateprogramming.com/knuthweb.pdf}{Knuth's opinion}
it separates ``literate programming'' from the alternative -
``illiterate programming.''

JOD provides a number of easy ways to document code. You can enter a
single sentence or a large dissertation. I would recommend the former.
See JOD documentation for more documentation options.

    \begin{tcolorbox}[breakable, size=fbox, boxrule=1pt, pad at break*=1mm,colback=cellbackground, colframe=cellborder]
\prompt{In}{incolor}{15}{\boxspacing}
\begin{Verbatim}[commandchars=\\\{\}]
\PY{c+c1}{NB. for new words try a single line of documentation.}
\PY{n+nv}{afterlaststr}\PY{o}{=:}\PY{o}{]} \PY{o}{\PYZcb{}}\PY{o}{.}\PY{o}{\PYZti{}} \PY{o}{\PYZsh{}}\PY{o}{@}\PY{o}{[} \PY{o}{+} \PY{l+m+mi}{1}\PY{o}{\PYZam{}}\PY{p}{(}\PY{n+nv}{i}\PY{o}{:}\PY{o}{\PYZti{}}\PY{p}{)}\PY{o}{@}\PY{p}{(}\PY{o}{[} \PY{n+nv}{E}\PY{o}{.} \PY{o}{]}\PY{p}{)}
\PY{n+nv}{put} \PY{l+s}{\PYZsq{}}\PY{l+s}{a}\PY{l+s}{f}\PY{l+s}{t}\PY{l+s}{e}\PY{l+s}{r}\PY{l+s}{l}\PY{l+s}{a}\PY{l+s}{s}\PY{l+s}{t}\PY{l+s}{s}\PY{l+s}{t}\PY{l+s}{r}\PY{l+s}{\PYZsq{}}

\PY{c+c1}{NB. insert sentence}
\PY{l+m+mi}{0} \PY{l+m+mi}{8} \PY{n+nv}{put} \PY{l+s}{\PYZsq{}}\PY{l+s}{a}\PY{l+s}{f}\PY{l+s}{t}\PY{l+s}{e}\PY{l+s}{r}\PY{l+s}{l}\PY{l+s}{a}\PY{l+s}{s}\PY{l+s}{t}\PY{l+s}{s}\PY{l+s}{t}\PY{l+s}{r}\PY{l+s}{\PYZsq{}}\PY{o}{;}\PY{l+s}{\PYZsq{}}\PY{l+s}{r}\PY{l+s}{e}\PY{l+s}{t}\PY{l+s}{a}\PY{l+s}{i}\PY{l+s}{n}\PY{l+s}{s}\PY{l+s}{ }\PY{l+s}{s}\PY{l+s}{t}\PY{l+s}{r}\PY{l+s}{i}\PY{l+s}{n}\PY{l+s}{g}\PY{l+s}{ }\PY{l+s}{(}\PY{l+s}{y}\PY{l+s}{)}\PY{l+s}{ }\PY{l+s}{a}\PY{l+s}{f}\PY{l+s}{t}\PY{l+s}{e}\PY{l+s}{r}\PY{l+s}{ }\PY{l+s}{l}\PY{l+s}{a}\PY{l+s}{s}\PY{l+s}{t}\PY{l+s}{ }\PY{l+s}{o}\PY{l+s}{c}\PY{l+s}{c}\PY{l+s}{u}\PY{l+s}{r}\PY{l+s}{r}\PY{l+s}{e}\PY{l+s}{n}\PY{l+s}{c}\PY{l+s}{e}\PY{l+s}{ }\PY{l+s}{o}\PY{l+s}{f}\PY{l+s}{ }\PY{l+s}{(}\PY{l+s}{x}\PY{l+s}{)}\PY{l+s}{\PYZsq{}}
\end{Verbatim}
\end{tcolorbox}

    \begin{Verbatim}[commandchars=\\\{\}]
+-+-------------------------------+------+
|1|1 word explanation(s) put in ->|bpcopy|
+-+-------------------------------+------+
    \end{Verbatim}

    JOD uses documentation in new contexts.

    \begin{tcolorbox}[breakable, size=fbox, boxrule=1pt, pad at break*=1mm,colback=cellbackground, colframe=cellborder]
\prompt{In}{incolor}{16}{\boxspacing}
\begin{Verbatim}[commandchars=\\\{\}]
\PY{c+c1}{NB. for tacits the docment sentence is fetched}
\PY{n+nv}{smoutput} \PY{n+nv}{disp} \PY{l+s}{\PYZsq{}}\PY{l+s}{a}\PY{l+s}{f}\PY{l+s}{t}\PY{l+s}{e}\PY{l+s}{r}\PY{l+s}{l}\PY{l+s}{a}\PY{l+s}{s}\PY{l+s}{t}\PY{l+s}{s}\PY{l+s}{t}\PY{l+s}{r}\PY{l+s}{\PYZsq{}}

\PY{c+c1}{NB. briefly describe sunmoon group  \PYZhy{} the payback for entering those sentences}
\PY{n+nv}{sbx} \PY{n+nv}{hlpnl} \PY{o}{\PYZcb{}}\PY{o}{.} \PY{n+nv}{grp} \PY{l+s}{\PYZsq{}}\PY{l+s}{s}\PY{l+s}{u}\PY{l+s}{n}\PY{l+s}{m}\PY{l+s}{o}\PY{l+s}{o}\PY{l+s}{n}\PY{l+s}{\PYZsq{}}
\end{Verbatim}
\end{tcolorbox}

    \begin{Verbatim}[commandchars=\\\{\}]
NB. retains string (y) after last occurrence of (x)
afterlaststr=:] \}.\textasciitilde{} \#@[ + 1\&(i:\textasciitilde{})@([ E. ])
+-----------+------------------------------------------------------------ {\ldots}
|NORISESET  |indicates sun never rises or sets in (sunriseset0) and (sunr {\ldots}
|arctan     |arc tangent                                                  {\ldots}
|calmoons   |calendar dates of new and full moons                         {\ldots}
|cos        |cosine radians                                               {\ldots}
|datecheck  |checks dates in YYYY MM DD format                            {\ldots}
|fromjulian |converts Julian day numbers to dates, converse (tojulian)    {\ldots}
|moons      |times of new and full moons for n calendar years             {\ldots}
|round      |round y to nearest x (e.g. 1000 round 12345)                 {\ldots}
|sin        |sine radians                                                 {\ldots}
|sunriseset0|computes sun rise and set times - see long documentation     {\ldots}
|sunriseset1|computes sun rise and set times - see long documentation     {\ldots}
|tabit      |promotes only atoms and lists to tables                      {\ldots}
|tan        |tan radians                                                  {\ldots}
|today      |returns todays date                                          {\ldots}
|yeardates  |returns all valid dates for n calendar years                 {\ldots}
+-----------+------------------------------------------------------------ {\ldots}
    \end{Verbatim}

    An example of long documentation.

    \begin{tcolorbox}[breakable, size=fbox, boxrule=1pt, pad at break*=1mm,colback=cellbackground, colframe=cellborder]
\prompt{In}{incolor}{17}{\boxspacing}
\begin{Verbatim}[commandchars=\\\{\}]
\PY{c+c1}{NB. long document}
\PY{n+nv}{stx} \PY{l+m+mi}{0} \PY{l+m+mi}{9} \PY{n+nv}{disp} \PY{l+s}{\PYZsq{}}\PY{l+s}{s}\PY{l+s}{u}\PY{l+s}{n}\PY{l+s}{r}\PY{l+s}{i}\PY{l+s}{s}\PY{l+s}{e}\PY{l+s}{s}\PY{l+s}{e}\PY{l+s}{t}\PY{l+s}{0}\PY{l+s}{\PYZsq{}}
\end{Verbatim}
\end{tcolorbox}

    \begin{Verbatim}[commandchars=\\\{\}]
*sunriseset0 v-- sunrise and sunset times.                                {\ldots}
                                                                          {\ldots}
This  verb has been adapted from a BASIC program submitted by             {\ldots}
Robin  G.  Stuart  Sky  \&  Telescope's  shortest  sunrise/set             {\ldots}
program  contest. Winning  entries were listed  in the  March             {\ldots}
1995 Astronomical Computing column.                                       {\ldots}
                                                                          {\ldots}
The  J version of this algorithm has been vectorized.  It can             {\ldots}
compute any number of sunrise and sunset times in one call.               {\ldots}
                                                                          {\ldots}
NB. verbatim:                                                             {\ldots}
                                                                          {\ldots}
    \end{Verbatim}

    \hypertarget{define-your-own-jod-shortcuts}{%
\subsubsection{Define your own JOD
shortcuts}\label{define-your-own-jod-shortcuts}}

JOD words can be used within arbitrary J programs. If you don't find a
JOD primitive that meets your needs do a little programming.

There are many examples of JOD programming in JOD's source code. Install
the \texttt{jodsource} addon to get JOD source code.

    \begin{tcolorbox}[breakable, size=fbox, boxrule=1pt, pad at break*=1mm,colback=cellbackground, colframe=cellborder]
\prompt{In}{incolor}{18}{\boxspacing}
\begin{Verbatim}[commandchars=\\\{\}]
\PY{c+c1}{NB. examples of using JOD words to define new facilities}

\PY{c+c1}{NB. describe a JOD group}
\PY{n+nv}{hg\PYZus{}ijod\PYZus{}}\PY{o}{=:} \PY{o}{[}\PY{o}{:} \PY{n+nv}{hlpnl} \PY{o}{[}\PY{o}{:} \PY{o}{\PYZcb{}}\PY{o}{.} \PY{n+nv}{grp}

\PY{c+c1}{NB. re\PYZhy{}reference put dictionary show any errors}
\PY{n+nv}{reref\PYZus{}ijod\PYZus{}}\PY{o}{=:} \PY{l+m+mi}{3} \PY{o}{:} \PY{l+s}{\PYZsq{}}\PY{l+s}{(}\PY{l+s}{n}\PY{l+s}{,}\PY{l+s}{.}\PY{l+s}{s}\PY{l+s}{)}\PY{l+s}{ }\PY{l+s}{\PYZsh{}}\PY{l+s}{\PYZti{}}\PY{l+s}{ }\PY{l+s}{\PYZhy{}}\PY{l+s}{.}\PY{l+s}{;}\PY{l+s}{0}\PY{l+s}{\PYZob{}}\PY{l+s}{\PYZdq{}}\PY{l+s}{1}\PY{l+s}{ }\PY{l+s}{s}\PY{l+s}{=}\PY{l+s}{.}\PY{l+s}{0}\PY{l+s}{ }\PY{l+s}{g}\PY{l+s}{l}\PY{l+s}{o}\PY{l+s}{b}\PY{l+s}{s}\PY{l+s}{\PYZam{}}\PY{l+s}{\PYZgt{}}\PY{l+s}{n}\PY{l+s}{=}\PY{l+s}{.}\PY{l+s}{\PYZcb{}}\PY{l+s}{.}\PY{l+s}{r}\PY{l+s}{e}\PY{l+s}{v}\PY{l+s}{o}\PY{l+s}{\PYZsq{}\PYZsq{}}\PY{l+s}{\PYZsq{}\PYZsq{}}\PY{l+s}{ }\PY{l+s}{[}\PY{l+s}{ }\PY{l+s}{y}\PY{l+s}{\PYZsq{}}

\PY{c+c1}{NB. show words referenced by words in a group that are not in the group}
\PY{n+nv}{jodg\PYZus{}ijod\PYZus{}}\PY{o}{=:} \PY{l+s}{\PYZsq{}}\PY{l+s}{a}\PY{l+s}{g}\PY{l+s}{r}\PY{l+s}{o}\PY{l+s}{u}\PY{l+s}{p}\PY{l+s}{\PYZsq{}}
\PY{n+nv}{nx\PYZus{}ijod\PYZus{}}\PY{o}{=:} \PY{l+m+mi}{3} \PY{o}{:} \PY{l+s}{\PYZsq{}}\PY{l+s}{(}\PY{l+s}{a}\PY{l+s}{l}\PY{l+s}{l}\PY{l+s}{r}\PY{l+s}{e}\PY{l+s}{f}\PY{l+s}{s}\PY{l+s}{ }\PY{l+s}{ }\PY{l+s}{\PYZcb{}}\PY{l+s}{.}\PY{l+s}{ }\PY{l+s}{g}\PY{l+s}{n}\PY{l+s}{)}\PY{l+s}{ }\PY{l+s}{\PYZhy{}}\PY{l+s}{.}\PY{l+s}{ }\PY{l+s}{g}\PY{l+s}{n}\PY{l+s}{=}\PY{l+s}{.}\PY{l+s}{ }\PY{l+s}{g}\PY{l+s}{r}\PY{l+s}{p}\PY{l+s}{ }\PY{l+s}{j}\PY{l+s}{o}\PY{l+s}{d}\PY{l+s}{g}\PY{l+s}{\PYZsq{}}

\PY{c+c1}{NB. missing from (agroup)}
\PY{n+nv}{nx} \PY{l+s}{\PYZsq{}}\PY{l+s}{\PYZsq{}}
\end{Verbatim}
\end{tcolorbox}

    \begin{Verbatim}[commandchars=\\\{\}]
+---------+------+---+---+-----+---+
|NORISESET|arctan|cos|sin|tabit|tan|
+---------+------+---+---+-----+---+
    \end{Verbatim}

    \hypertarget{customize-jod-edit-facilities}{%
\subsubsection{Customize JOD edit
facilities}\label{customize-jod-edit-facilities}}

The main JOD edit words (\texttt{nw}) and (\texttt{ed}) can be
customized by defining a \texttt{DOCUMENTCOMMAND} script.

\emph{Note: Verbs that spawn J editors may not work in Jupyter labs. The
following (\texttt{nw}) call opens a JQT or JHS editor in standard J
front ends but does nothing here. This is because the J kernal is
essentially a barebones \texttt{jconsole.exe} process that is is not
running JQT.}

    \begin{tcolorbox}[breakable, size=fbox, boxrule=1pt, pad at break*=1mm,colback=cellbackground, colframe=cellborder]
\prompt{In}{incolor}{19}{\boxspacing}
\begin{Verbatim}[commandchars=\\\{\}]
\PY{c+c1}{NB. define document command script \PYZhy{} \PYZob{}\PYZti{}N\PYZti{}\PYZcb{} is word name placeholder}
\PY{n+nv}{DOCUMENTCOMMAND\PYZus{}ijod\PYZus{}}\PY{o}{=:} \PY{n+ni}{0 : 0}
\PY{l+s}{s}\PY{l+s}{m}\PY{l+s}{o}\PY{l+s}{u}\PY{l+s}{t}\PY{l+s}{p}\PY{l+s}{u}\PY{l+s}{t}\PY{l+s}{ }\PY{l+s}{p}\PY{l+s}{r}\PY{l+s}{ }\PY{l+s}{\PYZsq{}}\PY{l+s}{\PYZob{}}\PY{l+s}{\PYZti{}}\PY{l+s}{N}\PY{l+s}{\PYZti{}}\PY{l+s}{\PYZcb{}}\PY{l+s}{\PYZsq{}}
\PY{n+nl}{)}

\PY{c+c1}{NB. create a new word \PYZhy{} opens an edit window in JQT and JHS }
\PY{c+c1}{NB. nw \PYZsq{}bpword\PYZsq{}}
\end{Verbatim}
\end{tcolorbox}

    Find the (\texttt{bpword}) edit window and note how
\texttt{DOCUMENTCOMMAND} text has been modified and insert. When the
script window is saved with CRTL-W \texttt{DOCUMENTCOMMAND} runs.

Run and close the (\texttt{bpword}) edit window.

Edit a word in the dictionary. \textbf{\emph{JOD is always editing
copies.}} Originals can only be changed with explicit (\texttt{put})
operations.

    \begin{tcolorbox}[breakable, size=fbox, boxrule=1pt, pad at break*=1mm,colback=cellbackground, colframe=cellborder]
\prompt{In}{incolor}{20}{\boxspacing}
\begin{Verbatim}[commandchars=\\\{\}]
\PY{c+c1}{NB. load dictionary word into edit window \PYZhy{} requires JQT or JHS front ends}
\PY{c+c1}{NB. also requires browser permissions and pop ups enabled}

\PY{c+c1}{NB. ed \PYZsq{}sunriseset0\PYZsq{}}

\PY{c+c1}{NB. find the (sunriseset0) edit window and note DOCUMENTCOMMAND }
\PY{c+c1}{NB. close edit window}
\end{Verbatim}
\end{tcolorbox}

    \hypertarget{define-jod-project-macros}{%
\subsubsection{Define JOD project
macros}\label{define-jod-project-macros}}

When programming with JOD you typically open a set of dictionaries. Load
system scripts and define some handy nouns. This can be done in a
project macro script.

    \begin{tcolorbox}[breakable, size=fbox, boxrule=1pt, pad at break*=1mm,colback=cellbackground, colframe=cellborder]
\prompt{In}{incolor}{21}{\boxspacing}
\begin{Verbatim}[commandchars=\\\{\}]
\PY{c+c1}{NB. define a project macro \PYZhy{} I use the prefix prj for such scripts}
\PY{n+nv}{prjsunmoon}\PY{o}{=:} \PY{n+ni}{0 : 0}

\PY{l+s}{N}\PY{l+s}{B}\PY{l+s}{.}\PY{l+s}{ }\PY{l+s}{s}\PY{l+s}{t}\PY{l+s}{a}\PY{l+s}{n}\PY{l+s}{d}\PY{l+s}{a}\PY{l+s}{r}\PY{l+s}{d}\PY{l+s}{ }\PY{l+s}{j}\PY{l+s}{ }\PY{l+s}{s}\PY{l+s}{c}\PY{l+s}{r}\PY{l+s}{i}\PY{l+s}{p}\PY{l+s}{t}\PY{l+s}{s}
\PY{l+s}{r}\PY{l+s}{e}\PY{l+s}{q}\PY{l+s}{u}\PY{l+s}{i}\PY{l+s}{r}\PY{l+s}{e}\PY{l+s}{ }\PY{l+s}{\PYZsq{}}\PY{l+s}{d}\PY{l+s}{e}\PY{l+s}{b}\PY{l+s}{u}\PY{l+s}{g}\PY{l+s}{ }\PY{l+s}{t}\PY{l+s}{a}\PY{l+s}{s}\PY{l+s}{k}\PY{l+s}{\PYZsq{}}

\PY{l+s}{N}\PY{l+s}{B}\PY{l+s}{.}\PY{l+s}{ }\PY{l+s}{l}\PY{l+s}{o}\PY{l+s}{c}\PY{l+s}{a}\PY{l+s}{l}\PY{l+s}{ }\PY{l+s}{s}\PY{l+s}{c}\PY{l+s}{r}\PY{l+s}{i}\PY{l+s}{p}\PY{l+s}{t}\PY{l+s}{ }\PY{l+s}{n}\PY{l+s}{o}\PY{l+s}{u}\PY{l+s}{n}\PY{l+s}{s}\PY{l+s}{ }
\PY{l+s}{j}\PY{l+s}{o}\PY{l+s}{d}\PY{l+s}{g}\PY{l+s}{\PYZus{}}\PY{l+s}{i}\PY{l+s}{j}\PY{l+s}{o}\PY{l+s}{d}\PY{l+s}{\PYZus{}}\PY{l+s}{=}\PY{l+s}{:}\PY{l+s}{ }\PY{l+s}{\PYZsq{}}\PY{l+s}{s}\PY{l+s}{u}\PY{l+s}{n}\PY{l+s}{m}\PY{l+s}{o}\PY{l+s}{o}\PY{l+s}{n}\PY{l+s}{\PYZsq{}}
\PY{l+s}{j}\PY{l+s}{o}\PY{l+s}{d}\PY{l+s}{s}\PY{l+s}{\PYZus{}}\PY{l+s}{i}\PY{l+s}{j}\PY{l+s}{o}\PY{l+s}{d}\PY{l+s}{\PYZus{}}\PY{l+s}{=}\PY{l+s}{:}\PY{l+s}{ }\PY{l+s}{\PYZsq{}}\PY{l+s}{s}\PY{l+s}{u}\PY{l+s}{n}\PY{l+s}{m}\PY{l+s}{o}\PY{l+s}{o}\PY{l+s}{n}\PY{l+s}{t}\PY{l+s}{e}\PY{l+s}{s}\PY{l+s}{t}\PY{l+s}{s}\PY{l+s}{\PYZsq{}}

\PY{l+s}{N}\PY{l+s}{B}\PY{l+s}{.}\PY{l+s}{ }\PY{l+s}{p}\PY{l+s}{u}\PY{l+s}{t}\PY{l+s}{/}\PY{l+s}{x}\PY{l+s}{r}\PY{l+s}{e}\PY{l+s}{f}
\PY{l+s}{D}\PY{l+s}{O}\PY{l+s}{C}\PY{l+s}{U}\PY{l+s}{M}\PY{l+s}{E}\PY{l+s}{N}\PY{l+s}{T}\PY{l+s}{C}\PY{l+s}{O}\PY{l+s}{M}\PY{l+s}{M}\PY{l+s}{A}\PY{l+s}{N}\PY{l+s}{D}\PY{l+s}{\PYZus{}}\PY{l+s}{z}\PY{l+s}{\PYZus{}}\PY{l+s}{ }\PY{l+s}{=}\PY{l+s}{:}\PY{l+s}{ }\PY{l+s}{\PYZsq{}}\PY{l+s}{s}\PY{l+s}{m}\PY{l+s}{o}\PY{l+s}{u}\PY{l+s}{t}\PY{l+s}{p}\PY{l+s}{u}\PY{l+s}{t}\PY{l+s}{ }\PY{l+s}{p}\PY{l+s}{r}\PY{l+s}{ }\PY{l+s}{\PYZsq{}}\PY{l+s}{\PYZsq{}}\PY{l+s}{\PYZob{}}\PY{l+s}{\PYZti{}}\PY{l+s}{N}\PY{l+s}{\PYZti{}}\PY{l+s}{\PYZcb{}}\PY{l+s}{\PYZsq{}}\PY{l+s}{\PYZsq{}}\PY{l+s}{\PYZsq{}}
\PY{n+nl}{)}

\PY{c+c1}{NB. store macro}
\PY{l+m+mi}{4} \PY{n+nv}{put} \PY{l+s}{\PYZsq{}}\PY{l+s}{p}\PY{l+s}{r}\PY{l+s}{j}\PY{l+s}{s}\PY{l+s}{u}\PY{l+s}{n}\PY{l+s}{m}\PY{l+s}{o}\PY{l+s}{o}\PY{l+s}{n}\PY{l+s}{\PYZsq{}}\PY{o}{;}\PY{n+nv}{JSCRIPT\PYZus{}ajod\PYZus{}}\PY{o}{;}\PY{n+nv}{prjsunmoon}
\end{Verbatim}
\end{tcolorbox}

    \begin{Verbatim}[commandchars=\\\{\}]
+-+--------------------+------+
|1|1 macro(s) put in ->|bpcopy|
+-+--------------------+------+
    \end{Verbatim}

    Once project macros are defined it's easy to configure your J session.
Open the required dictionaries and run the macor with (\texttt{rm}).

    \begin{tcolorbox}[breakable, size=fbox, boxrule=1pt, pad at break*=1mm,colback=cellbackground, colframe=cellborder]
\prompt{In}{incolor}{22}{\boxspacing}
\begin{Verbatim}[commandchars=\\\{\}]
\PY{c+c1}{NB. setup project}
\PY{n+nv}{rm} \PY{l+s}{\PYZsq{}}\PY{l+s}{p}\PY{l+s}{r}\PY{l+s}{j}\PY{l+s}{s}\PY{l+s}{u}\PY{l+s}{n}\PY{l+s}{m}\PY{l+s}{o}\PY{l+s}{o}\PY{l+s}{n}\PY{l+s}{\PYZsq{}}  \PY{o}{[} \PY{n+nv}{od} \PY{o}{;}\PY{o}{:}\PY{l+s}{\PYZsq{}}\PY{l+s}{b}\PY{l+s}{p}\PY{l+s}{c}\PY{l+s}{o}\PY{l+s}{p}\PY{l+s}{y}\PY{l+s}{ }\PY{l+s}{b}\PY{l+s}{p}\PY{l+s}{t}\PY{l+s}{e}\PY{l+s}{s}\PY{l+s}{t}\PY{l+s}{\PYZsq{}} \PY{o}{[} \PY{l+m+mi}{3} \PY{n+nv}{od} \PY{l+s}{\PYZsq{}}\PY{l+s}{\PYZsq{}}
\end{Verbatim}
\end{tcolorbox}

    \begin{Verbatim}[commandchars=\\\{\}]
   NB. standard j scripts
   require 'debug task'
   NB. local script nouns
   jodg\_ijod\_=: 'sunmoon'
   jods\_ijod\_=: 'sunmoontests'
   NB. put/xref
   DOCUMENTCOMMAND\_z\_ =: 'smoutput pr ''\{\textasciitilde{}N\textasciitilde{}\}'''

    \end{Verbatim}

    \hypertarget{edit-your-jodprofile.ijs}{%
\subsubsection{\texorpdfstring{Edit your
\texttt{jodprofile.ijs}}{Edit your jodprofile.ijs}}\label{edit-your-jodprofile.ijs}}

When JOD loads the script
(\texttt{\textasciitilde{}addons/general/jod/jodprofile.ijs}) is
executed. This script can be used to set up JOD the way you want. Note
how you can execute project macros when JOD loads with sentences like:

\begin{verbatim}
rm 'prjsunmoon' [ od ;:'bpcopy bptest'
\end{verbatim}

WARNING: store your profile in one of your dictionaries. This file is
reset when JAL updates the JOD addon.

    \begin{tcolorbox}[breakable, size=fbox, boxrule=1pt, pad at break*=1mm,colback=cellbackground, colframe=cellborder]
\prompt{In}{incolor}{23}{\boxspacing}
\begin{Verbatim}[commandchars=\\\{\}]
\PY{c+c1}{NB. display jodprofile.ijs}
\PY{n+nv}{stx} \PY{n+nv}{read\PYZus{}ajod\PYZus{}} \PY{n+nv}{jpath}\PY{l+s}{\PYZsq{}}\PY{l+s}{\PYZti{}}\PY{l+s}{a}\PY{l+s}{d}\PY{l+s}{d}\PY{l+s}{o}\PY{l+s}{n}\PY{l+s}{s}\PY{l+s}{/}\PY{l+s}{g}\PY{l+s}{e}\PY{l+s}{n}\PY{l+s}{e}\PY{l+s}{r}\PY{l+s}{a}\PY{l+s}{l}\PY{l+s}{/}\PY{l+s}{j}\PY{l+s}{o}\PY{l+s}{d}\PY{l+s}{/}\PY{l+s}{j}\PY{l+s}{o}\PY{l+s}{d}\PY{l+s}{p}\PY{l+s}{r}\PY{l+s}{o}\PY{l+s}{f}\PY{l+s}{i}\PY{l+s}{l}\PY{l+s}{e}\PY{l+s}{.}\PY{l+s}{i}\PY{l+s}{j}\PY{l+s}{s}\PY{l+s}{\PYZsq{}}
\end{Verbatim}
\end{tcolorbox}

    \begin{Verbatim}[commandchars=\\\{\}]
NB.*jodprofile s-- JOD dictionary profile.                                {\ldots}
NB.                                                                       {\ldots}
NB. An example JOD profile script. Save this script in                    {\ldots}
NB.                                                                       {\ldots}
NB. \textasciitilde{}addons/general/jod/                                                  {\ldots}
NB.                                                                       {\ldots}
NB. with the name jodprofile.ijs                                          {\ldots}
NB.                                                                       {\ldots}
NB. This script  is  executed  after all dictionary  objects have         {\ldots}
NB. been created. It  can  be used  to  set up  your default  JOD         {\ldots}
NB. working environment.                                                  {\ldots}
NB.                                                                       {\ldots}
    \end{Verbatim}

    \hypertarget{use-jod-help-and-documentation}{%
\subsubsection{Use JOD help and
documentation}\label{use-jod-help-and-documentation}}

Install the (\texttt{general/joddocument}) addon to use JOD PDF help.

    \begin{tcolorbox}[breakable, size=fbox, boxrule=1pt, pad at break*=1mm,colback=cellbackground, colframe=cellborder]
\prompt{In}{incolor}{24}{\boxspacing}
\begin{Verbatim}[commandchars=\\\{\}]
\PY{c+c1}{NB. location of jod.pdf \PYZhy{} install addon (general/joddocument) with JAL}
\PY{n+nv}{smoutput} \PY{n+nv}{jpath} \PY{l+s}{\PYZsq{}}\PY{l+s}{\PYZti{}}\PY{l+s}{a}\PY{l+s}{d}\PY{l+s}{d}\PY{l+s}{o}\PY{l+s}{n}\PY{l+s}{s}\PY{l+s}{/}\PY{l+s}{g}\PY{l+s}{e}\PY{l+s}{n}\PY{l+s}{e}\PY{l+s}{r}\PY{l+s}{a}\PY{l+s}{l}\PY{l+s}{/}\PY{l+s}{j}\PY{l+s}{o}\PY{l+s}{d}\PY{l+s}{d}\PY{l+s}{o}\PY{l+s}{c}\PY{l+s}{u}\PY{l+s}{m}\PY{l+s}{e}\PY{l+s}{n}\PY{l+s}{t}\PY{l+s}{/}\PY{l+s}{p}\PY{l+s}{d}\PY{l+s}{f}\PY{l+s}{d}\PY{l+s}{o}\PY{l+s}{c}\PY{l+s}{/}\PY{l+s}{j}\PY{l+s}{o}\PY{l+s}{d}\PY{l+s}{.}\PY{l+s}{p}\PY{l+s}{d}\PY{l+s}{f}\PY{l+s}{\PYZsq{}}

\PY{c+c1}{NB. opens jod.pdf if a pdf reader is available on your system}
\PY{n+nv}{jodhelp} \PY{l+m+mi}{0}
\end{Verbatim}
\end{tcolorbox}

    \begin{Verbatim}[commandchars=\\\{\}]
c:/j64/j903/addons/general/joddocument/pdfdoc/jod.pdf
+-+-------------------+
|1|starting PDF reader|
+-+-------------------+
    \end{Verbatim}

    \hypertarget{summary-and-final-remarks}{%
\subsubsection{Summary and final
remarks}\label{summary-and-final-remarks}}

\begin{enumerate}
\def\labelenumi{\arabic{enumi}.}
\item
  JOD DOES NOT BELONG IN THE J TREE
\item
  BACKUP BACKUP BACKUP
\item
  TAKE A SCRIPT DUMP
\item
  MAKE A MASTER RE-REGISTER SCRIPT
\item
  SET LIBRARY DICTIONARIES TO READONLY
\item
  KEEP REFERENCES UPDATED
\item
  DOCUMENT DICTIONARY OBJECTS
\item
  DEFINE YOUR OWN JOD SHORTCUTS
\item
  CUSTOMIZE JOD EDIT FACILITIES
\item
  DEFINE JOD PROJECT MACROS
\item
  EDIT YOUR \texttt{jodprofile.ijs}
\item
  USE JOD HELP AND DOCUMENTATION
\end{enumerate}

These are just some of the JOD practices I have found useful. As you use
JOD you will probably find your own methods.

If you find an a useful method please let me know. I can be reached at:

\begin{verbatim}
John Baker 
December 2021
bakerjd99@gmail.com
\end{verbatim}


    % Add a bibliography block to the postdoc
    
    
    
\end{document}
